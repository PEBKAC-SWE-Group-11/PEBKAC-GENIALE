
\documentclass[12pt, a4paper]{article}

\usepackage{graphicx}
\usepackage{xcolor}
\usepackage{float}
\usepackage{svg}
\usepackage[colorlinks=true, linkcolor=black, urlcolor=blue, citecolor=green]{hyperref}
\usepackage{enumitem}
\usepackage[italian]{babel}
\usepackage{lastpage}  % Pacchetto per ottenere il numero totale delle pagine
\usepackage{fancyhdr}  % Pacchetto per personalizzare l'intestazione e il piè di pagina
\usepackage{tabularx}
\graphicspath{ {images/} {../shared/images/} }
\definecolor{unipd}{HTML}{B5121B}

\addto\captionsitalian{\renewcommand{\contentsname}{Indice}}

\setcounter{secnumdepth}{5}
\setcounter{tocdepth}{5}
\makeatletter
\newcommand\subsubsubsection{\@startsection{paragraph}{4}{\z@}{-2.5ex\@plus -1ex \@minus -.25ex}{1.25ex \@plus .25ex}{\normalfont\normalsize\bfseries}}
\newcommand\subsubsubsubsection{\@startsection{subparagraph}{5}{\z@}{-2.5ex\@plus -1ex \@minus -.25ex}{1.25ex \@plus .25ex}{\normalfont\normalsize\bfseries}}
\makeatother

\pagestyle{fancy}% Imposta lo stile di pagina su "fancy"
\fancyhf{}% Cancella intestazioni e piè di pagina
\fancyfoot[C]{\thepage{} di \pageref{LastPage}} % Imposta il piè di pagina centrale come "numero pagina di totale pagine"
\renewcommand{\headrulewidth}{0pt} % Imposta la larghezza della linea di intestazione a 0 punti

\newcommand{\data}{}
\newcommand{\titolo}{Norme di progetto}
% VERBALE
%\textbf{Responsabile} &  \responsabile\\ 
%\textbf{Verificatore} &  \verificatore\\ 
%\textbf{Redattore} &     \redattore\\ 
\newcommand{\uso}{Interno}
\newcommand{\destinatari }{
   %& Vimar S.p.A.  \\
    & Tullio Vardanega  \\
    & Riccardo Cardin  }
\newcommand{\abstractcontent}{ ... }

\begin{document}

\begin{minipage}[]{0.3\textwidth}
\includesvg[width=\linewidth]{pebkac.svg} 
\end{minipage}
\hspace{0.1\textwidth}
\begin{minipage}[]{0.6\textwidth}
  {\Large \textbf{PEBKAC}} \\
  Email: \href{mailto:pebkacswe@gmail.com}{pebkacswe@gmail.com} \\
  Gruppo: 11
\end{minipage}

\bigskip

\begin{minipage}[]{0.3\textwidth}
\includesvg[width=\linewidth]{logo_unipd.svg} 
\end{minipage}
\hspace{0.1\textwidth}
\begin{minipage}[]{0.6\textwidth}
  \textcolor{unipd}{
    \textbf{Università degli Studi di Padova} \\
    Corso di Laurea: Informatica \\
    Corso: Ingegneria del Software \\
    Anno Accademico: 2024/2025
  }
\end{minipage}


\bigskip
\bigskip
\bigskip
\begin{center}
  \Huge\textbf{Verbale Interno}

  \Large\textbf{\data}
\end{center}

\bigskip


\begin{center}
\textbf{Informazioni sul documento}: \\
\vspace{0.5cm}

\begin{tabular}{r|l}
    \textbf{Responsabile} & Tommaso Zocche \\ 
    \textbf{Verificatore} & Alessandro Benin \\ 
    \textbf{Redattore} & Tommaso Zocche \\ 
    \textbf{Uso} & Interno \\ 
    \textbf{Destinatari} & Tullio Vardanega \\ & Riccardo Cardin \\ 
\end{tabular}

\vfill

\textbf{Abstract}: \\
\vspace{0.5cm}
L'obiettivo dell'incontro è stato definire l'ordine di preferenza dei capitolati a seguito degli incontri avuti con le aziende interessate e iniziare a redigere il prospetto orario del gruppo.
\end{center}


\bigskip
\newpage

\section*{Registro delle modifiche}
\begin{table}[H]
    \begin{tabular}{|c|c|c|c|p{5cm}|}
        \hline
         \textbf{Versione} &  \textbf{Data} &  \textbf{Autore} &  \textbf{Ruolo} & \textbf{Descrizione} \\
          \hline
          &  &  & Responsabile & Approvazione e rilascio\\
          \hline
          0.1.0 & 17/11/2024 & Alessandro Benin & Verificatore  & Verificato \\
          \hline
          0.0.1 & 13/11/2024 & Derek Gusatto & Amministratore  & Stesura iniziale \\
          \hline
    \end{tabular}
\end{table}
\newpage
\tableofcontents
\newpage
\listoffigures 
\newpage
\listoftables
\newpage
\section{Introduzione}
\subsection{Scopo del documento}
Il presente documento ha l’obiettivo di definire le \textit{best practices}\textsubscript{G}  e il \textit{way of working}\textsubscript{G} che i componente del team \textit{PEBKAC} hanno l’obbligo di rispettare per l’intero svolgimento del progetto. L'intento è di garantire un metodo di lavoro omogeneo, verificabile
e migliorabile nel tempo. La creazione delle norme è progressiva e incrementale nel tempo per consentire al team di apportare aggiornamenti continui in risposta alle esigenze e alle problematiche incorse durante lo sviluppo dell'intero progetto.

\subsection{Scopo del prodotto}
Il progetto "Vimar GENIALE" mira a sviluppare un'applicazione intelligente che supporti installatori elettrici nell'uso di dispositivi Vimar\textsubscript{G}, facilitando l'accesso alle informazioni tecniche sui prodotti, rispondendo a domande poste in linguaggio naturale.
La tecnologia alla base prevede l'uso di modelli di \textit{LLM}\textsubscript{G} e di tecniche \textit{RAG}\textsubscript{G}, con una struttura di gestione basata su container e integrata in un ambiente cloud.
Il sistema include tre componenti principali: una \textit{applicativo web responsive}\textsubscript{G}, un \textit{applicativo server}\textsubscript{G} e un'\textit{infrastruttura cloud-ready}\textsubscript{G}. 
\subsection{Glossario}
Per evitare ambiguità relative al linguaggio utilizzato nei documenti, viene fornito il Glossario V1.0.0, nel quale si possono trovare tutte le definizioni di termini che hanno un significato specifico che vuole essere disambiguato. Tali termini sono marcati con una G a pedice. 
\subsection{Riferimenti}
\subsubsection{Riferimenti normativi}
\begin{itemize}
    \item Regolamento del progetto didattico\\ \href{https://www.math.unipd.it/~tullio/IS-1/2024/Dispense/PD1.pdf}{https://www.math.unipd.it/~tullio/IS-1/2024/Dispense/PD1.pdf} \\ (Ultimo accesso 2024-11-14)
    \item ISO/IEC 12207:1995 Information technology - Software life cycle processes \\ \href{https://www.math.unipd.it/~tullio/IS-1/2010/Approfondimenti/A03.pdf}{https://www.math.unipd.it/~tullio/IS-1/2010/Approfondimenti/A03.pdf}\\ (Ultimo accesso 2024-11-14)

\end{itemize}

\subsubsection{Riferimenti informativi}
\begin{itemize}
    \item Capitolato C2 \\ \href{https://www.math.unipd.it/~tullio/IS-1/2024/Dispense/PD1.pdf}{https://www.math.unipd.it/~tullio/IS-1/2024/Dispense/PD1.pdf}\\ (Ultimo accesso 2024-11-14)
    \item Capitolato C2 - slides \\ \href{https://www.math.unipd.it/~tullio/IS-1/2024/Dispense/PD1.pdf}{https://www.math.unipd.it/~tullio/IS-1/2024/Dispense/PD1.pdf}\\ (Ultimo accesso 2024-11-14)
    \item Documentazione\textsubscript{G} GitHub\textsubscript{G} \\ \href{https://docs.github.com/en}{https://docs.github.com/en}\\ (Ultimo accesso 2024-11-14)
    
\end{itemize}
\newpage
\section{Standard ISO/IEC 12207:1995}
Il gruppo ha deciso di applicare nelle proprie modalità di lavoro e quindi nel presente documento di adottare lo \textit{standard ISO/IEC 12207:1995 Information technology - Software life cycle processes}. In questa sezione del documento si possono trovare i criteri di applicazione e i processi definiti nell'ambito di questo standard.
\subsection{Processi del ciclo di vita del software}
Questo documento è usato per normare il \textit{way of working}\textsubscript{G} del gruppo, in particolare l'organizzazione dei processi del ciclo di vita del software\textsubscript{G} secondo lo \textit{standard ISO/IEC 12207:1995 Information technology - Software life cycle processes}, questi sono organizzati in una organizzazione gerarchica in cui ogni processo\textsubscript{G} è costituitido da un insieme di attività, le quali possono prevedere delle procedure e un elenco di strumenti usati per lo svolgimento.
\subsubsection{Processi primari}
Lo standard adottato presenta cinque processi primari (\textit{Acquisizione, Fornitura, Sviluppo, Operazione, Manutenzione}), ma all’interno del contesto del progetto universitario in atto, il gruppo non ritiene i processi di \textit{Operazione} e \textit{Manutenzione} pertinendi, mentre il processo di \textit{Acquisizione} è di competenza del committente, pertanto, il gruppo decide di escluderli dalla presentazione nel documento.
I processi primari presentati nel presente documento sono:
\begin{itemize}
    \item \textbf{Fornitura}: definisce le attività del fornitore, l’organizzazione che fornisce il prodotto software all’acquirente;
    \item \textbf{Sviluppo}: definisce le attività dello sviluppatore, l’organizzazione che definisce e sviluppa il prodotto software.
\end{itemize}

\subsubsection{Processi di supporto}
I processi di supporto presentati nel presente documento sono:
\begin{itemize}
\item \textbf{Documentazione}\textsubscript{G}: definisce le attività per la registrazione delle informazioni prodotte da un processo del ciclo di vita;
\item \textbf{Configuration Management}\textsubscript{G}: definisce le attività di gestione della configurazione;
\item \textbf{Accertamento di qualità}: definisce le attività per assicurare in modo oggettivo che i prodotti e i processi software siano conformi ai requisiti\textsubscript{G} specificati;
\item \textbf{Verifica}\textsubscript{G}: definisce le attività per verificare il prodotto software;
\item \textbf{Validazione}\textsubscript{G}: definisce le attività per validare il prodotto software;
\item \textbf{Risoluzione dei problemi}: definisce un processo per analizzare e risolvere i problemi di qualsiasi natura o origine, sorti durante l’esecuzione di processi.
\end{itemize}


\subsubsection{Processi organizzativi}
I processi organizzativi presentati nel presente documento sono:
\begin{itemize}
\item \textbf{Gestione organizzativa}: definisce le attività dell’acquirente, l’organizzazione che acquisisce un prodotto software;
\item \textbf{Infrastruttura}: definisce le attività del fornitore, l’organizzazione che fornisce il prodotto software all’acquirente;
\item \textbf{Miglioramento}: definisce le attività dello sviluppatore, l’organizzazione;
\item \textbf{Formazione}: definisce le attività atte a provvedere una adeguata formazione del gruppo.
\end{itemize}

\subsubsection{Ruoli}
I ruoli definiti all’interno di questo progetto didattico universitario sono:
\begin{itemize}
\item \textbf{Docente del corso}: committente\textsubscript{G};
\item \textbf{Azienda proponente}: cliente e mentore;
\item \textbf{Gruppo di lavoro}: fornitore.
\end{itemize}
\newpage
\section{Processi Primari}
\subsection{Fornitura}
\subsubsection{Scopo}
La fornitura è il processo che descrive le attività svolte dal fornitore, coinvolge pianificazione, acquisizione e gestione delle risorse necessarie. Il processo determina le procedure e le risorse necessarie per gestire e garantire il progetto. L'obiettivo di questo processo è garantire l'efficienza\textsubscript{G} e la conformità ai requisiti del progetto per raggiungere gli obiettivi stabiliti dal proponente. 
\subsubsection{Implementazione}
Il processo di fornitura è composto delle seguenti fasi:
\begin{enumerate}
    \item \textbf{Risposta alla richiesta}: il fornitore, dopo aver analizzato i requisiti di una richiesta del proponente (il Capitolato) prepara in risposta una proposta;
    \item \textbf{Negoziazione}: il fornitore negozia e stipula un contratto con il proponente;
    \item \textbf{Pianificazione}: il fornitore rivede i requisiti e valuta le opzioni per lo sviluppo del prodotto software in base ad un'analisi dei rischi associati alle varie opzioni per definire la struttura di un piano di gestione del progetto al fine di garantire la qualità del prodotto finale;
    \item \textbf{Esecuzione e controllo}: il fornitore esegue il piano di gestione del progetto, monitorando il progresso e la qualità del prodotto per tutto il ciclo di vita del prodotto;
    \item \textbf{Revisione}: il fornitore coordina le comunicazioni con il proponente e partecipa a riunioni e revisioni. Il fornitore verifica e convalida il processo per dimostrare che i prodotti e i processi soddisfano i requisiti;
    \item \textbf{Consegna}: il fornitore consegna il prodotto finale, fornendo assistenza al proponente a supporto del prodotto consegnato.
\end{enumerate}

\subsubsection{Gestione}
Al fine di identificare e comprendere i bisogni del Proponente, per poter individuare i requisiti e i vincoli del progetto, deve essere mantenuta costante comunicazione con il Proponente, mediante riunioni SAL periodiche calendarizzate, in presenza o su Microsoft Teams\textsubscript{G} e con scambio di messaggi su Microsoft Teams\textsubscript{G} e mail qualora fosse necessario. Il dialogo continuo permette anche una valutazione costante dell'operato del fornitore, in modo da apportare correzioni, integrazioni e miglioramenti in modo tempestivo, incrementale e costruttivo.

\subsubsection{Documentazione fornita}
Sono di seguito elencati i documenti che PEBKAC si impegna a consegnare ai Committenti e al Proponente: 

\subsubsubsection{Piano di Progetto}
Il Piano di Progetto V1.0.0, redatto dal Responsabile con l'aiuto degli Amministratori, offre una guida per la pianificazione l'esecuzione e il controllo del progetto e viene utilizzato come punto di partenza principale per il monitoraggio del progresso del progetto, la gestione dei rischi e la comunicazione tra proponente e fornitore.
Il Piano di Progetto comprende:
    \begin{itemize}
        \item Calendario di Progetto;
        \item Stima dei costi di realizzazione;
        \item Rischi e relativa mitigazione;
        \item Pianificazione e modello di sviluppo;
        \item Preventivo e consuntivo;
        \item Retrospettiva.
    \end{itemize}

\subsubsubsection{Analisi dei requisiti}
L'Analisi dei Requisiti V1.0.0, redatto degli Analisti, è un documento fondamentale che ha l'obiettivo principale di definire nel dettaglio le funzionalità che il prodotto deve necessariamente avere per soddisfare a pieno le richieste del Proponente. 
Il documento di Analisi dei Requisiti è formato da una serie di definizioni essenziali:
\begin{itemize}
    \item \textbf{Attori}: vengono definite entità e persone che interagiscono col sistema\textsubscript{G};
    \item \textbf{Casi d'uso}: vengono descritti narrativamente degli scenari specifici che descrivono come gli attori interagiscono col sistema\textsubscript{G}. Lo scopo dei casi d'uso è offrire una visione semplice e chiara delle azioni eseguibili all'interno del sisitema e delle interazioni degli utenti con lo stesso. Per ciascun caso d'uso viene fornito un elenco delle azioni dell'attore\textsubscript{G} per attivare il caso d'uso, facilitando la comprensione dei requisiti corrispondenti;
    \item \textbf{Requisiti}: vengono individuati i requisiti obbligatori, desiderabili e opzionali e la loro categorizzazione in: 
    \begin{itemize}
        \item \textbf{Requisiti funzionali}: specificano le operazioni che il sistema deve essere in grado di eseguire; 
        \item \textbf{Requisiti di qualità}: definiscono gli standard e gli attributi che il software deve possedere per garantire prestazioni, affidabilità, sicurezza e usabilità ottimali;
        \item \textbf{Requisiti di vincolo}: definiscono vincoli e limitazioni che il sistema deve rispettare. Possono includere restrizioni tecnologiche, normative o di risorse.
    \end{itemize}
\end{itemize}
\subsubsubsection{Piano di Qualifica}
Il Piano di Qualifica V1.0.0, redatto dall'Amministratore, descrive gli approcci e le strategie che il gruppo ha adottato per garantire la qualità del prodotto. Lo scopo di questo documento è quello di definire le modalità di verifica e validazione, oltre che gli standard e le procedure di qualità che il gruppo ha deciso di adottare per il ciclo di vita del progetto. \\ 
Si compone delle sezioni riguardanti:
\begin{itemize}
    \item \textbf{Qualità di processo}: vengono definiti standard e procedure adottate per garantire la qualità durante tutto lo sviluppo del progetto. Vengono incluse anche informazioni sulle attività di gestione della qualità, i metodi utilizzati e le misurazioni dei processi stessi;
    \item \textbf{Qualità di prodotto}: vengono definiti standard, specifiche e caratteristiche che il prodotto deve soddisfare per essere considerato di qualità. Vengono incluse anche metriche e criteri di valutazione utilizzati per misurare la qualità del prodotto;
    \item \textbf{Specifiche dei test}: vengono definite specifiche dettagliate dei test che verranno condotti durante lo sviluppo del progetto;
    \item \textbf{Cruscotto delle metriche}: viene fatto un resoconto delle attività di valutazione effettuate durante il progetto per tracciare l'andamento dello stesso rispetto a obiettivi e aspettative e per identificare eventuali azioni correttive necessarie.
\end{itemize}

\subsubsubsection{Glossario}
Il Glossario V1.0.0 serve come un catalogo completo dei termini tecnici impiegati all'interno del progetto, fornendo definizioni chiare e precise. L'obiettivo di questo documento previene fraintendimenti a favore di una comprensione condivisa della terminologia specifica, migliorando la coerenza e la qualità della documentazione\textsubscript{G} prodotta dal gruppo.

\subsubsection{Strumenti}
Gli strumenti utilizzati per il processo di fornitura sono:
\begin{itemize}
    \item Google Calendar
    \item Google Sheets
    \item Microsoft PowerPoint
    \item Microsoft Teams
\end{itemize}
    \subsection{Sviluppo}
\subsubsection{Scopo}
Il processo di sviluppo rappresenta la serie di attività svolte dal team PEBKAC al fine di implementare il prodotto \textit{software}\textsubscript{G}, rispettando le scadenze e i \textit{requisiti}\textsubscript{G} concordati col Proponente. 
Il processo è suddiviso nelle seguenti attività:
\begin{itemize}
    \item Analisi dei requisiti,
    \item Progettazione;
    \item Codifica;
    \item Testing;
    \item Integrazione \textit{software}\textsubscript{G}.
\end{itemize}

\subsubsection{Analisi dei Requisiti}
\subsubsubsection{Scopo}
Lo scopo dell'analisi dei requisiti è comprendere e definire in modo chiaro e completo le necessità e le aspettative del Proponente e degli utenti relativamente al prodotto \textit{software}\textsubscript{G}.
\subsubsubsection{Implementazione}
L'analisi dei requisiti, raccolta nel documento Analisi dei Requisiti V1.0.0, viene svolta secondo le seguenti fasi:
\begin{enumerate}
    \item Studio del \textit{capitolato}\textsubscript{G} e delle esigenze del Proponente;
    \item Individuazione dei casi d'uso e dei \textit{requisiti}\textsubscript{G};
    \item Confronto con il Proponente su quanto prodotto;
    \item Divisione dei \textit{requisiti}\textsubscript{G} nelle categorie individuate e applicazione dei quanto emerso nella discussione col Proponente.
\end{enumerate}

L'attività di analisi può essere svolta in modo incrementale, quindi le sue fasi possono essere svolte più volte durante lo sviluppo del progetto. 
\\
L'Analisi dei Requisiti V1.0.0 contiene:
\begin{itemize}
    \item \textbf{Introduzione}: descrive lo scopo del documento, del prodotto e i riferimenti utilizzati;
    \item \textbf{Descrizione}: esplicita le funzionalità attese del prodotto;
    \item \textbf{Attori\textsubscript{G}}: descrive gli utilizzatori del prodotto;
    \item \textbf{Casi d'uso}: individua le possibili interazioni tra gli \textit{attori}\textsubscript{G} e il \textit{sistema}\textsubscript{G};
    \item \textbf{Requisiti\textsubscript{G}}: elenca le caratteristiche da soddisfare;
\end{itemize}
\subsubsubsection{Casi d'uso}
I casi d’uso sono strutturati nel seguente modo:
\begin{itemize}
    \item \textbf{Attore}\textsubscript{G}: l’\textit{attore}\textsubscript{G} che intende compiere lo scopo rappresentato dal caso d’uso;
    \item \textbf{Precondizioni}: stato in cui il \textit{sistema}\textsubscript{G} si deve trovare prima dell’avvio della funzionalità rappresentata dal caso d’uso;
    \item \textbf{Postcondizioni}: stato in cui il \textit{sistema}\textsubscript{G} si troverà dopo che l'utente avrà terminato lo scopo rappresentato dal caso d’uso;
    \item \textbf{Scenario principale}: descrizione della funzionalità rappresentata dal caso d’uso;
    \item \textbf{Scenari secondari} (se necessario);
    \item \textbf{Estensioni} (se presenti);
    \item \textbf{Specializzazioni} (se presenti).
\end{itemize}
\subsubsubsubsection{Notazione}
i casi d'uso seguono la seguente notazione: \textbf{UC[Codice] - [Titolo]} in cui:
\begin{itemize}
    \item \textbf{UC} sta per Use Case;
    \item \textbf{[Codice]} è l'identificativo univoco del caso d'uso. Si tratta di un numero intero progressivo assegnato in base all'ordine di descrizione, se il caso d'uso non ha padre, altrimenti se si tratta di un sottocaso d'uso si segue la notazione\textbf{ [Codice\_padre]-[Numero\_figlio]}, ricorsivamente senza porre limite alla profondità della gerarchia;
    \item \textbf{[Titolo]} è il titolo del caso d'uso.
\end{itemize}

\subsubsubsubsection{Diagrammi UML\textsubscript{G}}
Un \textit{diagramma dei casi d’uso}\textsubscript{G} è uno strumento di modellazione che rappresenta visivamente le funzionalità di un \textit{sistema}\textsubscript{G} e le modalità con cui gli utenti interagiscono con esso. È particolarmente utile nella progettazione di sistemi poiché offre una rappresentazione intuitiva delle dinamiche operative e delle interazioni tra \textit{attori}\textsubscript{G} e \textit{sistema}\textsubscript{G}, senza entrare nei dettagli implementativi.
I componenti principali di un \textit{diagramma dei casi d’uso}\textsubscript{G} sono: 
\begin{enumerate}
    \item \textbf{Attori}\textsubscript{G}: gli \textit{attori}\textsubscript{G} rappresentano entità esterne (umane o meno) che interagiscono con il \textit{sistema}\textsubscript{G} e sono raffigurati con un’icona stilizzata e un’etichetta identificativa. Possono essere generalizzati: un \textit{attore}\textsubscript{G} generico può avere \textit{attori}\textsubscript{G} più specifici che ne ereditano le funzionalità e aggiungono comportamenti contestuali;
    \item \textbf{Casi d'uso}: un caso d’uso descrive un'operazione che un utente può compiere attraverso il \textit{sistema}\textsubscript{G}. Ogni caso d’uso ha un'identificazione univoca e una breve descrizione della funzione. Può includere sequenze di azioni che illustrano le possibili interazioni con il \textit{sistema}\textsubscript{G} ed è collegato agli \textit{attori}\textsubscript{G} autorizzati tramite linee continue.
\end{enumerate}
Nei diagrammi in questione poi possono comparire delle relazioni:
\begin{enumerate}
    \item \textbf{Generalizzazioni}: le generalizzazioni possono riguardare sia gli \textit{attori}\textsubscript{G} che i casi d’uso. Gli \textit{attori}\textsubscript{G} o i casi figli ereditano le funzionalità dei genitori, aggiungendo aspetti specifici. La relazione è rappresentata con una freccia continua e un triangolo vuoto bianco;
    \item \textbf{Inclusioni}: si verificano quando un caso d’uso ne richiama un altro in modo obbligatorio. Questo favorisce la riduzione della duplicazione e il riutilizzo delle strutture. La relazione è indicata con una freccia tratteggiata e l’etichetta “include”;
    \item \textbf{Estensioni}: rappresentano relazioni condizionali in cui un caso d’uso aggiuntivo viene eseguito solo in circostanze particolari, interrompendo temporaneamente il flusso principale. La relazione è raffigurata con una freccia tratteggiata e l’etichetta “extend”.
\end{enumerate}

\subsubsubsection{Requisiti}
\subsubsubsubsection{Notazione}
Ogni \textit{requisito}\textsubscript{G} analizzato  sarà identificato univocamente da una sigla del tipo \\ \textbf{R[Tipo].[Importanza].[Codice]} nella quale:
\begin{itemize}
    \item \textbf{[R]} sta per \textit{Requisito}\textsubscript{G};
    \item \textbf{[Tipo]} può essere:
    \begin{itemize}
        \item \textbf{F} per Funzionale;
        \item \textbf{Q} per Qualità;
        \item \textbf{V} per Vincolo.
    \end{itemize}
    \item \textbf{[importanza]} classifica i \textit{requisiti}\textsubscript{G} in:
    \begin{itemize}
        \item \textbf{O} per Obbligatorio;
        \item \textbf{D} per Desiderabile;
        \item \textbf{P} per Opzionale.
    \end{itemize}
    \item \textbf{[Codice]} identifica univocamente i \textit{requisiti}\textsubscript{G} per ogni tipologia. È un numero intero progressivo univoco assegnato in ordine di importanza se il \textit{requisito}\textsubscript{G} non ha padre, se invece si tratta di un sotto-\textit{requisito}\textsubscript{G} segue il formato \textbf{[Codice\_padre].[Numero\_figlio]} e trattandosi di una struttura ricorsiva non c'è limite alla profondità della gerarchia.
\end{itemize}

\subsubsubsubsection{Suddivisione}
\begin{enumerate}
    \item \textbf{Requisiti\textsubscript{G} Funzionali}: descrivono le funzionalità del \textit{sistema}\textsubscript{G}, le azioni che il \textit{sistema}\textsubscript{G} può compiere e le informazioni che il \textit{sistema}\textsubscript{G} può fornire. Seguendo la notazione sopra riportata, si possono partizionare in:
    \begin{itemize}
        \item RF.O - \textit{Requisito}\textsubscript{G} Funzionale Obbligatorio;
        \item RF.D - \textit{Requisito}\textsubscript{G} Funzionale Desiderabile;
        \item RF.P - \textit{Requisito}\textsubscript{G} Funzionale Opzionale;
    \end{itemize}
     \item \textbf{Requisiti\textsubscript{G} di Qualità}: descrivono come un \textit{sistema}\textsubscript{G} deve essere, o come il \textit{sistema}\textsubscript{G} deve essere visualizzato, per soddisfare le esigenze dell’utente. Seguendo la notazione sopra riportata, si possono partizionare in:
    \begin{itemize}
        \item RQ.O - \textit{Requisito}\textsubscript{G} di Qualità Obbligatorio;
        \item RQ.D - \textit{Requisito}\textsubscript{G} di Qualità Desiderabile;
        \item RQ.P - \textit{Requisito}\textsubscript{G} di Qualità Opzionale;
    \end{itemize}
     \item \textbf{Requisiti\textsubscript{G} Funzionali}: descrivono i limiti e le restrizioni normative/legislative che un \textit{sistema}\textsubscript{G} deve rispettare per soddisfare le esigenze dell’utente. Seguendo la notazione sopra riportata, si possono partizionare in:
    \begin{itemize}
        \item RV.O - \textit{Requisito}\textsubscript{G} di Vincolo Obbligatorio;
        \item RV.D - \textit{Requisito}\textsubscript{G} di Vincolo Desiderabile;
        \item RV.P - \textit{Requisito}\textsubscript{G} di Vincolo Opzionale;
    \end{itemize}
\end{enumerate}

\newpage
\section{Processi di Supporto}
    \subsection{Documentazione}
\subsubsection{Scopo}
Il processo di documentazione procede sempre di pari passo con tutte le attività di sviluppo, con l'obiettivo di fornire tutte le informazioni necessarie, sotto forma di testo scritto facilmente consultabile, inerenti al prodotto e alle attività stesse. Oltre a svolgere un ruolo essenziale nella descrizione del prodotto per coloro che lo sviluppano, lo distribuiscono e lo utilizzano, la documentazione svolge un ruolo di storicizzazione e di supporto alla manutenzione. 

\subsubsection{Documenti}
In questa sezione viene descritto il piano che identifica i documenti da produrre durante il ciclo di vita del prodotto software. Tutti i documenti da redigere sono presentati nella tabella che segue, vengono esclusi i documenti presentati per la candidatura per il progetto didattico, quali \textit{Lettera di presentazione}, \textit{Preventivo dei costi e assunzione degli impegni} e \textit{Analisi dei capitolati}.


% Creazione della tabella
\begin{table}[H]
    \centering
   \begin{tabularx}{\textwidth}{X|X|>{\hsize=1.2\hsize}X|X|>{\hsize=0.8\hsize}X}
        \textbf{Nome} & \textbf{Scopo} & \textbf{Redattore} & \textbf{Destinatari} & \textbf{Consegne} \\ \hline
        Analisi dei requisiti    & Definizione dei requisiti utente    & Analista & Azienda proponente, Docenti & RTB\textsubscript{G}, PB\textsubscript{G}    \\ \hline
        Norme di progetto    & Regolamento normativo del gruppo    & Amministratore, Responsabile &  Docenti & RTB, PB    \\ \hline
        Piano di Progetto   & Definizione temporale scadenze e progressi    & Responsabile &  Docenti & RTB, PB   \\ \hline
        Piano di qualifica   & Definizione qualità e testing    & Amministratore &  Docenti & RTB, PB   \\ \hline
         Verbali esterni   & Tracciamento riunioni esterne   & Responsabile, Amministratore &  Azienda proponente, Docenti & Candidatura, RTB, PB   \\ \hline
         Verbali interni   & Tracciamento riunioni interne   & Responsabile, Amministratore &  Docenti & Candidatura, RTB, PB  
       
    \end{tabularx}
    \caption{Documenti del ciclo di vita del prodotto SW}
\end{table}


\subsubsection{Progettazione e sviluppo}
In questa sezione vengono presentati gli standard e le regole (nello specifico di stile) a cui i membri di PEBKAC si devono attenere per la stesura dei documenti relativi al progetto.

\subsubsubsection{Template}
Per la stesura dei documenti il gruppo ha creato un template in formato Latex\textsubscript{G}. Il template fornisce una struttura e un formato predefinito per semplificare la creazione di documenti, al fine di garantire coerenza, efficienza e standardizzazione della presentazione. 
Il template è progettato per essere facile da usare, dovendo inserire solo con piccole modifiche per rispecchiare le specificità di ciascun tipo di documento.\\
In particolare nel template è definite la pagina di copertina con intestazione contenente logo informazioni del gruppo e dell'Università di Padova, titolo del documento, informazioni sul documento (uso, destinatari) e un breve abstract del contenuto, oltre che altre specifiche di stile come il titolo dell'indice in italiano e il numero di pagina come \texttt{X di Tot}, dove \texttt{X} è il numero della pagina e \texttt{Tot} è il numero totale di pagine.
\subsubsubsubsection{Parametri}
Nel principale file Latex del template sono definiti una serie di comandi personalizzati per l'inserimento automatico delle informazioni come titolo, data, uso, destinatari e abstract. \\
Sono inoltre già presenti ma commentate le voci necessarie solo per i verbali (vedi  \hyperref[sec: struttura verbali]{§4.1.3.3 Verbali})
\subsubsubsection{Struttura del documento}
Tutti i documenti prodotti da PEBKAC presentano la medesima struttura, alla quale ogni membro si deve attenere durante la procedura di stesura e modifica.
\begin{itemize}
    \item \textbf{Pagina di copertina}: come nella sezione Template precedente;
    \item \textbf{Registro delle versioni}: questo registro è utilizzato per tenere traccia delle varie versioni per permettere di comprendere velocemente chi ha realizzato o modificato determinate sezioni della documentazione e quando. Il registro presenta le versioni ordinate a partire dalla versione più recente;
     \item \textbf{Indice}: presente per facilitare la consultazione del documento, dotato di sezioni. Il suo scopo è di facilitare e agevolare l’accesso ad un determinato contenuto all'interno nel documento;
     \item \textbf{Contenuto}: il contenuto vero e proprio del documento.
\end{itemize}

\subsubsubsection{Verbali}\label{sec: struttura verbali}
I verbali differiscono dalla struttura precedentemente esposta in quanto ad essi prevedono delle sezioni aggiuntive ed obbligatorie:
\begin{itemize}
    \item \textbf{Pagina di copertina}: nel caso di un verbale tra le informazioni sul documento compaiono anche i nominativi con i rispettivi ruoli dei membri che hanno lavorato alla loro produzione;
    \item \textbf{informazioni generali}: la prima sezione di un verbale deve sempre essere quella nominata ``Informazioni generali" che prevede, sotto forma di elenco puntato, le seguenti informazioni:
        \begin{itemize}
            \item Tipo di riunione,
            \item Luogo in cui si è tenuta la riunione (anche se telematica),
            \item Data in cui si è tenuta la riunione,
            \item Ora di inizio della riunione,
            \item Ora di fine della riunione,
            \item Membri presenti ed eventuali altre persone alla riunione,
            \item Membri assenti dalla riunione;
        \end{itemize}
     \item \textbf{Todo}: l'ultima sezione di un verbale deve sempre essere quella che elenca i task\textsubscript{G} emersi durante la riunione da aggiungere al backlog\textsubscript{G}. Questi vengono presentati sotto forma di tabella a due colonne:
     \begin{itemize}
         \item \textbf{Assegnatario}: il membro a cui quel task è stato assegnato, nel caso in cui non ve ne sia uso ma il task possa essere autoassegnato da uno dei membri si scriverà ``autoassegnazione" in corsivo;
         \item \textbf{Task Todo}: denominazione del task.
     \end{itemize}
     
\end{itemize}

\subsubsubsection{Nomenclatura}
La nomenclatura per i documenti si ottiene unendo il nome del file in \textit{Snake\_Case} quindi con le parole separate da  un underscore (\textit{\_}) (\texttt{Nome\_del\_File}), un underscore (\textit{\_}) e la sua versione (\texttt{1.2.3}), ottenendo per esempio \texttt{Norme\_di\allowbreak{}\_Progetto\_1.2.3.pdf}. Nel caso di documenti il cui nome contiene una data, essa si inserisce dopo il nome, ma prima della versione, sempre usando gli underscores come separatori, nella forma YYYY-MM-DD: YYYY rappresenta l'anno, MM il mese e DD il giorno, sempre scritto in due cifre.
\subsubsubsubsection{Verbali}
Per quanto riguarda i  verbali, per facilitarne l'ordinamento) il loro nome è la data in cui la riunione di è tenuta nella forma YYYY-MM-DD: YYYY rappresenta l'anno, MM il mese e DD il giorno, sempre scritto in due cifre. Nel caso si tratti di un verbale esterno viene aggiunta una \texttt{E}, sempre separata da underscores tra la data e la versione.


\subsubsubsection{Versionamento}
La versione di un documento è del tipo [\textbf{x}].[\textbf{y}].[\textbf{z}]:
\begin{itemize}
    \item \textbf{z}: è un numero intero che incrementato dal Redattore ad ogni modifica;
    \item \textbf{y}: è un numero intero incrementato dal Verificatore ad ogni verifica;
    \item \textbf{x}: è un numero intero che viene incrementato dal Responsabile dopo la sua approvazione (versione di produzione).
\end{itemize}

\subsubsubsection{Convenzioni stilistiche}
\begin{itemize}
    \item \textbf{Date}: tutte le date nella documentazione prevedono il seguente formato YYYY-MM-DD, dove DD indica il giorno a due cifre, MM il mese a due cifre e YYYY l'anno a 4 cifre;
    \item \textbf{Elenchi}: elenchi puntati o numerati, ogni punto inizia con la lettera maiuscola e termina con ``;" ad eccezione dell'ultimo che termina con ``.";
    \item \textbf{Menzioni}: ogni menzione ad una persona, interna o esterna, avviene nel formato Nome Cognome;
    \item \textbf{Riferimenti interni}: i riferimenti a sezioni interne allo stesso documento devono essere riportati seguendo la notazione \texttt{§1.2 Nome sezione}, dove \texttt{§1.2} è il numero della sezione. Inoltre questi riferimenti devono essere opportunamente collegati tramite link al paragrafo indicato, senza alterare lo stile del testo;
    \item \textbf{Riferimenti esterni}: i riferimenti a sezioni di documenti esterni devono essere riportati seguendo la notazione \texttt{Nome Documento (versione di riferimento), Nome sezione};
     \item \textbf{Link URL}: possono essere estesi o avere una visualizzazione abbreviata, ma sempre visualizzati di colore blu;
    \item \textbf{Caratteri maiuscoli}: devono essere utilizzati per
        \begin{itemize}
            \item Le iniziali dei nomi;
            \item Le lettere che compongono degli acronimi e le iniziali delle rispettive definizioni;
            \item Le iniziali dei ruoli svolti dai componenti del gruppo;
            \item Le iniziali dei ruoli definiti all'interno del progetto didattico;
            \item La prima lettera di ogni elenco puntato.
        \end{itemize}
    \item \textbf{Grassetto}: devono essere visualizzati in grassetto
        \begin{itemize}
            \item I titoli di sezioni/sottosezioni/paragrafi di un documento;
            \item Le parole che meritano enfasi;
            \item Le definizioni negli elenchi puntati.
        \end{itemize}
    \item \textbf{Caption}: ogni immagine o tabella deve avere una caption, utile a fornire una breve descrizione o spiegazione del contenuto visivo.
\end{itemize}

\subsubsection{Ciclo di vita dei documenti}
Ogni documento segue le fasi del seguente workflow\textsubscript{G}:
\begin{enumerate}
    \item \textbf{Assegnazione}: il gruppo assegna un documento a uno o più redattori, affiancati da uno o più verificatori;
    \item \textbf{Branch}: si crea un branch per lo sviluppo del documento nell’apposita repository\textsubscript{G} Docs;
    \item \textbf{Template}: si copia il Template all'interno della cartella appropriata;
    \item \textbf{Stesura}: si redige il documento o una sua sezione. Qualora serva un elevato parallelismo di lavoro è possibile usare Google Drive per la prima stesura e successivamente caricare il documento all’interno del branch;
    \item \textbf{Commit}: si esegue la commit sul branch creato;
    \item \textbf{Pull Request}: si apre una pull request dal branch appena creato verso il branch develop: se il documento non è pronto per la verifica, ma ha bisogno di ulteriori modifiche, si apre la pull request in modalità draft, per marcarla successivamente come “Ready to Review”, altrimenti in modalità normale;
    \item \textbf{Verifica}: se il verificatore richiede modifiche si ripete, in ordine, dal punto 3 al punto 5;
    \item \textbf{Chiusura branch}: si elimina, quando la pull request viene chiusa o risolta, il branch creato.
\end{enumerate}
Per la versione finale di un documento spetta al Responsabile conferire l’approvazione definitiva, annotando opportunamente nel registro delle versioni la versione \texttt{x.0.0} e la sua approvazione finale.


\subsubsection{Strumenti}
\begin{itemize}
    \item Latex
    \item Visual Studio Code
    \item GitHub
\end{itemize}

    \subsection{Configuration Management}
\subsubsection{Scopo}
In questa sezione vengono presentate le attività svolte da PEBKAC per il processo di \textit{Configuration Management}\textsubscript{G}. Il processo in questione consiste nell'applicazione di procedure amministrative e tecniche per l'intero ciclo di vita del \textit{software}\textsubscript{G}, al fine di:
\begin{itemize}
    \item Identificare, definire e stabilire una base per gli elementi \textit{software}\textsubscript{G} di un \textit{sistema}\textsubscript{G};
    \item Controllare le modifiche e le release degli elementi;
    \item Registrare lo stato degli elementi e delle richieste di modifica;
    \item Garantire la completezza, la coerenza e la correttezza degli elementi.
\end{itemize}
\subsubsection{Configuration control}
\subsubsubsection{Descrizione}
Il processo di configuration control è finalizzato a garantire il controllo e la coerenza delle configurazioni del \textit{sistema}\textsubscript{G}, assicurando che tutte le modifiche apportate a \textit{software}\textsubscript{G}, artefatti e documenti siano tracciate, gestite e allineate agli obiettivi e ai \textit{requisito}\textsubscript{G} del progetto.
\subsubsubsection{Scopo}
Il configuration control mira al raggiungimento dei seguenti punti:
\begin{itemize}
    \item \textbf{Gestire le modifiche}: assicurare un controllo e una gestione corretti e sistematici per qualsiasi modifica nel progetto o nel \textit{sistema}\textsubscript{G};
    \item \textbf{Documentare le richieste}: registrare tutte le richieste di modifica per mantenere una cronologia accurata e completa;
    \item \textbf{Valutare l'impatto}: analizzare tutte le conseguenze tecniche, economiche e operative di ogni modifica proposta;
    \item \textbf{Decidere sull'approvazione}: stabilire criteri chiari e inequivocabili per l'approvazione o il rifiuto delle modifiche con gli \textit{stakeholder}\textsubscript{G} rilevanti;
    \item \textbf{Assicurare la tracciabilità}: creare \textit{audit trail}\textsubscript{G} dettagliati per tracciare le modifiche e garantire la conformità alle politiche del progetto;
    \item \textbf{Evitare conflitti}: prevenire modifiche non autorizzate o che possano entrare in conflitto con il \textit{sistema}\textsubscript{G};
    \item \textbf{Mantenere la qualità}: garantire che le modifiche non compromettano l'integrità, la funzionalità o gli obiettivi generali del progetto.
\end{itemize}
\subsubsubsection{ITS\textsubscript{G}}
Per conseguire l'obiettivo di assicurare la tracciabilità delle modifiche è necessario creare degli \textit{audit trail}\textsubscript{G} dettagliati, ovvero dei registri che tracciano tutte le attività e le modifiche all'interno di un \textit{sistema}\textsubscript{G}. Per la creazione, la gestione ed il tracciamento di questi \textit{audit trail}\textsubscript{G}, PEBKAC utilizza l'Issue Tracking System \textit{Jira}\textsubscript{G}, sviluppato da \textit{Atlassian}\textsubscript{G}.
\subsubsubsubsection{Ticket}
Un \textit{ticket}\textsubscript{G} è una voce che rappresenta una singola attività, problema, richiesta o \textit{task}\textsubscript{G} all'interno di un progetto. \\
Esistono varie tipologie di ticket:
\begin{itemize}
    \item \textbf{Task\textsubscript{G}}: questa tipologia di ticket rappresenta una comune attività che deve essere completata all'interno del progetto;
    \item \textbf{Sub-task}: questa tipologia di ticket rappresenta una parte di un ticket più grande (come un \textit{task}\textsubscript{G}) che viene suddivisa in azioni più piccole e gestibili;
    \item \textbf{Story}: questa tipologia di ticket rappresenta un \textit{requisito}\textsubscript{G} ed è generalmente scritto in un formato che descrive il risultato atteso dal punto di vista dell'utente;
    \item \textbf{\textit{bug}\textsubscript{G}}: questa tipologia di ticket rappresenta un errore o difetto nel \textit{sistema}\textsubscript{G} che necessita di correzione;
\end{itemize}
Ogni ticket è dotato di campi per riportare i dettagli relativi all'attività, al problema, alla richiesta o alla \textit{task}\textsubscript{G} che rappresenta:
\begin{itemize}
    \item \textbf{Summary}: un riassunto breve in una sola riga del ticket;
    \item \textbf{Key}: un identificatore unico per ogni ticket, nella forma si SW-Key;
    \item \textbf{Epic}: \textit{epic}\textsubscript{G} a cui il ticket è associato;
    \item \textbf{Links}: un elenco di link a ticket correlati;
    \item \textbf{Assignee}: la persona o le persone a cui il ticket è attualmente assegnato;
    \item \textbf{Description}: una descrizione dettagliata del ticket;
    \item \textbf{Due}: la data entro cui questo ticket è programmato per essere completato;
    \item \textbf{Reporter}: la persona che ha inserito il ticket nel \textit{sistema}\textsubscript{G};
    \item \textbf{Links}: un elenco di link alle \textit{commit}\textsubscript{G} e alle \textit{pull request}\textsubscript{G} effettuati nella \textit{repository}\textsubscript{G} di \textit{GitHub}\textsubscript{G} correlate al ticket;
    \item \textbf{Status}: la fase in cui si trova attualmente il ticket nel suo ciclo di vita, che può essere "To Do", "In Process", "Verify" ed infine "Approve \& Release";
    \item \textbf{Sprint}: \textit{sprint}\textsubscript{G} a cui il ticket è associato;
    \item \textbf{Fix Version}: la versione del progetto in cui il ticket è stato (o sarà) risolto;
    \item \textbf{Priority}: l'importanza del ticket rispetto ad altri ticket.
\end{itemize}
\subsubsubsubsection{Epic}
Un'\textit{epic}\textsubscript{G} è una raccolta di \textit{ticket}\textsubscript{G} che rappresenta uno degli obiettivi più ampi e significativi verso cui è diretto l'intero progetto. Si tratta di un concetto che aiuta a gestire e strutturare il lavoro più complesso, suddividendolo in parti più piccole e gestibili. Le \textit{epic}\textsubscript{G} sono utili per monitorare i progressi rispetto a funzionalità che richiedono tempo o che coinvolgono diverse aree del progetto. Le \textit{epic}\textsubscript{G} sono particolarmente utili nei processi \textit{Agile}\textsubscript{G}, poiché offrono una visione a lungo termine del progetto, anche mentre si adatta e si pianifica in modo incrementale, fornendo strumenti di tracciamento, come una scorebord che presenta le percentuali di ticket presenti in ogni stato, che monitorano lo stato generale di ogni \textit{epic}\textsubscript{G}, misurano i progressi e identificano eventuali ritardi. Inoltre, un'\textit{epic}\textsubscript{G}, oltre ai ticket che comprende, possiede tutti i campi precedentemente elecati per i ticket.
\subsubsubsubsection{Versioni}
Le versioni sono la modalità di organizzazione, pianificazione e monitoraggio del lavoro in base alle specifiche milestone di un progetto. Ogni versione ha a che fare con le funzionalità, rappresentate da \textit{epic}\textsubscript{G} e relativi \textit{ticket}\textsubscript{G} ad essa associati, da realizzare entro una scadenza. In pratica, permette di sapere chiaramente quali \textit{requisito}\textsubscript{G} devono essere soddisfatti per la specifica milestone che rappresenta. Ciò rende semplice la tracciabilità dei progressi di ciascuna versione, oltre a ritardi o modifiche, usando una \textit{scoreboard}\textsubscript{G} che adotta la stessa logica di quella utilizzata dalle \textit{epic}\textsubscript{G}. In seguito al completamento di tutti i ticket associati ad una determina versione, questa può essere rilasciata. Inoltre, una versione, oltre ai ticket che comprende, possiede dei campi che specificano la data di inizio, la data di fine ed una breve descrizione.
\subsubsubsubsection{Backlog e Sprint}
In \textit{Jira}\textsubscript{G} sono integrati diversi strumenti per lo sviluppo secondo il metodo \textit{Agile}\textsubscript{G}, tra i quali è importante evidenziare \textit{backlog}\textsubscript{G} e \textit{sprint}\textsubscript{G}.\\
Il \textit{backlog}\textsubscript{G} contiene una lista di \textit{ticket}\textsubscript{G} da completare dal team, ed è ordinata in base alla priorità: i più importanti sono posti in cima, mentre i meno importanti sono disposti verso il fondo. La lista non è statica, ma è uno spazio dinamico all'interno del quale il team può aggiungere, eliminare, aggiornare e riorganizzare le priorità dei ticket al suo interno. Il \textit{backlog}\textsubscript{G} è inteso come punto di partenza per pianificare il lavoro: prima di ogni \textit{sprint}\textsubscript{G}, il team esamina il \textit{backlog}\textsubscript{G} per selezionare il lavoro da svolgere durante quello \textit{sprint}\textsubscript{G}.\\
Uno \textit{sprint}\textsubscript{G} è un periodo di tempo predefinito in cui vengono completati i ticket selezionati dal \textit{backlog}\textsubscript{G} prima del suo inizio. Ogni \textit{sprint}\textsubscript{G} è dotato di una data di inizio, una data di fine e di uno stato, che può essere "In Corso" o "Terminato".
\subsubsubsubsection{Timeline}
La \textit{timeline}\textsubscript{G} messa a diposizione in \textit{Jira}\textsubscript{G} è uno strumento realizzato tramite un \textit{diagramma di Gantt}\textsubscript{G} che aiuta a gestire le scadenze, le dipendenze e l'andamento del progetto, fornendo una panoramica d'insieme dello stato di avanzamento.\\
Essa mostra tutti i \textit{ticket}\textsubscript{G} associati ad una \textit{epic}\textsubscript{G} che sono stati inseriti al suo interno, evidenziandone le date di inizio e fine e le dipendenze con altri ticket. Ogni ticket viene visualizzato come un blocco che si estende lungo la \textit{timeline}\textsubscript{G} in base alla durata prevista.\\
Le dipendenze tra i vari ticket possono essere visualizzate tramite linee di collegamento, mostrando come il completamento di un'attività dipenda da un’altra. Inoltre, la \textit{timeline}\textsubscript{G} mostra chiamente anche lo stato delle attività, ovvero quali attività sono in corso, quali attività sono state completate e quali attività sono in ritardo.\\
Un'altra informazione mostrata nella \textit{timeline}\textsubscript{G} sono le versioni e, in particolare, quando sono state fissate le loro date di scadenza. Le versioni sono rappresentate graficamente come delle linee verticali posizionate proprio sulla data di scadenza corrispondente.\\
\'E inoltre possibile visualizzare gli \textit{sprint}\textsubscript{G} definiti all'interno di \textit{Jira}\textsubscript{G} nell'area superiore della \textit{timeline}\textsubscript{G}, permettendo così di determinare quali attività sono state svolte durante ciascuno \textit{sprint}\textsubscript{G}.\\
Infine, è utile notare che nella \textit{timeline}\textsubscript{G} e possibile effettuare delle operazioni di filtraggio dei ticket visualizzati, permettendo così di visualizzare anche l'organizzazione di specifici gruppi di attività.
\subsubsubsection{Pull Request}
Le \textit{pull request}\textsubscript{G} sono un meccanismo per la gestione controllata delle modifiche nei rilasci del prodotto. Quando un membro del team propone modifiche al \textit{repository}\textsubscript{G}, esse vengono raggruppate in una \textit{pull request}\textsubscript{G}, che consente ai verificatori di analizzarle prima di integrarle nel \textit{repository}\textsubscript{G}.\\
Il processo può essere riassunto nei seguenti passaggi:
\begin{itemize}
    \item \textbf{Creazione della \textit{pull request}\textsubscript{G}}: Non appena le modifiche vengono apportate in un \textit{branch}\textsubscript{G} dedicato, il contributore apre una \textit{pull request}\textsubscript{G} per proporre la loro integrazione nel \textit{branch}\textsubscript{G} develop o in un altro \textit{branch}\textsubscript{G} di destinazione;
    \item \textbf{Revisione}: I verificatori esaminano le modifiche proposte direttamente all'interno della \textit{pull request}\textsubscript{G}, visualizzando i dettagli di ciò che è stato modificato e avendo la possibilità di commentare su una singola riga o su più righe/sezioni. Se le modifiche richieste sono minime, il \textit{Verificatore}\textsubscript{G} può correggerle direttamente eseguendo una \textit{commit}\textsubscript{G} sul \textit{branch}\textsubscript{G} in cui sono state suggerite le modifiche;
    \item \textbf{Feedback e suggerimenti}: Se le modifiche sono significative, i verificatori solitamente forniscono suggerimenti per miglioramenti, evidenziano possibili errori ed elencano tutte le modifiche necessarie per allineare il lavoro agli standard e alle linee guida del progetto;
    \item \textbf{Applicazione delle correzioni}: Il contributore che ha creato la \textit{pull request}\textsubscript{G} risponde al \textit{feedback}\textsubscript{G} ricevuto ed apporta le modifiche necessarie. Successivamente, la \textit{pull request}\textsubscript{G} viene aggiornata con le modifiche revisionate;
    \item \textbf{Decisione finale}: Quando i verificatori approvano la proposta, o quando tutte le questioni sollevate da questi ultimi sono state risolte, la \textit{pull request}\textsubscript{G} può essere considerata accettata e viene "unita" al \textit{branch}\textsubscript{G} di destinazione, garantendo che solo contributi di alta qualità facciano parte del progetto.
\end{itemize}
\subsubsection{Configuration status accounting}
\subsubsubsection{Scopo}
Il \textit{configuration status accounting}\textsubscript{G} è il processo di \textit{documentazione}\textsubscript{G} e monitoraggio di verifiche e cambiamenti alle caratteristiche del prodotto \textit{software}\textsubscript{G}. Grazie ad esso si ha una visione complessiva della sua evoluzione e della sua aderenza ai \textit{requisito}\textsubscript{G} e agli standard.
\subsubsubsection{Version control}
Il \textit{sistema}\textsubscript{G} di version control adottato dal gruppo per tracciare l'evoluzione del \textit{software}\textsubscript{G} segue la seguente convenzione di numerazione delle versioni: [x].[y].[z]([build]). Si tratta di un \textit{sistema}\textsubscript{G} di versionamento a quattro componenti, ognuna delle quali ha un significato specifico:
\begin{itemize}
    \item \textbf{x}: Rappresenta la versione principale del \textit{software}\textsubscript{G}, e il suo valore viene incrementato per ogni fase significativa di revisione o avanzamento del progetto;
    \item \textbf{y}: Rappresenta cambiamenti significativi o aggiunte di nuove funzionalità, e il suo valore viene incrementato ogni volta che vengono apportati cambiamenti considerati rilevanti per il prodotto;
    \item \textbf{z}: Rappresenta piccole modifiche o correzioni di \textit{bug}\textsubscript{G}, e viene incrementato quando vengono svolte azioni come l'aggiornamento della \textit{documentazione}\textsubscript{G} o la correzione di errori minori;
    \item \textbf{build}: Indica il numero delle build eseguite per una determinata versione, e viene incrementato ogni volta che vengono apportate delle modifiche alla \textit{documentazione}\textsubscript{G}.
\end{itemize}
Il \textit{sistema}\textsubscript{G} prevede che la numerazione inizi dalla versione "0.0.1(0)". Ogni volta che il valore di x, y o z aumenta, tutti i valori alla sua destra vengono resettati a "0".
\subsubsection{Configuration evaluation}
\subsubsubsection{Scopo}
Il processo di configuration evaluation serve a garantire la correttezza, la coerenza e la conformità della configurazione del \textit{sistema}\textsubscript{G}, del \textit{software}\textsubscript{G} e dell'infrastruttura rispetto ai \textit{requisito}\textsubscript{G} definiti.
\subsubsubsection{Tracciamento dei requisiti}
Il team PEBKAC ha deciso di tracciare i \textit{requisiti}\textsubscript{G} direttamente nel codice del prodotto \textit{software}\textsubscript{G}. Questo approccio offre un collegamento diretto e verificabile tra i \textit{requisiti}\textsubscript{G} di progettazione e il codice che soddisfa tali \textit{requisiti}\textsubscript{G}. Il tracciamento viene effettuato includendo un commento specifico prima di ogni blocco di codice che implementa un determinato \textit{requisito}\textsubscript{G}. Il commento contiene l'ID univoco del \textit{requisito}\textsubscript{G}, in modo da facilitare l'associazione tra i \textit{requisiti}\textsubscript{G} e le corrispondenti implementazioni nel codice.
\subsubsection{Release Management}
\subsubsubsection{Scopo}
Il processo di \textit{release management}\textsubscript{G} serve a pianificare, coordinare e gestire il rilascio, per far sì che qualsiasi nuova versione di un prodotto venga distribuita dal \textit{sistema}\textsubscript{G} in modo controllato.
\subsubsubsection{Automazione compilazione documenti}
Il gruppo si è dotato di una \textit{GitHub Action}\textsubscript{G} che provvede all'\textit{automazione}\textsubscript{G} della generazione e trasferimento di documenti \textit{LaTeX}\textsubscript{G} in PDF.\\
Essa si attiva quando viene unita una \textit{pull request}\textsubscript{G} nel \textit{branch}\textsubscript{G} develop e funziona nel seguente modo: 
\begin{enumerate}
    \item Prepara l'ambiente ed installa gli strumenti necessari;
    \item Esplora la directory contenente i documenti \textit{LaTeX}\textsubscript{G};
    \item Converte eventuali immagini in PDF e compila i file \textit{LaTeX}\textsubscript{G};
    \item Carica i PDF generati in una cartella organizzata;
    \item Trasferisce la cartella sul \textit{branch}\textsubscript{G} main;
    \item Sposta i PDF all'interno della cartella corretta nel  \textit{branch}\textsubscript{G} main mediante un \textit{commit}\textsubscript{G} automatico.
\end{enumerate}
Il componente del gruppo che si è occupato della redazione di un documento dovrà semplicemente occuparsi di aprire una \textit{pull request}\textsubscript{G} verso il \textit{branch}\textsubscript{G} develop. Essa verrà verificata dal membro del gruppo che ricopre attualmente il ruolo di \textit{Verificatore}\textsubscript{G} e, in seguito, verrà approvata dal membro del gruppo che ricopre attualmente il ruolo di \textit{responsabile}\textsubscript{G}, che, infine, chiuderà la \textit{pull request}\textsubscript{G}, eseguendo l'\textit{automazione}\textsubscript{G}.\\
Questa \textit{automazione}\textsubscript{G} permette di eliminare il lavoro manuale ripetitivo e permette al team di concentrarsi sul contenuto invece che sulla gestione tecnica dei file.
    \subsection{Accertamento di qualità}
\subsubsection{Scopo}
L'accertamento della qualità ha l'obiettivo di prevenire i difetti nel prodotto software, garantendo la conformità ai processi adottati. Piuttosto che limitarsi a individuare errori a posteriori, questo approccio assicura che ogni fase dello sviluppo venga eseguita correttamente, monitorando l’applicazione delle best practice in modo non intrusivo attraverso strumenti di controllo. Questo processo può fare uso anche dei risultati di altri processi di supporto, come ad esempio del processo di verifica e del processo di validazione.

\subsubsection{Ciclo di Deming}
Il Ciclo di Deming, noto anche come Shewhart-Deming’s Learning-and-Quality Cycle, è un modello a quattro fasi ideato intorno al 1950 per introdurre miglioramenti specifici nei processi. Fondamentale per il miglioramento continuo, il ciclo si articola in:  
\begin{itemize}
    \item \textbf{Pianificare (Plan)}: definizione delle attività, delle scadenze, delle responsabilità e delle risorse necessarie per raggiungere obiettivi di miglioramento. Questa fase non riguarda la pianificazione di un progetto, ma l’organizzazione di azioni mirate all’ottimizzazione di un processo;
    \item \textbf{Eseguire (Do)}: implementazione delle attività pianificate, anche in modo esplorativo. Non si tratta dello sviluppo di un prodotto, ma dell’attuazione concreta delle azioni di miglioramento;
    \item \textbf{Valutare (Check)}: analisi dei risultati ottenuti per verificare se le azioni intraprese hanno prodotto gli effetti desiderati;
    \item \textbf{Agire (Act)}: consolidamento delle soluzioni efficaci, aggiornando il way of working, e individuazione di ulteriori opportunità di miglioramento.
\end{itemize}
Questo ciclo aiuta il gruppo a perfezionare continuamente la qualità del prodotto, garantendo che vengano raggiunti gli obiettivi di qualità che ci si era prefissati inizialmente e impedendo un peggioramento della stessa.

\subsubsection{Metriche}
Le metriche di qualità sono gli strumenti utilizzati per valutare la qualità di un prodotto software o di un processo di sviluppo. Servono a misurare il grado di conformità agli standard, identificare aree di miglioramento e garantire che il software soddisfi i requisiti attesi. \\
Si suddividono in diverse categorie, tra cui:
\begin{itemize}
    \item \textbf{Metriche di processo}: valutano l’efficienza dello sviluppo;
    \item \textbf{Metriche di prodotto}: misurano le caratteristiche del software.
\end{itemize}
Le metriche di qualità utilizzate in questo progetto sono riportate alla sezione §\ref{mdq}. Esse vengono rappresentate dal loro nome in lingua inglese, da un'abbreviazione, che genralmente consiste nella sequenza delle prime lettere delle parole del nome della metrica stessa, tutte in maiuscolo, e da una descrizione, che ne spiega brevemente il significato.

\subsubsection{Obiettivi di qualità}
Gli obiettivi di qualità sono riportati nel Piano di Qualifica e, come le metriche, sono suddivisi in obiettivi di qualità di processo e di prodotto.
Essi sono strutturati nel seguente modo:
\begin{itemize}
    \item \textbf{Metrica}: l'abbreviazione del nome della metrica;
    \item \textbf{Descrizione}: il nome in lingua inglese;
    \item \textbf{Valore accettabile}: valore per cui la metrica è da considerare soddisfatta, nonostante ci sia spazio per un miglioramento;
    \item \textbf{Valore ideale}: valore per cui la metrica viene considerata pienamente soddisfatta.
\end{itemize}

\subsubsection{Strumenti}
Gli strumenti utilizzati per il processo di accertamento di qualità sono:
\begin{itemize}
    \item \nameref{Google Sheet};
    \item \nameref{Jira}.
\end{itemize}
    \subsection{Verifica}
\subsubsection{Scopo}
Lo scopo del processo di \textit{verifica}\textsubscript{G} è fornire evidenza oggettiva che le uscite di un particolare segmento dello sviluppo \textit{software}\textsubscript{G} soddisfino tutti i requisiti specificati per esso. In altre parole, la \textit{verifica}\textsubscript{G} si concentra sull'accertarsi che lo sviluppo stia costruendo il \textit{sistema}\textsubscript{G} correttamente.
La \textit{verifica}\textsubscript{G} mira a ricercare la coerenza, la completezza e la correttezza di queste uscite. Essa fornisce inoltre supporto per la successiva conclusione che il \textit{software}\textsubscript{G} sia validato.

\subsubsection{Descrizione}
Questo processo viene effettuato almeno una volta prima del rilascio ufficiale di un prodotto in uno dei \textit{branch}\textsubscript{G} principali del \textit{repository}\textsubscript{G}. La \textit{verifica}\textsubscript{G} è affidata ai verificatori, che per garantire un'analisi obiettiva ed efficace, non devono essere gli stessi che hanno partecipato allo sviluppo del prodotto in questione.

\subsubsection{Analisi statica}
L'analisi statica è una forma di \textit{verifica}\textsubscript{G} del \textit{software}\textsubscript{G} che non richiede l'esecuzione del codice. Invece, esamina il codice sorgente, il codice oggetto o la \textit{documentazione}\textsubscript{G} per accertare la conformità a regole, l'assenza di difetti e la presenza di proprietà desiderate. Questa forma di \textit{verifica}\textsubscript{G} viene utilizzata anche per i documenti testuali. \\
L'analisi statica include tecniche che si basano sulla revisione manuale dell'oggetto di verifica, che può essere codice sorgente, \textit{documentazione}\textsubscript{G} o altri artefatti del processo di sviluppo. I due metodi di lettura utilizzati sono il walkthrough e l'inspection. \\
Nelle fasi iniziali è stato adottato l’approccio walkthrough. Tuttavia, man mano che acquisisce esperienza, il team potrà passare all’ispezione, rendendo il processo più rapido e ottimizzando l’uso delle \textit{risorse}\textsubscript{G}.

\subsubsubsection{Walkthrough}
Il walkthrough si basa su una lettura critica ad ampio spettro dell'oggetto in esame, che può essere codice sorgente, \textit{documentazione}\textsubscript{G} di progetto, specifiche o altri artefatti del processo di sviluppo. Esso coinvolge tipicamente gruppi misti composti da autori e verificatori, con ruoli distinti tra i partecipanti, dato che questa eterogeneità di prospettive aiuta a identificare un'ampia gamma di problemi. \\
Il processo tipico di un walkthrough si articola in diverse fasi:
\begin{enumerate}
    \item \textbf{Pianificazione}: gli autori e i verificatori si coordinano per organizzare la sessione di walkthrough;
    \item \textbf{Lettura}: i verificatori esaminano l'oggetto in esame individualmente prima della sessione, familiarizzando con il materiale. Durante la sessione, uno dei partecipanti può "guidare" la lettura, illustrando passo dopo passo il codice o il documento;
    \item \textbf{Discussione}: durante la sessione, i verificatori sollevano dubbi, pongono domande e identificano potenziali difetti o aree di incertezza. Autori e verificatori discutono i punti critici;
    \item \textbf{Correzione dei difetti}: dopo la sessione, la responsabilità di correggere i difetti identificati ricade sugli autori;
    \item \textbf{Documentazione}: ogni passo del walkthrough, incluse le attività svolte e i difetti identificati, viene documentato.
\end{enumerate}

\subsubsubsection{Ispezione}
L'ispezione è caratterizzata da un esame focalizzato su presupposti, che utilizza liste di controllo predefinite che guidano l'esame verso aree specifiche dove è più probabile trovare difetti, rendendo così non necessaria la completa lettura del prodotto in questione. \\
Il processo tipico di un'ispezione si articola in diverse fasi:
\begin{enumerate}
    \item \textbf{Pianificazione}: viene organizzata la sessione di ispezione e vengono identificati i partecipanti;
    \item \textbf{Definizione lista di controllo}: viene creata o selezionata una lista di controllo specifica per l'oggetto di verifica. Questa lista elenca gli aspetti specifici da esaminare selettivamente;
    \item \textbf{Lettura}: i verificatori esaminano l'oggetto di \textit{verifica}\textsubscript{G} individualmente, guidati dalla lista di controllo, annotando i potenziali difetti riscontrati;
    \item \textbf{Correzione dei difetti}: dopo la fase di ispezione, la responsabilità di correggere i difetti identificati ricade sugli autori dell'oggetto;
    \item \textbf{Documentazione}: ogni passo dell'ispezione, inclusa la lista di controllo utilizzata e i difetti identificati, viene documentato.
\end{enumerate}

\subsubsection{Analisi dinamica}
L'analisi dinamica è una forma di \textit{verifica}\textsubscript{G} del \textit{software}\textsubscript{G} che, a differenza dell'analisi statica, richiede l'esecuzione del codice per osservarne il comportamento ed evidenziare eventuali difetti. Questo processo prevede l'esecuzione di parti del \textit{software}\textsubscript{G} o dell'intero \textit{sistema}\textsubscript{G}, permettendo di valutarne il funzionamento su un insieme finito di casi di prova. Ogni caso di prova specifica i valori di ingresso, lo stato iniziale previsto e l'effetto atteso, che funge da riferimento per determinare l'esito del test. \\
L'analisi dinamica si realizza principalmente attraverso l'esecuzione di test, spesso automatizzati, per verificare la correttezza del \textit{software}\textsubscript{G} e identificare possibili errori. Questa tecnica è generalmente integrata con l'analisi statica, che esamina il codice senza eseguirlo, fornendo un quadro più completo della qualità del \textit{software}\textsubscript{G}. Esistono diverse tipologie di \textit{test}\textsubscript{G} nell'analisi dinamica, ciascuna con obiettivi specifici: i \textit{test}\textsubscript{G} di unità, i \textit{test}\textsubscript{G} di integrazione e i \textit{test}\textsubscript{G} di \textit{sistema}\textsubscript{G}. Un aspetto cruciale dell'analisi dinamica è la ripetibilità e l'\textit{automazione}\textsubscript{G} dei test, che consentono di eseguire verifiche sistematiche, ridurre i tempi di analisi e garantire coerenza nei risultati. L'\textit{automazione}\textsubscript{G} permette inoltre di eseguire i \textit{test}\textsubscript{G} in modo efficiente, ottimizzando il processo di \textit{verifica}\textsubscript{G} del \textit{software}\textsubscript{G}.
    \subsection{Validazione}
\subsubsection{Scopo}
Lo scopo del processo di \textit{validazione}\textsubscript{G} è di confermare, attraverso esami e fornitura di evidenze oggettive, che il prodotto \textit{software}\textsubscript{G} finale soddisfa i requisiti stabiliti e riportati nell'Analisi dei Requisiti. 

\subsubsection{Esecuzione del processo}
Il processo di \textit{validazione}\textsubscript{G} viene eseguito in presenza del proponente dopo aver avuto un esito positivo dai \textit{test}\textsubscript{G} di unità, di integrazione e di \textit{sistema}\textsubscript{G}. Lo svolgimento di questo processo è suddiviso in due fasi:
\begin{enumerate}
    \item \textbf{Verifica del tracciamento dei requisiti}: il team presenta all’azienda proponente l’analisi del tracciamento dei requisiti, dimostrando così che ogni \textit{requisito}\textsubscript{G} è stato effettivamente integrato nel prodotto sottoposto a \textit{validazione}\textsubscript{G};
    \item \textbf{Fase di collaudo}: in collaborazione con l’azienda proponente, il team procede all’esecuzione del prodotto. Durante questa fase, vengono effettuati i \textit{test}\textsubscript{G} di accettazione per verificare la conformità del \textit{sistema}\textsubscript{G} rispetto alle specifiche richieste e, se tutti i \textit{test}\textsubscript{G} di accettazione eseguiti forniscono un esito positivo, il proponente considererà il prodotto valido.
\end{enumerate}

\subsubsection{Test di accettazione}
\begin{itemize}
    \item \textbf{Redazione}: i \textit{test}\textsubscript{G} di accettazione sono stabiliti in contemporanea alla redazione dell'Analisi dei Requisiti e vengono svolti alla presenza del \textit{committente}\textsubscript{G};
    \item \textbf{Descrizione}: i \textit{test}\textsubscript{G} di accettazione hanno come obiettivo principale di verificare che il prodotto soddisfi i requisiti utente, cioè le aspettative del \textit{committente}\textsubscript{G} così come definite nel \textit{capitolato}\textsubscript{G}. Questa attività si svolge in modo formale alla presenza del \textit{committente}\textsubscript{G}, il quale supervisiona o esegue direttamente i casi di prova. Tali casi sono derivati dai requisiti utente e mirano a dimostrare la conformità del \textit{software}\textsubscript{G} alle specifiche concordate. Il buon esito di questa fase è determinante per l’accettazione finale del prodotto e ne consente il rilascio ufficiale.
\end{itemize}
    \subsection{Revisione congiunta con il proponente}
\subsubsection{Scopo}
Lo scopo del processo di revisione congiunta con il proponente è garantire che il prodotto \textit{software}\textsubscript{G} in sviluppo soddisfi le aspettative del proponente, coinvolgendolo direttamente nella \textit{verifica}\textsubscript{G} degli artefatti. Questo processo aiuta a individuare tempestivamente difetti, malintesi sui requisiti e possibili problematiche, permettendo di apportare modifiche in tempo utile. Inoltre, favorisce la comunicazione e la collaborazione tra team e proponente, riducendo il rischio di insoddisfazione e costose rilavorazioni nelle fasi finali del progetto.

\subsubsection{Implementazione}
Il processo di revisione congiunta con il proponente viene implementato attraverso una serie di fasi che permettono di verificare e migliorare il prodotto \textit{software}\textsubscript{G} in modo collaborativo:
\begin{itemize}
    \item \textbf{Pianificazione della revisione}: il primo passo della revisione è definire lo scopo e gli artefatti da esaminare. Si selezionano i partecipanti, tra cui membri del team di sviluppo, rappresentanti del proponente e \textit{stakeholder}\textsubscript{G}, e si organizzano i dettagli dell’incontro. Infine, i materiali vengono preparati e condivisi per l'analisi;
    \item \textbf{Preparazione}: per un’efficace revisione, i partecipanti ricevono in anticipo la \textit{documentazione}\textsubscript{G} da analizzare. Vengono assegnati ruoli chiave: il moderatore guida la discussione, i revisori esaminano gli artefatti e un segretario registra le osservazioni. Si definiscono inoltre criteri di valutazione come coerenza, chiarezza e qualità del codice;
    \item \textbf{Esecuzione della revisione}: durante l’incontro, il team di sviluppo presenta gli artefatti al proponente, che fornisce \textit{feedback}\textsubscript{G} su eventuali criticità. Tutti i commenti vengono raccolti e documentati per un'analisi successiva;
    \item \textbf{Analisi dei risultati e azioni correttive}: dopo la revisione, il team classifica i problemi per priorità e definisce le azioni correttive. Viene redatto un report con i risultati e le modifiche necessarie, che viene condiviso con il proponente per garantire trasparenza e allineamento alle aspettative;
    \item \textbf{Follow-up e validazione}: dopo le modifiche, si può fare un'ulteriore revisione per verificare la risoluzione dei problemi. L'obiettivo è ottenere la conferma del proponente e ridurre il rischio di rilavorazioni future;
\end{itemize}

\subsubsection{Strumenti}
Gli strumenti utilizzati per il processo di revisione congiunta con il proponente sono:
\begin{itemize}
    \item \nameref{Google Calendar};
    \item \nameref{Microsoft Teams}.
\end{itemize}
    \subsection{Verifiche ispettive interne}
\subsubsection{Scopo}
Lo scopo del processo di verifiche ispettive interne è valutare se le attività e i risultati relativi alla qualità soddisfano le disposizioni pianificate e se queste sono attuate efficacemente. Le verifiche ispettive interne contribuiscono a garantire la qualità del \textit{software}\textsubscript{G}, valutare l'efficacia dei processi, identificare non conformità e fornire \textit{feedback}\textsubscript{G} per il miglioramento continuo, supportando così la produzione di \textit{software}\textsubscript{G} di alta qualità.

\subsubsection{Implementazione}
Per implementare il processo di verifiche ispettive interne, è necessario seguire questi passi:
\begin{itemize}
    \item \textbf{Pianificazione}: definire l’ambito, gli obiettivi e gli artefatti da verificare, come \textit{documentazione}\textsubscript{G}, codice o processi. Stabilire un piano con tempistiche e \textit{risorse}\textsubscript{G} necessarie;
    \item \textbf{Assegnazione dei ruoli}: designare i membri del team di verifica, che dovrebbero essere indipendenti dalle attività verificate. Ogni membro avrà un \textit{ruolo}\textsubscript{G} specifico, come analizzare i processi, raccogliere dati e redigere il report;
    \item \textbf{Esecuzione della verifica}: condurre l'ispezione sistematica dei processi e dei risultati, confrontandoli con gli standard e le disposizioni pianificate. Raccogliere e documentare le non conformità o le aree di miglioramento;
    \item \textbf{Analisi dei risultati}: valutare i dati raccolti, classificando le non conformità e le deviazioni. Identificare le cause e le aree da migliorare;
    \item \textbf{Feedback\textsubscript{G} e azioni correttive}: redigere un report con i risultati e le raccomandazioni. Fornire un \textit{feedback}\textsubscript{G} alle parti coinvolte, suggerendo azioni correttive e preventive per migliorare i processi;
    \item \textbf{Monitoraggio e follow-up}: verificare che le azioni correttive siano state implementate e che le problematiche siano state risolte, promuovendo il miglioramento continuo.
\end{itemize}

\subsubsection{Strumenti}
Gli strumenti utilizzati per il processo di verifiche ispettive interne sono:
\begin{itemize}
    \item \nameref{Discord};
    \item \nameref{Jira}.
\end{itemize}
    \subsection{Risoluzione dei problemi}
\subsubsection{Scopo}
Lo scopo del processo di risoluzione dei problemi è identificare, analizzare e risolvere tempestivamente i problemi che emergono durante il ciclo di vita del software. Questo processo garantisce una gestione efficace ed efficiente dei problemi, dalla loro rilevazione fino alla completa risoluzione e verifica, contribuendo così alla qualità complessiva del prodotto software.

\subsubsection{Implementazione}
Il processo di risoluzione dei problemi viene implementato attraverso una serie di fasi strutturate, con l’obiettivo di identificare e risolvere tempestivamente le criticità che emergono durante lo sviluppo del software. L’implementazione di questo processo si articola nei seguenti passaggi:
\begin{itemize}
    \item \textbf{Sviluppo di una strategia di gestione dei problemi}: definizione di un approccio strutturato per affrontare i problemi, stabilendo ruoli e responsabilità, flussi di comunicazione e strumenti da utilizzare;
    \item \textbf{Registrazione e classificazione dei problemi}: ogni problema rilevato viene registrato in uno storico, includendo dettagli come descrizione, data di rilevamento, origine e segnalatore. I problemi vengono poi classificati per gravità e priorità, permettendo una gestione efficace e un'analisi delle tendenze nel tempo;
    \item \textbf{Analisi dei problemi e individuazione delle soluzioni}: dopo la registrazione, il problema viene analizzato per identificarne le cause radice. Sulla base di questa analisi, vengono proposte e valutate soluzioni adeguate. Questa fase può coinvolgere l'analisi del codice, dei log di sistema, della documentazione o il confronto con i membri del team;
    \item \textbf{Implementazione della soluzione}: dopo aver individuato la soluzione più adatta, questa viene applicata. L’intervento può riguardare modifiche al codice, configurazioni di sistema, aggiornamenti della documentazione o altre azioni correttive;
    \item \textbf{Verifica dell’efficacia della correzione}: una volta implementata la soluzione, è essenziale verificare che il problema sia stato effettivamente risolto e che non siano stati introdotti nuovi difetti. Questa verifica avviene attraverso test, revisioni o altri controlli di qualità;
    \item \textbf{Monitoraggio e gestione continua dei problemi}: il processo prevede un costante monitoraggio dello stato dei problemi aperti, assicurando che siano assegnati e gestiti fino alla loro completa risoluzione;
\end{itemize}

\subsubsection{Strumenti}
Gli strumenti utilizzati per il processo di risoluzione dei problemi sono:
\begin{itemize}
    \item \nameref{Jira}.
\end{itemize}

\end{document}
