\subsection{Fornitura}
\subsubsection{Scopo}
La fornitura è il processo che descrive le attività svolte dal fornitore, coinvolge pianificazione, acquizisione e gestione delle risorse necessarie. Il processo determina le procedure e le risorse necessarie per gestire e garantire il progetto. L'obiettivo di questo processo è garantire l'efficienza\textsubscript{G} e la conformità ai requisiti del progetto per raggiungere gli obiettivi stabiliti dal proponente. 
\subsubsection{Implementazione}
Il processo di fornitura è composto delle seguenti fasi:
\begin{enumerate}
    \item \textbf{Risposta alla richiesta}: il fornitore, dopo aver analizzato i requisiti di una richiesta del proponente (il Capitolato) prepara in risposta una proposta;
    \item \textbf{Negoziazione}: il fornitore negozia e stipula un contratto con il propoennte;
    \item \textbf{Pianificazione}: il fornitore rivede i requisiti e valuta le opzioni per lo sviluppo del prodotto software in base ad un'analisi dei rischi associati alle varie opzioni per definire la struttura di un piano di gestione del progetto al fine di garantire la qualità del prodotto finale;
    \item \textbf{Esecuzione e controllo}: il fornitore esegue il piano di gestione del progetto, monitorando il progresso e la qualità del prodotto per tutto il ciclo di vita del prodotto;
    \item \textbf{Revisione}: il fornitore coordina le cominicazioni con il proponente e partecipa a riunioni e revisioni. Il fornitore verifica e convalida il processo per dimostrare che i prodotti e i processi soddisfano i requisiti;
    \item \textbf{Consegna}: il fornitore consegna il prodotto finale, fornendo assistenza al proponente a supporto del prodotto consegnato.
\end{enumerate}

\subsubsection{Gestione}
Al fine di identificare e comprendere i bisogni del Proponente, per poter individuare i requisiti e i vinsoli del progetto, deve essere mantenuta costante comunicazione con il Proponente, mediante riunioni SAL periodiche calendarizzate, in presenza o su Microsoft Teams\textsubscript{G} e con scambio di messaggi su Microsoft Teams\textsubscript{G} e mail qualora fosse necessario. Il dialogo continuo permette anche una valutazione costante dell'operato del fornitore, in modo da apportare correzioni, integrazioni e miglioramenti in modo tempestivo, incrementale e costruttivo.

\subsubsection{Documentazione fornita}
Sono di seguito elencati i documenti che PEBKAC si impegna a consegnare ai Committenti e al Proponente: 

\subsubsubsection{Piano di Progetto}
Il Piano di Progetto V1.0.0, redatto dal Responsabile, offre 
\subsubsubsection{Analisi dei requisiti}

\subsubsubsection{Piano di Qualifica}

\subsubsubsection{Glossario}

\subsubsubsection{Lettera di Presentazione}


\subsubsection{Strumenti}