\subsection{Documentazione}
\subsubsection{Scopo}
Il processo di documentazione procede sempre di pari passo con tutte le attività di sviluppo, con l'obiettivo di fornire tutte le informazioni necessarie, sotto forma di testo scritto facilmente consultabile, inerenti al prodotto e alle attvità stesse. Oltre a svolgere un ruolo essenziale nella descrizione del prodotto per coloro che lo sviluppano, lo distribuiscono e lo utilizzano, la documentazione svoge un ruolo di storicizzazione e di supporto alla manutenzione. 

\subsubsection{Documenti}
In questa sezione viene descritto il piano che identificqa i documenti da produrre durante il ciclo di vita del prodotto software. Tutti i documenti da redigere sono presentati nella tabella che segue, vengono esclusi i documenti presentati per la candidatura per il progetto didattico, quali \textit{Lettera di presenzazione}, \textit{Preventivo dei costi e assunzione degli impegni} e \textit{Analisi dei capitolati}.


% Creazione della tabella
\begin{table}[H]
    \centering
   \begin{tabularx}{\textwidth}{X|X|X|X|X}
        \textbf{Nome} & \textbf{Scopo} & \textbf{Redattore} & \textbf{Destinatari} & \textbf{Consegne} \\ \hline
        Analisi dei requisiti    & Definizione dei requisiti utente    & Analista & Azienda proponente, Docenti & RTB\textsubscript{G}, PB\textsubscript{G}    \\ \hline
        Norme di progetto    & Regolamento normativo del gruppo    & Amministratore, Responsabile &  Docenti & RTB, PB    \\ \hline
        Piano di Progetto   & Definizione temporale scadenze e progressi    & Responsabile &  Docenti & RTB, PB   \\ \hline
        Piano di qualifica   & Definizione qualotà e testing    & Amministratore &  Docenti & RTB, PB   \\ \hline
         Verbali esterni   & Tracciamento riunioni esterne   & Responsabile, Amministratore &  Azienda proponente, Docenti & Candidatura, RTB, PB   \\ \hline
         Verbali interni   & Tracciamento riunioni interne   & Responsabile, Amministratore &  Docenti & Candidatura, RTB, PB  
       
    \end{tabularx}
    \caption{Documenti del ciclo di vita del prodotto SW}
\end{table}


\subsubsection{Progettazione e sviluppo}
In questa sezione vengono presentati gli standard e le regole (nello specifico di stile) a cui i membri di PEBKAC si devono attengono per la stesura dei documenti relativi al progetto.

\subsubsubsection{Template}
Per la stesura dei documenti il gruppo ha creato un template in formato Latex\textsubscript{G}. Il template fornisce una struttura e un formato predefinito per semplificare la creazione di documenti, al fine di garantire coerenza, efficienza e standardizzazione della presentazione. 
Il template è progettato per essere facile da usare, dovendo inserire solo con piccole modifiche per rispecchiare le specificità di ciascun tipo di documento.\\
In particolare nel template è definite la pagina di copertina con intestazione contenente logo informazioni del gruppo e dell'Università di Padova, titolo del docunto, informazioni sul documento (uso, destinatari) e un breve abstract del contenuto, oltre che altre specifiche di stile come il titolo dell'indice in italiano e il numero di pagina come \texttt{X di Tot}, dove \texttt{X} è il numero della pagina e \texttt{Tot} è il numero totale di pagine.
\subsubsubsubsection{Parametri}
Nel principale file Latex del template sono definiti una serie di comandi personalizzati per l'inserimento automatico delle informazioni come titolo, data, uso, destinatari e abstract. \\
Sono inoltre già presenti ma commentate le voci necessarie solo per i verbali (vedi  \hyperref[sec: struttura verbali]{§4.1.3.3 Verbali})
\subsubsubsection{Struttura del documento}
Tutti i documenti prodotti da PEBKAC presentano la medesima struttura, alla quale ogni membro di deve attenere durante la procedura di stesura e modifica.
\begin{itemize}
    \item \textbf{Pagina di copertina}: come nella sezione Template precedente;
    \item \textbf{Registro delle versioni}: questo registro è utilizzato per tenere traccia delle varie versioni per permettere di comprendere velocemente chi ha realizzato o modificato determinate sezioni della documentazione e quando.. Il registro presenta le versioni ordinate a partire dalla versione più recente;
     \item \textbf{Indice}: questo registro è utilizzato per tenere traccia delle varie versioni per permettere di comprendere velocemente chi ha realizzato o modificato determinate sezioni della documentazione e quando.. Il registro presenta le versioni ordinate a partire dalla versione più recente;
     \item \textbf{Indice}: presente per facilitare la consultazionedel documento, dotato di sezioni. Il suo scopo è di facilitare e agevolare l’accesso ad un determinato contenuto all'interno nel documento;
     \item \textbf{Contenuto}: il contenuto vero e proprio del documento.
\end{itemize}

\subsubsubsection{Verbali}\label{sec: struttura verbali}
I verbali differiscono dalla struttura precedentemente esposta in quanto ad essi prevedono delle sezioni aggiuntive ed obbligatorie:
\begin{itemize}
    \item \textbf{Pagina di copertina}: nel caso di un verbale tra le informazioni sul documento compaiono anche i nominativi con i rispettivi i ruoli dei membri che hanno lavorato alla alla loro produzione;
    \item \textbf{informazioni generali}: la prima sezione di un verbale deve sempre essere quella nominata "Informazioni generali" che prevede, sottoforma di elenco puntato, le seguenti informazioni:
        \begin{itemize}
            \item Tipo di riunione,
            \item Luogo in cui si è tenuta la riunione (anche se telematica),
            \item Data in cui si è tenuta la riunione,
            \item Ora di inizio della riunione,
            \item Ora di fine della riunione,
            \item Membri presenti ed eventuali altre persone alla riunione,
            \item Membri assenti dalla riunione;
        \end{itemize}
     \item \textbf{Todo}: l'ultima sezione di un verbale deve sempre essere quella che elenca i task\textsubscript{G} emersi durante la riunione da aggiungere al backlog\textsubscript{G}. Questi vengono presentati sotto forma di tabella a due colonne:
     \begin{itemize}
         \item \textbf{Assegnatario}: il membro a cui quel task è stato assegnato, nel caso in cui non ve ne sia uso ma il task possa essere autoassegnato da uno dei membri si scriverà "autoassegnazione" in corsivo;
         \item \textbf{Task Todo}: denominazione del task.
     \end{itemize}
     
\end{itemize}

\subsubsubsection{Nomenclatura}
La nomenclatura per i documenti si ottiene unendo il nome del file in \textit{Snake\_Case} quindi con le parole separate da  un underscore (\textit{\_}) (\texttt{Nome\_del\_File}), un underscore (\textit{\_}) e la sua versione (\texttt{1.2.3}), ottendendo per esempio \texttt{Norme\_di\allowbreak{}\_Progetto\_1.2.3.pdf}. Nel caso di documenti il cui nome contiene una data, essa si inserisce dopo il nome, ma prima della versione, sempre usando gli underscores come separatori, nella forma YYYY-MM-DD: YYYY rappresenta l'anno, MM il mese e DD il giorno, sempre scritto in due cifre.
\subsubsubsubsection{Verbali}
Per quanto riguarda i  verbali, per facilitarne l'ordinamento) il loro nome è la data in cui la riunione di è tenuta nella forma YYYY-MM-DD: YYYY rappresenta l'anno, MM il mese e DD il giorno, sempre scritto in due cifre. Nel caso si tratti di un verbale esterno viene aggiunta una \texttt{E}, sempre separata da underscores tra la data e la versione.


\subsubsubsection{Versionamento}
La versione di un documento è del tipo [\textbf{x}].[\textbf{y}].[\textbf{z}]:
\begin{itemize}
    \item \textbf{z}: è un numero intero che incrementato dal Redattore ad ogni modifica;
    \item \textbf{y}: è un numero intero incrementato dal Verificatore ad ogni verifica;
    \item \textbf{x}: è un numero intero che viene incrementato dal Responsansabile dopo la sua approvazione (versione di produzione).
\end{itemize}

\subsubsubsection{Convenzioni stilistiche}
\begin{itemize}
    \item \textbf{Date}: tutte le date nella documentazione prevedono il seguente formato YYYY-MM-DD, dove DD indica il giorno a due cifre, MM il mese a due cifre e YYYY l'anno a 4 cifre;
    \item \textbf{Elenchi}: elenchi puntati o numerati, ogni punto inizia con la lettera mauiscola e termina con ";" ad eccezione dell'ultimo che termina con ".";
    \item \textbf{Menzioni}: ogni menzione ad una persona, interna o esterna, avviene nel formato Nome Cognome;
    \item \textbf{Riferimenti interni}: i riferimenti a sezioni interne allo stesso documento devono essere ripostati seguendo la notazione \texttt{§1.2 Nome sezione}, dove \texttt{§1.2} è il numero della sezione. Inoltre questi riferimenti devono essere opportunamenti collegati tramite link al paragrafo indicato, senza alterare lo stile del testo;
    \item \textbf{Riferimenti interni}: i riferimenti a sezioni di documenti esterni devono essere ripostati seguendo la notazione \texttt{Nome Documento (versione di riferimento), Nome sezione};
     \item \textbf{Link URL}: possono essere estesi o avere una visualizzazione abbreviata, ma sempre visualizzati di colore blu;
    \item \textbf{Caratteri maiuscoli}: devono essere utilizzati per
        \begin{itemize}
            \item Le iniziali dei nomi;
            \item Le lettere che compongono degli acronimi e le iniziali delle rispettive definizioni;
            \item Le iniziali dei ruoli svolti dai componenti del gruppo;
            \item Le iniziali dei ruoli definiti all'interno del progetto didattico;
            \item La prima lettera di ogni elenco puntato.
        \end{itemize}
    \item \textbf{Grassetto}: devono essere visualizzati in grssetto
        \begin{itemize}
            \item I titoli di sezioni/sottosezioni/paragrafi di un documento;
            \item Le parore che meritano enfasi;
            \item Le defizioni negli elenchi puntati.
        \end{itemize}
    \item \textbf{Caption}: ogni immagine o tabella deve avere una caption, utile a fornire una breve descrizione o spiegazione del contenuto visivo.
\end{itemize}

\subsubsection{Ciclo di vita dei documenti}
Ogni documento segue le fasi del seguente workflow\textsubscript{G}:
\begin{enumerate}
    \item \textbf{Assegnazione}: il gruppo assegna un documento a uno o più redattori, affiancati da uno o più verificatori;
    \item \textbf{Branch}: si crea un branch per lo sviluppo del documento nell’apposita repository\textsubscript{G} Docs;
    \item \textbf{Template}: si copia il Template all'interno della cartella appropiata;
    \item \textbf{Stesura}: si redige il documento o una sua sezione. Qualora serva un elevato parallelismo di lavoro è possibile usare Google Drive per la prima stesura e successivamente caricare il documento all’interno del branch;
    \item \textbf{Commit}: si esegue la commit sul branch creato;
    \item \textbf{Pull Request}: si apre una pull request dal branch appena creato verso il branch develop: se il documento non è pronto per la verifica, ma ha bisogno di ulteriori modifiche, si apre la pull request in modalità draft, per marcarla successivamente come “Ready to Review”, altrimenti in modalità normale;
    \item \textbf{Verifica}: se il verificatore richiede modifiche si ripete, in ordine, dal punto 3 al punto 5;
    \item \textbf{Chiusura branch}: si elimina, quando la pull request viene chiusa o risolta, il branch creato.
\end{enumerate}
Per la versione finale di un documento spetta al Responsabile conferire l’approvazione definitiva, annotando opportunamente nel registro delle versioni la versione \texttt{x.0.0} e la sua approvazione finale.


\subsubsection{Strumenti}
\begin{itemize}
    \item Latex
    \item Visual Studio Code
    \item GitHub
\end{itemize}
