\section{Introduzione}
\subsection{Scopo del documento}
Il presente documento ha l'obiettivo di definire le strategie di verifica e validazione messe in atto per garantire la qualità del prodotto e dei processi associati al progetto. La creazione del piano di qualifica è progressiva e incrementale nel tempo
per consentire al team di riportare gli esiti delle verifiche effettuate nel tempo, con l'intento di mantenere ed eventualmente incrementare la qualità dell'intero progetto.
\subsection{Scopo del prodotto}
Il progetto "Vimar GENIALE" mira a sviluppare un'applicazione intelligente che supporti installatori elettrici nell'uso di dispositivi Vimar\textsubscript{G}, facilitando l'accesso alle informazioni tecniche sui prodotti, rispondendo a domande poste in linguaggio naturale.
La tecnologia alla base prevede l'uso di modelli di \textit{LLM}\textsubscript{G} e di tecniche \textit{RAG}\textsubscript{G}, con una struttura di gestione basata su \textit{container}\textsubscript{G} e integrata in un ambiente \textit{cloud\textsubscript{G}}.
Il sistema include tre componenti principali: una \textit{applicativo web responsive}\textsubscript{G}, un \textit{applicativo server}\textsubscript{G} e un'\textit{infrastruttura cloud-ready}\textsubscript{G}. 
\subsection{Glossario}
Per evitare ambiguità relative al linguaggio utilizzato nei documenti, viene fornito il Glossario V1.0.0, nel quale si possono trovare tutte le definizioni di termini che hanno un significato specifico che vuole essere disambiguato. Tali termini sono marcati con una G a pedice.
\subsection{Riferimenti}
\subsubsection{Riferimenti normativi}
\begin{itemize}
    \item \textbf{Norme di Progetto v1.0.0}
    \item \textbf{PD1 - Regolamento del progetto didattico} \\
    \url{https://www.math.unipd.it/~tullio/IS-1/2024/Dispense/PD1.pdf} 
    \item \textbf{Capitolato d'Appalto C2}: Vimar GENIALE \\
    \url{https://www.math.unipd.it/~tullio/IS-1/2024/Progetto/C2.pdf}
    \end{itemize}
\subsubsection{Riferimenti informativi}
\begin{itemize}
    \item \textbf{T7 - Qualità del Software} \\
    \url{https://www.math.unipd.it/~tullio/IS-1/2024/Dispense/T07.pdf}
    \item \textbf{T8 - Qualità del processo} \\
    \url{https://www.math.unipd.it/~tullio/IS-1/2024/Dispense/T08.pdf}
    \item \textbf{T9 - Verifica e Validazione: Introduzione} \\
    \url{ https://www.math.unipd.it/~tullio/IS-1/2024/Dispense/T09.pdf}
    \item \textbf{T10 - Verifica e Validazione: Analisi statica} \\
    \url{ https://www.math.unipd.it/~tullio/IS-1/2024/Dispense/T10.pdf}
    \item \textbf{T11 - Verifica e Validazione: Analisi dinamica (Testing)} \\
    \url{ https://www.math.unipd.it/~tullio/IS-1/2024/Dispense/T11.pdf}
    \item \textbf{Glossario v1.0.0} \\
    \end{itemize}