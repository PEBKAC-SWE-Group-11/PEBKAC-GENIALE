\section{Strategie di testing}
Per dimostare che i \textit{requisiti\textsubscript{G}} individuati dagli analisti ed elencati nella sezione omonima dell'Analisi dei Requisiti siano soddisfatti, è necessario che vengano realizzati dei \textit{test\textsubscript{G}} appositi che verranno eseguiti sul prodotto sia in fase di codifica che in fase di \textit{verifica\textsubscript{G}} e \textit{revisione\textsubscript{G}}.\\
I \textit{test\textsubscript{G}} realizzabili possono essere suddivisi in quattro categorie principali:
\begin{itemize}
    \item \textbf{\textit{Test\textsubscript{G}} di unità}: verificano il corretto funzionamento di una singola unità di codice indipendente (ad esempio una funzione), assicurandosi che produca i risultati attesi al variare dei possibili input\textsubscript{G}, e vengono generalmente automatizzati per facilitare l'individuazione degli errori durante la fase di sviluppo;
    \item \textbf{\textit{Test\textsubscript{G}} di integrazione}: verificano il corretto funzionamento delle interazioni tra diverse unità di codice o componenti di un sistema, assicurandosi che, una volta integrati, i vari moduli lavorino insieme senza problemi, rilevando eventuali errori nelle interfacce e nei flussi di dati tra di essi;
    \item \textbf{\textit{Test\textsubscript{G}} End-to-End}: verificano il funzionamento complessivo di un sistema eseguendo casi d'uso dall'inizio alla fine, comprese le interazioni con altri sistemi o applicazioni esterne, per garantire il corretto funzionamento di tutti gli elementi insieme, simulando l'esperienza dell'utente in un contesto realistico.
    
\end{itemize}

\subsection{Notazione dei test}
\'E stato decisa come notazione per identificare univocamente i \textit{test\textsubscript{G}} la seguente:
\begin{center}
    \textbf{T[Tipologia][Numero]}
\end{center}
\textbf{Tipologia} indica la tipologia del \textit{test\textsubscript{G}}:
\begin{itemize}
    \item \textbf{U}: di unità;
    \item \textbf{I}: di integrazione;
    \item \textbf{S}: di sistema;
%    \item \textbf{E}: End-to-End;
    \item \textbf{A}: di accettazione.
\end{itemize}
Ogni \textit{test\textsubscript{G}} si trova in uno \textbf{Stato}, che può essere:
\begin{itemize}
    \item \textbf{V}: verificato. Questo stato indica che il \textit{test\textsubscript{G}} ha fornito un esito positivo;
    \item \textbf{NV}: non verificato. Questo stato indica che il \textit{test\textsubscript{G}} ha fornito un esito negativo;
    \item \textbf{NI}: non implementato. Questo stato indica che il \textit{test\textsubscript{G}} non è ancora stato implementato, e quindi non fornisce nessun esito.
\end{itemize}

\subsection{\textit{Test\textsubscript{G}} di unità}
I \textit{test\textsubscript{G}} di unità sono una tipologia di \textit{test\textsubscript{G}} utilizzata per verificare singoli componenti o unità di codice in isolamento, al fine di garantire che funzionino correttamente. Un'unità di codice può essere una funzione, un metodo, una classe o un modulo, a seconda del livello di granularità scelto. I \textit{test\textsubscript{G}} di unità vengono solitamente scritti dagli sviluppatori durante o immediatamente dopo la scrittura del codice e vengono utilizzati per:
\begin{itemize}
    \item Validare il comportamento del codice, assicurandosi che ogni unità fornisca risultati corretti per un determinato insieme di input;
    \item Facilitare la manutenzione del \textit{software\textsubscript{G}}, individuando rapidamente errori introdotti da modifiche;
    \item Promuovere la modularità, progettando concettualmente componenti indipendenti e riutilizzabili.
\end{itemize}
Per la realizzazione di questa categoria di \textit{test\textsubscript{G}} per questo progetto sarà utilizzato il \textit{framework\textsubscript{G}} \textit{unittest\textsubscript{G}} per Python, dato che quest'ultimo è il linguaggio scelto per la realizzazione del \textit{backend\textsubscript{G}}.\\
I \textit{test\textsubscript{G}} di unità, insieme ai \textit{test\textsubscript{G}} di integrazione, come richiesto nel capitolato, devono avere un \textit{coverage\textsubscript{G}} minimo pari al 75\% (opzionalmente un \textit{coverage\textsubscript{G}} minimo pari al 90\%).
\begin{table}[H]
\centering
\begin{tabularx}{\textwidth}{>{\hsize=0.2\hsize}>{\centering\arraybackslash}X|X|>{\hsize=0.1\hsize}>{\centering\arraybackslash}X}
    \textbf{Codice} & \textbf{Descrizione} & \textbf{Stato} \\
        TU-1 & Verifica che il metodo create\_session() inserisca correttamente una nuova sessione nel database & V \\
        \hline
        TU-2 & Verifica che il metodo read\_session() recuperi correttamente una sessione dal database & V \\
        \hline
        TU-3 & Verifica che il metodo create\_conversation() crei correttamente una nuova conversazione nel database & V \\
        \hline
        TU-4 & Verifica che il metodo read\_conversations() recuperi correttamente tutte le conversazioni associate a una sessione dal database & V \\
        \hline
        TU-5 & Verifica che il metodo read\_conversation\_by\_id() recuperi correttamente una conversazione dal database tramite il suo ID & V \\
        \hline
        TU-6 & Verifica che il metodo delete\_conversation() elimini correttamente una conversazione dal database & V \\
        \hline
        TU-7 & Verifica che il metodo add\_message() aggiunga correttamente un nuovo messaggio a una conversazione & V \\
        \hline
        TU-8 & Verifica che il metodo read\_messages() recuperi correttamente tutti i messaggi associati a una conversazione & V \\
        \hline
        TU-9 & Verifica che il metodo read\_feedback() recuperi correttamente tutti i feedback associati a un messaggio & V \\
        \hline
        TU-10 & Verifica che il metodo add\_feedback() aggiunga correttamente un feedback a un messaggio & V \\
    \end{tabularx}
\end{table}
\begin{table}[H]
   \centering
   \begin{tabularx}{\textwidth}{>{\hsize=0.2\hsize}>{\centering\arraybackslash}X|X|>{\hsize=0.1\hsize}>{\centering\arraybackslash}X}
        TU-11 & Verifica che il metodo read\_num\_positive\_feedback() recuperi correttamente il numero di feedback positivi & V \\
        \hline
        TU-12 & Verifica che il metodo read\_num\_negative\_feedback() recuperi correttamente il numero di feedback negativi & V \\
        \hline
        TU-13 & Verifica che il metodo read\_num\_conversations() recuperi correttamente il numero totale di conversazioni & V \\
        \hline
        TU-14 & Verifica che l'endpoint restituisca correttamente gli header CORS (Cross-Origin Resource Sharing)  & V \\
        \hline
        TU-15 & Verifica che l'endpoint restituisca un codice di stato 404 se non definito & V \\
        \hline
        TU-16 & Verifica che il metodo execute\_query() venga chiamato correttamente con la query specificata & V \\
        \hline
        TU-17 & Verifica che il metodo fetch\_one() venga chiamato correttamente con la query e i parametri specificati & V \\
        \hline
        TU-18 & Verifica che il metodo fetch\_all() venga chiamato correttamente con la query specificata e restituisca tutti i risultati attesi & V \\
        \hline
        TU-19 & Verifica che il metodo close() venga chiamato correttamente & V \\
        \hline
        TU-20 & Verifica che una sessione venga creata correttamente senza specificare la data di creazione & V \\
        \hline
        TU-21 & Verifica che una sessione venga creata correttamente specificando la data di creazione & V \\
        \hline
        TU-22 & Verifica che una conversazione venga creata correttamente & V \\
        \hline
        TU-23 & Verifica che un messaggio venga creato correttamente & V \\
        \hline
        TU-24 & Verifica che venga generato correttamente un embedding per una query quando la richiesta HTTP ha successo & V \\
        \hline
        TU-25 & Verifica che venga generato correttamente un errore quando la risposta HTTP non contiene l'embedding & V \\
        \hline
        TU-26 & Verifica che venga generato correttamente un errore quando la richiesta HTTP fallisce & V \\
        \hline
        TU-27 & Verifica che una query venga eseguita correttamente & V \\
        \hline
        TU-28 & Verifica che una query venga eseguita correttamente e restituisca un solo risultato & V \\
        \hline
        TU-29 & Verifica che una query venga eseguita correttamente e restituisca tutti i risultati & V \\
        \hline
        TU-30 & Verifica che la connessione al database venga chiusa correttamente & V \\
        \hline
        TU-31 & Verifica che la connessione al database venga creata correttamente utilizzando psycopg2 & V \\
        \hline
        TU-32 & Verifica che l'endpoint: /api/test restituisca correttamente il messaggio di successo & V \\
        \hline
        TU-33 & Verifica che l'endpoint: /api/question/1 gestisca correttamente una richiesta di domanda & V \\
        \hline
        TU-34 & Verifica che l'endpoint: /api/session/1 recuperi correttamente una sessione & V \\
    \end{tabularx}
\end{table}
\begin{table}[H]
   \centering
   \begin{tabularx}{\textwidth}{>{\hsize=0.2\hsize}>{\centering\arraybackslash}X|X|>{\hsize=0.1\hsize}>{\centering\arraybackslash}X}
        TU-35 & Verifica che l'endpoint /api/conversation crei correttamente una nuova conversazione & V \\
        \hline
        TU-36 & Verifica che l'endpoint: /api/conversation recuperi correttamente tutte le conversazioni associate a una sessione & V \\
        \hline
        TU-37 & Verifica che l'endpoint: /api/conversation/1 recuperi correttamente una conversazione tramite il suo ID & V \\
        \hline
        TU-38 & Verifica che l'endpoint: \textbf{DELETE} /api/conversation/1 elimini correttamente una conversazione & V \\
        \hline
        TU-39 & Verifica che l'endpoint: \textbf{POST} /api/message aggiunga correttamente un nuovo messaggio a una conversazione & V \\
        \hline
        TU-40 & Verifica che l'endpoint: \textbf{GET} /api/message recuperi correttamente tutti i messaggi associati a una conversazione & V \\
        \hline
        TU-41 & Verifica che la vengano create correttamente le directory necessarie & V \\
        \hline
        TU-42 & Verifica che un PDF venga scaricato correttamente e salvato & V \\
        \hline
        TU-43 & Verifica che il download fallisca correttamente in caso di errore HTTP & V \\
        \hline
        TU-44 & Verifica che un PDF venga scaricato e convertito correttamente in un file di testo & V \\
        \hline
        TU-45 & Verifica che un file di testo venga suddiviso in chunk e che vengano generati i relativi vettori di embedding & V \\
        \hline
        TU-46 & Verifica che la connessione al database venga stabilita correttamente & V \\
        \hline
        TU-47 & Verifica che la funzione calcoli correttamente la dimensione del vettore di embedding & V \\
        \hline
        TU-48 & Verifica che vengano create correttamente le tabelle e i trigger nel database & V \\
        \hline
        TU-49 & Verifica cheVerifica che i chunk vengano inseriti correttamente nel database & V \\
        \hline
        TU-50 & Verifica che i prodotti vengano elaborati e inseriti correttamente nel database & V \\
        \hline
        TU-51 & Verifica che i documenti vengano inseriti correttamente nel database & V \\
        \hline
        TU-52 & Verifica che i prodotti vengano letti da un file JSON e scritti correttamente nel database & V \\
        \hline
        TU-53 & Verifica che la funzione restituisca un vettore di embedding valido per un input corretto & V \\
        \hline
        TU-54 & Verifica che la funzione gestisca correttamente un input vuoto restituendo un vettore vuoto & V \\
        \hline
        TU-55 & Verifica che la funzione gestisca correttamente un input molto grande restituendo un vettore di embedding valido & V \\
        \hline
        TU-56 & Verifica che le tabelle vengano create correttamente nel database & V \\
        \hline
        TU-57 & Verifica che venga gestito correttamente un errore durante la creazione delle tabelle & V \\
        \end{tabularx}
    \end{table}
\begin{table}[H]
   \centering
   \begin{tabularx}{\textwidth}{>{\hsize=0.2\hsize}>{\centering\arraybackslash}X|X|>{\hsize=0.1\hsize}>{\centering\arraybackslash}X}
        TU-58 & Verifica che i dati vengano importati correttamente nel database & V \\
        \hline
        TU-59 & Verifica che venga gestito correttamente un errore durante l'importazione dei dati & V \\
        \hline
        TU-60 & Verifica che i link vengano estratti correttamente dai prodotti e raggruppati per ID & V \\
        \hline
        TU-61 & Verifica che i prodotti vengano elaborati correttamente e che vengano generati i vettori di embedding & V \\
        \hline
        TU-62 & Verifica che venga fatta una richiesta di creare una sessione & V \\ 
        \hline
        TU-63 & Verifica che venga fatta una richiesta di aggiornamento della sessione & V \\
        \hline
        TU-64 & Verifica che venga fatta una richiesta per creare una conversazione & V \\
        \hline
        TU-65 & Verifica che venga fatta una richiesta per recuperare le conversazioni  & V \\
        \hline
        TU-66 & Verifica che venga fatta una richiesta per eliminare una conversazione secondo l'id  & V \\
        \hline
        TU-67 & Verifica che venga fatta una richiesta per inviare un messaggio & V \\
        \hline
        TU-68 & Verifica che venga fatta una richiesta per recuperare i messaggi  & V \\
        \hline
        TU-69 & Verifica che venga fatta una richiesta per inviare un feedback & V \\
        \hline
        TU-70 & Verifica che venga fatta una richiesta per recuperare il commento relativo al feedback & V \\
        \hline
        TU-71 & Verifica che venga fatta una richiesta per verificare che la password sia corretta & V \\
        \hline
        TU-72 & Verifica che venga fatta una richiesta per recuperare le statistiche degli admin & V \\
        \hline
        TU-73 & Verifica che la funzione askQuestion() funzioni correttamente & V \\
        \hline
        TU-74 & Verifica che la funzione readSession() funzioni correttamente & V \\
        \hline
        TU-75 & Verifica che venga creata un istanza di ChatService senza errori & V \\
        \hline
        TU-76 & Verifica che la sessione possa essere aggiornata correttamente & V \\
        \hline
        TU-77 & Verifica che le conversazioni possano essere correttamente recuperate& V \\
        \hline
        TU-78 & Verifica che non venga creata una conversazione senza la creazione di una sessione & V \\
        \hline
        TU-79 & Verifica che venga creata correttamente una conversazione & V \\
        \hline
        TU-90 & Verifica che non venga eliminata una conversazione attiva & V \\
        \hline
        TU-91 & Verifica che non venga eliminata alcuna conversazione senza la presenza di una sessione& V \\
        \hline
        TU-92 & Verifica che venga correttamente inviato un feedback & V \\
        \hline
        TU-93 & Verifica che non venga inviato un feedback senza la presenza di una sessione& V \\
        \hline
        TU-94 & Verifica del raggiungimento del limite di conversazioni& V \\
        \hline
        TU-95 & Verifica che venga inviato un messaggio & V \\
        \hline
        TU-96 & Verifica che venga creata correttamente un'istanza della componente Sidebar & V \\
        \hline
        \end{tabularx}
    \end{table}
\begin{table}[H]
   \centering
   \begin{tabularx}{\textwidth}{>{\hsize=0.2\hsize}>{\centering\arraybackslash}X|X|>{\hsize=0.1\hsize}>{\centering\arraybackslash}X}        
        TU-97 & Verifica che l'implementazione di ngOnInit() istanzi attributi e membri privati del componente Sidebar& V \\
        \hline
        TU-98 & Verifica che la funzione hasReachedLimit() segua il comportamento aspettato & V \\
        \hline
        TU-99 & Verifica che venga creata una conversazione nella Sidebar& V \\
        \hline
        TU-100 & Verifica che se l'app è in attesa di una risposta, non venga creata una nuova conversazione & V \\
        \hline
        TU-101 & Verifica che al raggiungimento del limite di conversazioni venga visualizzato un avviso interno al browser che notifica il problema all'utente & V \\
        \hline
        TU-102 & Verifica che l'app non si trovi nello stato in cui non sta aspettando una risposta dal LLM & V \\
        \hline
        TU-103 & Verifica che l'app si trovi nello stato in cui non sta aspettando una risposta dal LLM & V \\
        \hline
        TU-104 & Verifica che si possa selezionare una conversazione & V \\
        \hline
        TU-105 & Verifica che se l'app sta aspettando una risposta, non venga selezionata alcuna conversazione oltre a quella attiva c& V \\
        \hline
        TU-106 & Verifica che possa essere possibile eliminare una conversazione & V \\
        \hline
        TU-107 & Verifica che selezionando una conversazione essa diventi effettivamente quella attiva& V \\
        \hline
        TU-108 & Verifica che venga creata una istanza della componente Chatbox & V \\
        \hline
        TU-109 & Verifica che il metodo ngOnInit() inserisca dei messaggi fittizzi nella variabile messages & V \\
        \hline
        TU-110 & Verifica che il metodo ngOnInit() inserisca un vettore di messaggi vuoto nella variabile messages & V \\
        \hline
        TU-111 & Verifica che il metodo ngAfterViewChecked() assuma il comportamento atteso & V \\
        \hline
        TU-112 & Verifica che venga correttamente inviato un messaggio & V \\
        \hline
        TU-113 & Verifica che un messaggio vuoto non venga inviato & V \\ 
        \hline
        TU-114 & Verifica che un messaggio che sorpassa il limite di caratteri non venga inviato & V \\ 
        \hline
        TU-115 & Verifica che se l'app è in attesa di risposta, non venga inviato un messaggio & V \\
        \hline        
        TU-116 & Verifica che venga inviato un feedback positivo & V \\ 
        \hline
        TU-117 & Verifica che venga inviato un feedback negativo & V \\ 
        \hline
        TU-118 & Verifica che venga effettivamente inviato un messaggio di feedback & V \\ 
        \hline
        TU-119 & Verifica che non venga inviato un feedback senza che la variabile feedbackMessageId abbia un valore & V \\ 
        \hline
        TU-120 & Verifica che venga eseguito il metodo closeFeedbackPopup() & V \\ 
        \hline
        TU-121 & Verifica il numero di caratteri rimanenti per il messaggio di feedback & V \\ 
        \hline
        TU-122 & Verifica che venga creata correttame un'istanza della componente AdminDashboard & V \\ 
        \hline
        TU-123 & Verifica che l'implementazione del metodo ngOnInit() inizializzi statistiche e commenti di feedback & V \\ 
        \hline
        \end{tabularx}
    \end{table}
\begin{table}[H]
   \centering
   \begin{tabularx}{\textwidth}{>{\hsize=0.2\hsize}>{\centering\arraybackslash}X|X|>{\hsize=0.1\hsize}>{\centering\arraybackslash}X}             
        
        TU-124 & Verifica che avvenga il calcolo del rapporto di soddisfazione & V \\ 
        \hline
        TU-125 & Verifica che il calcolo del rapporto di soddisfazione sia uguale a 0 & V \\ 
        \hline
        TU-126 & Verifica che senza alcuna statistica caricata, il rapporto di soddisfazione sia uguale a 0 & V \\ 
        \hline
        TU-127 & Verifica che il metodo formatDate() funzioni correttamente & V \\ 
        \hline
        TU-128 & Verifica che venga creata una istanza della componente AdminLogin & V \\ 
        \hline
        TU-129 & Verifica che la password fornita venga sottoposta ad un processo di crittografia secondo hash & V \\ 
        \hline
        TU-130 & Verifica che il login avvenga correttamente & V \\ 
        \hline
        TU-131 & Verifica che il login non avvenga se non è presente la password & V \\ 
        \hline
        TU-132 & Verifica che con una risposta negativa da parte della relatica API, non venga effettuato il login & V \\ 
        \hline
        TU-133 & Verifica che il token dell'admin sia autenticato & V \\ 
        \hline
        TU-134 & Verifica che la guardia abbia un token e che esso sia autenticato & V \\ 
        \hline
        TU-135 & Verifica che la guardia ritorni false se non vi è alcun token& V \\ 
        \hline
        TU-136 & Verifica del corretto funzionamento di AuthInterceptor  & V \\ 
        \hline
        TU-137 & Verifica del corretto funzionamento di AuthInterceptor senza la presenza di un token & V \\
        \hline
        TU-138 & Verifica che venga creata correttamente un'istanza di AppComponent & V \\ 
        \hline
        TU-139 & Verifica che venga correttamente attivata la componente Sidebar & V \\ 
        \hline
        TU-140 & Verifica che venga correttamente chiusa la Sidebar entro una dimensione della finestra su dispositivi mobili & V \\ 
        \hline
        
        
        
   \end{tabularx}
   \caption{Stato dei \textit{test\textsubscript{G}} di unità}
\end{table}

\subsection{\textit{Test\textsubscript{G}} di integrazione}
I \textit{test\textsubscript{G}} di integrazione sono una tipologia di \textit{test\textsubscript{G}} progettata per verificare la capacità di diversi componenti o moduli di un sistema di funzionare insieme. I \textit{test\textsubscript{G}} di integrazione non mirano a testare singoli moduli in modo indipendente, come fa il \textit{test\textsubscript{G}} di unità, che si concentra su unità di codice isolate. Le caratteristiche principali dei \textit{test\textsubscript{G}} di integrazione sono:
\begin{itemize}
    \item Monitorare i problemi di comunicazione tra moduli;
    \item Garantire la corretta configurazione e gestione delle dipendenze tra moduli;
    Testare il sistema in condizioni più vicine a quelle reali rispetto a quanto avviene con i \textit{test\textsubscript{G}} di unità.
\end{itemize}
I \textit{test\textsubscript{G}} di integrazione, insieme ai \textit{test\textsubscript{G}} di unità, come richiesto nel capitolato, devono avere un \textit{coverage\textsubscript{G}} minimo pari al 75\% (opzionalmente un \textit{coverage\textsubscript{G}} minimo pari al 90\%).
%TODO: Inserire i test
\begin{table}[H]
   \centering
   \begin{tabularx}{\textwidth}{>{\hsize=0.2\hsize}>{\centering\arraybackslash}X|X|>{\hsize=0.1\hsize}>{\centering\arraybackslash}X}
       \textbf{Codice} & \textbf{Descrizione} & \textbf{Stato} \\
       \hline
       TI-1 & Verifica che l'endpoint restituisca correttamente il messaggio di successo & V \\
       \hline
       TI-2 & Verifica che l'endpoint gestisca correttamente una richiesta di domanda & V \\
       \hline
       TI-3 & Verifica che l'endpoint crei correttamente una nuova conversazione & V \\
       \hline
       TI-4 & Verifica che l'endpoint recuperi correttamente tutte le conversazioni associate a una sessione & V \\
       \hline
       TI-5 & Verifica che l'endpoint recuperi correttamente una conversazione tramite il suo ID & V \\
       \hline
       TI-6 & Verifica che l'endpoint elimini correttamente una conversazione & V \\
       \hline
       TI-7 & Verifica che l'endpoint aggiunga correttamente un nuovo messaggio a una conversazione & V \\
       \hline
       TI-8 & Verifica che l'endpoint recuperi correttamente tutti i messaggi associati a una conversazione & V \\
       \hline
       TI-9 & Verifica che l'endpoint restituisca correttamente il messaggio di successo & V \\
       \hline
       TI-10 & Verifica che l'endpoint gestisca correttamente una richiesta di domanda & V \\
       \hline
       TI-11 & Verifica che l'endpoint crei correttamente una nuova conversazione & V \\
       \hline
       TI-12 & Verifica che l'endpoint recuperi correttamente tutte le conversazioni associate a una sessione & V \\
       \hline
       TI-13 & Verifica che l'endpoint recuperi correttamente una conversazione tramite il suo ID & V \\
       \hline
       TI-14 & Verifica che l'endpoint elimini correttamente una conversazione & V \\
       \hline
       TI-15 & Verifica che l'endpoint aggiunga correttamente un nuovo messaggio a una conversazione & V \\
       \hline
       TI-16 & Verifica che l'endpoint recuperi correttamente tutti i messaggi associati a una conversazione & V \\
       \hline
       TI-17 & Verifica che una query venga eseguita correttamente & V \\
       \hline
       TI-18 & Verifica che una query venga eseguita correttamente e restituisca un solo risultato & V \\
       \hline
       TI-19 & Verifica che una query venga eseguita correttamente e restituisca tutti i risultati & V \\
       \hline
       TI-20 & Verifica che la funzione restituisca un vettore di embedding valido per un input corretto & V \\
       \hline
       TI-21 & Verifica che la funzione gestisca correttamente un input vuoto restituendo un vettore vuoto & V \\
       \hline
       TI-22 & Verifica che la funzione gestisca correttamente un input molto grande restituendo un vettore di embedding valido & V \\
   \end{tabularx}
   \caption{Stato dei \textit{test\textsubscript{G}} di integrazione}
\end{table}



\subsection{\textit{Test\textsubscript{G}} End-to-End}
I \textit{test\textsubscript{G}} End-to-End sono una tipologia di \textit{test\textsubscript{G}} che valida il funzionamento complessivo di un sistema, testando l'intero flusso dal punto A al punto B. Questo tipo di \textit{test\textsubscript{G}} riguarda l'interazione tra le parti del sistema stesso e con sistemi esterni, per assicurarsi che tutte le parti funzionino correttamente insieme, emulando una situazione online da testare tramite l'utente finale. In questo modo, si garantisce che il sistema sia privo di errori, soddisfi i \textit{requisiti\textsubscript{G}} funzionali e contribuisca a una buona esperienza utente.
I \textit{test\textsubscript{G}} End-to-End, come richiesto nel capitolato, devono avere un \textit{coverage\textsubscript{G}} minimo pari all'80\%.
%TODO: Inserire i test
\begin{table}[H]
    \centering
    \begin{tabularx}{\textwidth}{>{\hsize=0.4\hsize}>{\centering\arraybackslash}X|X|>{\centering\arraybackslash}X|>{\hsize=0.3\hsize}>{\centering\arraybackslash}X}
        \textbf{Codice} & \textbf{Descrizione} & \textbf{Casi d'uso} & \textbf{Stato} \\
        \hline
       TE-1 & Verifica della creazione di una nuova conversazione e dell'invio di una domanda con successiva visualizzazione della risposta & UC1, UC5 & V \\
        \hline
        TE-2 & Verifica della visualizzazione dello storico dei messaggi all'interno di una conversazione & UC3.4 & V \\
        \hline
        TE-3 & Verifica della possibilità di fare richieste in italiano e di ottenere risposte pertinenti & UC5, UC8  & V \\
        \hline
        TE-4 & Verifica della possibilità di fornire un feedback sulla risposta fornita dal sistema & UC10 & V \\
        \hline
        TE-5 & Verifica della funzionalità di eliminazione di una conversazione &  UC11 & V \\
        \hline
        TE-6 & Verifica dell'accesso al cruscotto informativo per amministratori e della visualizzazione delle informazioni nel cruscotto informativo& UC14, UC15, UC16 & V \\
    \end{tabularx}
    \caption{Stato dei \textit{test\textsubscript{G}} End-to-End}
\end{table}
