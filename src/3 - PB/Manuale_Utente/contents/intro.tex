\section{Introduzione}
\subsection{Scopo del documento}
Il Manuale Utente ha lo scopo di fornire agli utenti finali le informazioni necessarie per utilizzare il \textit{sistema}\textsubscript{G} in modo efficace. Il documento descrive le funzionalità principali, le modalità di interazione con l’interfaccia, le procedure operative e le eventuali configurazioni richieste. Inoltre, include istruzioni per la risoluzione di problemi comuni e best practice per un utilizzo ottimale del \textit{software}\textsubscript{G}. L'obiettivo è rendere l’esperienza dell’utente il più intuitiva possibile, riducendo la necessità di supporto tecnico.

\subsection{Scopo del prodotto}
Il progetto ``Vimar GENIALE" mira a sviluppare un'applicazione intelligente che supporti installatori elettrici nell'uso di dispositivi \textit{Vimar}\textsubscript{G}, facilitando l'accesso alle informazioni tecniche sui prodotti, rispondendo a domande poste in linguaggio naturale.
La tecnologia alla base prevede l'uso di modelli di \textit{LLM}\textsubscript{G} e di tecniche \textit{RAG}\textsubscript{G}, con una struttura di gestione basata su \textit{container}\textsubscript{G} e integrata in un ambiente \textit{cloud}\textsubscript{G}.
Il \textit{sistema}\textsubscript{G} include tre componenti principali: una \textit{applicativo web responsive}\textsubscript{G}, un \textit{applicativo server}\textsubscript{G} e un'\textit{infrastruttura cloud-ready}\textsubscript{G}. 

\subsection{Glossario}
Per evitare ambiguità relative al linguaggio utilizzato nei documenti, viene fornito il Glossario V1.0.0, nel quale si possono trovare tutte le definizioni di termini che hanno un significato specifico che vuole essere disambiguato. Tali termini sono marcati con una G a pedice. 

\subsection{Riferimenti}
\subsubsection{Riferimenti normativi}
\begin{itemize}
    \item \textbf{Norme di Progetto v2.0.0}\\
    (Ultimo accesso 2025-04-04)
    \item \textbf{PD1 - Regolamento del progetto didattico} \\
    \url{https://www.math.unipd.it/~tullio/IS-1/2024/Dispense/PD1.pdf} \\
    (Ultimo accesso 2024-11-20)
    \item \textbf{Capitolato d'Appalto C2}: Vimar GENIALE \\
    \url{https://www.math.unipd.it/~tullio/IS-1/2024/Progetto/C2.pdf}\\
    (Ultimo accesso 2024-03-26)
    \end{itemize}
\subsubsection{Riferimenti informativi}
\begin{itemize}
    \item \textbf{Glossario v1.0.0} \\
    (Ultimo acecsso 2025-04-01)
\end{itemize}