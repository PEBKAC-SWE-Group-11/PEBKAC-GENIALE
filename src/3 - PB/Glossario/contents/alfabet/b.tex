\section*{B}  
\addcontentsline{toc}{section}{B}
\begin{itemize}
    \item \textbf{Backend}: La parte amministrativa di un sito o web app, accessibile solo dagli amministratori, che gestisce le funzionalità server-side.
    \item \textbf{Backlog}: Artefatto di Scrum, consiste in un elenco ordinato di attività o funzionalità da sviluppare, mantenuto e aggiornato in base alle priorità del progetto.
    \item \textbf{Backup}: Copia di sicurezza di dati o informazioni cruciali, creata per garantire la loro integrità e disponibilità in caso di guasto, perdita o danneggiamento dei dati originali, spesso archiviata in un luogo separato o su supporti di memorizzazione esterni.
    \item \textbf{Baseline}: Punto di riferimento fisso utilizzato per confrontare le prestazioni del progetto nel tempo, riguardando ambito, tempi e costi.
    \item \textbf{Benchmarking}: Metodo per valutare le prestazioni di un sistema, come un modello \textit{LLM}\textsubscript{G}, confrontandolo con standard o metriche predefinite.
    \item \textbf{Bert}: Modello di linguaggio bidirezionale utilizzato per compiti di elaborazione del linguaggio naturale come analisi del sentimento o classificazione testuale.
    \item \textbf{Best practices}: Linee guida che rappresentano metodi o processi ottimali per raggiungere determinati obiettivi, generalmente adottati in ambito professionale o tecnico.
    \item \textbf{Branch}: Linea indipendente di sviluppo in un sistema di versionamento come \textit{Git}\textsubscript{G}, che consente modifiche parallele al codice sorgente.
    \item \textbf{Budget}: Pianificazione finanziaria che definisce le risorse economiche allocate per un determinato progetto, attività o periodo, utilizzata per monitorare e controllare le spese in relazione agli obiettivi e alle previsioni.
    \item \textbf{Budget At Completion (BAC)}: Budget totale pianificato per il completamento di un progetto.
    \item \textbf{Bug}: Errore o malfunzionamento nel software, spesso dovuto a problemi nel codice sorgente.
    %\item \textbf{Bug reporting}: Processo di registrazione e monitoraggio dei \textit{bug}\textsubscript{G} o errori rilevati durante i test software, con lo scopo di risolverli.
\end{itemize}
