\section*{D}  
\addcontentsline{toc}{section}{D}
\begin{itemize}
    \item \textbf{Dashboard}: Interfaccia grafica che fornisce una panoramica dei dati principali, spesso sotto forma di grafici e tabelle.
    \item \textbf{Database}: Sistema organizzato per la raccolta, la gestione e l'archiviazione di dati, che permette di memorizzare, recuperare e manipolare informazioni in modo strutturato, spesso utilizzando un sistema di gestione database (DBMS).
    \item \textbf{Database vettoriale}: Tipo di database progettato per archiviare e gestire dati geospaziali o geometrie, come punti, linee e poligoni.
    \item \textbf{Diagramma dei casi d’uso}: Rappresentazione grafica degli \textit{use case}\textsubscript{G}, che evidenzia gli \textit{attori}\textsubscript{G} e le interazioni con il sistema.
    \item \textbf{Diagramma di Gantt}: Strumento per rappresentare graficamente le attività di un progetto, indicando durata, sequenze e sovrapposizioni.
    \item \textbf{Diario di bordo}: Documento che monitora i progressi del progetto, evidenziando il rapporto tra costi sostenuti e risultati ottenuti.
    %\item \textbf{Discord}: Piattaforma di comunicazione che supporta messaggi, chiamate vocali e video, utilizzata per collaborazioni di team.
    \item \textbf{Docker}: Piattaforma \textit{open source}\textsubscript{G} per la creazione e gestione di container, garantendo la coerenza tra ambienti di sviluppo e produzione.
    \item \textbf{Docker Compose}: Strumento utilizzato per implementare il principio di Infrastructure as Code, consentendo la creazione e gestione di ambienti applicativi replicabili con un solo comando.
    \item \textbf{Documentazione}: Definisce le attività per la registrazione delle informazioni prodotte da un processo del ciclo di vita.
    \item \textbf{Drag-and-drop}: Interazione utente che consente di selezionare un oggetto (come un file o una finestra) e trascinarlo con il mouse o il touchpad in una nuova posizione, senza la necessità di utilizzare comandi da tastiera.
\end{itemize}

