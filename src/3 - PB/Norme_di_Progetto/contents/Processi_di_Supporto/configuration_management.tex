\subsection{Configuration Management}
\subsubsection{Scopo}
In questa sezione vengono presentate le attività svolte da PEBKAC per il processo di \textit{Configuration Management}\textsubscript{G}. Il processo in questione consiste nell'applicazione di procedure amministrative e tecniche per l'intero ciclo di vita del \textit{software}\textsubscript{G}, al fine di:
\begin{itemize}
    \item Identificare, definire e stabilire una base per gli elementi \textit{software}\textsubscript{G} di un \textit{sistema}\textsubscript{G};
    \item Controllare le modifiche e le release degli elementi;
    \item Registrare lo stato degli elementi e delle richieste di modifica;
    \item Garantire la completezza, la coerenza e la correttezza degli elementi.
\end{itemize}
\subsubsection{Configuration control}
\subsubsubsection{Descrizione}
Il processo di configuration control è finalizzato a garantire il controllo e la coerenza delle configurazioni del \textit{sistema}\textsubscript{G}, assicurando che tutte le modifiche apportate a \textit{software}\textsubscript{G}, artefatti e documenti siano tracciate, gestite e allineate agli obiettivi e ai \textit{requisito}\textsubscript{G} del progetto.
\subsubsubsection{Scopo}
Il configuration control mira al raggiungimento dei seguenti punti:
\begin{itemize}
    \item \textbf{Gestire le modifiche}: assicurare un controllo e una gestione corretti e sistematici per qualsiasi modifica nel progetto o nel \textit{sistema}\textsubscript{G};
    \item \textbf{Documentare le richieste}: registrare tutte le richieste di modifica per mantenere una cronologia accurata e completa;
    \item \textbf{Valutare l'impatto}: analizzare tutte le conseguenze tecniche, economiche e operative di ogni modifica proposta;
    \item \textbf{Decidere sull'approvazione}: stabilire criteri chiari e inequivocabili per l'approvazione o il rifiuto delle modifiche con gli \textit{stakeholder}\textsubscript{G} rilevanti;
    \item \textbf{Assicurare la tracciabilità}: creare \textit{audit trail}\textsubscript{G} dettagliati per tracciare le modifiche e garantire la conformità alle politiche del progetto;
    \item \textbf{Evitare conflitti}: prevenire modifiche non autorizzate o che possano entrare in conflitto con il \textit{sistema}\textsubscript{G};
    \item \textbf{Mantenere la qualità}: garantire che le modifiche non compromettano l'integrità, la funzionalità o gli obiettivi generali del progetto.
\end{itemize}
\subsubsubsection{ITS\textsubscript{G}}
Per conseguire l'obiettivo di assicurare la tracciabilità delle modifiche è necessario creare degli \textit{audit trail}\textsubscript{G} dettagliati, ovvero dei registri che tracciano tutte le attività e le modifiche all'interno di un \textit{sistema}\textsubscript{G}. Per la creazione, la gestione ed il tracciamento di questi \textit{audit trail}\textsubscript{G}, PEBKAC utilizza l'\textit{Issue Tracking System}\textsubscript{G} \textit{Jira}\textsubscript{G}, sviluppato da \textit{Atlassian}\textsubscript{G}.
\subsubsubsubsection{Ticket}
Un \textit{ticket}\textsubscript{G} è una voce che rappresenta una singola attività, problema, richiesta o \textit{task}\textsubscript{G} all'interno di un progetto. \\
Esistono varie tipologie di \textit{ticket}\textsubscript{G}:
\begin{itemize}
    \item \textbf{Task\textsubscript{G}}: questa tipologia di \textit{ticket}\textsubscript{G} rappresenta una comune attività che deve essere completata all'interno del progetto;
    \item \textbf{Sub-task}: questa tipologia di \textit{ticket}\textsubscript{G} rappresenta una parte di un \textit{ticket}\textsubscript{G} più grande (come un \textit{task}\textsubscript{G}) che viene suddivisa in azioni più piccole e gestibili;
    \item \textbf{Story}: questa tipologia di \textit{ticket}\textsubscript{G} rappresenta un \textit{requisito}\textsubscript{G} ed è generalmente scritto in un formato che descrive il risultato atteso dal punto di vista dell'utente;
    \item \textbf{\textit{bug}\textsubscript{G}}: questa tipologia di \textit{ticket}\textsubscript{G} rappresenta un errore o difetto nel \textit{sistema}\textsubscript{G} che necessita di correzione;
\end{itemize}
Ogni \textit{ticket}\textsubscript{G} è dotato di campi per riportare i dettagli relativi all'attività, al problema, alla richiesta o alla \textit{task}\textsubscript{G} che rappresenta:
\begin{itemize}
    \item \textbf{Summary}: un riassunto breve in una sola riga del \textit{ticket}\textsubscript{G};
    \item \textbf{Key}: un identificatore unico per ogni \textit{ticket}\textsubscript{G}, nella forma si SW-Key;
    \item \textbf{Epic}: \textit{epic}\textsubscript{G} a cui il \textit{ticket}\textsubscript{G} è associato;
    \item \textbf{Links}: un elenco di link a \textit{ticket}\textsubscript{G} correlati;
    \item \textbf{Assignee}: la persona o le persone a cui il \textit{ticket}\textsubscript{G} è attualmente assegnato;
    \item \textbf{Description}: una descrizione dettagliata del \textit{ticket}\textsubscript{G};
    \item \textbf{Due}: la data entro cui questo \textit{ticket}\textsubscript{G} è programmato per essere completato;
    \item \textbf{Reporter}: la persona che ha inserito il \textit{ticket}\textsubscript{G} nel \textit{sistema}\textsubscript{G};
    \item \textbf{Links}: un elenco di link alle \textit{commit}\textsubscript{G} e alle \textit{pull request}\textsubscript{G} effettuati nella \textit{repository}\textsubscript{G} di \textit{GitHub}\textsubscript{G} correlate al ticket;
    \item \textbf{Status}: la fase in cui si trova attualmente il \textit{ticket}\textsubscript{G} nel suo ciclo di vita, che può essere "To Do", "In Process", "Verify" ed infine "Approve \& Release";
    \item \textbf{Sprint}: \textit{sprint}\textsubscript{G} a cui il \textit{ticket}\textsubscript{G} è associato;
    \item \textbf{Fix Version}: la versione del progetto in cui il \textit{ticket}\textsubscript{G} è stato (o sarà) risolto;
    \item \textbf{Priority}: l'importanza del \textit{ticket}\textsubscript{G} rispetto ad altri ticket.
\end{itemize}
\subsubsubsubsection{Epic}
Un'\textit{epic}\textsubscript{G} è una raccolta di \textit{ticket}\textsubscript{G} che rappresenta uno degli obiettivi più ampi e significativi verso cui è diretto l'intero progetto. Si tratta di un concetto che aiuta a gestire e strutturare il lavoro più complesso, suddividendolo in parti più piccole e gestibili. Le \textit{epic}\textsubscript{G} sono utili per monitorare i progressi rispetto a funzionalità che richiedono tempo o che coinvolgono diverse aree del progetto. Le \textit{epic}\textsubscript{G} sono particolarmente utili nei processi \textit{Agile}\textsubscript{G}, poiché offrono una visione a lungo termine del progetto, anche mentre si adatta e si pianifica in modo incrementale, fornendo strumenti di tracciamento, come una scorebord che presenta le percentuali di \textit{ticket}\textsubscript{G} presenti in ogni stato, che monitorano lo stato generale di ogni \textit{epic}\textsubscript{G}, misurano i progressi e identificano eventuali ritardi. Inoltre, un'\textit{epic}\textsubscript{G}, oltre ai \textit{ticket}\textsubscript{G} che comprende, possiede tutti i campi precedentemente elecati per i ticket.
\subsubsubsubsection{Versioni}
Le versioni sono la modalità di organizzazione, pianificazione e monitoraggio del lavoro in base alle specifiche \textit{milestone}\textsubscript{G} di un progetto. Ogni versione ha a che fare con le funzionalità, rappresentate da \textit{epic}\textsubscript{G} e relativi \textit{ticket}\textsubscript{G} ad essa associati, da realizzare entro una scadenza. In pratica, permette di sapere chiaramente quali \textit{requisito}\textsubscript{G} devono essere soddisfatti per la specifica \textit{milestone}\textsubscript{G} che rappresenta. Ciò rende semplice la tracciabilità dei progressi di ciascuna versione, oltre a ritardi o modifiche, usando una \textit{scoreboard}\textsubscript{G} che adotta la stessa logica di quella utilizzata dalle \textit{epic}\textsubscript{G}. In seguito al completamento di tutti i \textit{ticket}\textsubscript{G} associati ad una determina versione, questa può essere rilasciata. Inoltre, una versione, oltre ai \textit{ticket}\textsubscript{G} che comprende, possiede dei campi che specificano la data di inizio, la data di fine ed una breve descrizione.
\subsubsubsubsection{Backlog e Sprint}
In \textit{Jira}\textsubscript{G} sono integrati diversi strumenti per lo sviluppo secondo il metodo \textit{Agile}\textsubscript{G}, tra i quali è importante evidenziare \textit{backlog}\textsubscript{G} e \textit{sprint}\textsubscript{G}.\\
Il \textit{backlog}\textsubscript{G} contiene una lista di \textit{ticket}\textsubscript{G} da completare dal team, ed è ordinata in base alla priorità: i più importanti sono posti in cima, mentre i meno importanti sono disposti verso il fondo. La lista non è statica, ma è uno spazio dinamico all'interno del quale il team può aggiungere, eliminare, aggiornare e riorganizzare le priorità dei \textit{ticket}\textsubscript{G} al suo interno. Il \textit{backlog}\textsubscript{G} è inteso come punto di partenza per pianificare il lavoro: prima di ogni \textit{sprint}\textsubscript{G}, il team esamina il \textit{backlog}\textsubscript{G} per selezionare il lavoro da svolgere durante quello \textit{sprint}\textsubscript{G}.\\
Uno \textit{sprint}\textsubscript{G} è un periodo di tempo predefinito in cui vengono completati i \textit{ticket}\textsubscript{G} selezionati dal \textit{backlog}\textsubscript{G} prima del suo inizio. Ogni \textit{sprint}\textsubscript{G} è dotato di una data di inizio, una data di fine e di uno stato, che può essere "In Corso" o "Terminato".
\subsubsubsubsection{Timeline}
La \textit{timeline}\textsubscript{G} messa a diposizione in \textit{Jira}\textsubscript{G} è uno strumento realizzato tramite un \textit{diagramma di Gantt}\textsubscript{G} che aiuta a gestire le scadenze, le dipendenze e l'andamento del progetto, fornendo una panoramica d'insieme dello stato di avanzamento.\\
Essa mostra tutti i \textit{ticket}\textsubscript{G} associati ad una \textit{epic}\textsubscript{G} che sono stati inseriti al suo interno, evidenziandone le date di inizio e fine e le dipendenze con altri ticket. Ogni \textit{ticket}\textsubscript{G} viene visualizzato come un blocco che si estende lungo la \textit{timeline}\textsubscript{G} in base alla durata prevista.\\
Le dipendenze tra i vari  possono essere visualizzate tramite linee di collegamento, mostrando come il completamento di un'attività dipenda da un’altra. Inoltre, la \textit{timeline}\textsubscript{G} mostra chiamente anche lo stato delle attività, ovvero quali attività sono in corso, quali attività sono state completate e quali attività sono in ritardo.\\
Un'altra informazione mostrata nella \textit{timeline}\textsubscript{G} sono le versioni e, in particolare, quando sono state fissate le loro date di scadenza. Le versioni sono rappresentate graficamente come delle linee verticali posizionate proprio sulla data di scadenza corrispondente.\\
\'E inoltre possibile visualizzare gli \textit{sprint}\textsubscript{G} definiti all'interno di \textit{Jira}\textsubscript{G} nell'area superiore della \textit{timeline}\textsubscript{G}, permettendo così di determinare quali attività sono state svolte durante ciascuno \textit{sprint}\textsubscript{G}.\\
Infine, è utile notare che nella \textit{timeline}\textsubscript{G} e possibile effettuare delle operazioni di filtraggio dei \textit{ticket}\textsubscript{G} visualizzati, permettendo così di visualizzare anche l'organizzazione di specifici gruppi di attività.
\subsubsubsection{Pull Request}
Le \textit{pull request}\textsubscript{G} sono un meccanismo per la gestione controllata delle modifiche nei rilasci del prodotto. Quando un membro del team propone modifiche al \textit{repository}\textsubscript{G}, esse vengono raggruppate in una \textit{pull request}\textsubscript{G}, che consente ai verificatori di analizzarle prima di integrarle nel \textit{repository}\textsubscript{G}.\\
Il processo può essere riassunto nei seguenti passaggi:
\begin{itemize}
    \item \textbf{Creazione della \textit{pull request}\textsubscript{G}}: Non appena le modifiche vengono apportate in un \textit{branch}\textsubscript{G} dedicato, il contributore apre una \textit{pull request}\textsubscript{G} per proporre la loro integrazione nel \textit{branch}\textsubscript{G} develop o in un altro \textit{branch}\textsubscript{G} di destinazione;
    \item \textbf{Revisione}: I verificatori esaminano le modifiche proposte direttamente all'interno della \textit{pull request}\textsubscript{G}, visualizzando i dettagli di ciò che è stato modificato e avendo la possibilità di commentare su una singola riga o su più righe/sezioni. Se le modifiche richieste sono minime, il \textit{Verificatore}\textsubscript{G} può correggerle direttamente eseguendo una \textit{commit}\textsubscript{G} sul \textit{branch}\textsubscript{G} in cui sono state suggerite le modifiche;
    \item \textbf{Feedback\textsubscript{G} e suggerimenti}: Se le modifiche sono significative, i verificatori solitamente forniscono suggerimenti per miglioramenti, evidenziano possibili errori ed elencano tutte le modifiche necessarie per allineare il lavoro agli standard e alle linee guida del progetto;
    \item \textbf{Applicazione delle correzioni}: Il contributore che ha creato la \textit{pull request}\textsubscript{G} risponde al \textit{feedback}\textsubscript{G} ricevuto ed apporta le modifiche necessarie. Successivamente, la \textit{pull request}\textsubscript{G} viene aggiornata con le modifiche revisionate;
    \item \textbf{Decisione finale}: Quando i verificatori approvano la proposta, o quando tutte le questioni sollevate da questi ultimi sono state risolte, la \textit{pull request}\textsubscript{G} può essere considerata accettata e viene "unita" al \textit{branch}\textsubscript{G} di destinazione, garantendo che solo contributi di alta qualità facciano parte del progetto.
\end{itemize}
\subsubsection{Configuration status accounting}
\subsubsubsection{Scopo}
Il \textit{configuration status accounting}\textsubscript{G} è il processo di \textit{documentazione}\textsubscript{G} e monitoraggio di verifiche e cambiamenti alle caratteristiche del prodotto \textit{software}\textsubscript{G}. Grazie ad esso si ha una visione complessiva della sua evoluzione e della sua aderenza ai \textit{requisito}\textsubscript{G} e agli standard.
\subsubsubsection{Version control}
Il \textit{sistema}\textsubscript{G} di version control adottato dal gruppo per tracciare l'evoluzione del \textit{software}\textsubscript{G} segue la seguente convenzione di numerazione delle versioni: [x].[y].[z]([build]). Si tratta di un \textit{sistema}\textsubscript{G} di versionamento a quattro componenti, ognuna delle quali ha un significato specifico:
\begin{itemize}
    \item \textbf{x}: Rappresenta la versione principale del \textit{software}\textsubscript{G}, e il suo valore viene incrementato per ogni fase significativa di revisione o avanzamento del progetto;
    \item \textbf{y}: Rappresenta cambiamenti significativi o aggiunte di nuove funzionalità, e il suo valore viene incrementato ogni volta che vengono apportati cambiamenti considerati rilevanti per il prodotto;
    \item \textbf{z}: Rappresenta piccole modifiche o correzioni di \textit{bug}\textsubscript{G}, e viene incrementato quando vengono svolte azioni come l'aggiornamento della \textit{documentazione}\textsubscript{G} o la correzione di errori minori;
    \item \textbf{build}: Indica il numero delle build eseguite per una determinata versione, e viene incrementato ogni volta che vengono apportate delle modifiche alla \textit{documentazione}\textsubscript{G}.
\end{itemize}
Il \textit{sistema}\textsubscript{G} prevede che la numerazione inizi dalla versione "0.0.1(0)". Ogni volta che il valore di x, y o z aumenta, tutti i valori alla sua destra vengono resettati a "0".
\subsubsection{Configuration evaluation}
\subsubsubsection{Scopo}
Il processo di configuration evaluation serve a garantire la correttezza, la coerenza e la conformità della configurazione del \textit{sistema}\textsubscript{G}, del \textit{software}\textsubscript{G} e dell'infrastruttura rispetto ai \textit{requisito}\textsubscript{G} definiti.
\subsubsubsection{Tracciamento dei requisiti}
Il team PEBKAC ha deciso di tracciare i \textit{requisiti}\textsubscript{G} direttamente nel codice del prodotto \textit{software}\textsubscript{G}. Questo approccio offre un collegamento diretto e verificabile tra i \textit{requisiti}\textsubscript{G} di progettazione e il codice che soddisfa tali \textit{requisiti}\textsubscript{G}. Il tracciamento viene effettuato includendo un commento specifico prima di ogni blocco di codice che implementa un determinato \textit{requisito}\textsubscript{G}. Il commento contiene l'ID univoco del \textit{requisito}\textsubscript{G}, in modo da facilitare l'associazione tra i \textit{requisiti}\textsubscript{G} e le corrispondenti implementazioni nel codice.
\subsubsection{Release Management}
\subsubsubsection{Scopo}
Il processo di \textit{release management}\textsubscript{G} serve a pianificare, coordinare e gestire il rilascio, per far sì che qualsiasi nuova versione di un prodotto venga distribuita dal \textit{sistema}\textsubscript{G} in modo controllato.
\subsubsubsection{Automazione compilazione documenti}
Il gruppo si è dotato di una \textit{GitHub Action}\textsubscript{G} che provvede all'\textit{automazione}\textsubscript{G} della generazione e trasferimento di documenti \textit{LaTeX}\textsubscript{G} in PDF.\\
Essa si attiva quando viene unita una \textit{pull request}\textsubscript{G} nel \textit{branch}\textsubscript{G} develop e funziona nel seguente modo: 
\begin{enumerate}
    \item Prepara l'ambiente ed installa gli strumenti necessari;
    \item Esplora la directory contenente i documenti \textit{LaTeX}\textsubscript{G};
    \item Converte eventuali immagini in PDF e compila i file \textit{LaTeX}\textsubscript{G};
    \item Carica i PDF generati in una cartella organizzata;
    \item Trasferisce la cartella sul \textit{branch}\textsubscript{G} main;
    \item Sposta i PDF all'interno della cartella corretta nel  \textit{branch}\textsubscript{G} main mediante un \textit{commit}\textsubscript{G} automatico.
\end{enumerate}
Il componente del gruppo che si è occupato della redazione di un documento dovrà semplicemente occuparsi di aprire una \textit{pull request}\textsubscript{G} verso il \textit{branch}\textsubscript{G} develop. Essa verrà verificata dal membro del gruppo che ricopre attualmente il \textit{ruolo}\textsubscript{G} di \textit{Verificatore}\textsubscript{G} e, in seguito, verrà approvata dal membro del gruppo che ricopre attualmente il \textit{ruolo}\textsubscript{G} di \textit{responsabile}\textsubscript{G}, che, infine, chiuderà la \textit{pull request}\textsubscript{G}, eseguendo l'\textit{automazione}\textsubscript{G}.\\
Questa \textit{automazione}\textsubscript{G} permette di eliminare il lavoro manuale ripetitivo e permette al team di concentrarsi sul contenuto invece che sulla gestione tecnica dei file.