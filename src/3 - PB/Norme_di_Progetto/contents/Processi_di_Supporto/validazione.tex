\subsection{Validazione}
\subsubsection{Scopo}
Lo scopo del processo di \textit{validazione}\textsubscript{G} è di confermare, attraverso esami e fornitura di evidenze oggettive, che il prodotto \textit{software}\textsubscript{G} finale soddisfa i requisiti stabiliti e riportati nell'Analisi dei Requisiti. 

\subsubsection{Esecuzione del processo}
Il processo di \textit{validazione}\textsubscript{G} viene eseguito in presenza del proponente dopo aver avuto un esito positivo dai \textit{test}\textsubscript{G} di unità, di integrazione e di \textit{sistema}\textsubscript{G}. Lo svolgimento di questo processo è suddiviso in due fasi:
\begin{enumerate}
    \item \textbf{Verifica del tracciamento dei requisiti}: il team presenta all’azienda proponente l’analisi del tracciamento dei requisiti, dimostrando così che ogni \textit{requisito}\textsubscript{G} è stato effettivamente integrato nel prodotto sottoposto a \textit{validazione}\textsubscript{G};
    \item \textbf{Fase di collaudo}: in collaborazione con l’azienda proponente, il team procede all’esecuzione del prodotto. Durante questa fase, vengono effettuati i \textit{test}\textsubscript{G} di accettazione per verificare la conformità del \textit{sistema}\textsubscript{G} rispetto alle specifiche richieste e, se tutti i \textit{test}\textsubscript{G} di accettazione eseguiti forniscono un esito positivo, il proponente considererà il prodotto valido.
\end{enumerate}

\subsubsection{Test di accettazione}
\begin{itemize}
    \item \textbf{Redazione}: i \textit{test}\textsubscript{G} di accettazione sono stabiliti in contemporanea alla redazione dell'Analisi dei Requisiti e vengono svolti alla presenza del \textit{committente}\textsubscript{G};
    \item \textbf{Descrizione}: i \textit{test}\textsubscript{G} di accettazione hanno come obiettivo principale di verificare che il prodotto soddisfi i requisiti utente, cioè le aspettative del \textit{committente}\textsubscript{G} così come definite nel \textit{capitolato}\textsubscript{G}. Questa attività si svolge in modo formale alla presenza del \textit{committente}\textsubscript{G}, il quale supervisiona o esegue direttamente i casi di prova. Tali casi sono derivati dai requisiti utente e mirano a dimostrare la conformità del \textit{software}\textsubscript{G} alle specifiche concordate. Il buon esito di questa fase è determinante per l’accettazione finale del prodotto e ne consente il rilascio ufficiale.
\end{itemize}