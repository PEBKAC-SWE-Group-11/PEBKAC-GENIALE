\subsection{Risoluzione dei problemi}
\subsubsection{Scopo}
Lo scopo del processo di risoluzione dei problemi è identificare, analizzare e risolvere tempestivamente i problemi che emergono durante il ciclo di vita del \textit{software}\textsubscript{G}. Questo processo garantisce una gestione efficace ed efficiente dei problemi, dalla loro rilevazione fino alla completa risoluzione e verifica, contribuendo così alla qualità complessiva del prodotto \textit{software}\textsubscript{G}.

\subsubsection{Implementazione}
Il processo di risoluzione dei problemi viene implementato attraverso una serie di fasi strutturate, con l’obiettivo di identificare e risolvere tempestivamente le criticità che emergono durante lo sviluppo del \textit{software}\textsubscript{G}. L’implementazione di questo processo si articola nei seguenti passaggi:
\begin{itemize}
    \item \textbf{Sviluppo di una strategia di gestione dei problemi}: definizione di un approccio strutturato per affrontare i problemi, stabilendo ruoli e responsabilità, flussi di comunicazione e strumenti da utilizzare;
    \item \textbf{Registrazione e classificazione dei problemi}: ogni problema rilevato viene registrato in uno storico, includendo dettagli come descrizione, data di rilevamento, origine e segnalatore. I problemi vengono poi classificati per gravità e priorità, permettendo una gestione efficace e un'analisi delle tendenze nel tempo;
    \item \textbf{Analisi dei problemi e individuazione delle soluzioni}: dopo la registrazione, il problema viene analizzato per identificarne le cause radice. Sulla base di questa analisi, vengono proposte e valutate soluzioni adeguate. Questa fase può coinvolgere l'analisi del codice, dei log di \textit{sistema}\textsubscript{G}, della \textit{documentazione}\textsubscript{G} o il confronto con i membri del team;
    \item \textbf{Implementazione della soluzione}: dopo aver individuato la soluzione più adatta, questa viene applicata. L’intervento può riguardare modifiche al codice, configurazioni di \textit{sistema}\textsubscript{G}, aggiornamenti della \textit{documentazione}\textsubscript{G} o altre azioni correttive;
    \item \textbf{Verifica dell’efficacia della correzione}: una volta implementata la soluzione, è essenziale verificare che il problema sia stato effettivamente risolto e che non siano stati introdotti nuovi difetti. Questa \textit{verifica}\textsubscript{G} avviene attraverso test, revisioni o altri controlli di qualità;
    \item \textbf{Monitoraggio e gestione continua dei problemi}: il processo prevede un costante monitoraggio dello stato dei problemi aperti, assicurando che siano assegnati e gestiti fino alla loro completa risoluzione;
\end{itemize}

\subsubsection{Strumenti}
Gli strumenti utilizzati per il processo di risoluzione dei problemi sono:
\begin{itemize}
    \item \nameref{Jira}.
\end{itemize}