\subsection{Revisione congiunta con il proponente}
\subsubsection{Scopo}
Lo scopo del processo di revisione congiunta con il proponente è garantire che il prodotto software in sviluppo soddisfi le aspettative del proponente, coinvolgendolo direttamente nella verifica degli artefatti. Questo processo aiuta a individuare tempestivamente difetti, malintesi sui requisiti e possibili problematiche, permettendo di apportare modifiche in tempo utile. Inoltre, favorisce la comunicazione e la collaborazione tra team e proponente, riducendo il rischio di insoddisfazione e costose rilavorazioni nelle fasi finali del progetto.

\subsubsection{Implementazione}
Il processo di revisione congiunta con il proponente viene implementato attraverso una serie di fasi che permettono di verificare e migliorare il prodotto software in modo collaborativo:
\begin{itemize}
    \item \textbf{Pianificazione della revisione}: il primo passo della revisione è definire lo scopo e gli artefatti da esaminare. Si selezionano i partecipanti, tra cui membri del team di sviluppo, rappresentanti del proponente e stakeholder, e si organizzano i dettagli dell’incontro. Infine, i materiali vengono preparati e condivisi per l'analisi;
    \item \textbf{Preparazione}: per un’efficace revisione, i partecipanti ricevono in anticipo la documentazione da analizzare. Vengono assegnati ruoli chiave: il moderatore guida la discussione, i revisori esaminano gli artefatti e un segretario registra le osservazioni. Si definiscono inoltre criteri di valutazione come coerenza, chiarezza e qualità del codice;
    \item \textbf{Esecuzione della revisione}: durante l’incontro, il team di sviluppo presenta gli artefatti al proponente, che fornisce feedback su eventuali criticità. Tutti i commenti vengono raccolti e documentati per un'analisi successiva;
    \item \textbf{Analisi dei risultati e azioni correttive}: dopo la revisione, il team classifica i problemi per priorità e definisce le azioni correttive. Viene redatto un report con i risultati e le modifiche necessarie, che viene condiviso con il proponente per garantire trasparenza e allineamento alle aspettative;
    \item \textbf{Follow-up e validazione}: dopo le modifiche, si può fare un'ulteriore revisione per verificare la risoluzione dei problemi. L'obiettivo è ottenere la conferma del proponente e ridurre il rischio di rilavorazioni future;
\end{itemize}

\subsubsection{Strumenti}
Gli strumenti utilizzati per il processo di revisione congiunta con il proponente sono:
\begin{itemize}
    \item \nameref{Google Calendar};
    \item \nameref{Microsoft Teams}.
\end{itemize}