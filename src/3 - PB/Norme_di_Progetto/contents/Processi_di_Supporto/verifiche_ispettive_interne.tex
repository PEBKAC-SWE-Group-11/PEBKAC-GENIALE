\subsection{Verifiche ispettive interne}
\subsubsection{Scopo}
Lo scopo del processo di verifiche ispettive interne è valutare se le attività e i risultati relativi alla qualità soddisfano le disposizioni pianificate e se queste sono attuate efficacemente. Le verifiche ispettive interne contribuiscono a garantire la qualità del \textit{software}\textsubscript{G}, valutare l'efficacia dei processi, identificare non conformità e fornire \textit{feedback}\textsubscript{G} per il miglioramento continuo, supportando così la produzione di \textit{software}\textsubscript{G} di alta qualità.

\subsubsection{Implementazione}
Per implementare il processo di verifiche ispettive interne, è necessario seguire questi passi:
\begin{itemize}
    \item \textbf{Pianificazione}: definire l’ambito, gli obiettivi e gli artefatti da verificare, come \textit{documentazione}\textsubscript{G}, codice o processi. Stabilire un piano con tempistiche e \textit{risorse}\textsubscript{G} necessarie;
    \item \textbf{Assegnazione dei ruoli}: designare i membri del team di verifica, che dovrebbero essere indipendenti dalle attività verificate. Ogni membro avrà un \textit{ruolo}\textsubscript{G} specifico, come analizzare i processi, raccogliere dati e redigere il report;
    \item \textbf{Esecuzione della verifica}: condurre l'ispezione sistematica dei processi e dei risultati, confrontandoli con gli standard e le disposizioni pianificate. Raccogliere e documentare le non conformità o le aree di miglioramento;
    \item \textbf{Analisi dei risultati}: valutare i dati raccolti, classificando le non conformità e le deviazioni. Identificare le cause e le aree da migliorare;
    \item \textbf{Feedback\textsubscript{G} e azioni correttive}: redigere un report con i risultati e le raccomandazioni. Fornire un \textit{feedback}\textsubscript{G} alle parti coinvolte, suggerendo azioni correttive e preventive per migliorare i processi;
    \item \textbf{Monitoraggio e follow-up}: verificare che le azioni correttive siano state implementate e che le problematiche siano state risolte, promuovendo il miglioramento continuo.
\end{itemize}

\subsubsection{Strumenti}
Gli strumenti utilizzati per il processo di verifiche ispettive interne sono:
\begin{itemize}
    \item \nameref{Discord};
    \item \nameref{Jira}.
\end{itemize}