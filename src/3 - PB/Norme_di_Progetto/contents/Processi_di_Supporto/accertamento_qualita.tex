\subsection{Accertamento di qualità}
\subsubsection{Scopo}
L'accertamento della qualità ha l'obiettivo di prevenire i difetti nel prodotto software, garantendo la conformità ai processi adottati. Piuttosto che limitarsi a individuare errori a posteriori, questo approccio assicura che ogni fase dello sviluppo venga eseguita correttamente, monitorando l’applicazione delle best practice in modo non intrusivo attraverso strumenti di controllo. Questo processo può fare uso anche dei risultati di altri processi di supporto, come ad esempio del processo di verifica e del processo di validazione.

\subsubsection{Ciclo di Deming}
Il Ciclo di Deming, noto anche come Shewhart-Deming’s Learning-and-Quality Cycle, è un modello a quattro fasi ideato intorno al 1950 per introdurre miglioramenti specifici nei processi. Fondamentale per il miglioramento continuo, il ciclo si articola in:  
\begin{itemize}
    \item \textbf{Pianificare (Plan)}: definizione delle attività, delle scadenze, delle responsabilità e delle risorse necessarie per raggiungere obiettivi di miglioramento. Questa fase non riguarda la pianificazione di un progetto, ma l’organizzazione di azioni mirate all’ottimizzazione di un processo;
    \item \textbf{Eseguire (Do)}: implementazione delle attività pianificate, anche in modo esplorativo. Non si tratta dello sviluppo di un prodotto, ma dell’attuazione concreta delle azioni di miglioramento;
    \item \textbf{Valutare (Check)}: analisi dei risultati ottenuti per verificare se le azioni intraprese hanno prodotto gli effetti desiderati;
    \item \textbf{Agire (Act)}: consolidamento delle soluzioni efficaci, aggiornando il way of working, e individuazione di ulteriori opportunità di miglioramento.
\end{itemize}
Questo ciclo aiuta il gruppo a perfezionare continuamente la qualità del prodotto, garantendo che vengano raggiunti gli obiettivi di qualità che ci si era prefissati inizialmente e impedendo un peggioramento della stessa.

\subsubsection{Metriche}
Le metriche di qualità sono gli strumenti utilizzati per valutare la qualità di un prodotto software o di un processo di sviluppo. Servono a misurare il grado di conformità agli standard, identificare aree di miglioramento e garantire che il software soddisfi i requisiti attesi. \\
Si suddividono in diverse categorie, tra cui:
\begin{itemize}
    \item \textbf{Metriche di processo}: valutano l’efficienza dello sviluppo;
    \item \textbf{Metriche di prodotto}: misurano le caratteristiche del software.
\end{itemize}
Le metriche di qualità utilizzate in questo progetto sono riportate alla sezione §\ref{mdq}. Esse vengono rappresentate dal loro nome in lingua inglese, da un'abbreviazione, che genralmente consiste nella sequenza delle prime lettere delle parole del nome della metrica stessa, tutte in maiuscolo, e da una descrizione, che ne spiega brevemente il significato.

\subsubsection{Obiettivi di qualità}
Gli obiettivi di qualità sono riportati nel Piano di Qualifica e, come le metriche, sono suddivisi in obiettivi di qualità di processo e di prodotto.
Essi sono strutturati nel seguente modo:
\begin{itemize}
    \item \textbf{Metrica}: l'abbreviazione del nome della metrica;
    \item \textbf{Descrizione}: il nome in lingua inglese;
    \item \textbf{Valore accettabile}: valore per cui la metrica è da considerare soddisfatta, nonostante ci sia spazio per un miglioramento;
    \item \textbf{Valore ideale}: valore per cui la metrica viene considerata pienamente soddisfatta.
\end{itemize}

\subsubsection{Strumenti}
Gli strumenti utilizzati per il processo di accertamento di qualità sono:
\begin{itemize}
    \item \nameref{Google Sheet};
    \item \nameref{Jira}.
\end{itemize}