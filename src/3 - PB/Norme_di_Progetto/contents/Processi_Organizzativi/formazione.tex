\subsection{Formazione}
\subsubsection{Scopo}
Lo scopo del processo di formazione è aiutare i membri del gruppo a sviluppare le competenze necessarie per lavorare in modo efficace e produttivo. Questo processo si propone di garantire che le persone abbiano le conoscenze giuste per raggiungere gli obiettivi del progetto e rispettare gli standard di qualità.

\subsubsection{Pianificazione}
Ogni membro del team ha la libertà di scegliere come formarsi, seguendo un approccio pratico e personalizzato, in base alle proprie esigenze e interessi. Inoltre, chi ha più esperienza e competenze in un determinato campo è incoraggiato ad aiutare e supportare i colleghi con meno esperienza, creando un ambiente di apprendimento collaborativo. Questo permette a tutti di crescere insieme, condividendo conoscenze e \textit{risorse}\textsubscript{G} in modo continuo e dinamico.

\subsubsection{Raccolta del materiale}
Il materiale per la formazione viene principalmente cercato e trovato \textit{online}\textsubscript{G}, sfruttando le \textit{risorse}\textsubscript{G} disponibili su internet. Ogni membro del team può accedere a tutorial, corsi, articoli, video e altre \textit{risorse}\textsubscript{G} gratuite per approfondire le proprie competenze in modo autonomo