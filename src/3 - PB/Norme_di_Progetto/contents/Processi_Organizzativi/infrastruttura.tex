\subsection{Gestione dell'Infrastruttura}
\subsubsection{Scopo}
Lo scopo del processo di gestione delle infrastrutture è garantire la disponibilità, la stabilità e l'efficienza degli strumenti informatici a supporto dei processi di progetto.

\subsubsection{Implementazione}
L’implementazione dell’infrastruttura è il processo di creazione e configurazione dell’ambiente hardware e software necessario per supportare lo sviluppo, il test e l’operatività di un sistema software. Questa fase è essenziale per fornire al team di sviluppo le risorse adeguate per lavorare in modo efficiente e senza ostacoli tecnici. Il ruolo centrale in questa attività è ricoperto dall’amministratore. \\
Le attività principali dell’implementazione dell’infrastruttura includono:
\begin{enumerate}
    \item \textbf{Selezione e messa in opera delle risorse informatiche}: il primo passo consiste nell’identificazione dei requisiti infrastrutturali del progetto, considerando il way of working adottato dal team. L'amministratore sceglie le componenti hardware e software, configurando la rete e assicurando la loro corretta installazione e operatività;
    \item \textbf{Definizione e controllo dell’ambiente}: vengono stabilite configurazioni standard per garantire coerenza, efficienza e sicurezza. Inoltre, il monitoraggio continuo dell’infrastruttura assicura il rispetto delle configurazioni e previene eventuali malfunzionamenti;
    \item \textbf{Analisi degli aspetti critici:} ogni scelta viene valutata considerando diversi fattori chiave:
    \begin{itemize}
        \item \textbf{Funzionalità}: le modifiche devono portare un miglioramento, introducendo nuove capacità o affinando quelle esistenti. È essenziale prevenire malfunzionamenti o comportamenti indesiderati;
        \item \textbf{Performance}: ogni intervento deve mantenere o migliorare le prestazioni dell'infrastruttura senza compromettere l'efficienza;
        \item \textbf{Sicurezza}: nessuna modifica deve ridurre il livello di protezione dei dati e delle risorse;
        \item \textbf{Disponibilità}: la configurazione deve essere applicabile su tutti i dispositivi, evitando discrepanze tra le infrastrutture;
        \item \textbf{Vincoli di spazio}: l'infrastruttura deve rispettare le limitazioni di archiviazione dei dispositivi utilizzati;
        \item \textbf{Compatibilità hardware}: le scelte devono essere compatibili con le risorse fisiche disponibili;
        \item \textbf{Costi}: viene data priorità a soluzioni open-source o gratuite rispetto a quelle a pagamento, ove possibile;
        \item \textbf{Tempistiche}: l'installazione e la configurazione devono essere rapide per non rallentare il progetto.
    \end{itemize}
    \item \textbf{Documentazione delle procedure:} se la verifica ha esito positivo, l’amministratore redige una guida dettagliata per l'installazione e la configurazione dell’infrastruttura;
    \item \textbf{Comunicazione al team:} una volta completata la documentazione, i membri del team vengono informati sulle procedure da seguire e devono implementarle autonomamente;
    \item \textbf{Supporto in caso di difficoltà:} se un componente riscontra problemi durante l'installazione o la configurazione, l’amministratore fornisce assistenza diretta per risolvere eventuali problematiche;
    \item \textbf{Sessioni di supporto aggiuntive:} nel caso di procedure particolarmente complesse, il responsabile del progetto, in accordo con l’amministratore, può organizzare incontri specifici per spiegare dettagliatamente le operazioni richieste;
\end{enumerate}

\subsubsection{Manutenzione}
La manutenzione dell’infrastruttura è un’attività essenziale per garantire un ambiente di lavoro stabile, sicuro ed efficiente. Questa responsabilità è affidata all’Amministratore. \\
Le principali attività di manutenzione dell'infrastruttura comprendono:
\begin{itemize}
    \item \textbf{Monitoraggio e gestione dell’ambiente}: l'amministratore si occupa della configurazione, controllo e manutenzione dell’infrastruttura, assicurando che hardware, software e reti siano sempre operativi e aggiornati per supportare al meglio le esigenze del team;
    \item \textbf{Gestione delle risorse informatiche}: la manutenzione non si limita alla configurazione iniziale, ma prevede un monitoraggio continuo delle prestazioni e della capacità delle risorse. Questo include l'aggiornamento dei sistemi, la sostituzione di componenti obsoleti o malfunzionanti e l’ottimizzazione delle configurazioni per garantire efficienza e scalabilità;
    \item \textbf{Prevenzione dei problemi e sicurezza}: oltre alla risoluzione dei guasti, la manutenzione prevede un approccio proattivo, basato su un monitoraggio costante dei log di sistema, sull’analisi delle prestazioni e sull’implementazione di misure di sicurezza per prevenire vulnerabilità e interruzioni del servizio;
    \item \textbf{Gestione delle segnalazioni e interventi correttivi}: in caso di malfunzionamenti, viene utilizzato un sistema di ticketing per raccogliere, analizzare e risolvere tempestivamente le segnalazioni.
\end{itemize}

\subsubsection{Strumenti}
Gli strumenti utilizzati per il processo di gestione dell'infrastruttura sono:
\begin{itemize}
    \item \nameref{Jira}.
\end{itemize}