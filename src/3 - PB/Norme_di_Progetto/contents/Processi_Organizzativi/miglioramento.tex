\subsection{Miglioramento}
\subsubsection{Scopo}
Il processo di miglioramento mira a ottimizzare continuamente il way of working, aumentando l’efficacia e l’efficienza delle attività senza compromettere la qualità.

\subsubsection{Determinazione del processo}
Il team definisce i processi del ciclo di vita del software basandosi sullo Standard ISO/IEC 12207:1997. Questi processi vengono documentati e, quando possibile, viene implementato un sistema di monitoraggio per garantire il controllo e il miglioramento continuo delle loro applicazioni.

\subsubsection{Valutazione del processo}
L'attività di valutazione di un processo consiste nell'analizzare e misurare sistematicamente come un processo viene eseguito per valutarne l'efficacia, l'efficienza e la conformità al way of working. L'obiettivo principale della valutazione è identificare i punti di forza e le aree di miglioramento del processo, fornendo così una base per ottimizzarlo continuamente. \\
La valutazione di un processo si concentra su aspetti chiave come:
\begin{itemize}
    \item Se il processo raggiunge gli obiettivi prefissati;
    \item Se il processo utilizza le risorse in modo ottimale;
    \item Se il processo è conforme al way of working stabilito;
    \item Il livello di maturità del processo;
    \item Le opportunità di miglioramento per ottimizzare il processo.
\end{itemize}

\subsubsection{Miglioramento del processo}
Il miglioramento di un processo deve seguire un approccio strutturato e ciclico (ciclo di Deming), che prevede una serie di fasi interconnesse:
\begin{itemize}
    \item \textbf{Pianificazione del miglioramento (Plan)}: sulla base della valutazione, si definiscono obiettivi specifici e azioni correttive per il miglioramento del processo, stabilendo scadenze, responsabilità e risorse necessarie, senza includere la pianificazione del progetto;
    \item \textbf{Esecuzione delle azioni di miglioramento (Do)}: le azioni pianificate vengono implementate attraverso modifiche alle procedure, l'introduzione di nuove tecnologie o la formazione del personale, con una documentazione accurata delle modifiche e delle azioni intraprese;
    \item \textbf{Valutazione dell'efficacia del miglioramento (Check)}: dopo l'implementazione, si valuta l'impatto delle modifiche confrontando i risultati con gli obiettivi prefissati, per verificare se i miglioramenti desiderati sono stati raggiunti;
    \item \textbf{Azione e consolidamento (Act)}: se i miglioramenti sono efficaci vengono consolidati, altrimenti si analizzano le cause e si pianificano nuove azioni correttive, riprendendo il ciclo di Deming.
\end{itemize}