\subsection{Gestione organizzativa}
\subsubsection{Scopo}
Lo scopo di questo processo è esporre le modalità e gli strumenti di coordinamento usati dal gruppo per la comunicazione, interna ed esterna, e normare l'assegnazione di \textit{ruoli}\textsubscript{G} e compiti, oltre che la gestione dei rischi.

\subsubsection{Ruoli}
Per ottimizzare la gestione delle attività e dei compiti da svolgere vengono definiti sei \textit{ruoli}\textsubscript{G} distinti, ciascuno con mansioni e responsabilità specifiche. Ogni componente del gruppo dovrà assumere ciascun \textit{ruolo}\textsubscript{G} per un numero di ore significativo.
\subsubsubsection{Responsabile}
Il \textit{responsabile}\textsubscript{G} è il punto di riferimento per tutto il gruppo e anche per le comunicazioni con il \textit{committente}\textsubscript{G} e con l'azienda proponente.
Inoltre il \textit{responsabile}\textsubscript{G} è la figura che ha il compito di coordinare le azioni dei membri del gruppo, perciò deve avere competenze tecniche in ogni ambito del progetto. Le responsabilità di questo \textit{ruolo}\textsubscript{G} sono:
\begin{itemize}
    \item Coordinamento tra gruppo ed enti esterni;
    \item Gestione delle comunicazioni interne;
    \item Pianificazione di progetto,
    \item Gestione dei \textit{task}\textsubscript{G} e delle risorse;
    \item Gestione dell'avanzamento del progetto.
\end{itemize}
\subsubsubsection{Amministratore}
L'\textit{amministratore}\textsubscript{G} è la figura che definisce, gestisce e mantiene l’ambiente e l’infrastruttura necessari per lo sviluppo del progetto facendo in modo che siano affidabili e sicuri. Si occupa della gestione della configurazione, del versionamento, delle varie \textit{automazioni}\textsubscript{G} e della \textit{documentazione}\textsubscript{G}. Si occupa di:
\begin{itemize}
    \item Selezionare e abilitare risorse informatiche a supporto del \textit{way of working}\textsubscript{G};
    \item Gestire errori e malfunzionamenti nei meccanismi nell’infrastruttura.
\end{itemize}

\subsubsubsection{Analista}
La funzione dell'\textit{analista}\textsubscript{G} è quella di analizzare il problema per definire i \textit{requisiti}\textsubscript{G} del prodotto, per questo deve avere buona conoscenza del dominio del problema. L'\textit{analista}\textsubscript{G} raccoglie le sue produzioni nel documento Analisi dei Requisiti. Si tratta di un \textit{ruolo}\textsubscript{G} fondamentale all'inizio del progetto, ma la cui utilità cala nelle seguenti fasi del progetto.
\subsubsubsection{Progettista}
Al progettista spettano le scelte realizzative e le specifiche architetturali del prodotto. Deve avere buone competenze tecniche e tecnologiche. Durante il processo di sviluppo la sua utilità è massima, ma tende a calare dalla fase di manutenzione in poi.
\subsubsubsection{Programmatore}
Quello del programmatore è un \textit{ruolo}\textsubscript{G} chiave nella fase di sviluppo. In particolare si occupa di :
\begin{itemize}
    \item Codificare ciò che è stato definito dai progettisti;
    \item Implementare i test;
    \item Redigere il Manuale utente.
\end{itemize}
\subsubsubsection{Verificatore}
Ha il compito di verificare il lavoro degli altri e per questo deve avere competenze tecniche ed essere presente per l'intera durata del progetto. Questa figura deve controllare che tutto ciò che viene prodotto sia conforme alle norme e alle aspettative di qualità del gruppo.


\subsubsection{Attività}
Ogni membro del gruppo può proporre attività da svolgere, ma è compito del \textit{responsabile}\textsubscript{G} stabilire la fattibilità rispetto alle risorse usufruibili. 
\subsubsubsection{Pianificazione}
Le attività definite devono poi essere pianificate in termini di tempo e risorse dal \textit{responsabile}\textsubscript{G}, stabilendo quindi:
\begin{itemize}
    \item La tempistica prevista per il completamento dell'attività;
    \item Il membro che dovrà eseguire l'attività, in base a \textit{ruolo}\textsubscript{G} e risorse disponibili;
    \item Il \textit{verificatore}\textsubscript{G};
    \item Il rischio associato.
\end{itemize}

\subsubsubsubsection{Strumenti}
Gli strumenti utilizzati per la pianificazione sono:
    \begin{itemize}
        \item \nameref{Jira}.
    \end{itemize}

\subsubsubsection{Esecuzione}
L'esecuzione delle attività avviene obbligatoriamente per mano dell'assegnatario, definito dal \textit{responsabile}\textsubscript{G}. L'esecuzione deve essere obbligatoriamente conforme alla \textit{documentazione}\textsubscript{G} associata precedentemente redatta. L'esecutore dovrà proporre la sua soluzione con una \textit{pull request}\textsubscript{G}.

\subsubsubsection{Revisione}
La revisione dell'attività è effettuata dal \textit{verificatore}\textsubscript{G} prima dell'effettivo inserimento delle modifiche nel \textit{repository}\textsubscript{G} \textit{GitHub}\textsubscript{G}: la \textit{pull request}\textsubscript{G} aperta dell'esecutore viene accettata o rifiutata, riportando le parti non valide ed eventuali accorgimenti possibili, a seconda dell'esito della \textit{verifica}\textsubscript{G}.
\subsubsubsubsection{Strumenti}
Gli strumenti utilizzati per la revisione sono:
    \begin{itemize}
        \item \textit{\nameref{GitHub}}\textsubscript{G}.
    \end{itemize}

\subsubsubsection{Chiusura}
Solo nel caso dell'esito positivo della \textit{verifica}\textsubscript{G}, con l'accettazione della \textit{pull request}\textsubscript{G}, viene chiuso il branch di cui è stato effettuato il merge e l'attività viene segnata come completata. 

\subsubsubsection{Tracciamento orario}
Il gruppo utilizza \textit{Google Sheets}\textsubscript{G} per avere un foglio di calcolo condiviso in cui tenere conto del tempo speso per svolgere le attività. Ogni membro è tenuto a registrare, alla fine di ogni sessione lavorativa, il numero di ore effettive di lavoro e il \textit{ruolo}\textsubscript{G} ricoperto.
\subsubsubsubsection{Strumenti}
Gli strumenti utilizzati per il tracciamento orario sono:
    \begin{itemize}
        \item \textit{\nameref{Google Sheets}}\textsubscript{G}.
    \end{itemize}


\subsubsection{Comunicazione}
\subsubsubsection{Comunicazioni interne}
\subsubsubsubsection{Comunicazioni sincrone}
Le riunioni interne si svolgeranno sulla piattaforma \textit{Slack}\textsubscript{G} oppure in presenza, dovranno in ogni caso decise e organizzate alcuni giorni prima per consentire la presenza di tutti i membri. In ogni riunione il gruppo predilige un approccio libero alla discussione, incentrato sulla crescita e allo scambio di opinioni. \\
La gestione delle riunioni interne viene affidata al \textit{responsabile}\textsubscript{G} che, coadiuvato dall'\textit{Amministratore}\textsubscript{G}, ha il compito di:
\begin{enumerate}
    \item Fissare data, ora e luogo della riunione;
    \item Stabilire un ordine del giorno per le riunioni;
    \item Fare da moderatore durante la discussione, per garantire a tutti l'opportunità di esprimersi;
    \item Se necessario, comunicare con l'esterno in base alle decisioni prese dal gruppo.
\end{enumerate}
\subsubsubsubsection{Comunicazioni asincrone}
Per le comunicazioni asincrone il gruppo utilizzerà:
\begin{itemize}
    \item \textit{Slack}\textsubscript{G}: in un area di lavoro sono stati creati:
            \begin{itemize}
                \item \textbf{Canali}: uno principale con tutti i membri e altri, alla necessità, in cui i componenti possono organizzarsi e lavorare su attività collaborative;
                \item \textbf{Canvas}: uno per ogni canale in cui sia necessario, per fissare messaggi importanti, documenti e risorse che richiedono facile accesso.
            \end{itemize}
    \item Whatsapp: solo per comunicazioni immediate e poco formali.
\end{itemize}
\subsubsubsubsection{Strumenti}
Gli strumenti utilizzati per le comunicazioni interne sono:
\begin{itemize}
    \item \textit{\nameref{Slack}}\textsubscript{G};
    \item \nameref{Whatsapp}.
\end{itemize}


\subsubsubsection{Comunicazioni esterne}
\subsubsubsubsection{Comunicazioni sincrone}
Per quanto riguarda gli incontri con l'azienda proponente, cruciali per discutere in modo semplice e immediata di argomenti anche complessi, potranno essere in presenza (presso la sede R\&D di Vimar S.p.A.) oppure da remoto sulla piattaforma \textit{Microsoft Teams}\textsubscript{G}. Per quanto riguarda le riunioni con l'azienda proponente, anche chiamate SAL - Stato di Avanzamento Lavori.
\begin{itemize}
    \item Il gruppo ha concordato con l'azienda un calendario di incontri bisettimanali della durata di 60 minuti fino alla prima revisione, poi settimanali della durata di 30 minuti;
    \item Il gruppo si impegna a presenziare in maniera assidua agli incontri, segnalando per tempo eventuali assenze o modifiche a quanto precedentemente concordato;
    \item Il gruppo si impegna a redigere un verbale per ogni incontro per documentarne il contenuto e farlo approvare all'azienda. 
\end{itemize}
\subsubsubsubsection{Comunicazioni asincrone}
Per le comunicazioni asincrone con il proponente o altri soggetti esterni vengono utilizzati:
\begin{itemize}
    \item \textbf{Microsoft Teams\textsubscript{G}}: per domande corte o piccoli chiarimenti si potrà usare la chat condivisa creata dall'azienda;
    \item \textbf{Posta elettronica}: per comunicare con soggetti esterni e con l'azienda proponente per domande articolate o modifiche agli appuntamenti fissati si utilizzerà la mail del gruppo: pebkacswe@gmail.com.
\end{itemize}
\subsubsubsubsection{Strumenti}
Gli strumenti utilizzati per le comunicazioni esterne sono:
\begin{itemize}
    \item \textit{\nameref{Microsoft Teams}}\textsubscript{G};
    \item \nameref{Google Gmail}.
\end{itemize}

\subsubsubsection{Norme comportamentali}
I membri del gruppo, per garantire il rispetto delle norme e degli altri membri del gruppo, sono obbligati a:
\begin{itemize}
    \item Essere sempre puntuali o almeno comunicare tempestivamente al \textit{responsabile}\textsubscript{G} eventuali problemi;
    \item Partecipare attivamente alla discussione;
    \item Mantenere un atteggiamento rispettoso, disciplinato, aperto alle discussione e disponibile.
\end{itemize}
Inoltre, per le riunioni SAL con l'azienda proponente i membri hanno concordato di rispettare le seguenti regole:
\begin{itemize}
    \item Tenere i telefoni spenti o silenziati (a meno di particolati motivi);
    \item Non utilizzare PC o Tablet, fatta eccezione per il \textit{responsabile}\textsubscript{G} o chi deve raccogliere degli appunti.
\end{itemize}

\subsubsubsection{Moderazione}
La gestione della comunicazione del gruppo di lavoro ha un ruolo fondamentale nel garantire interazioni efficaci, costruttive e orientate agli obiettivi. Il responsabile, che assume il ruolo di moderatore, si occupa di:
\begin{itemize}
    \item Facilitare le discussioni;
    \item Assicurare che tutti i membri abbiano l’opportunità di esprimersi;
    \item Mantenere il focus sugli argomenti rilevanti;
    \item Aiutare a gestire il tempo degli incontri;
    \item Sintetizzare i punti chiave;
    \item Promuovere una comunicazione chiara e collaborativa.
\end{itemize}

\subsubsubsection{Gestione dei rischi}
La gestione dei rischi in un progetto software è tipicamente responsabilità del responsabile e dell'amministratore. I risultati delle attività identificazione, analisi, pianificazione e controllo dei rischi sono inseriti all'interno del Piano di Progetto.

\subsubsubsubsection{Nomenclatura}
Per documentare i rischi viene utilizzata la notazione seguente:
\begin{center}
    \textbf{R[Tipologia][Numero]}
\end{center}
\textbf{Tipologia} indica la tipologia del rischio:
\begin{itemize}
    \item \textbf{T}: tecnologio;
    \item \textbf{O}: organizzativo;
    \item \textbf{G}: interno al gruppo;
\end{itemize}
\textbf{Numero} è un numero univoco progressivo correlato alla tipologia;

\subsubsubsubsection{Descrizione}
La definizione dei rischi presenta le seguenti informazioni:
\begin{itemize}
    \item \textbf{Id. Rischio}: codice identificativo assegnato al rischio;
    \item \textbf{Rischio}: nome completo del rischio;
    \item \textbf{Descrizione}: spiegazione dettagliata del rischio;
    \item \textbf{Pericolosità}: livello di gravità dell’impatto che il rischio potrebbe avere sul progetto;
    \item \textbf{Occorrenza}: probabilità che il rischio si verifichi;
    \item \textbf{Piano di intervento}: strategie e azioni previste per prevenire, mitigare o gestire il rischio nel caso in cui si manifesti.
\end{itemize}