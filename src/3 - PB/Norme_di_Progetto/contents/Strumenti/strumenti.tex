\section{Strumenti}
In questa sezione vengono presentati tutti gli strumenti utilizzati dal gruppo durante lo svolgimento del progetto.

\subsection{Draw.io}
\label{Diagrams.net}
\href{https://www.diagrams.net}{https://www.diagrams.net} (Ultimo accesso: 2025-04-01) \\ \\  
Diagrams.net è uno strumento gratuito basato su web per la creazione di diagrammi e grafici, utilizzato per creare diagrammi di flusso, mappe concettuali, organigrammi, wireframe e molti altri tipi di rappresentazioni visive. Permette di collaborare in tempo reale e offre una vasta libreria di forme predefinite.

\subsection{Discord}
\label{Discord}
\href{https://discord.com}{https://discord.com} (Ultimo accesso: 2025-04-01) \\ \\  
Discord è una piattaforma di comunicazione vocale, video e testuale, molto utilizzata da community, team di gioco e gruppi di lavoro. Consente la creazione di server, canali tematici, chat di testo e videoconferenze. Inoltre, è possibile integrare bot e automatizzare operazioni.

\subsection{Docker}
\label{Docker}
\href{https://www.docker.com}{https://www.docker.com} (Ultimo accesso: 2025-04-01) \\ \\  
\textit{Docker}\textsubscript{G} è una piattaforma che utilizza la virtualizzazione a livello di \textit{sistema}\textsubscript{G} operativo per distribuire applicazioni in contenitori isolati (\textit{container}\textsubscript{G}). \textit{Docker}\textsubscript{G} consente di costruire, testare e distribuire applicazioni in modo coerente e ripetibile su vari ambienti.

\subsection{GitHub}
\label{GitHub}
\href{https://github.com}{https://github.com} (Ultimo accesso: 2025-04-01) \\ \\  
\textit{GitHub}\textsubscript{G} è un servizio di hosting basato su \textit{Git}\textsubscript{G} per il controllo versione e la gestione di codice sorgente. Permette la collaborazione tra sviluppatori, gestione di \textit{pull request}\textsubscript{G}, gestione \textit{issue}\textsubscript{G} e \textit{documentazione}\textsubscript{G} dei progetti. È ampiamente usato per progetti open-source.

\subsection{Google Calendar}
\label{Google Calendar}
\href{https://calendar.google.com}{https://calendar.google.com} (Ultimo accesso: 2025-04-01) \\ \\  
Google Calendar è un servizio gratuito di Google per la gestione degli appuntamenti e la pianificazione degli eventi. Permette la creazione di calendari condivisi, l'invio di inviti, la sincronizzazione con dispositivi mobili e l'integrazione con altre applicazioni Google.

\subsection{Google Gmail}
\label{Google Gmail}
\href{https://mail.google.com}{https://mail.google.com} (Ultimo accesso: 2025-04-01) \\ \\  
Google Gmail è un servizio di posta elettronica gratuito fornito da Google, noto per la sua interfaccia pulita e funzionalità avanzate come la ricerca potente, l'archiviazione e la gestione dei filtri e delle etichette. Integrato con altri strumenti Google.

\subsection{Google Sheets}
\label{Google Sheets}
\href{https://sheets.google.com}{https://sheets.google.com} (Ultimo accesso: 2025-04-01) \\ \\  
\textit{Google Sheets}\textsubscript{G} è un'applicazione di fogli di calcolo \textit{online}\textsubscript{G}, parte di Google Drive, che consente di creare, modificare e collaborare su fogli di calcolo in tempo reale. È una valida alternativa a Microsoft Excel, con funzionalità di collaborazione e integrazione con altri strumenti Google.

\subsection{Jira}
\label{Jira}
\href{https://www.atlassian.com/software/jira}{https://www.atlassian.com/software/jira} (Ultimo accesso: 2025-04-01) \\ \\  
\textit{Jira}\textsubscript{G} è un \textit{software}\textsubscript{G} di gestione progetti e tracciamento dei problemi, ampiamente utilizzato in ambienti agili per gestire \textit{sprint}\textsubscript{G}, \textit{backlog}\textsubscript{G} e flussi di lavoro. Offre funzionalità avanzate per il monitoraggio delle attività, reportistica e integrazione con altri strumenti di sviluppo.

\subsection{LaTeX}
\label{LaTeX}
\href{https://www.latex-project.org}{https://www.latex-project.org} (Ultimo accesso: 2025-04-01) \\ \\  
\textit{LaTeX}\textsubscript{G} è un linguaggio di markup per la scrittura e la gestione di documenti complessi, particolarmente usato in ambito accademico e scientifico. \textit{LaTeX}\textsubscript{G} è perfetto per la produzione di documenti tecnici, con formule matematiche, bibliografie e strutture complesse.

\subsection{Microsoft PowerPoint}
\label{Microsoft PowerPoint}
\href{https://www.microsoft.com/en-us/microsoft-365/powerpoint}{https://www.microsoft.com/en-us/microsoft-365/powerpoint} (Ultimo accesso: 2025-04-01) \\ \\  
Microsoft PowerPoint è un \textit{software}\textsubscript{G} di presentazione parte della suite Microsoft Office. È utilizzato per creare diapositive visive con testo, immagini, grafici e video per presentazioni professionali. Ha strumenti avanzati per animazioni e transizioni.

\subsection{Microsoft Teams}
\label{Microsoft Teams}
\href{https://www.microsoft.com/en-us/microsoft-teams}{https://www.microsoft.com/en-us/microsoft-teams} (Ultimo accesso: 2025-04-01) \\ \\  
\textit{Microsoft Teams}\textsubscript{G} è una piattaforma di collaborazione che integra chat, videochiamate, archiviazione di file e integrazione con altre applicazioni Microsoft. Viene utilizzata per facilitare la comunicazione e la collaborazione a distanza in team di lavoro.

\subsection{Slack}
\label{Slack}
\href{https://slack.com}{https://slack.com} (Ultimo accesso: 2025-04-01) \\ \\  
\textit{Slack}\textsubscript{G} è uno strumento di comunicazione per team che unisce chat di testo, canali di discussione, integrazione con altre applicazioni e supporto per bot. È progettato per facilitare la collaborazione, la condivisione di informazioni e il lavoro a distanza.

\subsection{StarUML}
\label{StarUML}
\href{http://staruml.io}{http://staruml.io} (Ultimo accesso: 2025-04-01) \\ \\  
StarUML è uno strumento di modellazione \textit{UML}\textsubscript{G} (Unified Modeling Language) che consente la progettazione di diagrammi per rappresentare visivamente sistemi \textit{software}\textsubscript{G} e architetture. Supporta diversi tipi di diagrammi, inclusi quelli di classi, sequenze e attività.

\subsection{Visual Studio Code}
\label{Visual Studio Code}
\href{https://code.visualstudio.com}{https://code.visualstudio.com} (Ultimo accesso: 2025-04-01) \\ \\  
Visual Studio Code è un editor di codice sorgente gratuito sviluppato da Microsoft. Supporta numerosi linguaggi di programmazione e offre un'ampia gamma di estensioni per il debugging, il controllo versione e la gestione di progetti di sviluppo.

\subsection{Whatsapp}
\label{Whatsapp}
\href{https://www.whatsapp.com}{https://www.whatsapp.com} (Ultimo accesso: 2025-04-01) \\ \\  
Whatsapp è un'applicazione di messaggistica istantanea che consente di inviare messaggi di testo, foto, video, file e fare chiamate vocali e video. È ampiamente utilizzata per comunicazioni personali e professionali, supportando anche chat di gruppo.
