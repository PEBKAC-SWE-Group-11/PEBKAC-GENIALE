
\documentclass[12pt, a4paper]{article}
\usepackage{longtable}
\usepackage{graphicx}
\usepackage{xcolor}
\usepackage{float}
\usepackage{svg}
\usepackage[colorlinks=true, linkcolor=black, urlcolor=blue, citecolor=green]{hyperref}
\usepackage{enumitem}
\usepackage[italian]{babel}
\usepackage{lastpage}  % Pacchetto per ottenere il numero totale delle pagine
\usepackage{fancyhdr}  % Pacchetto per personalizzare l'intestazione e il piè di pagina
\usepackage[margin=1in]{geometry}
\usepackage{array}
\newcolumntype{C}[1]{>{\centering\arraybackslash}p{#1}}
\newcolumntype{L}[1]{>{\raggedright\arraybackslash}p{#1}}
\graphicspath{ {images/} {../shared/images/} }
\definecolor{unipd}{HTML}{B5121B}

\addto\captionsitalian{\renewcommand{\contentsname}{Indice}}

\setcounter{secnumdepth}{5}
\setcounter{tocdepth}{5}
\makeatletter
\newcommand\subsubsubsection{\@startsection{paragraph}{4}{\z@}{-2.5ex\@plus -1ex \@minus -.25ex}{1.25ex \@plus .25ex}{\normalfont\normalsize\bfseries}}
\newcommand\subsubsubsubsection{\@startsection{subparagraph}{5}{\z@}{-2.5ex\@plus -1ex \@minus -.25ex}{1.25ex \@plus .25ex}{\normalfont\normalsize\bfseries}}
\makeatother

\pagestyle{fancy}% Imposta lo stile di pagina su "fancy"
\fancyhf{}% Cancella intestazioni e piè di pagina
\fancyfoot[C]{\thepage{} di \pageref{LastPage}} % Imposta il piè di pagina centrale come "numero pagina di totale pagine"
\renewcommand{\headrulewidth}{0pt} % Imposta la larghezza della linea di intestazione a 0 punti

%\newcommand{\data}{GG mese AAAA}
\newcommand{\titolo}{Specifica Tecnica}
%\newcommand{\responsabile}{Responsabile}
\newcommand{\verificatore}{
    Tommaso Zocche \\
    & Derek Gusatto \\
    & Matteo Piron
}
\newcommand{\redattore}{
    Derek Gusatto \\
    & Tommaso Zocche \\
    & Ion Bourosu \\
    & Alessandro Benin 
}
\newcommand{\uso}{Esterno}
\newcommand{\destinatari }{
   %& Vimar S.p.A.  \\
    & Tullio Vardanega  \\
    & Riccardo Cardin  }
\newcommand{\abstractcontent}{abstract ...}

\begin{document}

\input{contents/header}
\input{contents/title}
\begin{center}
\textbf{Informazioni sul documento}: \\
\vspace{0.5cm}

\begin{tabular}{r|l}
% VERBALE
%\textbf{Responsabile} &  \responsabile\\ 
\textbf{Verificatori} &  \verificatore\\ 
\textbf{Redattori} &     \redattore\\ 
    \textbf{Uso} & \uso \\ 
    \textbf{Destinatari} \destinatari \\
\end{tabular}

\vfill

% \textbf{Abstract}: \\
% \vspace{0.5cm}
% \abstractcontent
\end{center}


\bigskip
\newpage
\section*{Registro delle versioni}
\begin{table}[H]
    \centering
    \begin{tabular}{|C{1.7cm}|C{2cm}|C{3.2cm}|C{2.8cm}|L{4cm}|}
        \hline
        \textbf{Versione} &  \textbf{Data} &  \textbf{Autore} &  \textbf{Ruolo} & \textbf{Descrizione} \\
        \hline
        1.4.1 & 2025-04-05 & Matteo Gerardin & Amministratore & Correzioni minori e aggiunta dei termini del glossario \\
        \hline
        1.4.0 & 2025-04-01 & Derek Gusatto & Verificatore & Verificato paragrafo §3.2.4.2, dal paragrafo §4.3 al §4.3.5, dal paragrafo §4.6 al §4.8.3, dal paragrafo §5.1.4.4 al §5.4.3 e dal paragrafo §7 a §7.15 \\
        \hline
        1.3.1 & 2025-04-01 & Matteo Gerardin & Amministratore & Aggiunto paragrafo §3.2.4.2, dal paragrafo §4.3 al §4.3.5, dal paragrafo §4.6 al §4.8.3, dal paragrafo §5.1.4.4 al §5.4.3 e dal paragrafo §7 a §7.15 \\
        \hline
        1.3.0 & 2025-03-31 & Derek Gusatto & Verificatore & Verifica versione 1.2.1 \\
        \hline
        1.2.1 & 2025-03-29 & Matteo Gerardin & Amministratore & Aggiunta da paragrafo §3.2.4 a §3.2.7.4 e da paragrafo §4.3 a §4.4.3 \\
        \hline
        1.2.0 & 2025-03-26 & Derek Gusatto & Verificatore & Verifica versione 1.1.1 \\
        \hline
        1.1.1 & 2025-03-12 & Matteo Gerardin & Amministratore & Aggiunta da paragrafo §3.2.3.6 a §3.2.3.9 \\
        \hline
        1.1.0 & 2025-03-09 & Davide Martinelli & Verificatore & Verifica paragrafi modificati nella versione 1.0.1 \\
        \hline
        1.0.1 & 2025-03-06 & Matteo Gerardin & Amministratore & Modifica §4.1.4 Ciclo di vita dei documenti e aggiunta da §3.2.2.5 a §3.2.3.5.2 \\
        \hline
        1.0.0 & 2025-01-26 & Derek Gusatto & Responsabile & Approvazione e rilascio \\
        \hline
        0.4.0 & 2025-01-26 & Davide Martinelli & Amministratore & Verifica intero documento\\
        \hline
        0.3.1 & 2025-01-23 & Matteo Gerardin & Amministratore & Inserimento redattori e verificatori nella prima pagina\\
        \hline
        0.3.0 & 2025-01-15 & Derek Gusatto & Verificatore & Verifica paragrafi aggiunti da V0.2.0\\
        \hline
        0.2.2 & 2025-01-11 & Matteo Gerardin & Amministratore & Configuration Control (fino a §4.2.5.2)\\
        \hline
        0.2.1 & 2024-12-28 & Matteo Gerardin & Amministratore & Inizio §4.2.2 Configuration Control (fino a §4.2.2.3.5)\\
        \hline
        0.2.0 & 2024-12-17 & Tommaso Zocche & Verificatore & Verifica parziale\\
        \hline
        0.1.5 & 2024-12-17 & Tommaso Zocche & Verificatore & Correzioni ``strutturali" e typo \\
        \hline
        0.1.4 & 2024-12-5 & Matteo Piron & Amministratore & §6 Metriche di qualità \\
        \hline
        0.1.3 & 2024-11-25 & Derek Gusatto & Amministratore & §4.2 Configuration Management (struttura e scopo)\\
        \hline
    \end{tabular}
\end{table}
\begin{table}[H]
    \centering
    \begin{tabular}{|C{1.7cm}|C{2cm}|C{3.2cm}|C{2.8cm}|L{4cm}|}
        \hline
        \textbf{Versione} &  \textbf{Data} &  \textbf{Autore} &  \textbf{Ruolo} & \textbf{Descrizione} \\
        \hline
        0.1.2 & 2024-11-22 & Derek Gusatto & Amministratore & §3.2.2 Analisi dei requisiti\\
        \hline
        0.1.1 & 2024-11-20 & Derek Gusatto & Amministratore & §3.1 Fornitura\\
        \hline
        0.1.0 & 2024-11-19 & Alessandro Benin & Verificatore & Verifica parziale\\
        \hline
        0.0.3 & 2024-11-17 & Derek Gusatto & Amministratore & §5.1 Gestione organizzativa\\
        \hline
        0.0.2 & 2024-11-16 & Derek Gusatto & Amministratore & §4.1 Documentazione\\
        \hline
        0.0.1 & 2024-11-14 & Derek Gusatto & Amministratore & §1 Introduzione,  §2 Standard ISO/IEC 12207:1195\\
        \hline
    \end{tabular}
\end{table}
\newpage
\tableofcontents
\newpage
% VERBALE
% \section{Informazioni generali}
\begin{itemize}
  \item \textbf{Tipo riunione}: esterna;
  \item \textbf{Luogo}: Vimar S.p.A. - Ricerca e Sviluppo;
  \item \textbf{Data}: 2025-01-15;
  \item \textbf{Ora inizio}: 14:00;
  \item \textbf{Ora fine}: 15:30;
  
  \item \textbf{Presenti}:
  \begin{itemize}
    \item Alessandro Benin
    \item Ion Bourosu
    \item Matteo Gerardin
    \item Derek Gusatto
    \item Davide Martinelli
    \item Matteo Piron
    \item Tommaso Zocche
    \item[$\star$] Alberto Pomella (Vimar S.p.A.)
    \item[$\star$] Alessandro Alzetta (Vimar S.p.A.)
    \item[$\star$] Mariano Sciacco (Vimar S.p.A.)
    \item[$\star$] Francesca Stival (Vimar S.p.A.)
  \end{itemize}

  \item \textbf{Assenti}:
 
\end{itemize}
% \newpage
% \section{Riassunto della riunione}
Il gruppo ha indetto questa riunione per discutere l'avanzamento delle attività in corso d'opera:
\begin{itemize}
    \item \textbf{Correzione Analisi dei Requisiti}: Davide Martinelli e Matteo Gerardin riportano che il documento sarà completato entro la giornata, così da poter procedere con verifica e validazione in data 2025-02-28, con successivo invio del documento corretto al prof. Cardin (come richiesto durante la revisione RTB);
    \item \textbf{Test per il Proponente}: in data 2025-02-26 il Proponente Vimar S.p.A. ha richiesto dei test sui modelli LLM ed embedding e dell'algoritmo di RAG scelto. Ci stanno lavorando Tommazo Zocche e Matteo Piron, che prevedono di poter avere i primi risultati dei suddetti test al massimo entro il 2025-03-01, anche per poterli inviare al Proponente in tempi utili per il SAL del 2025-01-04;
    \item \textbf{Specifica Tecnica}: i restanti membri del team si stanno impegnando nella stesura del documento Specifica Tecnica. La sezione §Tecnologie sarà terminata entro le 12:00 del 2025-02-28, per procedere alla successiva verifica. Successivamente i membri disponibili si impegneranno nella sezione §Architettura di sistema da presentale al successivo SAL del 2025-01-04.

\end{itemize}
% \newpage
% \section{Todo}
Durante la riunione sono emersi i seguenti task da svolgere.

\begin{center}
  \begin{tabular}{|L{5cm}|L{8cm}|}
    \hline
    \textbf{Assegnatario} & \textbf{Task Todo} \\ \hline
    Derek Gusatto &  \textit{PdQ} - Cruscotto delle metriche \\ \hline
    Derek Gusatto &  \textit{PdP} - Aggiornamento\\ \hline
    Matteo Gerardin &  Terminazione \textit{Glossario} \\ \hline
    I. Bourosu, A. Benin &  Terminazione e messa a punto del \textit{PoC} \\ \hline
    \textit{autoassegnazione} &  Verifica \textit{PdP} \\ \hline
    \textit{autoassegnazione} &  Verifica \textit{Glossario} \\ \hline
    \textit{autoassegnazione} &  Verifica \textit{AdR} \\ \hline
    \textit{autoassegnazione} &  Verifica \textit{PdQ} \\ \hline
    Derek Gusatto &  \textit{Verbale Interno} 2025-01-21\\ \hline
  \end{tabular}
\end{center}

%LISTA FIGURE
\listoffigures 
\newpage
%LISTA TABELLE
\listoftables
\newpage

\section{Introduzione}
\subsection{Scopo del documento}
Il presente documento ha l’obiettivo di definire in maniera esaustiva e comprensibile gli aspetti tecnici chiave del progetto "Vimar GENIALE". Il fine principale di questsa risorsa è fornire una descrizione dettagliata e approfondita dell'architettura\textsubscript{G} implementativa del sistema\textsubscript{G}. \\
Nel contesto dell'architettura\textsubscript{G} è prevista un'analisi approfondita che si estenda anche al livello di design più basso, includendo la definizione e la spiegazione dei design pattern\textsubscript{G} e dei linguaggi usati nel progetto.
Gli obiettivi del presente documento sono tre: motivare le scelte di sviluppo adottate, fornire al gruppo una guida fondamentale per l'attività di codifica e monitorare la copertura dei requisiti identificari nel documento Analisi dei Requisiti V2.0.0.

\subsection{Scopo del prodotto}
Il progetto ``Vimar GENIALE" mira a sviluppare un'applicazione intelligente che supporti installatori elettrici nell'uso di dispositivi \textit{Vimar}\textsubscript{G}, facilitando l'accesso alle informazioni tecniche sui prodotti, rispondendo a domande poste in linguaggio naturale.
La tecnologia alla base prevede l'uso di modelli di \textit{LLM}\textsubscript{G} e di tecniche \textit{RAG}\textsubscript{G}, con una struttura di gestione basata su \textit{container}\textsubscript{G} e integrata in un ambiente \textit{cloud}\textsubscript{G}.
Il sistema include tre componenti principali: una \textit{applicativo web responsive}\textsubscript{G}, un \textit{applicativo server}\textsubscript{G} e un'\textit{infrastruttura cloud-ready}\textsubscript{G}. 
\subsection{Glossario}
Per evitare ambiguità relative al linguaggio utilizzato nei documenti, viene fornito il Glossario V2.0.0, nel quale si possono trovare tutte le definizioni di termini che hanno un significato specifico che vuole essere disambiguato. Tali termini sono marcati con una G a pedice. 
\subsection{Riferimenti}
\subsubsection{Riferimenti normativi}
\begin{itemize}
    \item Regolamento del progetto didattico\\ \href{https://www.math.unipd.it/~tullio/IS-1/2024/Dispense/PD1.pdf}{https://www.math.unipd.it/~tullio/IS-1/2024/Dispense/PD1.pdf} \\ (Ultimo accesso 2024-11-14)
    \item ISO/IEC 12207:1995 Information technology - Software life cycle processes \\ \href{https://www.math.unipd.it/~tullio/IS-1/2010/Approfondimenti/A03.pdf}{https://www.math.unipd.it/~tullio/IS-1/2010/Approfondimenti/A03.pdf}\\ (Ultimo accesso 2024-11-14)

\end{itemize}

\subsubsection{Riferimenti informativi}
\begin{itemize}
    \item Capitolato C2 \\ \href{https://www.math.unipd.it/~tullio/IS-1/2024/Dispense/PD1.pdf}{https://www.math.unipd.it/~tullio/IS-1/2024/Dispense/PD1.pdf}\\ (Ultimo accesso 2024-11-14)
    \item Capitolato C2 - slides \\ \href{https://www.math.unipd.it/~tullio/IS-1/2024/Dispense/PD1.pdf}{https://www.math.unipd.it/~tullio/IS-1/2024/Dispense/PD1.pdf}\\ (Ultimo accesso 2024-11-14)
    \item Documentazione\textsubscript{G} GitHub\textsubscript{G} \\ \href{https://docs.github.com/en}{https://docs.github.com/en}\\ (Ultimo accesso 2024-11-14)
    
\end{itemize}
\newpage
\section{Tecnologie }
In questa sezione si espone una panoramica delle tecnologie adottate. Si ha la descrizione delle tecnologie e dei linguaggi di programmazione utilizzati, delle librerie\textsubscript{G} e dei framework\textsubscript{G} necessari, oltre che delle infrastrutture realizzate. 
\subsection{Docker}
Docker\textsubscript{G} è una piattaforma \textit{open-source} che automatizza la distribuzione, la scalabilità e l’isolamento delle applicazioni utilizzando la virtualizzazione a livello di sistema operativo. È stato scelto per la sua efficienza e portabilità. Rispetto ad altre soluzioni (e.g. Vagrant), Docker offre una maggiore efficienza e facilità d’uso.
\subsubsection{Vantaggi}
\begin{itemize}
    \item primo vantaggio
\end{itemize}
\subsubsection{Svantaggi}
\begin{itemize}
    \item primo svantaggio
\end{itemize}
\subsection{Flask}
Flask\textsubscript{G} è un micro-framework web per Python, progettato per facilitare lo sviluppo di applicazioni web in modo semplice e veloce. Non richiede strumenti o librerie particolari per funzionare e non impone una struttura rigida al progetto. Abbiamo deciso di utilizzarlo per implementare un'API RESTful\textsubscript{G}, che segue i prinipi REST\textsubscript{G}, per gestire sessioni, conversazioni, messaggi e feedback, integrando anche funzionalità di intelligenza artificiale per generare risposte a domande attraverso un modello di linguaggio LLM\textsubscript{G} e un sistema di embedding\textsubscript{G}. Nel nostro sistema, Flask è la parte del back-end\textsubscript{G} che permette la connessione con un front-end\textsubscript{G} per la gestione delle interazioni con l'utente, grazie a un sistema di routing\textsubscript{G} che definisce delle rotte per gestire le richieste HTTP\textsubscript{G} (GET, POST, PUT e DELETE). 
\subsubsection{Vantaggi}
\begin{itemize}
    \item Non impone una struttura rigida, permettendo al team di sviluppo di organizzare il codice senza troppi vincoli e di integrare librerie strettamente necessarie;
    \item È semplice, quindi adatto a che si interfaccia per la prima volta allo sviluppo web con Python\textsubscript{G};
    \item L'integrazione con tecnologie avanzate come LLM\textsubscript{G} e sistemi di embedding\textsubscript{G}, dimostra la sua capacità di supportare soluzioni complesse.
\end{itemize}
\subsubsection{Svantaggi}
\begin{itemize}
    \item La flessibilità d'altro canto può risultare un rallentamento, essendo che manca la standardizzazione nel codice, specialmente in un contesto in cui il team è inesperto.
\end{itemize}

\subsection{Scrapy}
Scrapy\textsubscript{G} è un framework\textsubscript{G} open-source per Python\textsubscript{G} progettato specificamente per il web scraping\textsubscript{G}, ovvero l'estrazione di dati da siti web, nel nostro caso dal sito di Vimar. È basato su un'architettura asincrona\textsubscript{G}, che lo rende adatto per il crawling\textsubscript{G} di grandi volumi di dati. Per il nostro progetto abbiamo deciso di adottare un approccio che vede lo scraping come componente a sè stante. Dunque, l'inserimento dei dati avviene da un file JSON\textsubscript{G} dove sono stati caricati e aggiornati precedentemente.
\subsubsection{Vantaggi}
\begin{itemize}
    \item L'architettura asincrona ci permette di gestire un grande numero di richieste contemporaneamente;
    \item La seprazione fra la gestione delle richieste, l'elaborazione dei dati e il salvataggio dei risultati facilitano lo sviluppo e la manutenzione del codice;
    \item Permette di salvare i dati in formati diversi (JSON\textsubscript{G}, CSV\textsubscript{G} o XML\textsubscript{G}) e di integrarli facilmente con database o altre applicazioni.
\end{itemize}
\subsubsection{Svantaggi}
\begin{itemize}
    \item La scarsa esperienza nell'ambito del web scraping\textsubscript{G} può rallentare il processo, richiedendo più tempo per l'apprendimento di tale tecnologia;
    \item Si possono incotrare rallentamenti nel caso di politiche di crawling\textsubscript{G} specifiche o blocchi da parte dei siti web.
\end{itemize}

\subsection{Ollama}
Ollama\textsubscript{G} è una piattaforma leggera ed efficace per eseguire modelli di intelligenza artificiale in locale, con un'architettura ottimizzata che garantisce prestazioni elevate con un utilizzo ridotto di risorse. È più semplice e intuitiva rispetto ad LLM Studio, risultando ideale per il nostro progetto.
\subsubsection{Vantaggi}
\begin{itemize}
    \item Più leggero ed efficiente, consumando meno risorse e offrendo migliori prestazioni su hardware\textsubscript{G} standard;
    \item Interfaccia\textsubscript{G} più semplice e intuitiva, facile da usare, anche senza configurazioni complesse.
\end{itemize}
\subsubsection{Svantaggi}
\begin{itemize}
    \item Meno opzioni avanzate di personalizzazione per l'addestramento;
    \item Meno strumenti integrati per il monitoraggio e l'analisi delle prestazioni del modello.
\end{itemize}

\subsection{PostgreSQL}

\subsection{Angular}
Angular\textsubscript{G} è un framework open-source per lo sviluppo di applicazioni web single-page (SPA\textsubscript{G}). È basato su TypeScript\textsubscript{G}, un superset di JavaScript\textsubscript{G}. Nel nostro sistema è utilizzato per il front-end\textsubscript{G}, permettendo una gestione efficiente delle interazioni utente e una comunicazione fluida con il back-end\textsubscript{G} tramite API RESTful\textsubscript{G}. Si tratta di una comunicazione asincrona\textsubscript{G} (di default), grazie all'uso di RxJS\textsubscript{G} e degli Observables\textsubscript{G}. Quest'approccio sfrutta il pattern reattivo\textsubscript{G} per gestire le oprazioni come le richieste HTTP\textsubscript{G}. 
\subsubsection{Vantaggi}
\begin{itemize}
    \item Utilizza un sistema di moduli e componenti che favorisce il riciclo del codice e una struttura organizzata;
    \item Si basa sul two-way of data binding\textsubscript{G}, ovvero la sincronizzazione automatica dei dati tra modello\textsubscript{G} e vista\textsubscript{G};
    \item Grazie all'utilizzo di Angular CLI\textsubscript{G} alcune attività come la creazione del progetto e la generazione delle componenti, vengono automatizzate;
    \item Si integra facilmente con il back-end tramite sevizi HTTP, redendolo un'ottima scelta per applicazioni che richiedono una comunicazione costante con il server.
\end{itemize}
\subsubsection{Svantaggi}
\begin{itemize}
    \item Richiede molto tempo per l'apprendimento a causa della sua complessità e della necessità di imparare concetti come moduli, servizi e dependency injection\textsubscript{G};
    \item Le app sviluppate con Angular possono richiedere molto spazio;
    \item Il ciclo di aggiornamente è molto frequente perciò è possibile che ci sia la necessità di intervenire per mantenrlo compatibile alle nuove versioni.
\end{itemize}

\subsection{Linguaggi e formato dati}
\subsubsection{Python}

\subsubsection{SQL}
SQL\textsubscript{G} è un linguaggio standard per la gestione di database relazionali, ideale per organizzare e interrogare dati strutturati in modo efficiente. Abbiamo scelto SQL\textsubscript{G} rispetto a NoSQL\textsubscript{G} perché garantisce affidabilità, coerenza e query avanzate, fondamentali per il nostro progetto.
\subsubsection{Vantaggi}
\begin{itemize}
    \item Strutturato e organizzato, Ideale per dati relazionali con schemi ben definiti;
    \item Query\textsubscript{G} potenti e precise, grazie a JOIN\textsubscript{G}, filtri e funzioni avanzate;
    \item Ampio supporto, Usato in molti database\textsubscript{G}.
\end{itemize}
\subsubsection{Svantaggi}
\begin{itemize}
    \item Meno flessibile con dati non strutturati, NoSQL\textsubscript{G} è più adatto per dati dinamici e documentali;
    \item Scalabilità orizzontale complessa, NoSQL\textsubscript{G} può essere più efficiente per sistemi distribuiti su larga scala;
    \item Rigidità nella modifica dello schema, cambiare la struttura del database\textsubscript{G} può essere complicato in ambienti SQL\textsubscript{G}.
\end{itemize}

\subsubsection{YAML}
YAML\textsubscript{G} è un formato di serializzazione dati leggibile dall’uomo, utilizzato principalmente per configurazioni e automazione, come nei file Docker Compose\textsubscript{G}. La sua sintassi basata sull’indentazione lo rende più leggibile rispetto ad altri formati, ma richiede attenzione alla formattazione.
\subsubsection{Vantaggi}
\begin{itemize}
    \item Maggiore leggibilità grazie alla sintassi senza parentesi e virgolette;
    \item Supporta commenti, facilitando la documentazione delle configurazioni;
    \item Permette riferimenti e ancoraggi per riutilizzare parti del file\textsubscript{G}.
\end{itemize}
\subsubsection{Svantaggi}
\begin{itemize}
    \item Sensibile all’indentazione, aumentando il rischio di errori difficili da individuare;
    \item Parsing\textsubscript{G} più lento rispetto ad altri formati come JSON\textsubscript{G};
    \item Meno diffuso per l’interscambio dati rispetto ad altri standard.
\end{itemize}

\subsubsection{JSON}
JSON\textsubscript{G} è un formato leggero per lo scambio di dati tra applicazioni, ampiamente utilizzato nelle API\textsubscript{G} e nella comunicazione tra servizi. La sua sintassi basata su coppie chiave-valore e array\textsubscript{G} lo rende facile da elaborare e compatibile con la maggior parte dei linguaggi di programmazione.
\begin{itemize}
    \item Struttura semplice e facilmente interpretabile da macchine e sviluppatori;
    \item Parsing\textsubscript{G} più lento rispetto ad altri formati come JSON;
    \item Standard per lo scambio di dati tra sistemi diversi.
\end{itemize}
\subsubsection{Svantaggi}
\begin{itemize}
    \item Meno leggibile rispetto a YAML\textsubscript{G}, specialmente in configurazioni complesse;
    \item Non supporta commenti, rendendo più difficile documentare il codice;
    \item Più rigido nella sintassi, con obbligo di virgolette e parentesi.
\end{itemize}
\newpage


\section{Architettura di sistema }
\subsection{Modello Architetturale}
Il progetto Vimar GENIALE rappresenta un’applicazione moderna e avanzata per la gestione di conversazioni intelligenti, sviluppata con un’architettura distribuita e tecnologie all’avanguardia. La sua struttura è pensata per garantire flessibilità, scalabilità e manutenibilità nel tempo, integrando diversi componenti che collaborano per offrire un’esperienza fluida ed efficiente.Per la progettazione del software, si possono adottare due diverse strategie architetturali: l’architettura a strati e l’architettura esagonale.
\begin{itemize}
\item \textbf{L’architettura a strati} è paragonabile a un edificio tradizionale, in cui ogni piano rappresenta un livello ben definito e la comunicazione avviene esclusivamente tra piani adiacenti. Solitamente, si suddivide in:

Presentation Layer: l’interfaccia utente che interagisce con gli utenti.
Business Layer: il livello che contiene la logica di business e le regole applicative.
Persistence Layer: lo strato che si occupa della gestione e dell’accesso ai dati.
Questa organizzazione offre una struttura semplice e ben definita, ideale per applicazioni meno complesse e con pochi collegamenti esterni. Tuttavia, nel caso di un progetto che deve gestire molte integrazioni e garantire una forte modularità, questa soluzione può risultare rigida e difficile da modificare senza impattare altri livelli.
\end{itemize}
\begin{itemize}
\item \textbf{L’architettura esagonale}, invece, è pensata per rendere il sistema più flessibile e indipendente dalle sue dipendenze esterne. Immaginiamola come un hub centrale con diverse porte di accesso: il cuore dell’applicazione (core) è isolato e protetto, mentre tutti i servizi esterni, come il database o il motore AI, si collegano attraverso interfacce ben definite, chiamate ports. Queste porte vengono poi implementate concretamente dagli adapters, che gestiscono le interazioni con il mondo esterno.

Un esempio pratico di questa architettura può essere visto come un aeroporto, dove ogni compagnia aerea ha un terminale dedicato, ma tutti condividono la stessa infrastruttura centrale. Questo permette di mantenere un’organizzazione chiara e modulare, evitando che modifiche a un componente possano impattare il resto del sistema.
\end{itemize}

L’\textbf{architettura esagonale} è la scelta ideale per \textbf{ Vimar GENIALE} perché garantisce una maggiore flessibilità e modularità rispetto all’\textbf{architettura a strati}. La sua capacità di isolare le dipendenze esterne, migliorare la testabilità, facilitare la scalabilità e mantenere il codice ben organizzato la rende perfetta per un progetto che integra tecnologie avanzate e che potrebbe evolversi nel tempo.
\begin{figure}[H]
    \centering
    \includegraphics[width=\textwidth]{Esagonale.png}
    \caption{Schema architetturale del sistema}
    \label{fig:architettura}
\end{figure}
    \item \textbf{Data Extraction & Indexing (Scrapy e Nomic)}: raccoglie dati dal web (es. Vimar.com), li elabora e li indicizza, rendendoli facilmente consultabili dal database e dall'AI.
    \item \textbf{AI (Meta Llama 3)}: utilizzato per elaborazioni avanzate, supporta il sistema nell'analisi dei dati, nella generazione di risposte intelligenti e nell'arricchimento delle informazioni estratte.
    \item \textbf{Database (PostgreSQL)}: memorizza i dati raccolti ed elaborati, garantendo un accesso rapido ed efficiente tramite il backend.
    \item \textbf{Backend (API Flask e Python)}: funge da intermediario tra il frontend e gli altri componenti, gestendo le richieste dell’utente, eseguendo la logica di business e comunicando con il database e i servizi AI.
    \item \textbf{Frontend (WebApp)}: sviluppato in Angular, permette agli utenti di interagire con il sistema attraverso un’interfaccia intuitiva, inviando richieste e visualizzando i dati elaborati.
    \subsection{Modello Architetturale}
    \item Il diagramma illustra il flusso dei dati all'interno del sistema, evidenziando le varie fasi di acquisizione, elaborazione, archiviazione e visualizzazione. Mostra le entità coinvolte e le loro interconnessioni, offrendo una visione chiara di come le informazioni vengono gestite e trasformate lungo il processo.
    \item immagine
    
    
    
%\newpage
%\section{Architettura delle compomenenti }


Il sistema è un'applicazione modulare basata su un'architettura Docker a microservizi\textsubscript{G}. Ogni componente è eseguito in un container separato, mantenendo indipendenza e flessibilità, garantendo al contempo un ambiente isolato e sicuro.

\subsection{Descrizione dell'Architettura}

L'architettura si compone di quattro container principali, ciascuno con un compito ben definito:

\begin{itemize}
    \item \textbf{Container App (Flask)}
    \begin{itemize}
        \item Espone l'applicazione Flask\textsubscript{G} sulla porta 5001.
        \item Include un healthcheck che verifica la disponibilità del servizio Ollama\textsubscript{G}.
        \item Ha accesso ai seguenti volumi:
        \begin{itemize}
            \item Directory dell'applicazione
            \item Modelli Ollama\textsubscript{G}
            \item Script di avvio
            \item Requisiti dell'applicazione
            \item Volume dati condiviso
        \end{itemize}
        \item Dipende dal database\textsubscript{G} per il suo funzionamento.
    \end{itemize}

    \item \textbf{Container Database (PostgreSQL con pgvector)}
    \begin{itemize}
        \item Utilizza PostgreSQL\textsubscript{G} con l'estensione \texttt{pgvector} per la gestione dei vettori.
        \item Espone la porta 54321 (mappata internamente alla porta 5432).
        \item Credenziali di default:
        \begin{itemize}
            \item \texttt{Utente: postgres}
            \item \texttt{Password: pebkac}
            \item \texttt{Database: postgres}
        \end{itemize}
        \item Healthcheck per verificare la disponibilità del database.
        \item Dati persistenti tramite il volume \texttt{postgres\_data}.
        \item Include l'estensione \texttt{vector} preinstallata.
    \end{itemize}

    \item \textbf{Container Angular}
    \begin{itemize}
        \item Ospita l'applicazione frontend Angular.
        \item Espone la porta 4200.
        \item Collegato alla rete \texttt{app\_network}.
    \end{itemize}

    \item \textbf{Container Data Processing}
    \begin{itemize}
        \item Gestisce l'elaborazione dei dati.
        \item Accesso alla directory \texttt{data\_processing}\textsubscript{G} tramite un volume dedicato.
        \item Dipende sia dal database che dall'applicazione principale.
    \end{itemize}
\end{itemize}

\subsection{Rete e Persistenza dei Dati}

Tutti i container\textsubscript{G} sono collegati tramite una rete bridge chiamata \texttt{app\_network}, che permette loro di comunicare in modo sicuro e isolato. La persistenza dei dati è garantita da due volumi:

\begin{itemize}
    \item \texttt{postgres\_data}: Memorizza i dati del database.
    \item \texttt{data\_volume}: Contiene i dati condivisi dell'applicazione.
\end{itemize}

\subsection{Diagramma dell'Architettura}
\begin{figure}[H]
    \centering
    \includegraphics[width=\textwidth]{images/esagonale.png}
    \caption{Schema dell'architettura Docker}
\end{figure}

\subsection{Resilienza e Manutenzione}

Questa architettura offre numerosi vantaggi:
\begin{itemize}
    \item \textbf{Scalabilità:} Ogni componente può essere scalato indipendentemente in base al carico.
    \item \textbf{Isolamento dei Servizi:} Un problema in un container\textsubscript{G} non influenza gli altri.
    \item \textbf{Portabilità:} Facilità di distribuzione in ambienti diversi.
    \item \textbf{Resilienza:} I volumi persistenti assicurano che i dati non vadano persi anche in caso di riavvio dei container\textsubscript{G}.
\end{itemize}

\subsection{Conclusione}

L'architettura Docker\textsubscript{G} proposta è flessibile e scalabile. La separazione dei componenti garantisce sicurezza e affidabilità, mentre l'utilizzo dei container\textsubscript{G} facilita il deployment\textsubscript{G} e la manutenzione del sistema.
\newpage
\section{Tracciamento dei requisiti}
 Si riportano i requisiti funzionali della tabella
presente nel documento Analisi dei Requisiti v2.0.0 - Sez. Req. Funzionali, qui è presente una
colonna indicante la soddisfazione di tale requisito.

\subsection{Tabella dei requisiti funzionali}
\begin{table}[H]
\centering
    \begin{tabular}{|C{2.7cm}|L{6.5cm}|C{2.7cm}|C{2.7cm}|}
        \hline
        \textbf{ID requisito} & \textbf{Descrizione} & \textbf{Importanza} & \textbf{Stato}  \\
        \hline
        RF.O.001 & Il \textit{sistema}\textsubscript{G} deve permettere all'installatore di effettuare ricerche testuali e ricevere informazioni dettagliate sui prodotti \textit{Vimar}\textsubscript{G}. & Obbligatorio & Soddisfatto \\
        \hline
        RF.O.002 & Il \textit{sistema}\textsubscript{G} deve prevedere l'autenticazione tramite \textit{password}\textsubscript{G} per l'accesso alla \textit{dashboard}\textsubscript{G} per amministratori. & Obbligatorio & Soddisfatto \\
        \hline
        RF.O.003 & Il \textit{cruscotto informativo}\textsubscript{G} deve includere una sezione per la visualizzazione di statistiche di utilizzo. & Obbligatorio & Soddisfatto \\
        \hline
        RF.O.004 & Il \textit{sistema}\textsubscript{G} deve permettere agli utenti di fornire un \textit{feedback}\textsubscript{G} positivo o negativo dopo ogni risposta ricevuta. & Obbligatorio & Soddisfatto \\
        \hline
        RF.O.005 & Il \textit{sistema}\textsubscript{G} deve essere in grado di identificare e bloccare le richieste che riguardano argomenti non pertinenti ai prodotti \textit{VIMAR}\textsubscript{G}. & Obbligatorio & Soddisfatto \\
        \hline
        RF.D.006 & Il \textit{sistema}\textsubscript{G} deve includere la possibilità di visualizzare \textit{link}\textsubscript{G} di riferimento alle fonti delle informazioni fornite. & Desiderabile & Non Soddisfatto \\
        \hline
        RF.P.007 & Il \textit{sistema}\textsubscript{G} deve fornire un'interfaccia con menu e sottomenu per costruire richieste specifiche in conversazioni guidate. & Opzionale & Non Soddisfatto\\
        \hline
        RF.D.008 & Il componente di interrogazione deve prevedere un controllo sull’\textit{output}\textsubscript{G}
        per verificare che il contenuto non vada in conflitto con argomenti proibiti. & Desiderabile &  Soddisfatto \\
        \hline
        \end{tabular}
    \caption{Requisiti di funzionalità (1\textsuperscript{a}  parte)}
\end{table}
\begin{table}[H]
\centering
    \begin{tabular}{|C{2.7cm}|L{6.5cm}|C{2.7cm}|C{2.7cm}|}
        \hline
        RF.O.009 & L’utente deve poter fare richieste testuali limitate a un certo numero di caratteri. & Obbligatorio & Soddisfatto \\
        \hline
        RF.O.010 & L’utente deve poter visualizzare uno storico dei messaggi nella stessa
        conversazione. & Obbligatorio & Soddisfatto \\
        \hline
        RF.D.011 & L’utente può fare richieste in italiano e/o in inglese.
         & Desiderabile & Soddisfatto \\
         \hline
         RF.O.012 & Le conversazioni avute possono essere salvate al termine della \textit{sessione}\textsubscript{G}. & Obbligatorio & Soddisfatto \\
        \hline
        RF.O.013 & Le conversazioni devono poter essere cancellate. & Obbligatorio & Soddisfatto \\
        \hline
        RF.O.014 & Ogni utente dovrà avere un limite massimo di conversazioni.
         & Obbligatorio & Soddisfatto \\
        \hline
        RF.P.015 & L'\textit{amministratore}\textsubscript{G} deve poter visualizzare nel \textit{cruscotto informativo}\textsubscript{G} il numero totale di richieste effettuate con \textit{conversazione libera}\textsubscript{G} o \textit{guidata}\textsubscript{G}.
         & Opzionale & Non Soddisfatto \\
        \hline
        RF.P.016 & L'\textit{amministratore}\textsubscript{G} deve poter visualizzare nel \textit{cruscotto informativo}\textsubscript{G} la classifica sul numero di termini usati maggiormente nelle richieste.
         & Opzionale & Non Soddisfatto \\
        \hline
        RF.D.017 & L'\textit{amministratore}\textsubscript{G} deve poter visualizzare nel \textit{cruscotto informativo}\textsubscript{G} il numero di risposte positive o negative ricevute dal sistema di \textit{feedback}\textsubscript{G}.
         & Desiderabile & Soddisfatto \\
        \hline
         RF.D.018 & L'utente deve poter scaricare i file di istruzioni dei prodotti in formato PDF.
         & Desiderabile & Non Soddisfatto \\
        \hline
        
         RF.O.019 & Il \textit{sistema}\textsubscript{G} deve essere in grado di ricavare le informazioni utili dai PDF ed estrarre le immagini degli schemi elettrici.
         & Desiderabile & Non Soddisfatto \\
        \hline
         RF.O.020 & Il \textit{sistema}\textsubscript{G} deve impiegare un \textit{database}\textsubscript{G} per collezionare le informazioni relative ai prodotti.
         & Obbligatorio & Soddisfatto \\
        \hline
        \end{tabular}
    \caption{Requisiti di funzionalità (1\textsuperscript{a}  parte)}
\end{table}
\begin{table}[H]
\centering
    \begin{tabular}{|C{2.7cm}|L{6.5cm}|C{2.7cm}|C{2.7cm}|}
        \hline
        RF.O.021 & Il componente di interrogazione deve fornire in \textit{output}\textsubscript{G} la risposta del modello \textit{AI}\textsubscript{G} (\textit{LLM}\textsubscript{G}).
         & Obbligatorio & Soddisfatto \\
        \hline
        RF.O.022 & Il modello \textit{AI}\textsubscript{G} (\textit{LLM}\textsubscript{G}) deve essere in grado di rispondere a domande sui prodotti appartenenti agli impianti Smart.
         & Obbligatorio & Soddisfatto \\
        \hline
        RF.O.023 & Il modello \textit{AI}\textsubscript{G} (\textit{LLM}\textsubscript{G}) deve essere in grado di rispondere a domande sui prodotti appartenenti agli impianti Domotici.
         & Obbligatorio & Soddisfatto \\
         \hline
        RF.D.024 & Il modello \textit{AI}\textsubscript{G} (\textit{LLM}\textsubscript{G}) deve essere in grado di rispondere a domande sui prodotti appartenenti agli impianti Tradizionali.
         & Desiderabile & Non Soddisfatto \\
         \hline
        RF.D.025 & L’utente deve poter visualizzare il numero della pagina del documento relativo alla risposta.
        & Desiderabile & Non Soddisfatto \\
        \hline
        RF.D.026 & L’utente deve poter visualizzare il nome del documento relativo alla
        risposta.
         & Desiderabile & Non Soddisfatto \\
        \hline
        RF.O.027 & L’utente deve poter confermare l'eliminazione di una conversazione con il \textit{chatbot}\textsubscript{G}.
         & Obbligatorio & Soddisfatto \\
        \hline

        RF.O.028 & L’utente deve poter visualizzare una lista delle
        \textit{sessioni}\textsubscript{G} di conversazione attive col \textit{chatbot}\textsubscript{G}.
         & Obbligatorio & Soddisfatto \\
        \hline
        RF.O.029 & L’utente deve poter creare una nuova \textit{sessione}\textsubscript{G} di conversazione col \textit{chatbot}\textsubscript{G}.
         & Obbligatorio & Soddisfatto \\
        \hline
        RF.O.030 & L’utente deve ricevere una risposta di cortesia quando pone una domanda relativa a dei contenuti proibiti.
         & Obbligatorio & Soddisfatto \\
        \hline
        RF.O.031 & L'utente deve essere notificato da un errore se non ci sono informazioni sul prodotto ricercato. & Obbligatorio & Soddisfatto \\
        \hline
        RF.O.032 & L’utente deve poter digitare la domanda da porgere al \textit{chatbot}\textsubscript{G} tramite tastiera.
         & Obbligatorio & Soddisfatto \\
        \hline
        \end{tabular}
    \caption{Requisiti di funzionalità (1\textsuperscript{a}  parte)}
\end{table}
\begin{table}[H]
\centering
    \begin{tabular}{|C{2.7cm}|L{6.5cm}|C{2.7cm}|C{2.7cm}|}
        \hline
        RF.O.033 & L’utente deve poter inviare le domande da porgere al \textit{chatbot}\textsubscript{G}.
         & Obbligatorio & Soddisfatto \\
        \hline
        RF.P.034 & L’utente deve poter visualizzare una conversazione salvata in precedenza.
         & Opzionale &  Soddisfatto \\
        \hline
        RF.P.035 & L’utente può visualizzare la data di invio di un messaggio.
         & Opzionale & Non Soddisfatto \\
        \hline
        RF.P.036 & L’utente può visualizzare l'ora di invio di un messaggio.
         & Opzionale & Non Soddisfatto \\
        \hline
        RF.P.037 & L’utente deve poter consultare il contenuto del documento di interesse.
         & Opzionale & Non Soddisfatto \\
        \hline
        RF.D.038 & L'interfaccia utente del \textit{sistema}\textsubscript{G} deve essere responsive, adattandosi a diversi dispositivi. & Desiderabile &  Soddisfatto \\
        \hline
        RF.O.039 & L'utente deve poter creare una \textit{conversazione libera}\textsubscript{G} o \textit{guidata}\textsubscript{G}. & Obbligatorio & Soddisfatto \\ \hline
        RF.O.040 & L'utente deve essere notificato da un errore quando supera il numero limite di conversazioni. & Obbligatorio & Soddisfatto \\ \hline
        RF.O.041 & L'installatore deve poter visualizzare la singola \textit{conversazione libera}\textsubscript{G}. & Obbligatorio & Soddisfatto \\ \hline 
        RF.O.042 & L'installatore deve poter visualizzare la singola \textit{conversazione guidata}\textsubscript{G}. & Obbligatorio & Soddisfatto \\ \hline
        RF.P.043 & L'installatore deve essere in grado di modellare il \textit{prompt}\textsubscript{G} tramite una serie di menù nella \textit{conversazione guidata}\textsubscript{G}. & Opzionale & Non Soddisfatto \\
        \hline
        RF.P.044 & L'installatore deve visualizzare un suggerimento sulla domanda successiva nella \textit{conversazione libera}\textsubscript{G}. & Opzionale & Non Soddisfatto \\ \hline
        RF.O.045 & L'installatore deve poter visualizzare l'area per scrivere il messaggio da inviare al \textit{sistema}\textsubscript{G}. & Obbligatorio & Soddisfatto \\ \hline
        
        RF.O.046 & L'installatore deve visualizzare data e ora sia dei messaggi inviati sia dei messaggi ricevuti. & Obbligatorio & Soddisfatto\\ \hline
        RF.O.047 & L'utente deve poter salvare in formato PDF una conversazione. & Obbligatorio & Soddisfatto \\ \hline
        \end{tabular}
    \caption{Requisiti di funzionalità (1\textsuperscript{a}  parte)}
\end{table}
\begin{table}[H]
\centering
    \begin{tabular}{|C{2.7cm}|L{6.5cm}|C{2.7cm}|C{2.7cm}|}
        \hline
        RF.O.048 & L'amministratore deve essere notificato da un errore se le credenziali inserite sono errate. & Obbligatorio & Soddisfatto \\ \hline
        RF.O.049 & L'installatore deve essere notificato dell'assenza di alcuni elementi della conversazione, non presenti a causa della mancanza di messaggi pregressi nella conversazione. & Obbligatorio & Soddisfatto \\ \hline
        RF.O.050 & L'installatore deve essere notificato dell'impossibilità di salvare una conversazione vuota. & Obbligatorio & Soddisfatto
        \\ \hline
        RF.O.051 & L'installatore deve poter identificare le conversazioni presenti nel \textit{sistema}\textsubscript{G} tramite un nome. & Obbligatorio & Soddisfatto
        \\ \hline
        RF.D.052 & L'installatore deve poter rinominare le conversazioni presenti nel \textit{sistema}\textsubscript{G}. & Desiderabile & Non Soddisfatto
        \\ \hline
    \end{tabular}
    \caption{Requisiti di funzionalità (5\textsuperscript{a}  parte)}
\end{table}


\subsection{Tabella dei requisiti Vincolo}

\begin{table}[H]
\centering
    \begin{tabular}{|C{2.7cm}|L{6.5cm}|C{2.7cm}|C{2.7cm}|}
        \hline
    \textbf{ID requisito} & \textbf{Descrizione} & \textbf{Importanza} & \textbf{Stato}  \\
    \hline
           RV.O.001 & Il \textit{sistema}\textsubscript{G} deve integrare il modello \textit{AI}\textsubscript{G} (\textit{LLM}\textsubscript{G}) \textit{Open Source}\textsubscript{G} Llama 3.1 8B. & Obbligatorio & Soddisfatto \\
          \hline 
          RV.O.002 & L’infrastruttura \textit{Cloud}\textsubscript{G} deve utilizzare \textit{Docker}\textsubscript{G} insieme a \textit{Docker}\textsubscript{G} Compose, al fine di rispettare il principio di Infrastructure as Code. & Obbligatorio & Soddisfatto \\
           \hline
          RV.D.003 & L'applicativo può essere ospitato su \textit{AWS}\textsubscript{G}. & Desiderabile & Soddisfatto \\
          \hline
          RV.O.004 & Il modello \textit{AI}\textsubscript{G} (\textit{LLM}\textsubscript{G}) deve essere \textit{Open Source}\textsubscript{G}.
         & Obbligatorio & Soddisfatto \\
        \hline
        RV.O.005 & Il componente di interrogazione deve essere in grado di interfacciarsi con il \textit{sistema}\textsubscript{G} di indicizzazione e con il modello \textit{AI}\textsubscript{G} (\textit{LLM}\textsubscript{G}).
         & Obbligatorio & Soddisfatto \\
        \hline
        RV.O.006 & Il componente di interrogazione deve poter essere contattato da un altro servizio sotto-forma di \textit{API}\textsubscript{G} autenticata (ad esempio tramite \textit{API}\textsubscript{G}-KEY)
         & Obbligatorio & Soddisfatto \\
         \hline
        RV.O.007 &  L’infrastruttura deve utilizzare la tecnologia dei \textit{container}\textsubscript{G}.
         & Obbligatorio & Soddisfatto \\
        \hline
         RV.O.008 & Il risultato atteso è che la parte applicativa possa essere costruita e replicata con un solo comando.
         & Obbligatorio & Soddisfatto \\
        \hline
        RV.O.009 & Il \textit{repository}\textsubscript{G} di lavoro deve essere versionato tramite Git e deve essere pubblicamente accessibile.
         & Obbligatorio & Soddisfatto \\
         \hline
        RV.O.010 & La licenza per i sorgenti dovrà essere \textit{Open Source}\textsubscript{G}.
         & Obbligatorio & Soddisfatto \\
        \hline
        RV.O.011 & Il modello \textit{AI}\textsubscript{G} (\textit{LLM}\textsubscript{G}) dovrà fare uso dell’approccio \textit{RAG}\textsubscript{G}.
         & Obbligatorio & Soddisfatto \\
        \hline
        RV.O.012 & L’applicazione deve essere compatibile con il browser Chrome dalla
        versione 108.
         & Obbligatorio & Soddisfatto \\
         \hline
        RV.O.013 & L’applicazione deve essere compatibile con il browser Edge dalla versione 94.0.992.31.
         & Obbligatorio & Soddisfatto \\
        \hline
        RV.O.014 & L’applicazione deve essere compatibile con il browser Opera dalla
        versione 95.
         & Obbligatorio & Soddisfatto \\
         \hline
    \end{tabular}
    \caption{Requisiti di vincolo (1\textsuperscript{a}  parte)}
\end{table}
\begin{table}[H]
\centering
    \begin{tabular}{|C{2.7cm}|L{6.5cm}|C{2.7cm}|C{2.7cm}|}
        \hline
        RV.O.015 & L’applicazione deve essere compatibile con il browser Firefox dalla
versione 109.
         & Obbligatorio & Soddisfatto \\
        \hline
        RV.O.016 & L’applicazione deve essere compatibile con il browser Safari dalla
versione 16.
         & Obbligatorio & Soddisfatto \\
          \hline
        RV.O.017 &  L’applicativo deve prevedere un \textit{sistema}\textsubscript{G} di estrazione e raccolta delle informazioni dal sito web dell'azienda.
         & Obbligatorio & Soddisfatto \\
         \hline
         RV.O.018 & Il \textit{sistema}\textsubscript{G} deve essere in grado di navigare un elenco di prodotti, estrarre le informazioni utili e immagazzinare le informazioni correlandole in modo opportuno.
         & Obbligatorio & Soddisfatto \\
         \hline
         RV.O.019 & Il \textit{sistema}\textsubscript{G} di estrazione e raccolta deve essere realizzato sotto-forma di \textit{pipeline}\textsubscript{G} e automatizzato.
         & Obbligatorio & Soddisfatto \\
        \hline
         RV.O.020 & L’applicativo deve prevedere un \textit{sistema}\textsubscript{G} di indicizzazione delle informazioni a partire dal
        \textit{database}\textsubscript{G} in cui sono stati salvati i dati precedentemente estratti dal sito web.
         & Obbligatorio & Soddisfatto \\
        \hline
    \end{tabular}
    \caption{Requisiti di vincolo (2\textsuperscript{a}  parte)}
\end{table}


\subsection{Tabella dei requisiti Qualità}
\begin{table}[H]
\centering
    \begin{tabular}{|C{2.7cm}|L{6.5cm}|C{2.7cm}|C{2.7cm}|}
        \hline
        \textbf{ID requisito} & \textbf{Descrizione} & \textbf{Importanza} & \textbf{Stato}  \\
        \hline
        
        \hline
        RQ.O.001 & \`E necessario fornire un documento che descriva le attività di \textit{bug}\textsubscript{G} reporting effettuate. & Obbligatorio & Soddisfatto \\
        \hline
        RQ.O.002 & Il progetto deve essere svolto seguendo le regole contenute nel documento Norme di Progetto. & Obbligatorio & Soddisfatto \\
        \hline
        RQ.O.003 & \`E necessario fornire al proponente il codice sorgente dell'applicativo in un
        \textit{repository}\textsubscript{G}
        \textit{GitHub}\textsubscript{G}. & Obbligatorio & Soddisfatto \\
        \hline
        RQ.O.004 & \`E necessario fornire il Manuale Utente dell'applicativo. & Obbligatorio & Soddisfatto \\
        \hline
    \end{tabular}
    \caption{Requisiti di qualità}
\end{table}

\newpage

\subsection{Grafici requisiti soddisfatti}
Riguardo alla soddisfazione dei requisiti il gruppo PEBKAC ha soddisfatto 61 su 76,
arrivando ad una copertura del 80\%
\begin{figure}[H]
    \centering
    \includegraphics[width=\textwidth]{images/requisiti_tot.png}
    \caption{Schema architetturale del backend}
    \label{fig:Requisiti Totali}
\end{figure}
\newpage
Per quanto riguarda la copertura dei requisiti obbligatori, la copertura rilevata è di 55 su 55
requisiti, arrivando quindi ad un 100\% sul totale

\begin{figure}[H]
    \centering
    \includegraphics[width=\textwidth]{images/requisiti_o.png}
    \caption{Schema Requisiti Obbligatori}
    \label{fig:Requisiti Obbligatori}
\end{figure}

\newpage

Per quanto riguarda la copertura dei requisiti Desiderabili, la copertura rilevata è di 5 su 12
requisiti, arrivando quindi ad un 41\% sul totale
\begin{figure}[H]
    \centering
    \includegraphics[width=\textwidth]{images/requisiti_d.png}
    \caption{Schema Requisiti Desiderabili}
    \label{fig:Requisiti Desiderabili}
\end{figure}

\newpage

Per quanto riguarda la copertura dei requisiti Desiderabili, la copertura rilevata è di 1 su 9
requisiti, arrivando quindi ad un 11\% sul totale
\begin{figure}[H]
    \centering
    \includegraphics[width=\textwidth]{images/requisiti_op.png}
    \caption{Schema Requisiti Opzionali}
    \label{fig:Requisiti Opzionali}
\end{figure}






\end{document}
