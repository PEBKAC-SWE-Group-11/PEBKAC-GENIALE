\section{Architettura delle compomenenti }


Il sistema è un'applicazione modulare basata su un'architettura Docker a microservizi\textsubscript{G}. Ogni componente è eseguito in un container separato, mantenendo indipendenza e flessibilità, garantendo al contempo un ambiente isolato e sicuro.

\subsection{Descrizione dell'Architettura}

L'architettura si compone di quattro container principali, ciascuno con un compito ben definito:

\begin{itemize}
    \item \textbf{Container App (Flask)}
    \begin{itemize}
        \item Espone l'applicazione Flask\textsubscript{G} sulla porta 5001.
        \item Include un healthcheck che verifica la disponibilità del servizio Ollama\textsubscript{G}.
        \item Ha accesso ai seguenti volumi:
        \begin{itemize}
            \item Directory dell'applicazione
            \item Modelli Ollama\textsubscript{G}
            \item Script di avvio
            \item Requisiti dell'applicazione
            \item Volume dati condiviso
        \end{itemize}
        \item Dipende dal database\textsubscript{G} per il suo funzionamento.
    \end{itemize}

    \item \textbf{Container Database (PostgreSQL con pgvector)}
    \begin{itemize}
        \item Utilizza PostgreSQL\textsubscript{G} con l'estensione \texttt{pgvector} per la gestione dei vettori.
        \item Espone la porta 54321 (mappata internamente alla porta 5432).
        \item Credenziali di default:
        \begin{itemize}
            \item \texttt{Utente: postgres}
            \item \texttt{Password: pebkac}
            \item \texttt{Database: postgres}
        \end{itemize}
        \item Healthcheck per verificare la disponibilità del database.
        \item Dati persistenti tramite il volume \texttt{postgres\_data}.
        \item Include l'estensione \texttt{vector} preinstallata.
    \end{itemize}

    \item \textbf{Container Angular}
    \begin{itemize}
        \item Ospita l'applicazione frontend Angular.
        \item Espone la porta 4200.
        \item Collegato alla rete \texttt{app\_network}.
    \end{itemize}

    \item \textbf{Container Data Processing}
    \begin{itemize}
        \item Gestisce l'elaborazione dei dati.
        \item Accesso alla directory \texttt{data\_processing}\textsubscript{G} tramite un volume dedicato.
        \item Dipende sia dal database che dall'applicazione principale.
    \end{itemize}
\end{itemize}

\subsection{Rete e Persistenza dei Dati}

Tutti i container\textsubscript{G} sono collegati tramite una rete bridge chiamata \texttt{app\_network}, che permette loro di comunicare in modo sicuro e isolato. La persistenza dei dati è garantita da due volumi:

\begin{itemize}
    \item \texttt{postgres\_data}: Memorizza i dati del database.
    \item \texttt{data\_volume}: Contiene i dati condivisi dell'applicazione.
\end{itemize}

\subsection{Diagramma dell'Architettura}
\begin{figure}[H]
    \centering
    \includegraphics[width=\textwidth]{images/esagonale.png}
    \caption{Schema dell'architettura Docker}
\end{figure}

\subsection{Resilienza e Manutenzione}

Questa architettura offre numerosi vantaggi:
\begin{itemize}
    \item \textbf{Scalabilità:} Ogni componente può essere scalato indipendentemente in base al carico.
    \item \textbf{Isolamento dei Servizi:} Un problema in un container\textsubscript{G} non influenza gli altri.
    \item \textbf{Portabilità:} Facilità di distribuzione in ambienti diversi.
    \item \textbf{Resilienza:} I volumi persistenti assicurano che i dati non vadano persi anche in caso di riavvio dei container\textsubscript{G}.
\end{itemize}

\subsection{Conclusione}

L'architettura Docker\textsubscript{G} proposta è flessibile e scalabile. La separazione dei componenti garantisce sicurezza e affidabilità, mentre l'utilizzo dei container\textsubscript{G} facilita il deployment\textsubscript{G} e la manutenzione del sistema.