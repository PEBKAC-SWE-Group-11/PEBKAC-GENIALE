\section{Riassunto della riunione}
Nella riunione svoltasi, il gruppo come da usuale ordine del giorno ha presentato al Proponente le attività svolte nel periodo precedente e sulla pianificazione delle attività future. 
Per quanto riguarda il periodo trascorso, il gruppo ha:
\begin{itemize}
    \item Aggiornato i Proponente sugli attuali ruoli dei membri;
    \item Presentato una retrospettiva sullo svolgimento e sull'esito della prima revisione RTB;
    \item Descritto al Proponente le attività svolte dall'ultimo SAL e le difficoltà riscontrate:
\end{itemize}

 
\noindent Sono poi stati trattati i seguenti punti:
\begin{enumerate}
    \item \textbf{Test degli LLM}: sono stati esposti al Proponente i primi risultati dei test sui modelli (ancora in corso) e con il Proponente si concorda per un modello più "grande" si quello usato nel PoC, anche a discapito di velocità di risposta e risorse hardware necessarie. Inoltre è stato proposta dal gruppo l'ipotesi della quantizzazione degli LLM.
    \item \textbf{Automazione della pipeline}: il Proponente ha chiesto al gruppo di iniziare a ragionare sull'automatizzazione della pipeline di estrazione, indicizzazione e salvataggio dei dati, su come automatizzarla e in che modo scatenare la partenza di questa pipeline (temporizzata, event-driven o altro).
    \item \textbf{PoC}: per quanto riguarda il PoC e le successive necessità in vista del MVP il Proponente evidenzia le successive correzioni o aggiunte necessarie rispetto al PoC realizzato:
        \begin{enumerate}
            \item Personalità del chatbot: il chatbot, a differenza di quanto dimostrato nel PoC non deve rivolgersi per forza a un dipendente Vimar, ma anche a un installatore esterno. Inoltre il tono deve essere più impersonale.
            \item UI e UX: per quanto riguarda l'interfaccia e l'esperienza utente il Proponente richiede le seguenti modifiche:
                \begin{itemize}
                \item Mostrare data e ora del messaggio (inviato o ricevuto);
                \item Mostrare il limite di caratteri rimanenti per messaggio (ed eventualmente di messaggi rimanenti per conversazione)
                \item Mostrare attivamente all'utente che il sistema sta generando una risposta, chiedendo di rimanere in attesa.
                \end{itemize}
            \item \textbf{Feedback}: il Proponente richiede al gruppo di implementare quanto prima il feedback sui messaggi (anche per consentire un periodo di prova del prodotto da parte del Proponente stesso)
        \end{enumerate}
    \item \textbf{Analisi dei Requisiti}: il gruppo ha voluto discutere col Proponente una correzione emersa dalla RTB sulla necessità di autenticazione dell'utilizzatore. È emerso che come interpretato dal gruppo inizialmente l'utilizzatore (non amministratore) del prodotto è un utente ospite a cui viene assegnato un cookie di sessione, senza necessità di autenticazione.
    \item \textbf{Prossimi SAL}: i prossimi SAL (come già discusso) si svolgeranno settimanalmente e avranno durata di 30 minuti. Viene scelto come appuntamento fisso martedì alle 15.00.
\end{enumerate}
