\section{Riassunto della riunione}
Nel SAL odierno abbiamo svolto un allineamento e abbiamo discusso in particolare di testing del codice e di qualità della risposta del modello LLM, come richiesto qualche giorno prima tramite email. \\
Vengono riportati di seguito i punti salienti emersi durante la riunione:
\begin{itemize}
    \item Abbiamo ottenuto un avanzamento significativo sia per quanto riguarda lo sviluppo dei test di unità sia per quanto riguarda lo sviluppo dei test di integrazione. Attualmente stiamo sviluppando i test di unità per il componente per il processing dei dati. Nel momento in cui termineremo o arriveremo a poco dalla fine dei test di unità, procederemo con i test di integrazione. \'E stato anche sollevata la necessità di un migliore controllo di tutti i rami condizionali e di verificare anche altri codici di stato oltre al codice 200;
    \item Abbiamo specificato che i test unitari vengono realizzati tramite la creazione di mock, che permettono di valutare ogni pacchetto o metodo in modo isolato. I test end-to-end, invece, sono definiti come test basati sul punto di vista dell'utente, ad esempio testando le interazioni con la UI. Possiamo valutare di automatizzare anche questa tipologia di test, nonostante non sia strettamente necessario;
    \item Abbiamo implementato la componente che si occupa di fornire le risposte. Tuttavia abbiamo riscontrato che se le domande sono specifiche allora le risposte sono relativamente corrette, mentre se le domande sono più generiche è molto improbabile che le risposte siano corrette. Attualmente, per ottenere un margine di miglioramento in questo aspetto, sarebbe necessario investire parecchie ore produttive, di cui però non disponiamo. \'E necessario, dunque, determinare la quantità di lavoro necessaria, la quantità di tempo che abbiamo a disposizione ed il livello di qualità che vogliamo ottenere (ad esempio che la risposta ad almeno metà delle domande sia corretta). Per verificare il livello di qualità delle risposte fornite dal nostro applicativo, ci verrà fornita dal proponente una lista di test composta dalle domande e dalle rispettive risposte attese, e in cui sarà necessario inserire per ogni domanda la risposta generata dall'applicativo, una valutazione di similarità tra la nostra risposta e quella fornita dal proponente ed un commento testuale;
    \item \'E necessario documentare il percorso e le motivazioni che ci hanno portato ad avere l'applicativo di cui disponiamo oggi;
    \item Abbiamo implementato il web design responsive;
    \item Abbiamo confermato che non verrà fatto uso di AWS, visto l'avanzamento riportato;
    \item Abbiamo espresso la preferenza di saltare la settimana di testing ed eventuale feedback fornito dall'azienda proponente;
    \item \'E stata sollecitata la risoluzione delle questioni aperte dalla riunione SAL precedente.
\end{itemize}
Infine abbiamo stabilito che, se possibile, entro il weekend la lista di domande dovrà essere compilata ed inviata all'azienda proponente, così che durante la prossima riunione SAL (con data che dovrà essere comunicata al proponente tramite email) si possa determinare se è necessario migliorare degli aspetti o se si possa chiudere il progetto