\section{Riassunto della riunione}
Dopo la candidatura alla RTB il gruppo ha iniziato le attività di progettazione e redazione della documentazione necessaria in vista della PB. Nel presente SAL si è infatti discusso in merito alle prospettive di completamento delle varie componenti del sistema e relativa documentazione. Di seguito si riportano le specifiche componenti con annessa la percentuale di completamento calcolata dal gruppo e le attività previste per il loro completamento:
\begin{itemize}
    \item Progettazione dell'infrastruttura (70\%): il design delle componenti attualmente si trova in buono stato in quanto manca il design solo di alcune di esse;
    \item Sviluppo del frontend (50\%): necessario effettuare un design accurato con implementazione di design pattern;
    \item API (60\%): occorre implementare tutte quelle relative al sistema di feedback;
    \item Scraping e indicizzazione (30\%): questa parte deve essere inserita nella pipeline automatizzata;
    \item Database (90\%): lo sviluppo di esso non richiede molto lavoro aggiuntivo rispetto a quello fatto finora;
    \item La scelta del LLM per l'interrogazione utilizzato è stata confermata in Llama 3.1 8B. Per quanto riguarda il modello di embedding, invece, sono ancora in corso dei test per determinarne la scelta definitiva..
    \item RAG (60\%): una volta confermata la scelta dei modelli LLM è necessario ottimizzare questa parte al fine di ottenere risposte migliori.
    \item Inoltre, per tutte le parti dove viene sviluppato codice, bisogna prevedere l'implementazione dei relativi test.
\end{itemize}