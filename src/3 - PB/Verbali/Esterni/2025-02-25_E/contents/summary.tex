\section{Riassunto della riunione}
Dopo la candidatura alla RTB il gruppo ha iniziato le attività di progettazione e redazione della documentazione necessaria in vista della PB. Nel presente SAL si è infatti discusso in merito alle prospettive di completamento delle varie componenti del sistema e relativa documentazione. Di seguito si riportano le specifiche componenti con annessa la percentuale di completamento calcolata dal gruppo e le attività previste per il loro completamento:
\begin{itemize}
    \item Progettazione dell'infrastruttura (70\%): necessario ultimare il design nonostante le componenti progettate finora rispecchiano quelle del PoC;
    \item Sviluppo del frontend (50\%): necessario effettuare un design accurato con implementazione di design pattern;
    \item API (60\%): sviluppate già in maniera corretta nel PoC, rimane da implementare la parte relativa all'invio del feedback della risposta.
    \item Scraping e indicizzazione (30\%): questa parte deve essere inserita nella pipeline automatizzata;
    \item Database (90\%): la maggior parte del lavoro svolto nel PoC è valido anche per la MVP, non è previsto molto lavoro aggiuntivo;
    \item La scelta del LLM per l'interrogazione utilizzato è stata confermata in Llama 3.1 8B. Per quanto riguarda quello di embedding invece si è scelto mxbai-embed-large.
    \item RAG (60\%): una volta confermata la scelta dei modelli LLM è necessario ottimizzare questa parte al fine di ottenere risposte migliori.
    \item Inoltre, per tutte le parti dove viene sviluppato codice, bisogna prevedere l'implementazione dei relativi test.
\end{itemize}