\section{Riassunto della riunione}
Durante la riunione è stato fatto il punto delle attività in corso e si sono decise le prossime attività da svolgere:
\begin{itemize}
    \item \textbf{Architettura di sistema}: l'architettura esagonale scelta per il sistema è stata descritta nel documento di Specifica Tecnica, mancano gli ultimi ritocchi. Il documento verrà sottoposto a verifica entro la serata del 2025-03-03;
    \item \textbf{Test per il Proponente}: sono stati riportati diversi errori nelle domande di test richieste dal Proponente, che è già stato avvisato e ha già provveduto in parte alle correzioni. Questo ha rallentato, ovviamente, i test sul sistema RAG implementato. Saranno comunque presentati dei primi risultati al SAL del 2025-03-04:
    \item \textbf{Correzione Analisi dei Requisiti}: il documento è stato corretto, verificato, approvato e sottoposto alla revisione del professor Cardin;
    \item \textbf{Implementazione Architettura Esagonale}: entro il 2025-03-09 sarà implementato, in un nuovo branch del repository principale, la containerizzazione e la prima bozza di architettura esagonale come descritto da Specifica Tecnica, per mano di Progettisti e Programmatori;
    \item \textbf{Norme di Progetto}: verranno inserite nuove norme nel relativo documento per quanto riguarda la connessione di verifica-modifica e la codifica. Data prevista 2025-03-11. 

\end{itemize}
Viene notificato inoltre che riprendono i Diari di Bordo e sarà il Responsabile a occuparsene, comunicando preventivamente eventuali problemi.
\\
Si è poi discusso del periodo concluso (retrospettiva inserita nel Piano di Progetto) e sono stati cambiati i ruoli per il prossimo periodo come segue:
\begin{itemize}
    \item Alessandro Benin: Programmatore;
    \item Ion Bourosu: Responsabile;
    \item Matteo Gerardin: Amministratore;
    \item Derek Gusatto: Progettista;
   \item Davide Martinelli: Verificatore;
    \item Matteo Piron: Verificatore;
    \item Tommaso Zocche: Programmatore.
  \end{itemize}
