\section{Riassunto della riunione}
La presente riunione ha l'obiettivo di rivere con tutto il team l'esito della prima revisione RTB, consultabile a \href{https://www.math.unipd.it/~tullio/IS-1/2024/Progetto/RTB/PEBKAC.pdf}{questo link}, dopo che ogni membro ne ha preso visione singolarmente.
\\
Il gruppo prende quindi atto di dover intervenire al più presto nei seguenti modi:
\begin{itemize}
    \item Rivedendo il template LaTeX dei verbali per eliminare il registro di versionamento, non necessario perché si ha  che la
loro versione ufficiale, sovrascriva, cancellandola, ogni revisione
precedente;
\item Correggendo in maniera precisa e tempestiva il documento Analisi dei Requisiti;
    \item Aggiornamndo la GitHub Page di presentazione dei documenti in modalità \textit{stack} (con i documenti più recenti visualizzati per prima);
    \item Rimuovendo la numerazione di sezione dal Glossario;
    \item Inserendo un preventivo a finire al termine di ogni periodo nel Piano di Progetto, aggiornato dopo la retrospettiva del periodo.

\end{itemize}

Al termine del quinto periodo le attività completate sono meno del previsto, e bisognerà sopperire nel prossimo periodo. 

Emergono poi le seguenti decisioni:
\begin{itemize}
\item Pur essendosi concluso il perido, il gruppo manterrà i ruoli attuali per un altra settima,  con lo scopo di migliorare la produttività del nuovo periodo e sopperire al ritardo del periodo appena concluso;
\item Iniziare i test per la scelta definitiva degli LLM, nello specifico del modello che genera la risposta. I test, come anche da suggerimento del Proponente, avverranno nelle seguenti modalità:
    \begin{enumerate}
        \item Porre una serie di domande su temi generici, che richiedano una risposta vaga/descrittiva oppure precisa:
        \item Valutare:
            \begin{itemize}
                \item Tempi di risposta;
                \item Consumo di risorse hardware;
                \item Correttezza di una risposta (anche a confronto con le altre) giudicata da un assistente AI esterno (e.g. ChatGPT);
            \end{itemize}
       \item Discutere, tra i membri del gruppo a cui è affidato il compito, dei risultati per tratte le giuste conclusioni;    

       \item (\textbf{Opzionalmente}, su idea del Proponente) Provare a porre domande già poste in lingua italiana in lingua inglese, per valutare se un modello più piccolo e che consuma meno risorse sia preferibile al costo di dover utilizzare la lingua inglese.
    \end{enumerate}
\item Chiedere al Proponente di fissare il prossimo SAL (previsto al momento durante la settimana del 17 febbraio) il giorno giovedì 20 febbraio.

\end{itemize}