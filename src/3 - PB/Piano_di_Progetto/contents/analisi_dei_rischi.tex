\section{Analisi dei Rischi}

\subsection{Rischi Tecnologici}

\hypertarget{RT1}{}
\begin{tabular}{|L{4cm}|L{9.7cm}|}
    \hline
    \textbf{Id. Rischio} & \textbf{RT1} \\
    \hline
    \textbf{Rischio} & Inesperienza \\
    \hline
    \textbf{Descrizione} & La mancanza di esperienza con le tecnologie richieste dal progetto può comportare ritardi nello sviluppo, errori nel codice e difficoltà nell'utilizzo degli strumenti. \\
    \hline
    \textbf{Pericolosità} & Alta \\
    \hline
    \textbf{Occorrenza} & Alta \\
    \hline
    \textbf{Piano di intervento} & I membri ``inesperti" avranno tempo di studiare le nuove tecnologie ed eventualmente saranno affiancati a membri più esperti. \\
    \hline
\end{tabular}
\\[30pt]
\hypertarget{RT2}{}
\begin{tabular}{|L{4cm}|L{9.7cm}|}
    \hline
    \textbf{Id. Rischio} & \textbf{RT2} \\
    \hline
    \textbf{Rischio} & Problemi con software di terze parti \\
    \hline
    \textbf{Descrizione} & L'utilizzo di software o librerie esterne può comportare malfunzionamenti, incompatibilità e difficoltà di integrazione. \\
    \hline
    \textbf{Pericolosità} & Alta \\
    \hline
    \textbf{Occorrenza} & Bassa \\
    \hline
    \textbf{Piano di intervento} & L'adozione di un sistema di versionamento dei file, ``backup" regolari e la replica dei dati. \\
    \hline
\end{tabular}
\\[30pt]
\hypertarget{RT3}{}
\begin{tabular}{|L{4cm}|L{9.7cm}|}
    \hline
    \textbf{Id. Rischio} & \textbf{RT3} \\
    \hline
    \textbf{Rischio} & Basse prestazioni hardware \\
    \hline
    \textbf{Descrizione} & Le risorse hardware limitate dei PC personali potrebbero risultare insufficienti per condurre test approfonditi. \\
    \hline
    \textbf{Pericolosità} & Media \\
    \hline
    \textbf{Occorrenza} & Media \\
    \hline
    \textbf{Piano di intervento} & Ottimizzazione del codice, semplificazione delle funzionalità del progetto, adozione di strategie di test meno onerose. \\
    \hline
\end{tabular}

\subsection{Rischi Organizzativi}

\hypertarget{RO1}{}
\begin{tabular}{|L{4cm}|L{9.7cm}|}
    \hline
    \textbf{Id. Rischio} & \textbf{RO1} \\
    \hline
    \textbf{Rischio} & Imprecisioni nella pianificazione \\
    \hline
    \textbf{Descrizione} & La sottostima o sovrastima dei tempi e delle risorse per completare le attività può portare a ritardi nello sviluppo, sforamento del budget e stress. \\
    \hline
    \textbf{Pericolosità} & Alta \\
    \hline
    \textbf{Occorrenza} & Alta \\
    \hline
    \textbf{Piano di intervento} & Pianificazione flessibile, revisioni periodiche del Piano di Progetto, confronto con il proponente, riassegnazione di compiti e suddivisione delle attività in task più piccole. \\
    \hline
\end{tabular}
\\[30pt]
\hypertarget{RO2}{}
\begin{tabular}{|L{4cm}|L{9.7cm}|}
    \hline
    \textbf{Id. Rischio} & \textbf{RO2} \\
    \hline
    \textbf{Rischio} & Impegni personali (e accademici) \\
    \hline
    \textbf{Descrizione} & La difficoltà nel conciliare gli impegni del progetto con quelli personali, in particolare durante gli esami, può comportare ritardi e assenze. \\
    \hline
    \textbf{Pericolosità} & Alta \\
    \hline
    \textbf{Occorrenza} & Alta \\
    \hline
    \textbf{Piano di intervento} & Comunicazione tempestiva, pianificazione di periodi di ``stallo", ridistribuzione dei compiti con successivo recupero dell'eventuale ``debito lavorativo". \\
    \hline
\end{tabular}
\\[30pt]
\hypertarget{RO3}{}
\begin{tabular}{|L{4cm}|L{9.7cm}|}
    \hline
    \textbf{Id. Rischio} & \textbf{RO3} \\
    \hline
    \textbf{Rischio} & Variazione dei requisiti di progetto \\
    \hline
    \textbf{Descrizione} & Durante l'implementazione del progetto potrebbero emergere cambiamenti nei requisiti, i quali potrebbero determinare una deviazione delle attività pianificate. \\
    \hline
    \textbf{Pericolosità} & Media \\
    \hline
    \textbf{Occorrenza} & Alta \\
    \hline
    \textbf{Piano di intervento} & Preparare un'analisi dettagliata dei requisiti all'inizio del progetto e attuare tempestivamente le eventuali misure correttive necessarie. \\
    \hline
\end{tabular}
\\[30pt]
\hypertarget{RO4}{}
\begin{tabular}{|L{4cm}|L{9.7cm}|}
    \hline
    \textbf{Id. Rischio} & \textbf{RO4} \\
    \hline
    \textbf{Rischio} & Mal interpretazione dei requisiti \\
    \hline
    \textbf{Descrizione} & Un'interpretazione errata dei requisiti del progetto può portare alla realizzazione di un prodotto non conforme alle aspettative. \\
    \hline
    \textbf{Pericolosità} & Media \\
    \hline
    \textbf{Occorrenza} & Bassa \\
    \hline
    \textbf{Piano di intervento} & Documentazione chiara e dettagliata, comunicazione costante con il cliente. \\
    \hline
\end{tabular}

\subsection{Rischi Interni al Gruppo}

\hypertarget{RG1}{}
\begin{tabular}{|L{4cm}|L{9.7cm}|}
    \hline
    \textbf{Id. Rischio} & \textbf{RG1} \\
    \hline
    \textbf{Rischio} & Problemi di comunicazione interna \\
    \hline
    \textbf{Descrizione} & La difficoltà di comunicazione efficace all'interno del team può portare a incomprensioni, errori e ritardi. \\
    \hline
    \textbf{Pericolosità} & Alta \\
    \hline
    \textbf{Occorrenza} & Alta \\
    \hline
    \textbf{Piano di intervento} & Strumenti di comunicazione adeguati, meeting regolari, clima collaborativo e definizione chiara dei ruoli. \\
    \hline
\end{tabular}
\\[30pt]
\hypertarget{RG2}{}
\begin{tabular}{|L{4cm}|L{9.7cm}|}
    \hline
    \textbf{Id. Rischio} & \textbf{RG2} \\
    \hline
    \textbf{Rischio} & Rischio di conflitti interni \\
    \hline
    \textbf{Descrizione} & Data la durata del progetto, conflitti tra i membri del team possono sorgere. \\
    \hline
    \textbf{Pericolosità} & Bassa \\
    \hline
    \textbf{Occorrenza} & Bassa \\
    \hline
    \textbf{Piano di intervento} & Clima di rispetto reciproco, un mediatore per le controversie. \\
    \hline
\end{tabular}