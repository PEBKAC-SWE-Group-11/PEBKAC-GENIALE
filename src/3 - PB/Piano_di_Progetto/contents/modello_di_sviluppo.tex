\section{Modello di Sviluppo}
Dopo un'approfondita analisi e un confronto, il gruppo ha deciso di adottare il modello Agile$_G$ per la gestione del progetto.\\
A differenza di metodologie più tradizionali, questo approccio ampiamente utilizzato nello sviluppo software$_G$, si basa in un ciclo continuo di pianificazione, implementazione e verifica.\\
Nello specifico, il team ha stabilito una struttura operativa organizzata in cicli bisettimanali. La scelta di Agile$_G$ è stata motivata dai seguenti benefici principali: 
\begin{itemize}[topsep=1pt, itemsep=1pt]
    \item \textbf{Gestione efficace dei rischi}: la breve durata dei cicli consente di individuare eventuali criticità in tempi rapidi, riducendo sia l’impatto delle problematiche sia il rischio complessivo di fallimento;
    \item \textbf{Partecipazione del team}: la struttura di questo modello favorisce il coinvolgimento attivo del team, grazie alla frequente trasformazione delle attività svolte in risultati concreti;
    \item \textbf{Massima ``trasparenza"}: il modello consente di presentare regolarmente i progressi agli stakeholder$_G$, facilitando il controllo sull'avanzamento dei lavori;
    \item \textbf{Adattabilità ai cambiamenti}: grazie alla struttura iterativa, Agile$_G$ permette di gestire agevolmente modifiche ai requisiti del progetto, rispondendo tempestivamente a nuovi scenari o esigenze impreviste.
\end{itemize}