\section{V}
\begin{itemize}
    %\item \textbf{V model}: Modello di sviluppo software che correla ogni fase di progettazione con la rispettiva fase di test$_G$.
    \item \textbf{Validazione}: Definisce le attività per validare il prodotto \textit{software}$_G$.
    \item \textbf{Verifica}: Definisce le attività per verificare il prodotto \textit{software}$_G$.
    \item \textbf{Verificatore}: Figura incaricata di controllare la coerenza del lavoro svolto rispetto ai requisiti e alla documentazione$_G$.
    \item \textbf{Vimar}: Azienda proponente del \textit{capitolato}$_G$ "Knowledge Management$_G$ Ai$_G$", che mette a disposizione le sue competenze per sviluppare soluzioni innovative$_G$.
    \item \textbf{VueJS}: Framework JavaScript \textit{open-source}$_G$ utilizzato per la costruzione di interfacce utente e applicazioni web single-page, che si distingue per la sua reattività, semplicità e capacità di integrare facilmente componenti modulari$_G$.


\end{itemize}
