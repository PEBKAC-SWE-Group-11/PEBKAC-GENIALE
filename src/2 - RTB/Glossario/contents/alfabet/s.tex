\section{S}
\begin{itemize}
    \item \textbf{Schedule Performance Index (SPI)}: Indicatore di efficienza temporale di un progetto.
    \item \textbf{Scoreboard}: Visualizzazione grafica dello stato di avanzamento di un progetto, generalmente con percentuali e stati dei \textit{task}\textsubscript{G}.
    \item \textbf{Scrapy}: Framework \textit{open source}\textsubscript{G} per il web scraping, utile per estrarre informazioni da siti web.
    %\item \textbf{Scrum}: Framework di \textit{Agile}\textsubscript{G} per gestire progetti complessi, basato su iterazioni e feedback continui\textsubscript{G}.
    %\item \textbf{SEMAT}: Sistema a sei livelli per misurare il progresso di un progetto\textsubscript{G}.
    \item \textbf{Sessione}: Periodo di tempo durante il quale un utente interagisce con un sistema o applicazione, in cui vengono memorizzate informazioni temporanee (come credenziali o preferenze) per garantire una continuità nelle operazioni, solitamente gestita tramite un identificatore univoco.
    \item \textbf{Sistema}: Insieme di componenti interconnessi che lavorano insieme per raggiungere uno scopo comune. In ambito software, può includere \textit{hardware}\textsubscript{G}, \textit{software}\textsubscript{G}, persone e processi.
    \item \textbf{Slack}: Strumento di collaborazione che consente messaggistica istantanea e gestione delle attività di team.
    \item \textbf{Software}: Insieme di programmi, dati e istruzioni che permettono a un computer di eseguire specifiche funzioni o compiti, comprese applicazioni, sistemi operativi e strumenti di sviluppo.
    \item \textbf{Software Development Lifecycle (SDLC)}: Processo strutturato che guida la progettazione, lo sviluppo, il testing, la distribuzione e la manutenzione di software, con l'obiettivo di produrre software di alta qualità attraverso fasi ben definite e pratiche sistematiche.
    %\item \textbf{Sottotask}: Parte più granulare di un’attività principale, rappresentata da un \textit{ticket}\textsubscript{G}.
    \item \textbf{Spring}: Framework \textit{open-source}\textsubscript{G} per lo sviluppo di applicazioni Java, che fornisce infrastrutture per la gestione della configurazione, la gestione delle dipendenze, la sicurezza e l'accesso ai dati, semplificando la creazione di applicazioni scalabili e modulari\textsubscript{G}.
    \item \textbf{Spring Boot}: Framework \textit{open-source}\textsubscript{G} basato su \textit{Spring}\textsubscript{G}, utilizzato per creare applicazioni Java autonome e pronte per la produzione, con configurazioni minime e una struttura predefinita che semplifica la creazione di applicazioni basate su microservizi\textsubscript{G}.
    \item \textbf{Sprint}: Iterazione a tempo fisso di due settimane in Scrum, durante la quale il team sviluppa un incremento del prodotto.
    \item \textbf{SQL}: Acronimo di Structured Query Language, linguaggio standard per la gestione di database relazionali.
    \item \textbf{Stakeholder}: Persona o gruppo interessato al successo di un progetto o iniziativa\textsubscript{G}.
    \item \textbf{Superset}: Strumento \textit{open source}\textsubscript{G} per la visualizzazione e l'analisi dei dati, che consente di creare dashboard interattive e report complessi a partire da diverse fonti di dati.
    \item \textbf{Swift}: Linguaggio di programmazione sviluppato da Apple, utilizzato per lo sviluppo di applicazioni iOS, macOS, watchOS e tvOS, noto per la sua velocità e sicurezza.
    \item \textbf{SwiftUI}: Framework sviluppato da Apple per la costruzione di interfacce utente dichiarative per le piattaforme Apple, che semplifica il design delle UI attraverso una sintassi Swift integrata.
\end{itemize}
