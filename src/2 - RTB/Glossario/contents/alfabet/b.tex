\section{B}
\begin{itemize}
    \item \textbf{Back end}: La parte amministrativa di un sito o web app, accessibile solo dagli \textit{amministratori}$_G$, che gestisce le funzionalità server-side.
    \item \textbf{Backlog}: Artefatto di \textit{Scrum}$_G$, consiste in un elenco ordinato di attività o funzionalità da sviluppare, mantenuto e aggiornato in base alle priorità del progetto.
    \item \textbf{Baseline}: Punto di riferimento fisso utilizzato per confrontare le prestazioni del progetto nel tempo, riguardando ambito, tempi e costi.
    \item \textbf{Branch}: Linea indipendente di sviluppo in un sistema di versionamento come \textit{Git}$_G$, che consente modifiche parallele al codice sorgente.
    \item \textbf{Bug}: Errore o malfunzionamento nel software, spesso dovuto a problemi nel codice sorgente.
    \item \textbf{Bug reporting}: Processo di registrazione e monitoraggio dei \textit{bug}$_G$ o errori rilevati durante i test software, con lo scopo di risolverli.
    \item \textbf{Bert}: Modello di linguaggio bidirezionale utilizzato per compiti di elaborazione del linguaggio naturale come analisi del sentimento o classificazione testuale.
    \item \textbf{Benchmarking}: Metodo per valutare le prestazioni di un sistema, come un modello \textit{LLM}$_G$, confrontandolo con standard o metriche predefinite.
\end{itemize}
