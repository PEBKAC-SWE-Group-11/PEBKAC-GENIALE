\section{C}
\begin{itemize}
    \item \textbf{Capitolato}: Documento che descrive in dettaglio lavorazioni, modalità e tempi di esecuzione per un progetto, incluso il relativo costo.
    \item \textbf{Chat history}: Cronologia delle conversazioni precedenti con un interlocutore, utile per contestualizzare le interazioni in un \textit{chatbot}$_G$.
    \item \textbf{Chatbot}: Programma che utilizza \textit{AI}$_G$ e \textit{LLM}$_G$ per simulare conversazioni umane e automatizzare risposte.
    \item \textbf{ChatGPT}: Chatbot sviluppato da \textit{OpenAI}$_G$, basato su modelli \textit{GPT} (Generative Pre-trained Transformer), specializzato nel dialogo umano.
    \item \textbf{Code coverage}: Metodologia per misurare quanta parte del codice è stata testata durante l'esecuzione dei test software.
    \item \textbf{Commit}: Operazione in \textit{Git}$_G$ per salvare le modifiche apportate ai file in un \textit{repository}$_G$.
    \item \textbf{Committente}: Persona o organizzazione che commissiona la realizzazione di un'opera o servizio. In questo contesto, sono i professori Tullio Vardanega e Riccardo Cardin.
    \item \textbf{Consulenza}: Attività svolta da un esperto per fornire supporto e soluzioni in un determinato ambito.
    \item \textbf{Consuntivo}: Rendiconto dei risultati ottenuti in un determinato periodo o progetto.
    \item \textbf{CSS}: Acronimo di Cascading Style Sheets, linguaggio usato per definire il design e la formattazione di documenti HTML.
    \item \textbf{Cloud}: Infrastruttura tecnologica che permette l'archiviazione, l'elaborazione e la distribuzione di dati e applicazioni tramite internet.
    \item \textbf{C\#}: Linguaggio di programmazione orientato agli oggetti sviluppato da Microsoft, principalmente utilizzato per lo sviluppo di applicazioni su Windows, con forte integrazione nell'ambiente .NET.
    \item \textbf{ClickHouse}: Database analitico open-source progettato per l'elaborazione di grandi volumi di dati in tempo reale, particolarmente adatto per applicazioni che richiedono analisi veloci e scalabili.
\end{itemize}
