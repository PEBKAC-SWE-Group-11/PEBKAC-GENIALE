\section{C}
\begin{itemize}
    \item \textbf{CA (Customer Acceptance)}: Fase del processo di sviluppo in cui il cliente verifica e approva il prodotto o il servizio fornito, assicurandosi che soddisfi i requisiti$_G$ e le specifiche concordate prima della consegna finale o del rilascio.
    \item \textbf{Capitolato}: Documento che descrive in dettaglio lavorazioni, modalità e tempi di esecuzione per un progetto, incluso il relativo costo.
    \item \textbf{Chat history}: Cronologia delle conversazioni precedenti con un interlocutore, utile per contestualizzare le interazioni in un \textit{chatbot}$_G$.
    \item \textbf{Chatbot}: Programma che utilizza \textit{AI}$_G$ e \textit{LLM}$_G$ per simulare conversazioni umane e automatizzare risposte.
    \item \textbf{ChatGPT}: Chatbot sviluppato da \textit{OpenAI}$_G$, basato su modelli \textit{GPT} (Generative Pre-trained Transformer), specializzato nel dialogo umano.
    \item \textbf{ClickHouse}: Database analitico \textit{open-source}$_G$ progettato per l'elaborazione di grandi volumi di dati in tempo reale, particolarmente adatto per applicazioni che richiedono analisi veloci e scalabili.
    \item \textbf{Cloud}: Infrastruttura tecnologica che permette l'archiviazione, l'elaborazione e la distribuzione di dati e applicazioni tramite internet.
    \item \textbf{Cloud-ready}: Infrastruttura$_G$ progettata per essere compatibile e facilmente integrabile con servizi di \textit{cloud computing}$_G$, offrendo scalabilità e flessibilità per l'implementazione di applicazioni e servizi.
    \item \textbf{Code coverage}: Metodologia per misurare quanta parte del codice è stata testata durante l'esecuzione dei test software.
    \item \textbf{Commit}: Operazione in \textit{Git}$_G$ per salvare le modifiche apportate ai file in un \textit{repository}$_G$.
    \item \textbf{Committente}: Persona o organizzazione che commissiona la realizzazione di un'opera o servizio.
    \item \textbf{Configuration Management}: Processo che definisce e controlla le modifiche agli elementi di un sistema software, garantendo coerenza e tracciabilità.
    \item \textbf{Configuration Status Accounting}: Processo di documentazione e monitoraggio delle verifiche e dei cambiamenti apportati alle caratteristiche di un prodotto software, garantendo la tracciabilità e l'aggiornamento continuo.
    \item \textbf{Consulenza}: Attività svolta da un esperto per fornire supporto e soluzioni in un determinato ambito.
    \item \textbf{Consuntivo}: Rendiconto dei risultati ottenuti in un determinato periodo o progetto.
    \item \textbf{conversazione guidata}:Il sistema deve prevedere una serie di menù e sottomenù con delle opzioni
    selezionabili dall’utente che costruiscano un prompt per le esigenze principali di un
    possibile installatore.
    \item \textbf{conversazione libera}:Sistema attraverso cui l’installatore può
    fare domande e ricevere risposte in linguaggio naturale.
    \item \textbf{Cost Performance Index (CPI)}: Indicatore di efficienza dei costi di un progetto rispetto al budget.
    \item \textbf{Cruscotto informativo}: Strumento di visualizzazione dei dati che aggrega e presenta informazioni chiave attraverso grafici, tabelle e indicatori, consentendo il monitoraggio e l'analisi delle performance di un sistema o processo in tempo reale$_G$.
    \item \textbf{CSS}: Acronimo di \textit{Cascading Style Sheets}, linguaggio usato per definire il design e la formattazione di documenti HTML.
    \item \textbf{C\#}: Linguaggio di programmazione orientato agli oggetti sviluppato da Microsoft, principalmente utilizzato per lo sviluppo di applicazioni su Windows, con forte integrazione nell'ambiente .NET.
    
\end{itemize}
