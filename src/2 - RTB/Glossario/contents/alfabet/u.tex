\section{U}
\begin{itemize}
    \item \textbf{UI}: Acronimo di User Interface, rappresenta l'interfaccia grafica attraverso cui l'utente interagisce con un sistema$_G$.
    \item \textbf{UML}: Acronimo di Unified Modeling Language, linguaggio per modellare sistemi basati sul paradigma orientato agli oggetti$_G$.
    \item \textbf{Use case}: Insieme di scenari che descrivono interazioni tra \textit{attori}$_G$ e sistema per raggiungere un obiettivo$_G$.
    \item \textbf{User-friendly}: Termine che descrive un sistema, un'interfaccia o un'applicazione progettati per essere semplici da usare, intuitivi e accessibili anche per utenti con poca esperienza tecnica$_G$.
    \item \textbf{User Stories}: Descrizioni brevi e semplici di funzionalità o requisiti da parte degli utenti finali, utilizzate nel processo di sviluppo agile per definire ciò che un sistema dovrebbe fare, solitamente redatte nel formato \textit{Come} [utente], \textit{voglio} [funzionalità], \textit{così che} [beneficio]$_G$.
    \item \textbf{Username}: Identificativo univoco assegnato a un utente per accedere a un sistema, applicazione o servizio, spesso combinato con una password per autenticare l'accesso e garantire la sicurezza$_G$.
\end{itemize}
