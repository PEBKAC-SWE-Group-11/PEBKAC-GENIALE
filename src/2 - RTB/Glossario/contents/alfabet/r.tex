\section{R}
\begin{itemize}
    \item \textbf{React}: Libreria JavaScript \textit{open source}$_G$ utilizzata per creare interfacce utente interattive.
    \item \textbf{Repository}: Archivio digitale che conserva codice sorgente, file e documenti relativi a un progetto software.
    \item \textbf{Requisito}: Descrizione di ciò che un sistema deve fare o vincoli che deve rispettare.
    \item \textbf{Responsabile}: Figura che guida il team di progetto, pianificando attività e gestendo le relazioni esterne.
    \item \textbf{Risorsa}: Qualsiasi elemento utilizzato dal sistema per eseguire operazioni o attività.
    \item \textbf{RTB}: Acronimo di Requirements and Technology Baseline, revisione iniziale per fissare requisiti e tecnologie di un progetto.
    \item \textbf{Ruolo}: Incarico assunto da un membro del team, come \textit{responsabile}$_G$, \textit{amministratore}$_G$, o \textit{analista}$_G$.
    \item \textbf{REST}: Acronimo di Representational State Transfer, stile architetturale per la progettazione di servizi web basati su risorse.
    \item \textbf{RAG}: Acronimo di Retrieval-Augmented Generation, tecnica utilizzata per migliorare i modelli \textit{LLM}$_G$ combinando la generazione di testo con la ricerca di informazioni pertinenti da una base di conoscenza esterna.
\end{itemize}
