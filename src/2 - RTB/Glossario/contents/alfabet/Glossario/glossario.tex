
\documentclass[12pt, a4paper]{article}

\usepackage{graphicx}
\usepackage{xcolor}
\usepackage{float}
\usepackage{svg}
\usepackage[colorlinks=true, linkcolor=black, urlcolor=blue, citecolor=green]{hyperref}
\usepackage{enumitem}
\usepackage[italian]{babel}
\usepackage{lastpage}  % Pacchetto per ottenere il numero totale delle pagine
\usepackage{fancyhdr}  % Pacchetto per personalizzare l'intestazione e il piè di pagina
\usepackage[margin=1in]{geometry}
\usepackage{array}
\newcolumntype{C}[1]{>{\centering\arraybackslash}p{#1}}
\newcolumntype{L}[1]{>{\raggedright\arraybackslash}p{#1}}
\graphicspath{ {images/} {../shared/images/} }
\definecolor{unipd}{HTML}{B5121B}

\addto\captionsitalian{\renewcommand{\contentsname}{Indice}}


\pagestyle{fancy}% Imposta lo stile di pagina su "fancy"
\fancyhf{}% Cancella intestazioni e piè di pagina
\fancyfoot[C]{\thepage{} di \pageref{LastPage}} % Imposta il piè di pagina centrale come "numero pagina di totale pagine"
\renewcommand{\headrulewidth}{0pt} % Imposta la larghezza della linea di intestazione a 0 punti

%\newcommand{\data}{GG mese AAAA}
\newcommand{\titolo}{Glossario}
%\newcommand{\responsabile}{Responsabile}
\newcommand{\verificatore}{
    & Matteo Gerardin
}
\newcommand{\redattore}{
    & Alessandro Benin \\
    & Matteo Piron \\
    & Tommaso Zocche
}
\newcommand{\uso}{Esterno}
\newcommand{\destinatari }{
    & Tullio Vardanega  \\
    & Riccardo Cardin \\  
    & Vimar S.p.A.
}
\newcommand{\abstractcontent}{abstract ...}

\begin{document}
\begin{minipage}[]{0.3\textwidth}
\includesvg[width=\linewidth]{pebkac.svg} 
\end{minipage}
\hspace{0.1\textwidth}
\begin{minipage}[]{0.6\textwidth}
  {\Large \textbf{PEBKAC}} \\
  Email: \href{mailto:pebkacswe@gmail.com}{pebkacswe@gmail.com} \\
  Gruppo: 11
\end{minipage}

\bigskip

\begin{minipage}[]{0.3\textwidth}
\includesvg[width=\linewidth]{logo_unipd.svg} 
\end{minipage}
\hspace{0.1\textwidth}
\begin{minipage}[]{0.6\textwidth}
  \textcolor{unipd}{
    \textbf{Università degli Studi di Padova} \\
    Corso di Laurea: Informatica \\
    Corso: Ingegneria del Software \\
    Anno Accademico: 2024/2025
  }
\end{minipage}


\bigskip
\bigskip
\bigskip
\begin{center}
  \Huge\textbf{Verbale Interno}

  \Large\textbf{\data}
\end{center}

\bigskip


\begin{center}
\textbf{Informazioni sul documento}: \\
\vspace{0.5cm}

\begin{tabular}{r|l}
    \textbf{Responsabile} & Tommaso Zocche \\ 
    \textbf{Verificatore} & Alessandro Benin \\ 
    \textbf{Redattore} & Tommaso Zocche \\ 
    \textbf{Uso} & Interno \\ 
    \textbf{Destinatari} & Tullio Vardanega \\ & Riccardo Cardin \\ 
\end{tabular}

\vfill

\textbf{Abstract}: \\
\vspace{0.5cm}
L'obiettivo dell'incontro è stato definire l'ordine di preferenza dei capitolati a seguito degli incontri avuti con le aziende interessate e iniziare a redigere il prospetto orario del gruppo.
\end{center}


\bigskip
\newpage

\section*{Registro delle modifiche}
\begin{table}[H]
    \begin{tabular}{|c|c|c|c|p{5cm}|}
        \hline
         \textbf{Versione} &  \textbf{Data} &  \textbf{Autore} &  \textbf{Ruolo} & \textbf{Descrizione} \\
          \hline
          &  &  & Responsabile & Approvazione e rilascio\\
          \hline
          0.1.0 & 17/11/2024 & Alessandro Benin & Verificatore  & Verificato \\
          \hline
          0.0.1 & 13/11/2024 & Derek Gusatto & Amministratore  & Stesura iniziale \\
          \hline
    \end{tabular}
\end{table}
\newpage
\tableofcontents
\newpage
\section{Introduzione}
Questo documento mira a garantire chiarezza e uniformità nella terminologia utilizzata all'interno della documentazione del progetto, riducendo al minimo le ambiguità o potenziali incomprensioni. A tal fine, include un glossario che fornisce definizioni precise per i termini specifici del dominio d'uso.
Per facilitare la consultazione, i termini sono organizzati in ordine alfabetico, consentendo una navigazione semplice e immediata. Inoltre, in tutti gli altri documenti del progetto, i termini presenti nel Glossario saranno identificati da un simbolo di riferimento, rappresentato da una ``G" in pedice.
\newpage
\section{A}
\begin{itemize}
    \item \textbf{Accertamento di qualità}: Definisce le attività per assicurare in modo oggettivo che i prodotti ed i processi \textit{software}\textsubscript{G} siano conformi ai \textit{requisiti}\textsubscript{G} specificati.
    \item \textbf{Actual Cost (AC)}: Costo effettivo, rappresenta la spesa reale sostenuta per completare il lavoro fino ad un dato momento.
    \item \textbf{Agile}: \textit{Modello di sviluppo}\textsubscript{G} che privilegia la flessibilità e la comunicazione efficace sia all'interno del team che con lo \textit{stakeholder}\textsubscript{G}, ponendo maggiore enfasi su un prodotto funzionante rispetto ad una \textit{documentazione}\textsubscript{G} completa, e favorendo l'adattamento ai cambiamenti piuttosto che l'aderenza rigorosa ad un piano prestabilito.
    \item \textbf{AI}: Acronimo di \textit{Artificial Intelligence}, disciplina che studia come progettare sistemi intelligenti in grado di emulare il ragionamento ed il comportamento umano.
    \item \textbf{Amministratore}: Figura che si occupa della configurazione e manutenzione dell'ambiente IT di lavoro, selezionando ed implementando le \textit{risorse}\textsubscript{G} necessarie per supportare il \textit{way of working}\textsubscript{G}.
    \item \textbf{Amazon Q}: Servizio di cloud computing che supporta lo sviluppo \textit{software}\textsubscript{G}, fornendo strumenti per il testing e l’analisi.
    \item \textbf{Angular}: \textit{Framework}\textsubscript{G} \textit{open source}\textsubscript{G} utilizzato per creare applicazioni web dinamiche e reattive, sfruttando componenti modulari e strumenti avanzati per lo sviluppo \textit{frontend}\textsubscript{G}.
    \item \textbf{Analista}: Persona esperta nel dominio del problema che identifica i \textit{requisiti}\textsubscript{G} necessari per avviare una progettazione efficiente e mirata.
    \item \textbf{API}: Acronimo di \textit{Application Programming Interface}, indica un insieme di procedure e regole per permettere la comunicazione tra sistemi \textit{software}\textsubscript{G} diversi o tra componenti di uno stesso \textit{sistema}\textsubscript{G}.
    %\item \textbf{Applicazioni distribuite}: Sistemi \textit{software}\textsubscript{G} in cui i componenti sono eseguiti su nodi diversi all'interno di una rete, che collaborano tra loro per raggiungere obiettivi comuni, offrendo scalabilità, tolleranza ai guasti ed alta disponibilità\textsubscript{G}.
    \item \textbf{Applicativo server}: \textit{Software}\textsubscript{G} progettato per essere eseguito su un server, responsabile di gestire richieste da client, elaborare dati e fornire risposte adeguate attraverso una rete.
    \item \textbf{AS-IS / TO-BE}: Metodologia utilizzata per analizzare e rappresentare la situazione attuale (\textit{AS-IS}) e la situazione futura desiderata (\textit{TO-BE}) in un processo di cambiamento o implementazione di un prodotto.
    \item \textbf{Atlassian}: Azienda specializzata nello sviluppo di strumenti \textit{software}\textsubscript{G} per il project management e la collaborazione, come \textit{Jira}\textsubscript{G} e Confluence.
    \item \textbf{Attore}: Nel contesto degli \textit{use case}\textsubscript{G}, rappresenta il ruolo svolto da un utente o \textit{sistema}\textsubscript{G} che interagisce con il \textit{software}\textsubscript{G}.
    \item \textbf{Audit trail}: Registro dettagliato delle attività svolte su un \textit{sistema}\textsubscript{G} per garantire la tracciabilità e l’integrità.
    \item \textbf{Automazione}: Processo di sostituzione delle attività manuali con istruzioni che consentono l'esecuzione automatica di operazioni ripetitive sui \textit{sistemi}\textsubscript{G}. informatici
    \item \textbf{AWS}: \textit{Amazon Web Services}, piattaforma di servizi \textit{cloud}\textsubscript{G} che offre un'infrastruttura IT scalabile, sicura ed altamente disponibile.
    \item \textbf{AWS EC2}: Servizio di \textit{Amazon Web Services}\textsubscript{G} che fornisce capacità di calcolo scalabile nel \textit{cloud}\textsubscript{G}, consentendo di lanciare e gestire istanze virtuali.
    %\item \textbf{AWS LightSail}: Servizio di \textit{cloud computing}\textsubscript{G} offerto da \textit{Amazon Web Services}\textsubscript{G}, che fornisce soluzioni semplici per applicazioni e siti web.
\end{itemize}

\newpage
\section{B}
\begin{itemize}
    \item \textbf{Back end}: La parte amministrativa di un sito o web app, accessibile solo dagli \textit{amministratori}$_G$, che gestisce le funzionalità server-side.
    \item \textbf{Backlog}: Artefatto di \textit{Scrum}$_G$, consiste in un elenco ordinato di attività o funzionalità da sviluppare, mantenuto e aggiornato in base alle priorità del progetto.
    \item \textbf{Baseline}: Punto di riferimento fisso utilizzato per confrontare le prestazioni del progetto nel tempo, riguardando ambito, tempi e costi.
    \item \textbf{Branch}: Linea indipendente di sviluppo in un sistema di versionamento come \textit{Git}$_G$, che consente modifiche parallele al codice sorgente.
    \item \textbf{Bug}: Errore o malfunzionamento nel software, spesso dovuto a problemi nel codice sorgente.
    \item \textbf{Bug reporting}: Processo di registrazione e monitoraggio dei \textit{bug}$_G$ o errori rilevati durante i test software, con lo scopo di risolverli.
    \item \textbf{Bert}: Modello di linguaggio bidirezionale utilizzato per compiti di elaborazione del linguaggio naturale come analisi del sentimento o classificazione testuale.
    \item \textbf{Benchmarking}: Metodo per valutare le prestazioni di un sistema, come un modello \textit{LLM}$_G$, confrontandolo con standard o metriche predefinite.
\end{itemize}

\newpage
\section{C}
\begin{itemize}
    \item \textbf{CA (Customer Acceptance)}: Fase del processo di sviluppo in cui il cliente verifica e approva il prodotto o il servizio fornito, assicurandosi che soddisfi i requisiti$_G$ e le specifiche concordate prima della consegna finale o del rilascio.
    \item \textbf{Capitolato}: Documento che descrive in dettaglio lavorazioni, modalità e tempi di esecuzione per un progetto, incluso il relativo costo.
    %\item \textbf{Chat history}: Cronologia delle conversazioni precedenti con un interlocutore, utile per contestualizzare le interazioni in un \textit{chatbot}$_G$.
    \item \textbf{Chatbot}: Programma che utilizza \textit{AI}$_G$ e \textit{LLM}$_G$ per simulare conversazioni umane e automatizzare risposte.
    %\item \textbf{ChatGPT}: Chatbot sviluppato da \textit{OpenAI}$_G$, basato su modelli \textit{GPT} (Generative Pre-trained Transformer), specializzato nel dialogo umano.
    \item \textbf{ClickHouse}: Database analitico \textit{open-source}$_G$ progettato per l'elaborazione di grandi volumi di dati in tempo reale, particolarmente adatto per applicazioni che richiedono analisi veloci e scalabili.
    \item \textbf{Cloud}: Infrastruttura tecnologica che permette l'archiviazione, l'elaborazione e la distribuzione di dati e applicazioni tramite internet.
    \item \textbf{Cloud-ready}: Infrastruttura$_G$ progettata per essere compatibile e facilmente integrabile con servizi di \textit{cloud computing}$_G$, offrendo scalabilità e flessibilità per l'implementazione di applicazioni e servizi.
    \item \textbf{Code coverage}: Metodologia per misurare quanta parte del codice è stata testata durante l'esecuzione dei test software.
    \item \textbf{Commit}: Operazione in \textit{Git}$_G$ per salvare le modifiche apportate ai file in un \textit{repository}$_G$.
    \item \textbf{Committente}: Persona o organizzazione che commissiona la realizzazione di un'opera o servizio.
    \item \textbf{Configuration Management}: Processo che definisce e controlla le modifiche agli elementi di un sistema software, garantendo coerenza e tracciabilità.
\item \textbf{Container}: unità di software che raggruppa il codice di un'applicazione e tutte le sue dipendenze (librerie, configurazioni e altre risorse necessarie) affinché possa funzionare in maniera isolata in un ambiente di runtime.
    \item \textbf{Configuration Status Accounting}: Processo di documentazione e monitoraggio delle verifiche e dei cambiamenti apportati alle caratteristiche di un prodotto software, garantendo la tracciabilità e l'aggiornamento continuo.
    %\item \textbf{Consulenza}: Attività svolta da un esperto per fornire supporto e soluzioni in un determinato ambito.
    \item \textbf{Consuntivo}: Rendiconto dei risultati ottenuti in un determinato periodo o progetto.
    \item \textbf{Conversazione guidata}: Sistema con una serie di menù e sottomenù con delle opzioni
    selezionabili dall’utente che costruiscano un prompt per le esigenze principali di un
    possibile installatore.
    \item \textbf{Conversazione libera}: Sistema attraverso cui l’installatore può
    fare domande e ricevere risposte in linguaggio naturale.
    \item \textbf{Cost Performance Index (CPI)}: Indicatore di efficienza dei costi di un progetto rispetto al budget.
    \item \textbf{Cruscotto informativo}: Strumento di visualizzazione dei dati che aggrega e presenta informazioni chiave attraverso grafici, tabelle e indicatori, consentendo il monitoraggio e l'analisi delle performance di un sistema o processo in tempo reale$_G$.
    \item \textbf{CSS}: Acronimo di \textit{Cascading Style Sheets}, linguaggio usato per definire il design e la formattazione di documenti HTML.
    %\item \textbf{C\#}: Linguaggio di programmazione orientato agli oggetti sviluppato da Microsoft, principalmente utilizzato per lo sviluppo di applicazioni su Windows, con forte integrazione nell'ambiente .NET.
    
\end{itemize}

\newpage
\section*{D}  
\addcontentsline{toc}{section}{D}
\begin{itemize}
    \item \textbf{Dashboard}: Interfaccia grafica che fornisce una panoramica dei dati principali, spesso sotto forma di grafici e tabelle.
    \item \textbf{Database}: Sistema organizzato per la raccolta, la gestione e l'archiviazione di dati, che permette di memorizzare, recuperare e manipolare informazioni in modo strutturato, spesso utilizzando un sistema di gestione database (DBMS).
    \item \textbf{Database vettoriale}: Tipo di database progettato per archiviare e gestire dati geospaziali o geometrie, come punti, linee e poligoni.
    \item \textbf{Diagramma dei casi d’uso}: Rappresentazione grafica degli \textit{use case}\textsubscript{G}, che evidenzia gli \textit{attori}\textsubscript{G} e le interazioni con il sistema.
    \item \textbf{Diagramma di Gantt}: Strumento per rappresentare graficamente le attività di un progetto, indicando durata, sequenze e sovrapposizioni.
    \item \textbf{Diario di bordo}: Documento che monitora i progressi del progetto, evidenziando il rapporto tra costi sostenuti e risultati ottenuti.
    %\item \textbf{Discord}: Piattaforma di comunicazione che supporta messaggi, chiamate vocali e video, utilizzata per collaborazioni di team.
    \item \textbf{Docker}: Piattaforma \textit{open source}\textsubscript{G} per la creazione e gestione di container, garantendo la coerenza tra ambienti di sviluppo e produzione.
    \item \textbf{Docker Compose}: Strumento utilizzato per implementare il principio di Infrastructure as Code, consentendo la creazione e gestione di ambienti applicativi replicabili con un solo comando.
    \item \textbf{Documentazione}: Definisce le attività per la registrazione delle informazioni prodotte da un processo del ciclo di vita.
    \item \textbf{Drag-and-drop}: Interazione utente che consente di selezionare un oggetto (come un file o una finestra) e trascinarlo con il mouse o il touchpad in una nuova posizione, senza la necessità di utilizzare comandi da tastiera.
\end{itemize}


\newpage
\section*{E}  
\addcontentsline{toc}{section}{E}
\begin{itemize}
    \item \textbf{Earned Value (EV)}: Valore guadagnato, rappresenta il costo stimato del lavoro effettivamente completato.
    \item \textbf{Embedding}: Metodo per rappresentare parole o frasi come vettori numerici in uno spazio semantico, utile per analisi linguistiche.
    %\item \textbf{End-to-end test}: Metodo di test che verifica il funzionamento completo di un'applicazione simulando scenari realistici dal punto di vista dell'utente.
    \item \textbf{Epic}: Una raccolta di \textit{ticket}\textsubscript{G} che rappresenta obiettivi significativi di un progetto.
    \item \textbf{Estimated At Completion (EAC)}: Stima aggiornata del costo totale del progetto, basata sull'andamento attuale.
\end{itemize}

\newpage
\section{F}
\begin{itemize}
    \item 
\end{itemize}
\newpage
\section{G}
\begin{itemize}
    \item \textbf{Gantt}: Diagramma che rappresenta graficamente la pianificazione temporale delle attività di un progetto.
    \item \textbf{Git}: Sistema distribuito per il controllo di versione che consente di tracciare modifiche e collaborare su progetti software.
    \item \textbf{GitHub}: Piattaforma che integra \textit{Git}$_G$, offrendo strumenti per la collaborazione e la gestione di codice sorgente.
    \item \textbf{GitHub Action}: Strumento di automazione che consente di creare flussi di lavoro per test, compilazione e distribuzione del codice.
    \item \textbf{GitHub Runners}: Servizi utilizzati per eseguire azioni di \textit{GitHub Action}$_G$, come test automatici e analisi del codice.
    \item \textbf{Generative AI}: Tipo di intelligenza artificiale che utilizza modelli per generare nuovi dati, contenuti o soluzioni, come testo, immagini, musica o codice, basandosi su esempi precedenti e apprendendo dalle caratteristiche e pattern dei dati di addestramento.
    \item \textbf{Grafana}: Piattaforma open-source per il monitoraggio e la visualizzazione di metriche, dati di log e \textit{serie temporali$_G$}, utilizzata per creare cruscotti personalizzati e per l'analisi in tempo reale di performance e dati operativi.

\end{itemize}

\newpage
\section{H}
\begin{itemize}
    \item \textbf{Hardware}: Componenti fisici di un sistema informatico, inclusi dispositivi come processori, memoria, dischi rigidi, schede madri e periferiche, che supportano l'esecuzione di software e l'interazione con l'utente$_G$.
    \item \textbf{Hugging Face}: Piattaforma \textit{open-source}$_G$ focalizzata sullo sviluppo e la distribuzione di modelli di \textit{machine learning}$_G$ per l'elaborazione del linguaggio naturale (NLP), che offre una libreria di modelli pre-addestrati e strumenti per il training e l'implementazione$_G$.
\end{itemize}

\newpage
\section{I}
\begin{itemize}
    \item 
\end{itemize}
\newpage
\section{J}
\begin{itemize}
    \item \textbf{JavaScript}: Linguaggio di programmazione utilizzato per creare applicazioni web dinamiche, sia lato client che lato server.
    \item \textbf{JSON}: Acronimo di JavaScript Object Notation, formato di interscambio dati leggero e facilmente leggibile da macchine e umani.
    \item \textbf{Jira Software}: Sistema di tracciamento di \textit{issue}$_G$, utilizzato per il project management.
\end{itemize}

\newpage
\section{K}
\begin{itemize}
    \item \textbf{Knowledge Management}: Sistema per facilitare la creazione, la condivisione e l'archiviazione di conoscenza all'interno di un'organizzazione.
\end{itemize}

\newpage
\section{L}
\begin{itemize}
    \item \textbf{LangChain}: \textit{Framework}\textsubscript{G} per collegare modelli linguistici a fonti di dati e consentire loro di interagire con l'ambiente.
    \item \textbf{LaTeX}: Linguaggio di markup utilizzato per la composizione tipografica di documenti\textsubscript{G}.
    \item \textbf{LLM}: Acronimo di \textit{Large Language Model}, modelli addestrati su grandi quantità di dati testuali per generare linguaggio naturale.
    %\item \textbf{Log}: Registrazione cronologica delle operazioni effettuate da un \textit{sistema}\textsubscript{G}.
    \item \textbf{Llama 3.1 / Llama 3.2}: Modello di linguaggio di grandi dimensioni (\textit{LLM}\textsubscript{G}) \textit{open source}\textsubscript{G}, progettato per generare testo naturale e rispondere a domande.
    \item \textbf{LightSail Containers}: Servizio di \textit{Amazon Web Services}\textsubscript{G} che consente di eseguire applicazioni containerizzate su istanze di \textit{container}\textsubscript{G} completamente gestite, offrendo un'infrastruttura scalabile e semplice da configurare per lo sviluppo ed il deployment di microservizi ed applicazioni distribuite.
    \item \textbf{Link}: Elemento interattivo all'interno di una pagina web che, quando selezionato, consente di navigare verso un'altra pagina o risorsa, tramite un URL (Uniform Resource Locator), utilizzato per facilitare l'accesso a contenuti e \textit{risorse}\textsubscript{G} \textit{online}\textsubscript{G}.
\end{itemize}

\newpage
\section{M}
\begin{itemize}
    \item \textbf{Merge}: Comando in \textit{Git}$_G$ per unire più sequenze di modifiche in una cronologia unificata$_G$.
    \item \textbf{Milestone}: Obiettivo intermedio importante nel ciclo di vita di un progetto$_G$.
    \item \textbf{Modello di sviluppo}: Principio teorico che guida la progettazione e implementazione di un programma$_G$.
    \item \textbf{MVP}: Acronimo di Minimum Viable Product, una versione minima di un prodotto con funzionalità essenziali$_G$.
    \item \textbf{Machine Learning}: Disciplina dell'\textit{Artificial Intelligence}$_G$ che utilizza algoritmi per apprendere dai dati e migliorare le prestazioni di un sistema senza programmazione esplicita$_G$.
    \item \textbf{Mistral}: Modello di linguaggio \textit{open source}$_G$ focalizzato sull'efficienza e sulla generazione di contenuti testuali accurati$_G$.
    \item \textbf{MongoDB}: Database NoSQL orientato ai documenti, utilizzato per memorizzare dati non strutturati, che consente una gestione flessibile delle informazioni grazie alla sua architettura basata su documenti JSON-like$_G$.
    \item \textbf{Meeting}: Incontro o sessione di discussione tra membri di un team o parti interessate, finalizzato a prendere decisioni, risolvere problemi o condividere informazioni su progetti, attività o obiettivi comuni$_G$.
    \item \textbf{Milestone}: Obiettivo intermedio che segna un punto significativo nel ciclo di vita di un progetto$_G$.
    \item \textbf{Microsoft Teams}: Piattaforma$_G$ di collaborazione sviluppata da Microsoft, che consente la comunicazione tramite chat, videoconferenze e condivisione di file, integrandosi con altri strumenti della suite Microsoft 365$_G$.
\end{itemize}

\newpage
\section{N}
\begin{itemize}
    \item \textbf{Node.js}: Runtime multipiattaforma per l'esecuzione di JavaScript lato server.
        \item \textbf{NoSQL}: Tipologia di database progettati per memorizzare e gestire grandi volumi di dati non relazionali.
    \item \textbf{Notion}: Strumento per la gestione di progetti, note e attività.

\end{itemize}

\newpage
\section*{O}  
\addcontentsline{toc}{section}{O}
\begin{itemize}
    \item \textbf{OCRmyPDF}: Strumento che utilizza la tecnologia OCR (Optical Character Recognition) per estrarre testo da documenti PDF.
    %\item \textbf{Ollama}: Framework per la gestione di modelli linguistici con librerie pre-costruite per varie applicazioni\textsubscript{G}.
    \item \textbf{Online}: Riferito a qualsiasi servizio, applicazione o risorsa che è accessibile tramite Internet, permettendo interazioni e scambi di dati in tempo reale tra utenti e \textit{sistemi}\textsubscript{G}.
    \item \textbf{Open Source}: Software il cui codice sorgente è accessibile e modificabile liberamente.
    \item \textbf{OpenAI}: Laboratorio di ricerca che sviluppa \textit{intelligenze artificiali}\textsubscript{G} avanzate come i modelli GPT.
    \item \textbf{Output}: Dati o risultati prodotti da un sistema, programma o dispositivo in risposta ad un input, che possono essere visualizzati, elaborati o utilizzati per ulteriori operazioni.
\end{itemize}

\newpage
\section{P}
\begin{itemize}
    \item \textbf{Password}: Sequenza di caratteri utilizzata come mezzo di autenticazione per garantire l'accesso sicuro a sistemi, applicazioni o risorse, generalmente combinata con un nome utente per verificarne l'identità$_G$.
    \item \textbf{PB}: Acronimo di Product Baseline, punto di revisione per valutare l'avanzamento di un progetto$_G$.
    \item \textbf{Phi}: Modello \textit{LLM} specializzato per applicazioni avanzate di generazione e comprensione del linguaggio$_G$.
    \item \textbf{Pipeline}: Sequenza di fasi o processi interconnessi in un flusso di lavoro automatizzato, in cui i dati vengono elaborati passo dopo passo, consentendo l'esecuzione continua e la trasformazione dei dati attraverso diverse operazioni, come nel contesto del \textit{continuous integration} e \textit{continuous delivery}$_G$.
    \item \textbf{Planed Value (PV)}: Valore pianificato$_G$, costo stimato del lavoro previsto in un dato momento$_G$.
    \item \textbf{PoC}: \textit{Proof of Concept}, prototipo o realizzazione iniziale di un'idea o tecnologia per verificarne la fattibilità, il funzionamento o il potenziale, spesso utilizzato per testare se una soluzione può risolvere un determinato problema prima di impegnarsi in un progetto più ampio$_G$.
    \item \textbf{PostGIS}: Estensione spaziale per il database \textit{PostgreSQL}$_G$, che aggiunge il supporto per dati geografici, permettendo di eseguire query geospaziali e operazioni su dati spaziali$_G$.
    \item \textbf{PostgreSQL}: Sistema di gestione di database relazionale open-source, altamente estensibile e compatibile con SQL, che supporta funzionalità avanzate come transazioni ACID, gestione di dati non strutturati e indexing avanzato$_G$.
    \item \textbf{Problem-solving}: Processo cognitivo che implica l'identificazione, l'analisi e la risoluzione di problemi, attraverso l'uso di metodi logici, creativi e analitici, frequentemente applicato in contesti informatici e nello sviluppo di software$_G$.
    \item \textbf{Pull Request}: Proposta di modifica al codice sorgente in un sistema di controllo di versione$_G$.
    \item \textbf{Python}: Linguaggio di programmazione ad alto livello, noto per la sua sintassi semplice ed elegante, utilizzato in numerosi ambiti, tra cui lo sviluppo web, l'automazione, l'analisi dei dati e l'intelligenza artificiale$_G$.
\end{itemize}


\newpage
\section*{Q}  
\addcontentsline{toc}{section}{Q}
\begin{itemize}
    \item 
\end{itemize}
\newpage
\section{R}
\begin{itemize}
    \item 
\end{itemize}
\newpage
\section{S}
\begin{itemize}
    \item \textbf{Scrum}: Framework di \textit{Agile}$_G$ per gestire progetti complessi, basato su iterazioni e feedback continui.
    \item \textbf{SEMAT}: Sistema a sei livelli per misurare il progresso di un progetto.
    \item \textbf{Slack}: Strumento di collaborazione che consente messaggistica istantanea e gestione delle attività di team.
    \item \textbf{Sottotask}: Parte più granulare di un’attività principale, rappresentata da un \textit{ticket}$_G$.
    \item \textbf{Sprint}: Iterazione a tempo fisso di due settimane in \textit{Scrum}$_G$, durante la quale il team sviluppa un incremento del prodotto.
    \item \textbf{Stakeholder}: Persona o gruppo interessato al successo di un progetto o iniziativa.
    \item \textbf{SQL}: Acronimo di Structured Query Language, linguaggio standard per la gestione di database relazionali.
    \item \textbf{Scrapy}: Framework \textit{open source}$_G$ per il web scraping, utile per estrarre informazioni da siti web.
    \item \textbf{Swift}: Linguaggio di programmazione sviluppato da Apple, utilizzato per lo sviluppo di applicazioni iOS, macOS, watchOS e tvOS, noto per la sua velocità e sicurezza.
\item \textbf{SwiftUI}: Framework sviluppato da Apple per la costruzione di interfacce utente dichiarative per le piattaforme Apple, che semplifica il design delle UI attraverso una sintassi Swift integrata.
\item \textbf{Superset}: Strumento open-source per la visualizzazione e l'analisi dei dati, che consente di creare dashboard interattive e report complessi a partire da diverse fonti di dati.
\item \textbf{Spring Boot}: Framework open-source basato su \textit{Spring}$_G$, utilizzato per creare applicazioni Java autonome e pronte per la produzione, con configurazioni minime e una struttura predefinita che semplifica la creazione di applicazioni basate su microservizi.
\item \textbf{Software}: Insieme di programmi, dati e istruzioni che permettono a un computer di eseguire specifiche funzioni o compiti, comprese applicazioni, sistemi operativi e strumenti di sviluppo$_G$.
\item \textbf{Spring}: Framework open-source per lo sviluppo di applicazioni Java, che fornisce infrastrutture per la gestione della configurazione, la gestione delle dipendenze, la sicurezza e l'accesso ai dati, semplificando la creazione di applicazioni scalabili e modulari$_G$.

\end{itemize}

\newpage
\section{T}
\begin{itemize}
    \item \textbf{Telegram}: Applicazione di messaggistica istantanea e broadcasting$_G$.
    \item \textbf{Typescript}: Linguaggio di programmazione \textit{open source}$_G$, estensione di JavaScript con tipizzazione statica$_G$.
    \item \textbf{Timescale}: Database \textit{open-source}$_G$ progettato per la gestione di \textit{serie temporali}$_G$, utilizzato per raccogliere e analizzare dati che cambiano nel tempo, come metriche, sensori o dati IoT$_G$.
    \item \textbf{Tableau}: Software di business intelligence e \textit{data visualization}$_G$ che permette di creare report interattivi e grafici per analizzare e comprendere i dati in modo semplice e visuale$_G$.
    \item \textbf{Ticket}: Rappresentazione di un’attività, problema o funzionalità da gestire in sistemi di tracciamento come \textit{Jira}$_G$.
    \item \textbf{Task}: Un'unità di lavoro o attività da completare, solitamente assegnata a una persona o gruppo, che può essere tracciata e gestita in sistemi di gestione progetti$_G$.
    \item \textbf{TimescaleDB}: Estensione di \textit{PostgreSQL}$_G$ progettata per gestire serie temporali, consentendo l'archiviazione e l'analisi di grandi volumi di dati con timestamp, ottimizzando le query e le operazioni su dati temporali attraverso funzionalità avanzate di partizionamento e compressione$_G$.
    \item \textbf{Timeline}: Diagramma temporale$_G$ utilizzato per rappresentare scadenze, dipendenze e progressi di un progetto$_G$.
\end{itemize}

\newpage
\section{U}
\begin{itemize}
    \item \textbf{UML}: Acronimo di Unified Modeling Language, linguaggio per modellare sistemi basati sul paradigma orientato agli oggetti.
    \item \textbf{Use case}: Insieme di scenari che descrivono interazioni tra \textit{attori}$_G$ e sistema per raggiungere un obiettivo.
    \item \textbf{UI}: Acronimo di User Interface, rappresenta l'interfaccia grafica attraverso cui l'utente interagisce con un sistema.
    \item \textbf{User-friendly}: Termine che descrive un sistema, un'interfaccia o un'applicazione progettati per essere semplici da usare, intuitivi e accessibili anche per utenti con poca esperienza tecnica.
    \item \textbf{User Stories}: Descrizioni brevi e semplici di funzionalità o requisiti da parte degli utenti finali, utilizzate nel processo di sviluppo agile per definire ciò che un sistema dovrebbe fare, solitamente redatte nel formato \textit{Come} [utente], \textit{voglio} [funzionalità], \textit{così che} [beneficio]$_G$.
    \item \textbf{Username}: Identificativo univoco assegnato a un utente per accedere a un sistema, applicazione o servizio, spesso combinato con una password per autenticare l'accesso e garantire la sicurezza$_G$.

\end{itemize}

\newpage
\section{V}
\begin{itemize}
    \item 
\end{itemize}
\newpage
\section{W}
\begin{itemize}
    \item \textbf{Way of working}: Metodo organizzativo per gestire in modo professionale le attività di un progetto$_G$.
    \item \textbf{Workflow}: Sequenza di stati e transizioni che descrivono il ciclo di vita di un’attività$_G$.
    \item \textbf{Web Responsive}: Approccio al design di siti web che consente la visualizzazione ottimale dei contenuti su una vasta gamma di dispositivi e risoluzioni, adattando il layout e le immagini in base alla dimensione dello schermo per garantire una buona esperienza utente$_G$.
    \item \textbf{Web Scraping}: Tecnica di estrazione automatica di dati da siti web, utilizzando script o software per raccogliere informazioni strutturate da pagine web, comunemente impiegata per l'analisi dei dati e l'automazione di processi di raccolta$_G$.
\end{itemize}

\newpage
\section*{X}  
\addcontentsline{toc}{section}{X}
\begin{itemize}
    \item 
\end{itemize}
\newpage
\section*{Y}  
\addcontentsline{toc}{section}{Y}
\begin{itemize}
    \item 
\end{itemize}
\newpage
\section{Z}
\begin{itemize}
    \item 
\end{itemize}
\newpage


\end{document}
