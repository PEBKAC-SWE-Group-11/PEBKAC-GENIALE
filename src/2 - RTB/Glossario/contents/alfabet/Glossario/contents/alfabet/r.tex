\section{R} 
\begin{itemize}
\item \textbf{RAG}: Acronimo di \textit{Retrieval-Augmented Generation}, tecnica utilizzata per migliorare i modelli 
\textit{LLM}\textsubscript{G} combinando la generazione di testo con la ricerca di informazioni pertinenti da una base di conoscenza esterna. 
\item \textbf{React}: Libreria JavaScript \textit{open source}\textsubscript{G} utilizzata per creare interfacce utente interattive. 
\item \textbf{Release Management}: Processo di pianificazione, coordinamento e gestione del rilascio di nuove versioni di un prodotto, al fine di garantire che la distribuzione del \textit{software}\textsubscript{G} avvenga in modo controllato, riducendo i rischi ed assicurando la qualità del prodotto rilasciato.
\item \textbf{Repository}: Archivio digitale che conserva codice sorgente, file e documenti relativi ad un progetto \textit{software}\textsubscript{G}.
\item \textbf{Requisito}: Descrizione di ciò che un \textit{sistema}\textsubscript{G} deve fare o vincoli che deve rispettare.
\item \textbf{Responsabile}: Figura che guida il team di progetto, pianificando attività e gestendo le relazioni esterne.
\item \textbf{Risorse}: Qualsiasi elemento utilizzato dal \textit{sistema}\textsubscript{G} per eseguire operazioni o attività.
\item \textbf{REST}: Acronimo di \textit{Representational State Transfer}, stile architetturale per la progettazione di servizi web basati su \textit{risorse}\textsubscript{G}.
\item \textbf{RTB}: Acronimo di \textit{Requirements and Technology Baseline}, revisione iniziale per fissare requisiti e tecnologie di un progetto.
\item \textbf{Retrieval}: Processo di recupero di informazioni da un archivio o \textit{database}\textsubscript{G}, basato su query o richieste, utilizzato in vari contesti, tra cui la ricerca di dati, la gestione delle informazioni e l'elaborazione del linguaggio naturale (NLP).
\item \textbf{Ruolo}: Incarico assunto da un membro del team, come \textit{responsabile}\textsubscript{G}, \textit{amministratore}\textsubscript{G}, o \textit{analista}\textsubscript{G}.
\end{itemize}