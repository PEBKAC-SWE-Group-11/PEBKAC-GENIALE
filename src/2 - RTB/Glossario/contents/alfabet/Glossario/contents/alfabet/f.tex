\section{F}
\begin{itemize}
    \item \textbf{Feedback}: Risposta o commento ricevuto in seguito ad un'azione, processo o comportamento, utilizzato per valutare, migliorare o ottimizzare sistemi, prodotti o performance, spesso applicato in contesti di sviluppo \textit{software}\textsubscript{G} o interazione utente.
    \item \textbf{Fine-Tuning}: Processo di ottimizzazione di un modello pre-addestrato (\textit{LLM}\textsubscript{G}) per adattarlo a compiti specifici tramite ulteriori cicli di addestramento con un dataset mirato.
    \item \textbf{FLow}: Piattaforma \textit{open source}\textsubscript{G} per la gestione e l'orchestrazione di flussi di lavoro, utilizzata per integrare modelli di linguaggio con altre applicazioni e processi.
    \item \textbf{Framework}: Insieme di strumenti e librerie che forniscono una struttura di base per lo sviluppo \textit{software}\textsubscript{G}.
    \item \textbf{Framework Pytest}:  \textit{Framework}\textsubscript{G} per il testing in \textit{Python}\textsubscript{G}, che supporta la scrittura di \textit{test}\textsubscript{G} semplici e scalabili, con funzionalità avanzate come fixture, marker e gestione parametrizzata dei \textit{test}\textsubscript{G}.
    
    \item \textbf{Frontend}: Parte visibile di un'applicazione con cui l'utente interagisce direttamente, composta da interfacce grafiche.
\end{itemize}
