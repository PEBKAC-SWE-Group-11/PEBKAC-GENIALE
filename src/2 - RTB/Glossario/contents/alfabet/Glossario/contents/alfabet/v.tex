\section{V}
\begin{itemize}
    %\item \textbf{V model}: \textit{Modello di sviluppo}\textsubscript{G} \textit{software}\textsubscript{G} che correla ogni fase di progettazione con la rispettiva fase di \textit{test}\textsubscript{G}.
    \item \textbf{Validazione}: Definisce le attività per validare il prodotto \textit{software}\textsubscript{G}.
    \item \textbf{Verifica}: Definisce le attività per verificare il prodotto \textit{software}\textsubscript{G}.
    \item \textbf{Verificatore}: Figura incaricata di controllare la coerenza del lavoro svolto rispetto ai \textit{requisiti}\textsubscript{G} ed alla \textit{documentazione}\textsubscript{G}.
    \item \textbf{Vimar}: Azienda proponente del \textit{capitolato}\textsubscript{G} "Vimar Geniale", che mette a disposizione le sue competenze per sviluppare soluzioni innovative.
    \item \textbf{VueJS}: \textit{Framework}\textsubscript{G} JavaScript \textit{open source}\textsubscript{G} utilizzato per la costruzione di interfacce utente ed applicazioni web single-page, che si distingue per la sua reattività, semplicità e capacità di integrare facilmente componenti modulari.
\end{itemize}
