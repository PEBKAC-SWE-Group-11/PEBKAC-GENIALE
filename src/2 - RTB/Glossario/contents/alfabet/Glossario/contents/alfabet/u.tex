\section{U}
\begin{itemize}
    \item \textbf{UI}: Acronimo di \textit{User Interface}\textsubscript{G}, rappresenta l'interfaccia grafica attraverso cui l'utente interagisce con un \textit{sistema}\textsubscript{G}.
    \item \textbf{UML}: Acronimo di \textit{Unified Modeling Language}\textsubscript{G}, linguaggio per modellare sistemi basati sul paradigma orientato agli oggetti.
    \item \textbf{Use case}: Insieme di scenari che descrivono interazioni tra \textit{attori}\textsubscript{G} e \textit{sistema}\textsubscript{G} per raggiungere un obiettivo.
    \item \textbf{User-friendly}: Termine che descrive un \textit{sistema}\textsubscript{G}, un'interfaccia o un'applicazione progettati per essere semplici da usare, intuitivi ed accessibili anche per utenti con poca esperienza tecnica.
    \item \textbf{User Stories}: Descrizioni brevi e semplici di funzionalità o requisiti da parte degli utenti finali, utilizzate nel processo di sviluppo \textit{Agile}\textsubscript{G} per definire ciò che un \textit{sistema}\textsubscript{G} dovrebbe fare, solitamente redatte nel formato \textit{Come} [utente], \textit{voglio} [funzionalità], \textit{così che} [beneficio].
    \item \textbf{Username}: Identificativo univoco assegnato ad un utente per accedere ad un \textit{sistema}\textsubscript{G}, applicazione o servizio, spesso combinato con una \textit{password}\textsubscript{G} per autenticare l'accesso e garantire la sicurezza.
\end{itemize}
