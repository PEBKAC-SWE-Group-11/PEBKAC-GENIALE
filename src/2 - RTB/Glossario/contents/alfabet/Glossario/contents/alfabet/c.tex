\section{C}
\begin{itemize}
    \item \textbf{CA (Customer Acceptance)}: Fase del processo di sviluppo in cui il cliente verifica ed approva il prodotto o il servizio fornito, assicurandosi che soddisfi i \textit{requisiti}\textsubscript{G} e le specifiche concordate prima della consegna finale o del rilascio.
    \item \textbf{Capitolato}: Documento che descrive in dettaglio lavorazioni, modalità e tempi di esecuzione per un progetto, incluso il relativo costo.
    %\item \textbf{Chat history}: Cronologia delle conversazioni precedenti con un interlocutore, utile per contestualizzare le interazioni in un \textit{chatbot}\textsubscript{G}.
    \item \textbf{Chatbot}: Programma che utilizza \textit{AI}\textsubscript{G} e \textit{LLM}\textsubscript{G} per simulare conversazioni umane ed automatizzare risposte.
    %\item \textbf{ChatGPT}: \textit{Chatbot}\textsubscript{G} sviluppato da \textit{OpenAI}\textsubscript{G}, basato su modelli \textit{GPT} (Generative Pre-trained Transformer), specializzato nel dialogo umano.
    \item \textbf{ClickHouse}: \textit{Database}\textsubscript{G} analitico \textit{open source}\textsubscript{G} progettato per l'elaborazione di grandi volumi di dati in tempo reale, particolarmente adatto per applicazioni che richiedono analisi veloci e scalabili.
    \item \textbf{Cloud}: Infrastruttura tecnologica che permette l'archiviazione, l'elaborazione e la distribuzione di dati ed applicazioni tramite internet.
    \item \textbf{Cloud-ready}: Infrastruttura progettata per essere compatibile e facilmente integrabile con servizi di cloud computing, offrendo scalabilità e flessibilità per l'implementazione di applicazioni e servizi.
    \item \textbf{Code coverage}: Metodologia per misurare quanta parte del codice è stata testata durante l'esecuzione dei \textit{test}\textsubscript{G} \textit{software}\textsubscript{G}.
    \item \textbf{Commit}: Operazione in \textit{Git}\textsubscript{G} per salvare le modifiche apportate ai file in un \textit{repository}\textsubscript{G}.
    \item \textbf{Committente}: Persona o organizzazione che commissiona la realizzazione di un'opera o servizio.
    \item \textbf{Configuration Management}: Processo che definisce e controlla le modifiche agli elementi di un \textit{sistema}\textsubscript{G} \textit{software}\textsubscript{G}, garantendo coerenza e tracciabilità.
    \item \textbf{Container}: Unità di \textit{software}\textsubscript{G} che raggruppa il codice di un'applicazione e tutte le sue dipendenze (librerie, configurazioni ed altre \textit{risorse}\textsubscript{G} necessarie) affinché possa funzionare in maniera isolata in un ambiente di runtime.
    \item \textbf{Configuration Status Accounting}: Processo di \textit{documentazione}\textsubscript{G} e monitoraggio delle verifiche e dei cambiamenti apportati alle caratteristiche di un prodotto \textit{software}\textsubscript{G}, garantendo la tracciabilità e l'aggiornamento continuo.
    %\item \textbf{Consulenza}: Attività svolta da un esperto per fornire supporto e soluzioni in un determinato ambito.
    \item \textbf{Consuntivo}: Rendiconto dei risultati ottenuti in un determinato periodo o progetto.
    \item \textbf{Conversazione guidata}: \textit{Sistema}\textsubscript{G} con una serie di menù e sottomenù con delle opzioni selezionabili dall’utente che costruiscano un prompt per le esigenze principali di un possibile installatore.
    \item \textbf{Conversazione libera}: \textit{Sistema}\textsubscript{G} attraverso cui l’installatore può fare domande e ricevere risposte in linguaggio naturale.
    \item \textbf{Cost Performance Index (CPI)}: Indicatore di efficienza dei costi di un progetto rispetto al \textit{budget}\textsubscript{G}.
    \item \textbf{Cruscotto informativo}: Strumento di visualizzazione dei dati che aggrega e presenta informazioni chiave attraverso grafici, tabelle ed indicatori, consentendo il monitoraggio e l'analisi delle performance di un \textit{sistema}\textsubscript{G} o processo in tempo reale.
    \item \textbf{CSS}: Acronimo di \textit{Cascading Style Sheets}, linguaggio usato per definire il design e la formattazione di documenti HTML.
    %\item \textbf{C\#}: Linguaggio di programmazione orientato agli oggetti sviluppato da Microsoft, principalmente utilizzato per lo sviluppo di applicazioni su Windows, con forte integrazione nell'ambiente .NET.
    
\end{itemize}
