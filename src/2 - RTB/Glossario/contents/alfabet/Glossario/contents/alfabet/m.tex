\section{M}
\begin{itemize}
    \item \textbf{Merge}: Comando in \textit{Git}\textsubscript{G} per unire più sequenze di modifiche in una cronologia unificata.
    \item \textbf{Modello di sviluppo}: Principio teorico che guida la progettazione ed implementazione di un programma.
    %\item \textbf{MVP}: Acronimo di \textit{Minimum Viable Product}, una versione minima di un prodotto con funzionalità essenziali\textsubscript{G}.
    \item \textbf{Machine Learning}: Disciplina dell'\textit{Artificial Intelligence}\textsubscript{G} che utilizza algoritmi per apprendere dai dati e migliorare le prestazioni di un \textit{sistema}\textsubscript{G} senza programmazione esplicita.
    \item \textbf{Mistral}: Modello di linguaggio \textit{open source}\textsubscript{G} focalizzato sull'efficienza e sulla generazione di contenuti testuali accurati.
    \item \textbf{MongoDB}: \textit{Database}\textsubscript{G} \textit{NoSQL}\textsubscript{G} orientato ai documenti, utilizzato per memorizzare dati non strutturati, che consente una gestione flessibile delle informazioni grazie alla sua architettura basata su documenti \textit{JSON}\textsubscript{G}-like.
    \item \textbf{Meeting}: Incontro o sessione di discussione tra membri di un team o parti interessate, finalizzato a prendere decisioni, risolvere problemi o condividere informazioni su progetti, attività o obiettivi comuni.
    \item \textbf{Milestone}: Obiettivo intermedio che segna un punto significativo nel ciclo di vita di un progetto.
    \item \textbf{Microsoft Teams}: Piattaforma di collaborazione sviluppata da Microsoft, che consente la comunicazione tramite chat, videoconferenze e condivisione di file, integrandosi con altri strumenti della suite Microsoft 365.
\end{itemize}
