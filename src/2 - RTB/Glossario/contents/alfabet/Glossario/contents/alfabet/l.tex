\section{L}
\begin{itemize}
    \item \textbf{LangChain}: \textit{Framework}\textsubscript{G} per collegare modelli linguistici a fonti di dati e consentire loro di interagire con l'ambiente.
    \item \textbf{LaTeX}: Linguaggio di markup utilizzato per la composizione tipografica di documenti\textsubscript{G}.
    \item \textbf{LLM}: Acronimo di \textit{Large Language Model}, modelli addestrati su grandi quantità di dati testuali per generare linguaggio naturale.
    %\item \textbf{Log}: Registrazione cronologica delle operazioni effettuate da un \textit{sistema}\textsubscript{G}.
    \item \textbf{Llama 3.1 / Llama 3.2}: Modello di linguaggio di grandi dimensioni (\textit{LLM}\textsubscript{G}) \textit{open source}\textsubscript{G}, progettato per generare testo naturale e rispondere a domande.
    \item \textbf{LightSail Containers}: Servizio di \textit{Amazon Web Services}\textsubscript{G} che consente di eseguire applicazioni containerizzate su istanze di \textit{container}\textsubscript{G} completamente gestite, offrendo un'infrastruttura scalabile e semplice da configurare per lo sviluppo ed il deployment di microservizi ed applicazioni distribuite.
    \item \textbf{Link}: Elemento interattivo all'interno di una pagina web che, quando selezionato, consente di navigare verso un'altra pagina o risorsa, tramite un URL (Uniform Resource Locator), utilizzato per facilitare l'accesso a contenuti e \textit{risorse}\textsubscript{G} \textit{online}\textsubscript{G}.
\end{itemize}
