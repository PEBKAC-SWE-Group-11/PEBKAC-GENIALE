\section{G}
\begin{itemize}
    \item \textbf{Gantt}: Diagramma che rappresenta graficamente la pianificazione temporale delle attività di un progetto.
    \item \textbf{Generative AI}: Tipo di \textit{intelligenza artificiale\textsubscript{G}} che utilizza modelli per generare nuovi dati, contenuti o soluzioni, come testo, immagini, musica o codice, basandosi su esempi precedenti ed apprendendo dalle caratteristiche e pattern dei dati di addestramento.
    \item \textbf{Git}: \textit{Sistema}\textsubscript{G} distribuito per il controllo di versione che consente di tracciare modifiche e collaborare su progetti \textit{software}\textsubscript{G}.
    \item \textbf{GitHub}: Piattaforma che integra \textit{Git}\textsubscript{G}, offrendo strumenti per la collaborazione e la gestione di codice sorgente.
    \item \textbf{GitHub Action}: Strumento di \textit{automazione}\textsubscript{G} che consente di creare flussi di lavoro per \textit{test}\textsubscript{G}, compilazione e distribuzione del codice.
    \item \textbf{GitHub Runners}: Servizi utilizzati per eseguire azioni di \textit{GitHub Action}\textsubscript{G}, come \textit{test}\textsubscript{G} automatici ed analisi del codice.
    \item \textbf{Google Sheets}: Applicazione di foglio di calcolo basata su \textit{cloud}\textsubscript{G}, parte della suite Google Workspace, che permette la creazione, modifica e collaborazione in tempo reale su documenti di calcolo.
    \item \textbf{Grafana}: Piattaforma \textit{open source}\textsubscript{G} per il monitoraggio e la visualizzazione di metriche, dati di log e serie temporali, utilizzata per creare cruscotti personalizzati e per l'analisi in tempo reale di performance e dati operativi.
    \item \textbf{GUI (Graphical User Interface)}: Interfaccia utente grafica che consente l'interazione con un \textit{sistema}\textsubscript{G} informatico attraverso elementi visivi come finestre, icone, menu e pulsanti, rendendo l'uso del \textit{software}\textsubscript{G} più intuitivo ed accessibile.

\end{itemize}
