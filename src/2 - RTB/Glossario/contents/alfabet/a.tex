\section{A}
\begin{itemize}
    \item \textbf{Agile}: \textit{Modello di sviluppo}$_G$ che privilegia la flessibilità e la comunicazione efficace sia all'interno del team che con lo \textit{stakeholder}$_G$, ponendo maggiore enfasi su un prodotto funzionante rispetto a una documentazione completa, e favorendo l'adattamento ai cambiamenti piuttosto che l'aderenza rigorosa a un piano prestabilito.
    \item \textbf{AI}: Acronimo di  \textit{Artificial Intelligence}, disciplina che studia come progettare \textit{sistemi intelligenti}$_G$ in grado di emulare il ragionamento e il comportamento umano.
    \item \textbf{Amministratore}: Figura che si occupa della configurazione e manutenzione dell'ambiente IT di lavoro, selezionando e implementando le \textit{risorse}$_G$ necessarie per supportare il \textit{way of working}$_G$.
    \item \textbf{Analista}: Persona esperta nel dominio del problema che identifica i \textit{requisiti}$_G$ necessari per avviare una progettazione efficiente e mirata.
    \item \textbf{Angular}: \textit{Framework}$_G$ \textit{open source}$_G$ utilizzato per creare applicazioni web dinamiche e reattive, sfruttando componenti modulari e strumenti avanzati per lo sviluppo front-end.
    \item \textbf{API}: Acronimo di Application Programming Interface, indica un insieme di procedure e regole per permettere la comunicazione tra \textit{sistemi software}$_G$ diversi o tra componenti di uno stesso sistema.
    \item \textbf{Attore}: Nel contesto degli \textit{use case}$_G$, rappresenta il \textit{ruolo}$_G$ svolto da un utente o sistema che interagisce con il software.
    \item \textbf{Automazione}: Processo di sostituzione delle attività manuali con istruzioni che consentono l'esecuzione automatica di operazioni ripetitive sui \textit{sistemi informatici}$_G$.
    \item \textbf{Amazon Q}: Servizio di \textit{cloud computing}$_G$ che supporta lo sviluppo software, fornendo strumenti per il testing e l’analisi.
    \item \textbf{AWS}: \textit{Amazon Web Services}, piattaforma di servizi cloud che offre infrastruttura IT scalabile, sicura e altamente disponibile, utilizzata per ospitare applicazioni, gestire storage, e fornire risorse per il machine learning e l'intelligenza artificiale.
    \item \textbf{Apache Airflow}: Piattaforma open-source per la gestione e l'automazione di flussi di lavoro, utilizzata per orchestrare e monitorare processi complessi in ambito di \textit{data engineering$_G$} e \textit{machine learning$_G$}.
    \item \textbf{Apache NiFi}: Strumento open-source per lo \textit{stream processing$_G$}, che consente di automatizzare il flusso di dati tra sistemi eterogenei, garantendo il controllo e la gestione dei flussi in tempo reale.
\end{itemize}
