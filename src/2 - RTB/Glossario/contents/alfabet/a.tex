\section{A}
\begin{itemize}
    \item \textbf{Accertamento di qualità}: Definisce le attività per assicurare in modo oggettivo che i prodotti e i processi \textit{software}\textsubscript{G} siano conformi ai \textit{requisiti}\textsubscript{G} specificati.
    \item \textbf{Actual Cost (AC)}: Costo effettivo, rappresenta la spesa reale sostenuta per completare il lavoro fino ad un dato momento.
    \item \textbf{Agile}: \textit{Modello di sviluppo}\textsubscript{G} che privilegia la flessibilità e la comunicazione efficace sia all'interno del team che con lo \textit{stakeholder}\textsubscript{G}, ponendo maggiore enfasi su un prodotto funzionante rispetto ad una documentazione completa, e favorendo l'adattamento ai cambiamenti piuttosto che l'aderenza rigorosa ad un piano prestabilito.
    \item \textbf{AI}: Acronimo di \textit{Artificial Intelligence}, disciplina che studia come progettare sistemi intelligenti in grado di emulare il ragionamento ed il comportamento umano.
     \item \textbf{Amazon Q}: Servizio di cloud computing che supporta lo sviluppo software, fornendo strumenti per il testing e l’analisi.
    \item \textbf{Amministratore}: Figura che si occupa della configurazione e manutenzione dell'ambiente IT di lavoro, selezionando ed implementando le \textit{risorse}\textsubscript{G} necessarie per supportare il \textit{way of working}\textsubscript{G}.
    \item \textbf{Analista}: Persona esperta nel dominio del problema che identifica i \textit{requisiti}\textsubscript{G} necessari per avviare una progettazione efficiente e mirata.
    \item \textbf{Angular}: \textit{Framework}\textsubscript{G} \textit{open source}\textsubscript{G} utilizzato per creare applicazioni web dinamiche e reattive, sfruttando componenti modulari e strumenti avanzati per lo sviluppo frontend.
    \item \textbf{API}: Acronimo di \textit{Application Programming Interface}, indica un insieme di procedure e regole per permettere la comunicazione tra sistemi \textit{software}\textsubscript{G} diversi o tra componenti di uno stesso sistema.
    \item \textbf{Applicativo server}: \textit{Software}\textsubscript{G} progettato per essere eseguito su un server, responsabile di gestire richieste da client, elaborare dati e fornire risposte adeguate attraverso una rete.
    %\item \textbf{Applicazioni distribuite}: Sistemi software in cui i componenti sono eseguiti su nodi diversi all'interno di una rete, che collaborano tra loro per raggiungere obiettivi comuni, offrendo scalabilità, tolleranza ai guasti e alta disponibilità\textsubscript{G}.
    \item \textbf{AS-IS / TO-BE}: Metodologia utilizzata per analizzare e rappresentare la situazione attuale (\textit{AS-IS}) e la situazione futura desiderata (\textit{TO-BE}) in un processo di cambiamento o implementazione di un prodotto.
    \item \textbf{Atlassian}: Azienda specializzata nello sviluppo di strumenti software per il project management e la collaborazione, come \textit{Jira}\textsubscript{G} e Confluence.
    \item \textbf{Attore}: Nel contesto degli \textit{use case}\textsubscript{G}, rappresenta il ruolo svolto da un utente o sistema che interagisce con il software.
    \item \textbf{Audit trail}: Registro dettagliato delle attività svolte su un sistema per garantire la tracciabilità e l’integrità.
    \item \textbf{Automazione}: Processo di sostituzione delle attività manuali con istruzioni che consentono l'esecuzione automatica di operazioni ripetitive sui \textit{sistemi}\textsubscript{G}. informatici
    \item \textbf{AWS}: \textit{Amazon Web Services}, piattaforma di servizi \textit{cloud}\textsubscript{G} che offre un'infrastruttura IT scalabile, sicura e altamente disponibile.
    \item \textbf{AWS EC2}: Servizio di \textit{Amazon Web Services}\textsubscript{G} che fornisce capacità di calcolo scalabile nel \textit{cloud}\textsubscript{G}, consentendo di lanciare e gestire istanze virtuali.
    %\item \textbf{AWS LightSail}: Servizio di \textit{cloud computing}\textsubscript{G} offerto da \textit{Amazon Web Services}\textsubscript{G}, che fornisce soluzioni semplici per applicazioni e siti web.
\end{itemize}
