\section{A}
\begin{itemize}
    \item \textbf{Accertamento di qualità}: Definisce le attività per assicurare in modo oggettivo che i prodotti e i processi software$_G$ siano conformi ai requisiti$_G$ specificati.
    \item \textbf{Actual Cost (AC)}: Costo effettivo$_G$, rappresenta la spesa reale sostenuta per completare il lavoro fino a un dato momento.
    \item \textbf{Agile}: \textit{Modello di sviluppo}$_G$ che privilegia la flessibilità e la comunicazione efficace sia all'interno del team che con lo \textit{stakeholder}$_G$, ponendo maggiore enfasi su un prodotto funzionante rispetto a una documentazione completa, e favorendo l'adattamento ai cambiamenti piuttosto che l'aderenza rigorosa a un piano prestabilito.
    \item \textbf{AI}: Acronimo di \textit{Artificial Intelligence}, disciplina che studia come progettare \textit{sistemi intelligenti}$_G$ in grado di emulare il ragionamento e il comportamento umano.
    \item \textbf{Amministratore}: Figura che si occupa della configurazione e manutenzione dell'ambiente IT di lavoro, selezionando e implementando le risorse$_G$ necessarie per supportare il \textit{way of working}$_G$.
    \item \textbf{Amazon Q}: Servizio di \textit{cloud computing}$_G$ che supporta lo sviluppo software, fornendo strumenti per il testing e l’analisi.
    \item \textbf{Angular}: \textit{Framework}$_G$ \textit{open source}$_G$ utilizzato per creare applicazioni web dinamiche e reattive, sfruttando componenti modulari e strumenti avanzati per lo sviluppo front-end.
    \item \textbf{Analista}: Persona esperta nel dominio del problema che identifica i requisiti$_G$ necessari per avviare una progettazione efficiente e mirata.
    \item \textbf{API}: Acronimo di \textit{Application Programming Interface}, indica un insieme di procedure e regole per permettere la comunicazione tra \textit{sistemi software}$_G$ diversi o tra componenti di uno stesso sistema.
    \item \textbf{Applicazioni distribuite}: Sistemi software in cui i componenti sono eseguiti su nodi diversi all'interno di una rete, che collaborano tra loro per raggiungere obiettivi comuni, offrendo scalabilità, tolleranza ai guasti e alta disponibilità$_G$.
    \item \textbf{Applicativo server}: Software$_G$ progettato per essere eseguito su un server, responsabile di gestire richieste da client, elaborare dati e fornire risposte adeguate attraverso una rete.
    \item \textbf{AS-IS / TO-BE}: Metodologia utilizzata per analizzare e rappresentare la situazione attuale (\textit{AS-IS}) e la situazione futura desiderata (\textit{TO-BE}) in un processo di cambiamento o implementazione di un prodotto.
    \item \textbf{Atlassian}: Azienda specializzata nello sviluppo di strumenti software per il \textit{project management}$_G$ e la collaborazione, come Jira$_G$ e Confluence.
    \item \textbf{Attore}: Nel contesto degli \textit{use case}$_G$, rappresenta il ruolo$_G$ svolto da un utente o sistema che interagisce con il software.
    \item \textbf{Audit trail}: Registro dettagliato delle attività svolte su un sistema per garantire la tracciabilità e l’integrità.
    \item \textbf{Automazione}: Processo di sostituzione delle attività manuali con istruzioni che consentono l'esecuzione automatica di operazioni ripetitive sui sistemi informatici$_G$.
    \item \textbf{AWS}: \textit{Amazon Web Services}, piattaforma di servizi \textit{cloud}$_G$ che offre infrastruttura IT scalabile, sicura e altamente disponibile.
    \item \textbf{AWS EC2}: Servizio di \textit{Amazon Web Services}$_G$ che fornisce capacità di calcolo scalabile nel \textit{cloud}$_G$, consentendo di lanciare e gestire istanze virtuali.
    \item \textbf{AWS LightSail}: Servizio di \textit{cloud computing}$_G$ offerto da \textit{Amazon Web Services}$_G$, che fornisce soluzioni semplici per applicazioni e siti web.
\end{itemize}
