\section{I}
\begin{itemize}
    \item \textbf{Issue}: Un'unità di lavoro in un sistema di tracciamento che può rappresentare un problema, una richiesta di funzionalità o un'attività.
    \item \textbf{ITS}: Acronimo di Issue Tracking System, un software utilizzato per gestire e monitorare problemi o richieste in un progetto.
    \item \textbf{InfluxDB}: Database open-source specializzato nella gestione di serie temporali, progettato per raccogliere, archiviare e analizzare grandi volumi di dati con timestamp, utilizzato principalmente per applicazioni di monitoraggio e IoT$_G$.
    \item \textbf{Interfaccia web}: Pannello di controllo o applicazione accessibile tramite un browser, che consente agli utenti di interagire con un sistema o servizio online, utilizzando elementi grafici e interattivi come moduli, pulsanti e menù$_G$.
    \item \textbf{IoT (Internet of Things)}: Rete di dispositivi fisici interconnessi, come sensori, attuatori e apparecchiature, che comunicano e scambiano dati tra loro tramite Internet, permettendo l'automazione e il monitoraggio remoto in tempo reale$_G$.
    \item \textbf{Input}: Dati, comandi o informazioni forniti a un sistema, programma o dispositivo, che vengono elaborati per generare un output o per avviare un'azione specifica$_G$.

\end{itemize}
