\section{P}
\begin{itemize}
    \item \textbf{Password}: Sequenza di caratteri utilizzata come mezzo di autenticazione per garantire l'accesso sicuro a sistemi, applicazioni o risorse, generalmente combinata con un nome utente per verificarne l'identità$_G$.
    \item \textbf{PB}: Acronimo di Product Baseline, punto di revisione per valutare l'avanzamento di un progetto$_G$.
    \item \textbf{Phi}: Modello \textit{LLM} specializzato per applicazioni avanzate di generazione e comprensione del linguaggio$_G$.
    \item \textbf{Pipeline}: Sequenza di fasi o processi interconnessi in un flusso di lavoro automatizzato, in cui i dati vengono elaborati passo dopo passo, consentendo l'esecuzione continua e la trasformazione dei dati attraverso diverse operazioni, come nel contesto del \textit{continuous integration} e \textit{continuous delivery}$_G$.
    \item \textbf{Planed Value (PV)}: Valore pianificato$_G$, costo stimato del lavoro previsto in un dato momento$_G$.
    \item \textbf{PoC}: \textit{Proof of Concept}, prototipo o realizzazione iniziale di un'idea o tecnologia per verificarne la fattibilità, il funzionamento o il potenziale, spesso utilizzato per testare se una soluzione può risolvere un determinato problema prima di impegnarsi in un progetto più ampio$_G$.
    \item \textbf{PostGIS}: Estensione spaziale per il database \textit{PostgreSQL}$_G$, che aggiunge il supporto per dati geografici, permettendo di eseguire query geospaziali e operazioni su dati spaziali$_G$.
    \item \textbf{PostgreSQL}: Sistema di gestione di database relazionale open-source, altamente estensibile e compatibile con SQL, che supporta funzionalità avanzate come transazioni ACID, gestione di dati non strutturati e indexing avanzato$_G$.
    \item \textbf{Problem-solving}: Processo cognitivo che implica l'identificazione, l'analisi e la risoluzione di problemi, attraverso l'uso di metodi logici, creativi e analitici, frequentemente applicato in contesti informatici e nello sviluppo di software$_G$.
    \item \textbf{Pull Request}: Proposta di modifica al codice sorgente in un sistema di controllo di versione$_G$.
    \item \textbf{Python}: Linguaggio di programmazione ad alto livello, noto per la sua sintassi semplice ed elegante, utilizzato in numerosi ambiti, tra cui lo sviluppo web, l'automazione, l'analisi dei dati e l'intelligenza artificiale$_G$.
\end{itemize}

