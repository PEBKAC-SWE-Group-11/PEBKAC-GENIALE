\section{P}
\begin{itemize}
    \item \textbf{PB}: Acronimo di Product Baseline, punto di revisione per valutare l'avanzamento di un progetto.
    \item \textbf{PDCA}: Ciclo Plan-Do-Check-Act, metodo per migliorare i processi.
    \item \textbf{Pull Request}: Proposta di modifica al codice sorgente in un sistema di controllo di versione.
    \item \textbf{Phi}: Modello \textit{LLM} specializzato per applicazioni avanzate di generazione e comprensione del linguaggio.
    \item \textbf{PoC}: \textit{Proof of Concept}, prototipo o realizzazione iniziale di un'idea o tecnologia per verificarne la fattibilità, il funzionamento o il potenziale, spesso utilizzato per testare se una soluzione può risolvere un determinato problema prima di impegnarsi in un progetto più ampio.
    \item \textbf{Python}: Linguaggio di programmazione ad alto livello, noto per la sua sintassi semplice ed elegante, utilizzato in numerosi ambiti, tra cui lo sviluppo web, l'automazione, l'analisi dei dati e l'intelligenza artificiale.
    \item \textbf{PostGIS}: Estensione spaziale per il database \textit{PostgreSQL$_G$}, che aggiunge il supporto per dati geografici, permettendo di eseguire query geospaziali e operazioni su dati spaziali.
    \item \textbf{Problem-solving}: Processo cognitivo che implica l'identificazione, l'analisi e la risoluzione di problemi, attraverso l'uso di metodi logici, creativi e analitici, frequentemente applicato in contesti informatici e nello sviluppo di software$_G$.

\end{itemize}
