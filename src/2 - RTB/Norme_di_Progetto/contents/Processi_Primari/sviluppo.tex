\subsection{Sviluppo}
\subsubsection{Scopo}
Il processo di sviluppo rappresenta la serie di attività svolte dal team PEBKAC al fine di implementare il prodotto \textit{software}\textsubscript{G}, rispettando le scadenze e i \textit{requisiti}\textsubscript{G} concordati col Proponente. 
Il processo è suddiviso nelle seguenti attività:
\begin{itemize}
    \item Analisi dei requisiti,
    \item Progettazione;
    \item Codifica;
    \item Testing;
    \item Integrazione \textit{software}\textsubscript{G}.
\end{itemize}

\subsubsection{Analisi dei Requisiti}
\subsubsubsection{Scopo}
Lo scopo dell'analisi dei requisiti è comprendere e definire in modo chiaro e completo le necessità e le aspettative del Proponente e degli utenti relativamente al prodotto \textit{software}\textsubscript{G}.
\subsubsubsection{Implementazione}
L'analisi dei requisiti, raccolta nel documento Analisi dei Requisiti V1.0.0, viene svolta secondo le seguenti fasi:
\begin{enumerate}
    \item Studio del \textit{capitolato}\textsubscript{G} e delle esigenze del Proponente;
    \item Individuazione dei casi d'uso e dei \textit{requisiti}\textsubscript{G};
    \item Confronto con il Proponente su quanto prodotto;
    \item Divisione dei \textit{requisiti}\textsubscript{G} nelle categorie individuate e applicazione dei quanto emerso nella discussione col Proponente.
\end{enumerate}

L'attività di analisi può essere svolta in modo incrementale, quindi le sue fasi possono essere svolte più volte durante lo sviluppo del progetto. 
\\
L'Analisi dei Requisiti V1.0.0 contiene:
\begin{itemize}
    \item \textbf{Introduzione}: descrive lo scopo del documento, del prodotto e i riferimenti utilizzati;
    \item \textbf{Descrizione}: esplicita le funzionalità attese del prodotto;
    \item \textbf{Attori\textsubscript{G}}: descrive gli utilizzatori del prodotto;
    \item \textbf{Casi d'uso}: individua le possibili interazioni tra gli \textit{attori}\textsubscript{G} e il \textit{sistema}\textsubscript{G};
    \item \textbf{Requisiti\textsubscript{G}}: elenca le caratteristiche da soddisfare;
\end{itemize}
\subsubsubsection{Casi d'uso}
I casi d’uso sono strutturati nel seguente modo:
\begin{itemize}
    \item \textbf{Attore}\textsubscript{G}: l’\textit{attore}\textsubscript{G} che intende compiere lo scopo rappresentato dal caso d’uso;
    \item \textbf{Precondizioni}: stato in cui il \textit{sistema}\textsubscript{G} si deve trovare prima dell’avvio della funzionalità rappresentata dal caso d’uso;
    \item \textbf{Postcondizioni}: stato in cui il \textit{sistema}\textsubscript{G} si troverà dopo che l'utente avrà terminato lo scopo rappresentato dal caso d’uso;
    \item \textbf{Scenario principale}: descrizione della funzionalità rappresentata dal caso d’uso;
    \item \textbf{Scenari secondari} (se necessario);
    \item \textbf{Estensioni} (se presenti);
    \item \textbf{Specializzazioni} (se presenti).
\end{itemize}
\subsubsubsubsection{Notazione}
i casi d'uso seguono la seguente notazione: \textbf{UC[Codice] - [Titolo]} in cui:
\begin{itemize}
    \item \textbf{UC} sta per Use Case;
    \item \textbf{[Codice]} è l'identificativo univoco del caso d'uso. Si tratta di un numero intero progressivo assegnato in base all'ordine di descrizione, se il caso d'uso non ha padre, altrimenti se si tratta di un sottocaso d'uso si segue la notazione\textbf{ [Codice\_padre]-[Numero\_figlio]}, ricorsivamente senza porre limite alla profondità della gerarchia;
    \item \textbf{[Titolo]} è il titolo del caso d'uso.
\end{itemize}

\subsubsubsubsection{Diagrammi UML\textsubscript{G}}
Un \textit{diagramma dei casi d’uso}\textsubscript{G} è uno strumento di modellazione che rappresenta visivamente le funzionalità di un \textit{sistema}\textsubscript{G} e le modalità con cui gli utenti interagiscono con esso. È particolarmente utile nella progettazione di sistemi poiché offre una rappresentazione intuitiva delle dinamiche operative e delle interazioni tra \textit{attori}\textsubscript{G} e \textit{sistema}\textsubscript{G}, senza entrare nei dettagli implementativi.
I componenti principali di un \textit{diagramma dei casi d’uso}\textsubscript{G} sono: 
\begin{enumerate}
    \item \textbf{Attori}\textsubscript{G}: gli \textit{attori}\textsubscript{G} rappresentano entità esterne (umane o meno) che interagiscono con il \textit{sistema}\textsubscript{G} e sono raffigurati con un’icona stilizzata e un’etichetta identificativa. Possono essere generalizzati: un \textit{attore}\textsubscript{G} generico può avere \textit{attori}\textsubscript{G} più specifici che ne ereditano le funzionalità e aggiungono comportamenti contestuali;
    \item \textbf{Casi d'uso}: un caso d’uso descrive un'operazione che un utente può compiere attraverso il \textit{sistema}\textsubscript{G}. Ogni caso d’uso ha un'identificazione univoca e una breve descrizione della funzione. Può includere sequenze di azioni che illustrano le possibili interazioni con il \textit{sistema}\textsubscript{G} ed è collegato agli \textit{attori}\textsubscript{G} autorizzati tramite linee continue.
\end{enumerate}
Nei diagrammi in questione poi possono comparire delle relazioni:
\begin{enumerate}
    \item \textbf{Generalizzazioni}: le generalizzazioni possono riguardare sia gli \textit{attori}\textsubscript{G} che i casi d’uso. Gli \textit{attori}\textsubscript{G} o i casi figli ereditano le funzionalità dei genitori, aggiungendo aspetti specifici. La relazione è rappresentata con una freccia continua e un triangolo vuoto bianco;
    \item \textbf{Inclusioni}: si verificano quando un caso d’uso ne richiama un altro in modo obbligatorio. Questo favorisce la riduzione della duplicazione e il riutilizzo delle strutture. La relazione è indicata con una freccia tratteggiata e l’etichetta “include”;
    \item \textbf{Estensioni}: rappresentano relazioni condizionali in cui un caso d’uso aggiuntivo viene eseguito solo in circostanze particolari, interrompendo temporaneamente il flusso principale. La relazione è raffigurata con una freccia tratteggiata e l’etichetta “extend”.
\end{enumerate}

\subsubsubsection{Requisiti}
\subsubsubsubsection{Notazione}
Ogni \textit{requisito}\textsubscript{G} analizzato  sarà identificato univocamente da una sigla del tipo \\ \textbf{R[Tipo].[Importanza].[Codice]} nella quale:
\begin{itemize}
    \item \textbf{[R]} sta per \textit{Requisito}\textsubscript{G};
    \item \textbf{[Tipo]} può essere:
    \begin{itemize}
        \item \textbf{F} per Funzionale;
        \item \textbf{Q} per Qualità;
        \item \textbf{V} per Vincolo.
    \end{itemize}
    \item \textbf{[importanza]} classifica i \textit{requisiti}\textsubscript{G} in:
    \begin{itemize}
        \item \textbf{O} per Obbligatorio;
        \item \textbf{D} per Desiderabile;
        \item \textbf{P} per Opzionale.
    \end{itemize}
    \item \textbf{[Codice]} identifica univocamente i \textit{requisiti}\textsubscript{G} per ogni tipologia. È un numero intero progressivo univoco assegnato in ordine di importanza se il \textit{requisito}\textsubscript{G} non ha padre, se invece si tratta di un sotto-\textit{requisito}\textsubscript{G} segue il formato \textbf{[Codice\_padre].[Numero\_figlio]} e trattandosi di una struttura ricorsiva non c'è limite alla profondità della gerarchia.
\end{itemize}

\subsubsubsubsection{Suddivisione}
\begin{enumerate}
    \item \textbf{Requisiti\textsubscript{G} Funzionali}: descrivono le funzionalità del \textit{sistema}\textsubscript{G}, le azioni che il \textit{sistema}\textsubscript{G} può compiere e le informazioni che il \textit{sistema}\textsubscript{G} può fornire. Seguendo la notazione sopra riportata, si possono partizionare in:
    \begin{itemize}
        \item RF.O - \textit{Requisito}\textsubscript{G} Funzionale Obbligatorio;
        \item RF.D - \textit{Requisito}\textsubscript{G} Funzionale Desiderabile;
        \item RF.P - \textit{Requisito}\textsubscript{G} Funzionale Opzionale;
    \end{itemize}
     \item \textbf{Requisiti\textsubscript{G} di Qualità}: descrivono come un \textit{sistema}\textsubscript{G} deve essere, o come il \textit{sistema}\textsubscript{G} deve essere visualizzato, per soddisfare le esigenze dell’utente. Seguendo la notazione sopra riportata, si possono partizionare in:
    \begin{itemize}
        \item RQ.O - \textit{Requisito}\textsubscript{G} di Qualità Obbligatorio;
        \item RQ.D - \textit{Requisito}\textsubscript{G} di Qualità Desiderabile;
        \item RQ.P - \textit{Requisito}\textsubscript{G} di Qualità Opzionale;
    \end{itemize}
     \item \textbf{Requisiti\textsubscript{G} Funzionali}: descrivono i limiti e le restrizioni normative/legislative che un \textit{sistema}\textsubscript{G} deve rispettare per soddisfare le esigenze dell’utente. Seguendo la notazione sopra riportata, si possono partizionare in:
    \begin{itemize}
        \item RV.O - \textit{Requisito}\textsubscript{G} di Vincolo Obbligatorio;
        \item RV.D - \textit{Requisito}\textsubscript{G} di Vincolo Desiderabile;
        \item RV.P - \textit{Requisito}\textsubscript{G} di Vincolo Opzionale;
    \end{itemize}
\end{enumerate}

\subsubsubsection{Strumenti}
Gli strumenti utilizzati per il processo di sviluppo dell'Analisi dei Requisiti sono:
\begin{itemize}
    \item Diagrams.net;
    \item StarUML.
\end{itemize}

\subsubsection{Progettazione}
\subsubsubsection{Scopo}
Lo scopo della progettazione è definire l'architettura del prodotto software e fornire i componenti e le interazioni del sistema per garantire che funzioni in modo efficiente ed efficacie rispetto ai requisiti funzionali e non funzionali determinati durante l'attività di analisi dei requisiti.

\subsubsubsection{Documentazione}
La progettazione porterà alla redazione del documento di Specifica Tecnica. Questo documento ha principalmente lo scopo di descrivere l'architettura del prodotto software e di mettere a disposizione una linea guida per garantire che il sistema venga implementato secondo i requisiti del progetto. Gli argomenti trattati in questo documento sono:
\begin{itemize}
    \item \textbf{Tecnologie}: espone un'analisi delle tecnologie e dei linguaggi di programmazione utilizzati, delle librerie\textsubscript{G} e dei framework\textsubscript{G} necessari, oltre che delle infrastrutture realizzate, riportando in particolare vantaggi e svantaggi di ognuna;
    \item \textbf{Architettura di sistema}: descrive la struttura generale del software, la suddivisione in moduli o livelli e le interazioni tra i componenti;
    \item \textbf{Architettura delle componenti}: fornisce i dettagli dei singoli moduli o componenti del sistema, descrivendone responsabilità, interfacce, flussi di dati, modalità di interazione ed esplicitando eventuali sottocomponenti;
    \item \textbf{Progettazione di dettaglio}: definisce nel dettaglio gli algoritmi, le strutture dati, i flussi operativi e l'implementazione dei singoli componenti.
\end{itemize}

\subsubsubsection{Principi di progettazione dei componenti}
Per ogni componente devono essere specificati i seguenti punti:
\begin{enumerate}
    \item Una breve descrizione delle funzionalità del componente;
    \item Caratteristiche del componente:
    \begin{itemize}
        \item \textbf{Route API}: il percorso dell’endpoint esposto dal componente per interagire con il sistema;
        \item \textbf{Metodo}: il tipo di richiesta HTTP supportata;
        \item \textbf{Lista parametri HTTP}: l'elenco dei parametri richiesti o opzionali, inclusi nome, tipo e scopo di ciascun parametro.
    \end{itemize}
    \item Una lista che comprenda i possibili risultati dell'esecuzione del componente, accompagnati da:
    \begin{itemize}
        \item Il codice che porta al verificarsi di quel particolare esito;
        \item Una descrizione di quel particolare esito;
        \item Il messaggio che viene visualizzato nel caso in cui si verifichi quel particolare esito.
    \end{itemize}
    \item Un sistema che consenta di effettuare il tracciamento dei requisiti soddisfatti;
    \item Una dichiarazione esaustiva di tutte le sottocomponenti, corredate dal loro scopo.
\end{enumerate}

\subsubsubsection{Principi di progettazione di dettaglio}
La sezione dedicata alla progettazione di dettaglio del documento di Specifica Tecnica definisce nel dettaglio gli algoritmi, le strutture dati, i flussi operativi e l'implementazione dei singoli componenti di sistema.\\
Per ogni componente è necessario realizzare un diagramma delle classi e per ogni classe è necessario elencare le proprietà, che comprendono:
\begin{itemize}
    \item \textbf{Attributi}: un elenco degli attributi della classe;
    \item \textbf{Implementazione}: un campo che specifica se la classe implementa un'interfaccia;
    \item \textbf{Estensione}: un campo che specifica se la classe estende un'altra classe;
    \item \textbf{Metodi}: un elenco dei metodi della classe, dove per ogni metodo vengono specificati la firma che lo identifica e una descrizione;
    \item \textbf{Valori}: un campo che caratterizza le enumerazioni.
\end{itemize}

\subsubsubsection{Convenzioni nella progettazione di dettaglio}
\subsubsubsubsection{Convenzioni di denominazione}
\begin{itemize}
    \item \textbf{Nomi delle classi}: utilizzano il PascalCase (es. NomeClasse);
    \item \textbf{Nomi degli attributi}: utilizzano il camelCase (es. nomeAttributo);
    \item \textbf{Nomi dei metodi}: utilizzano il camelCase con verbi all'infinito (es. nomeMetodo());
    \item \textbf{Nomi delle costanti}: utilizzano lo SCREAMING\_SNAKE\_CASE (es. NOME\_COSTANTE).
\end{itemize}

\subsubsubsubsection{Convenzioni di rappresentazione}
Le classi sono rappresentate con tre sezioni:
\begin{itemize}
    \item Nome della classe;
    \item Attributi, rappresentati come segue:\\
    \texttt{visibilità nomeAttributo: tipo}
    \item Metodi, rappresentati con la loro firma come segue:\\
    \texttt{visibilità nomeMetodo(parametro0: tipo0, parametro1: tipo1): tipo}
\end{itemize}
