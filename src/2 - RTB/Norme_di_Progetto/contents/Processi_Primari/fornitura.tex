\subsection{Fornitura}
\subsubsection{Scopo}
La fornitura è il processo che descrive le attività svolte dal fornitore, coinvolge pianificazione, acquisizione e gestione delle risorse necessarie. Il processo determina le procedure e le risorse necessarie per gestire e garantire il progetto. L'obiettivo di questo processo è garantire l'\textit{efficienza}\textsubscript{G} e la conformità ai \textit{requisiti}\textsubscript{G} del progetto per raggiungere gli obiettivi stabiliti dal proponente. 
\subsubsection{Implementazione}
Il processo di fornitura è composto delle seguenti fasi:
\begin{enumerate}
    \item \textbf{Risposta alla richiesta}: il fornitore, dopo aver analizzato i \textit{requisiti}\textsubscript{G} di una richiesta del proponente (il \textit{Capitolato}\textsubscript{G}) prepara in risposta una proposta;
    \item \textbf{Negoziazione}: il fornitore negozia e stipula un contratto con il proponente;
    \item \textbf{Pianificazione}: il fornitore rivede i \textit{requisiti}\textsubscript{G} e valuta le opzioni per lo sviluppo del prodotto \textit{software}\textsubscript{G} in base ad un'analisi dei rischi associati alle varie opzioni per definire la struttura di un piano di gestione del progetto al fine di garantire la qualità del prodotto finale;
    \item \textbf{Esecuzione e controllo}: il fornitore esegue il piano di gestione del progetto, monitorando il progresso e la qualità del prodotto per tutto il ciclo di vita del prodotto;
    \item \textbf{Revisione}: il fornitore coordina le comunicazioni con il proponente e partecipa a riunioni e revisioni. Il fornitore \textit{verifica}\textsubscript{G} e convalida il processo per dimostrare che i prodotti e i processi soddisfano i \textit{requisiti}\textsubscript{G};
    \item \textbf{Consegna}: il fornitore consegna il prodotto finale, fornendo assistenza al proponente a supporto del prodotto consegnato.
\end{enumerate}

\subsubsection{Gestione}
Al fine di identificare e comprendere i bisogni del Proponente, per poter individuare i \textit{requisiti}\textsubscript{G} e i vincoli del progetto, deve essere mantenuta costante comunicazione con il Proponente, mediante riunioni SAL periodiche calendarizzate, in presenza o su \textit{Microsoft Teams}\textsubscript{G} e con scambio di messaggi su \textit{Microsoft Teams}\textsubscript{G} e mail qualora fosse necessario. Il dialogo continuo permette anche una valutazione costante dell'operato del fornitore, in modo da apportare correzioni, integrazioni e miglioramenti in modo tempestivo, incrementale e costruttivo.

\subsubsection{Documentazione fornita}
Sono di seguito elencati i documenti che PEBKAC si impegna a consegnare ai \textit{Committenti}\textsubscript{G} e al Proponente: 

\subsubsubsection{Piano di Progetto}
Il Piano di Progetto V1.0.0, redatto dal \textit{Responsabile}\textsubscript{G} con l'aiuto degli Amministratori, offre una guida per la pianificazione l'esecuzione e il controllo del progetto e viene utilizzato come punto di partenza principale per il monitoraggio del progresso del progetto, la gestione dei rischi e la comunicazione tra proponente e fornitore.
Il Piano di Progetto comprende:
    \begin{itemize}
        \item Calendario di Progetto;
        \item Stima dei costi di realizzazione;
        \item Rischi e relativa mitigazione;
        \item Pianificazione e modello di sviluppo;
        \item Preventivo e \textit{consuntivo}\textsubscript{G};
        \item Retrospettiva.
    \end{itemize}

\subsubsubsection{Analisi dei requisiti}
L'Analisi dei Requisiti V1.0.0, redatto degli \textit{Analisti}\textsubscript{G}, è un documento fondamentale che ha l'obiettivo principale di definire nel dettaglio le funzionalità che il prodotto deve necessariamente avere per soddisfare a pieno le richieste del Proponente. 
Il documento di Analisi dei Requisiti è formato da una serie di definizioni essenziali:
\begin{itemize}
    \item \textbf{Attori\textsubscript{G}}: vengono definite entità e persone che interagiscono col \textit{sistema}\textsubscript{G};
    \item \textbf{Casi d'uso}: vengono descritti narrativamente degli scenari specifici che descrivono come gli attori interagiscono col \textit{sistema}\textsubscript{G}. Lo scopo dei casi d'uso è offrire una visione semplice e chiara delle azioni eseguibili all'interno del sisitema e delle interazioni degli utenti con lo stesso. Per ciascun caso d'uso viene fornito un elenco delle azioni dell'\textit{attore}\textsubscript{G} per attivare il caso d'uso, facilitando la comprensione dei requisiti corrispondenti;
    \item \textbf{Requisiti}: vengono individuati i requisiti obbligatori, desiderabili e opzionali e la loro categorizzazione in: 
    \begin{itemize}
        \item \textbf{Requisiti\textsubscript{G} funzionali}: specificano le operazioni che il \textit{sistema}\textsubscript{G} deve essere in grado di eseguire; 
        \item \textbf{Requisiti\textsubscript{G} di qualità}: definiscono gli standard e gli attributi che il \textit{software}\textsubscript{G} deve possedere per garantire prestazioni, affidabilità, sicurezza e usabilità ottimali;
        \item \textbf{Requisiti\textsubscript{G} di vincolo}: definiscono vincoli e limitazioni che il \textit{sistema}\textsubscript{G} deve rispettare. Possono includere restrizioni tecnologiche, normative o di risorse.
    \end{itemize}
\end{itemize}
\subsubsubsection{Piano di Qualifica}
Il Piano di Qualifica V1.0.0, redatto dall'\textit{amministratore}\textsubscript{G}, descrive gli approcci e le strategie che il gruppo ha adottato per garantire la qualità del prodotto. Lo scopo di questo documento è quello di definire le modalità di \textit{verifica}\textsubscript{G}e \textit{validazione}\textsubscript{G}, oltre che gli standard e le procedure di qualità che il gruppo ha deciso di adottare per il ciclo di vita del progetto. \\ 
Si compone delle sezioni riguardanti:
\begin{itemize}
    \item \textbf{Qualità di processo}: vengono definiti standard e procedure adottate per garantire la qualità durante tutto lo sviluppo del progetto. Vengono incluse anche informazioni sulle attività di gestione della qualità, i metodi utilizzati e le misurazioni dei processi stessi;
    \item \textbf{Qualità di prodotto}: vengono definiti standard, specifiche e caratteristiche che il prodotto deve soddisfare per essere considerato di qualità. Vengono incluse anche metriche e criteri di valutazione utilizzati per misurare la qualità del prodotto;
    \item \textbf{Specifiche dei test}: vengono definite specifiche dettagliate dei test che verranno condotti durante lo sviluppo del progetto;
    \item \textbf{Cruscotto delle metriche}: viene fatto un resoconto delle attività di valutazione effettuate durante il progetto per tracciare l'andamento dello stesso rispetto a obiettivi e aspettative e per identificare eventuali azioni correttive necessarie.
\end{itemize}

\subsubsubsection{Glossario}
Il Glossario V1.0.0 serve come un catalogo completo dei termini tecnici impiegati all'interno del progetto, fornendo definizioni chiare e precise. L'obiettivo di questo documento previene fraintendimenti a favore di una comprensione condivisa della terminologia specifica, migliorando la coerenza e la qualità della \textit{documentazione}\textsubscript{G} prodotta dal gruppo.

\subsubsection{Strumenti}
Gli strumenti utilizzati per il processo di fornitura sono:
\begin{itemize}
    \item Google Calendar;
    \item \textit{Google Sheets}\textsubscript{G};
    \item Microsoft PowerPoint;
    \item \textit{Microsoft Teams}\textsubscript{G}.
\end{itemize}