\subsection{Gestione organizzativa}
\subsubsection{Scopo}
Lo scopo di questo processo è esporre le modalità e gli strumenti di coordinamento usati dal gruppo per la comunicazione, interna ed esterna, e normare l'assegnazione di ruoli e compiti, oltre che la gestione dei rischi.

\subsubsection{Ruoli}
Per ottimizzare la gesione delle attività e dei compiti da svolgere vengono definiti sei ruoli distinti, ciascuno con mansioni e responsabilità specifiche. Ogni componente del gruppo dovrà assumere ciascun ruolo per un numero di ore significativo.
\subsubsubsection{Responsabile}
Il responsabile è il punto di riferimento per tutto il gruppo e anche per le comunicazioni con il committente e con l'azienda proponente.
Inoltre il responsabile è la figura che ha il compito di coordinare le azioni dei membri del gruppo, perciò deve avere competenze tecniche in ogni ambito del progetto. Le responsabilità di questo ruolo sono:
\begin{itemize}
    \item Coordinamento tra gruppo ed enti esterni;
    \item Gestione delle comunicazioni interne;
    \item Pianificazione di progetto,
    \item Gestione dei task e delle risorse;
    \item Gestione dell'avanzamento del progetto;
\end{itemize}
\subsubsubsection{Amministratore}
L'amministrazione è la figura che definisce, gestisce e mantiene l’ambiente e l’infrastruttura necessari per lo
sviluppo del progetto facendo in modo che siano affidabili e sicuri. Si occupa della gestione della configurazione, del versionamento, delle varie automazioni e della documentazione. Si occupa di:
\begin{itemize}
    \item Selezionare e abilitare risorse informatiche a supporto del \textit{way of working}\textsubscript{G};
    \item Gestire errori e malfunzionamenti nei meccanismi nell’infrastruttura.
\end{itemize}

\subsubsubsection{Analista}
La funzione dell'analista è quella di analizzare il problema per definire i requisiti del prodotto, per questo deve avere buona conoscenza del dominio del problema. L'analista raccoglie le sue produzioni nel documento Analisi dei Requisiti. Si tratta di un ruolo fondamentale all'inizio del progetto, ma la cui utilità cala nelle seguenti fasi del progetto.
\subsubsubsection{Progettista}
Al progettista spettano le scelte realizzative e le specifiche architetturali del prodotto. Deve avere buone competenze tecniche e tecnologiche. Durante il processo di sviluppo la sua utilità è massima, ma tende a calare dalla fase di manutenzione in poi.
\subsubsubsection{Programmatore}
Quello del programmatore è un ruolo chiave nella fase di sviluppo. In particolare si occupa di :
\begin{itemize}
    \item Codificare ciò che è stato definito dai progettisti;
    \item Implementare i test;
    \item Redigere il Manuale utente.
\end{itemize}
\subsubsubsection{Verificatore}
Ha il compito di verificare il lavoro degli altri e per questo deve avere competenze tecniche ed essere presente per l'intera durata del progetto. Questa figura deve controllare che tutto ciò che viene prodotto sia conforme alle norme e alle aspettative di qualità del gruppo.


\subsubsection{Attività}
Ogni membro del gruppo può proporre attività da svolgere, ma è compito del responsabile stabilire la fattibilità rispetto alle risorse usufruibili. 
\subsubsubsection{Pianificazione}
\subsubsubsubsection{Strumenti}
    \begin{itemize}
        \item Trello
    \end{itemize}

\subsubsubsection{Esecuzione}
L'esecuzione delle attività avviene obbligatoriamente per mano dell'assegnatario, definito dal responsabile. L'esecuzione deve essere obbligatoriamente conforme alla documentazione associata precedentemente redatta. L'esecutore dovrà proporre la sua soluzione con una pull request\textsubscript{G}.

\subsubsubsection{Revisione}
La revisione dell'attività è effettuata dal verificatore prima dell'effettivo inserimento delle modifiche nel repository GitHub: la pull request\textsubscript{G} aperta dell'esecutore viene accettata o rifiutata, riportando le parti non valide ed eventiali accorgimenti possibili, a seconda dell'esito della verifica.
\subsubsubsubsection{Strumenti}
    \begin{itemize}
        \item GitHub
    \end{itemize}

\subsubsubsection{Chiusura}
Solo nel caso dell'esito positivo della verifica, con l'accettazione della pull request, viene chiuso il branch di cui è stato effettuato il merge e l'attività viene segnata come completata. 

\subsubsubsection{Tracciamento orario}
Il gruppo utilizza Google Sheets\textsubscript{G} per avere un foglio di calcolo condiviso in cui tenere conto del tempo speso per svolgere le attività. Ogni membro è tenuto a registrare, alla fine di ogni sessione lavorativa, il numero di ore effettive di lavoro e il ruolo ricoperto.
\subsubsubsubsection{Strumenti}
    \begin{itemize}
        \item Google Sheets
    \end{itemize}


\subsubsection{Comunicazione}
...
\subsubsubsection{Comunicazioni interne}
... 
\subsubsubsubsection{Comunicazioni sincrone}
... 
\subsubsubsubsection{Comunicazioni asincrone}
... 
\subsubsubsection{Comunicazioni esterne}
... 
\subsubsubsubsection{Comunicazioni sincrone}
... 
\subsubsubsubsection{Comunicazioni asincrone}
... 