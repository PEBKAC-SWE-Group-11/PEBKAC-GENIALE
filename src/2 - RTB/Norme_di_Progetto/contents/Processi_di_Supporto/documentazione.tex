\subsection{Documentazione}
\subsubsection{Scopo}
Il processo di \textit{documentazione}\textsubscript{G} procede sempre di pari passo con tutte le attività di sviluppo, con l'obiettivo di fornire tutte le informazioni necessarie, sotto forma di testo scritto facilmente consultabile, inerenti al prodotto e alle attività stesse. Oltre a svolgere un \textit{ruolo}\textsubscript{G} essenziale nella descrizione del prodotto per coloro che lo sviluppano, lo distribuiscono e lo utilizzano, la \textit{documentazione}\textsubscript{G} svolge un \textit{ruolo}\textsubscript{G} di storicizzazione e di supporto alla manutenzione. 

\subsubsection{Documenti}
In questa sezione viene descritto il piano che identifica i documenti da produrre durante il ciclo di vita del prodotto \textit{software}\textsubscript{G}. Tutti i documenti da redigere sono presentati nella tabella che segue, vengono esclusi i documenti presentati per la candidatura per il progetto didattico, quali \textit{Lettera di presentazione}, \textit{Preventivo dei costi e assunzione degli impegni} e \textit{Analisi dei capitolati}.


% Creazione della tabella
\begin{table}[H]
    \centering
   \begin{tabularx}{\textwidth}{X|X|>{\hsize=1.2\hsize}X|X|>{\hsize=0.8\hsize}X}
        \textbf{Nome} & \textbf{Scopo} & \textbf{Redattore} & \textbf{Destinatari} & \textbf{Consegne} \\ \hline
        Analisi dei requisiti    & Definizione dei requisiti utente    & \textit{Analista}\textsubscript{G} & Azienda proponente, Docenti & \textit{RTB}\textsubscript{G}, \textit{PB}\textsubscript{G}    \\ \hline
        Norme di progetto    & Regolamento normativo del gruppo    & \textit{Amministratore}\textsubscript{G}, \textit{Responsabile}\textsubscript{G} &  Docenti & \textit{RTB}\textsubscript{G}, \textit{PB}\textsubscript{G}   \\ \hline
        Piano di Progetto   & Definizione temporale scadenze e progressi    & \textit{Responsabile}\textsubscript{G} &  Docenti & \textit{RTB}\textsubscript{G}, \textit{PB}\textsubscript{G}   \\ \hline
        Piano di qualifica   & Definizione qualità e testing    & \textit{Amministratore}\textsubscript{G} &  Docenti & \textit{RTB}\textsubscript{G}, \textit{PB}\textsubscript{G}   \\ \hline
         Verbali esterni   & Tracciamento riunioni esterne   & \textit{Responsabile}\textsubscript{G}, \textit{Amministratore}\textsubscript{G} &  Azienda proponente, Docenti & Candidatura, \textit{RTB}\textsubscript{G}, \textit{PB}\textsubscript{G}   \\ \hline
         Verbali interni   & Tracciamento riunioni interne   & \textit{Responsabile}\textsubscript{G}, \textit{Amministratore}\textsubscript{G} &  Docenti & Candidatura, \textit{RTB}\textsubscript{G}, \textit{PB}\textsubscript{G}  
       
    \end{tabularx}
    \caption{Documenti del ciclo di vita del prodotto \textit{software}\textsubscript{G}.}
\end{table}


\subsubsection{Progettazione e sviluppo}
In questa sezione vengono presentati gli standard e le regole (nello specifico di stile) a cui i membri di PEBKAC si devono attenere per la stesura dei documenti relativi al progetto.

\subsubsubsection{Template}
Per la stesura dei documenti il gruppo ha creato un template in formato \textit{LaTeX}\textsubscript{G}. Il template fornisce una struttura e un formato predefinito per semplificare la creazione di documenti, al fine di garantire coerenza, efficienza e standardizzazione della presentazione. 
Il template è progettato per essere facile da usare, dovendo inserire solo con piccole modifiche per rispecchiare le specificità di ciascun tipo di documento.\\
In particolare nel template è definite la pagina di copertina con intestazione contenente logo informazioni del gruppo e dell'Università di Padova, titolo del documento, informazioni sul documento (uso, destinatari) e un breve abstract del contenuto, oltre che altre specifiche di stile come il titolo dell'indice in italiano e il numero di pagina come \texttt{X di Tot}, dove \texttt{X} è il numero della pagina e \texttt{Tot} è il numero totale di pagine.
\subsubsubsubsection{Parametri}
Nel principale file \textit{LaTex}\textsubscript{G} del template sono definiti una serie di comandi personalizzati per l'inserimento automatico delle informazioni come titolo, data, uso, destinatari e abstract. \\
Sono inoltre già presenti ma commentate le voci necessarie solo per i verbali (vedi  \hyperref[sec: struttura verbali]{§4.1.3.3 Verbali})
\subsubsubsection{Struttura del documento}
Tutti i documenti prodotti da PEBKAC presentano la medesima struttura, alla quale ogni membro si deve attenere durante la procedura di stesura e modifica.
\begin{itemize}
    \item \textbf{Pagina di copertina}: come nella sezione Template precedente;
    \item \textbf{Registro delle versioni}: questo registro è utilizzato per tenere traccia delle varie versioni per permettere di comprendere velocemente chi ha realizzato o modificato determinate sezioni della \textit{documentazione}\textsubscript{G} e quando. Il registro presenta le versioni ordinate a partire dalla versione più recente;
     \item \textbf{Indice}: presente per facilitare la consultazione del documento, dotato di sezioni. Il suo scopo è di facilitare e agevolare l’accesso ad un determinato contenuto all'interno nel documento;
     \item \textbf{Contenuto}: il contenuto vero e proprio del documento.
\end{itemize}

\subsubsubsection{Verbali}\label{sec: struttura verbali}
I verbali differiscono dalla struttura precedentemente esposta in quanto ad essi prevedono delle sezioni aggiuntive ed obbligatorie:
\begin{itemize}
    \item \textbf{Pagina di copertina}: nel caso di un verbale tra le informazioni sul documento compaiono anche i nominativi con i rispettivi ruoli dei membri che hanno lavorato alla loro produzione;
    \item \textbf{informazioni generali}: la prima sezione di un verbale deve sempre essere quella nominata ``Informazioni generali" che prevede, sotto forma di elenco puntato, le seguenti informazioni:
        \begin{itemize}
            \item Tipo di riunione,
            \item Luogo in cui si è tenuta la riunione (anche se telematica),
            \item Data in cui si è tenuta la riunione,
            \item Ora di inizio della riunione,
            \item Ora di fine della riunione,
            \item Membri presenti ed eventuali altre persone alla riunione,
            \item Membri assenti dalla riunione;
        \end{itemize}
     \item \textbf{Todo}: l'ultima sezione di un verbale deve sempre essere quella che elenca i \textit{task}\textsubscript{G} emersi durante la riunione da aggiungere al \textit{backlog}\textsubscript{G}. Questi vengono presentati sotto forma di tabella a due colonne:
     \begin{itemize}
         \item \textbf{Assegnatario}: il membro a cui quel \textit{task}\textsubscript{G} è stato assegnato, nel caso in cui non ve ne sia uso ma il \textit{task}\textsubscript{G} possa essere autoassegnato da uno dei membri si scriverà ``autoassegnazione" in corsivo;
         \item \textbf{Task\textsubscript{G} Todo}: denominazione del \textit{task}\textsubscript{G}.
     \end{itemize}
     
\end{itemize}

\subsubsubsection{Nomenclatura}
La nomenclatura per i documenti si ottiene unendo il nome del file in \textit{Snake\_Case} quindi con le parole separate da un underscore (\textit{\_}) (\texttt{Nome\_del\_File}), un underscore (\textit{\_}) e la sua versione (\texttt{1.2.3}), ottenendo per esempio \texttt{Norme\_di\allowbreak{}\_Progetto\_1.2.3.pdf}. Nel caso di documenti il cui nome contiene una data, essa si inserisce dopo il nome, ma prima della versione, sempre usando gli underscores come separatori, nella forma YYYY-MM-DD: YYYY rappresenta l'anno, MM il mese e DD il giorno, sempre scritto in due cifre.
\subsubsubsubsection{Verbali}
Per quanto riguarda i  verbali, per facilitarne l'ordinamento) il loro nome è la data in cui la riunione di è tenuta nella forma YYYY-MM-DD: YYYY rappresenta l'anno, MM il mese e DD il giorno, sempre scritto in due cifre. Nel caso si tratti di un verbale esterno viene aggiunta una \texttt{E}, sempre separata da underscores tra la data e la versione.


\subsubsubsection{Versionamento}
La versione di un documento è del tipo [\textbf{x}].[\textbf{y}].[\textbf{z}]:
\begin{itemize}
    \item \textbf{z}: è un numero intero che incrementato dal Redattore ad ogni modifica;
    \item \textbf{y}: è un numero intero incrementato dal \textit{Verificatore}\textsubscript{G} ad ogni \textit{verifica}\textsubscript{G};
    \item \textbf{x}: è un numero intero che viene incrementato dal \textit{Responsabile}\textsubscript{G} dopo la sua approvazione (versione di produzione).
\end{itemize}

\subsubsubsection{Convenzioni stilistiche}
\begin{itemize}
    \item \textbf{Date}: tutte le date nella \textit{documentazione}\textsubscript{G} prevedono il seguente formato YYYY-MM-DD, dove DD indica il giorno a due cifre, MM il mese a due cifre e YYYY l'anno a 4 cifre;
    \item \textbf{Elenchi}: elenchi puntati o numerati, ogni punto inizia con la lettera maiuscola e termina con ``;" ad eccezione dell'ultimo che termina con ``.";
    \item \textbf{Menzioni}: ogni menzione ad una persona, interna o esterna, avviene nel formato Nome Cognome;
    \item \textbf{Riferimenti interni}: i riferimenti a sezioni interne allo stesso documento devono essere riportati seguendo la notazione \texttt{§1.2 Nome sezione}, dove \texttt{§1.2} è il numero della sezione. Inoltre questi riferimenti devono essere opportunamente collegati tramite link al paragrafo indicato, senza alterare lo stile del testo;
    \item \textbf{Riferimenti esterni}: i riferimenti a sezioni di documenti esterni devono essere riportati seguendo la notazione \texttt{Nome Documento (versione di riferimento), Nome sezione};
     \item \textbf{Link URL}: possono essere estesi o avere una visualizzazione abbreviata, ma sempre visualizzati di colore blu;
    \item \textbf{Caratteri maiuscoli}: devono essere utilizzati per
        \begin{itemize}
            \item Le iniziali dei nomi;
            \item Le lettere che compongono degli acronimi e le iniziali delle rispettive definizioni;
            \item Le iniziali dei ruoli svolti dai componenti del gruppo;
            \item Le iniziali dei ruoli definiti all'interno del progetto didattico;
            \item La prima lettera di ogni elenco puntato.
        \end{itemize}
    \item \textbf{Grassetto}: devono essere visualizzati in grassetto
        \begin{itemize}
            \item I titoli di sezioni/sottosezioni/paragrafi di un documento;
            \item Le parole che meritano enfasi;
            \item Le definizioni negli elenchi puntati.
        \end{itemize}
    \item \textbf{Caption}: ogni immagine o tabella deve avere una caption, utile a fornire una breve descrizione o spiegazione del contenuto visivo.
\end{itemize}

\subsubsection{Ciclo di vita dei documenti}
Ogni documento segue le fasi del seguente \textit{workflow}\textsubscript{G}:
\begin{enumerate}
    \item \textbf{Assegnazione}: il gruppo assegna un documento a uno o più redattori, affiancati da uno o più verificatori;
    \item \textbf{Branch\textsubscript{G}}: si crea un \textit{branch}\textsubscript{G} per lo sviluppo del documento nell’apposita \textit{repository}\textsubscript{G} Docs;
    \item \textbf{Template}: si copia il Template all'interno della cartella appropriata;
    \item \textbf{Feature branch\textsubscript{G}}: si crea un \textit{branch}\textsubscript{G} per la realizzazione di una specifica modifica da apportare al relativo documento nell’apposita \textit{repository}\textsubscript{G} Docs;
    \item \textbf{Stesura}: si redige una o più sezioni del documento. Qualora serva un elevato parallelismo di lavoro è possibile usare Google Drive per la prima stesura e successivamente caricare il documento all’interno del \textit{branch}\textsubscript{G};
    \item \textbf{Commit\textsubscript{G}}: si esegue la \textit{commit}\textsubscript{G} sul feature \textit{branch}\textsubscript{G} creato;
    \item \textbf{Pull Request\textsubscript{G} verso il branch\textsubscript{G} del documento}: si apre una \textit{pull request}\textsubscript{G} dal feature \textit{branch}\textsubscript{G} appena creato verso il \textit{branch}\textsubscript{G} del documento;
    \item \textbf{Verifica\textsubscript{G} della feature}: se il \textit{verificatore}\textsubscript{G} richiede modifiche si ripetono, in ordine, il punto 5 e il punto 6;
    \item \textbf{Chiusura feature branch\textsubscript{G}}: si elimina, quando la \textit{pull request}\textsubscript{G} viene chiusa o risolta, il feature \textit{branch}\textsubscript{G} creato.
    \item \textbf{Pull Request\textsubscript{G} verso il branch\textsubscript{G} develop}: si apre una \textit{pull request}\textsubscript{G} dal \textit{branch}\textsubscript{G} del documento verso il \textit{branch}\textsubscript{G} develop: se il documento non è pronto per la \textit{verifica}\textsubscript{G}, ma ha bisogno di ulteriori modifiche, si apre la \textit{pull request}\textsubscript{G} in modalità draft, per marcarla successivamente come “Ready to Review”, altrimenti in modalità normale;
    \item \textbf{Verifica\textsubscript{G} del documento}: se il \textit{verificatore}\textsubscript{G} richiede modifiche si ripete, in ordine, dal punto 4 al punto 9;
    \item \textbf{Chiusura branch\textsubscript{G}}: si elimina, quando la \textit{pull request}\textsubscript{G} viene chiusa o risolta, il \textit{branch}\textsubscript{G} del documento.
\end{enumerate}
Per la versione finale di un documento spetta al \textit{Responsabile}\textsubscript{G} conferire l’approvazione definitiva, annotando opportunamente nel registro delle versioni la versione \texttt{x.0.0} e la sua approvazione finale.\\ \\
Per i verbali è sufficiente solamente la creazione del branch del documento, in quanto costituisce esso stesso un feature branch, dato che i verbali vengono redatti interamente in un'unica volta. In conclusione, per questi è possibile utilizzare un ciclo di vita semplificato, che è rppresentato dal seguente \textit{workflow}\textsubscript{G}:
\begin{enumerate}
    \item \textbf{Assegnazione}: il gruppo assegna un documento a uno o più redattori, affiancati da uno o più verificatori;
    \item \textbf{Branch\textsubscript{G}}: si crea un \textit{branch}\textsubscript{G} per lo sviluppo del documento nell’apposita \textit{repository}\textsubscript{G} Docs;
    \item \textbf{Template}: si copia il Template all'interno della cartella appropriata;
    \item \textbf{Stesura}: si redige il documento o una sua sezione. Qualora serva un elevato parallelismo di lavoro è possibile usare Google Drive per la prima stesura e successivamente caricare il documento all’interno del \textit{branch}\textsubscript{G};
    \item \textbf{Commit\textsubscript{G}}: si esegue la \textit{commit}\textsubscript{G} sul \textit{branch}\textsubscript{G} creato;
    \item \textbf{Pull Request\textsubscript{G}}: si apre una \textit{pull request}\textsubscript{G} dal \textit{branch}\textsubscript{G} appena creato verso il \textit{branch}\textsubscript{G} develop: se il documento non è pronto per la \textit{verifica}\textsubscript{G}, ma ha bisogno di ulteriori modifiche, si apre la \textit{pull request}\textsubscript{G} in modalità draft, per marcarla successivamente come “Ready to Review”, altrimenti in modalità normale;
    \item \textbf{Verifica\textsubscript{G}}: se il \textit{verificatore}\textsubscript{G} richiede modifiche si ripete, in ordine, dal punto 3 al punto 5;
    \item \textbf{Chiusura branch\textsubscript{G}}: si elimina, quando la \textit{pull request}\textsubscript{G} viene chiusa o risolta, il \textit{branch}\textsubscript{G} creato.
\end{enumerate}


\subsubsection{Strumenti}
\begin{itemize}
    \item \textit{LaTex}\textsubscript{G}
    \item Visual Studio Code
    \item \textit{GitHub}\textsubscript{G}
\end{itemize}
