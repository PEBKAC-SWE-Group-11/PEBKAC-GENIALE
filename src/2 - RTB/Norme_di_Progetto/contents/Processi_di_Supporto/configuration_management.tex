\subsection{Configuration Management}
\subsubsection{Scopo}
In questa sezione vengono presentate le attività svolte da PEBKAC per il processo di Configuration Management. Il processo in questione consiste nell'applicazione di procedure amministrative e tecniche per l'intero ciclo di vita del software, al fine di:
\begin{itemize}
    \item Identificare, definire e stabilire una base per gli elementi software di un sistema\textsubscript{G};
    \item controllare le modifiche e le release degli elementi;
    \item Registrare lo stato degli elementi e delle richieste di modifica;
    \item Garantire la completezza, la coerenza e la correttezza degli elementi.
\end{itemize}

\subsubsection{Configuration control}
\subsubsubsection{Scopo}
Il processo di configuration management è finalizzato a garantire il controllo e la coerenza delle configurazioni del sistema, assicurando che tutte le modifiche apportate a software, artefatti e documenti siano tracciate, gestite e allineate agli obiettivi e ai requisiti del progetto.
\subsubsubsection{Descrizione}
In questa
\subsubsubsection{ITS\textsubscript{G}}
In questa 