\subsection{Configuration Management}
\subsubsection{Scopo}
In questa sezione vengono presentate le attività svolte da PEBKAC per il processo di Configuration Management. Il processo in questione consiste nell'applicazione di procedure amministrative e tecniche per l'intero ciclo di vita del software, al fine di:
\begin{itemize}
    \item Identificare, definire e stabilire una base per gli elementi software di un sistema\textsubscript{G};
    \item Controllare le modifiche e le release degli elementi;
    \item Registrare lo stato degli elementi e delle richieste di modifica;
    \item Garantire la completezza, la coerenza e la correttezza degli elementi.
\end{itemize}
\subsubsection{Configuration control}
\subsubsubsection{Descrizione}
Il processo di configuration control è finalizzato a garantire il controllo e la coerenza delle configurazioni del sistema, assicurando che tutte le modifiche apportate a software, artefatti e documenti siano tracciate, gestite e allineate agli obiettivi e ai requisiti del progetto.
\subsubsubsection{Scopo}
Il configuration control mira al raggiungimento dei seguenti punti:
\begin{itemize}
    \item \textbf{Gestire le modifiche}: assicurare un controllo e una gestione corretti e sistematici per qualsiasi modifica nel progetto o nel sistema;
    \item \textbf{Documentare le richieste}: registrare tutte le richieste di modifica per mantenere una cronologia accurata e completa;
    \item \textbf{Valutare l'impatto}: analizzare tutte le conseguenze tecniche, economiche e operative di ogni modifica proposta;
    \item \textbf{Decidere sull'approvazione}: stabilire criteri chiari e inequivocabili per l'approvazione o il rifiuto delle modifiche con gli stakeholder rilevanti;
    \item \textbf{Assicurare la tracciabilità}: creare audit trail dettagliati per tracciare le modifiche e garantire la conformità alle politiche del progetto;
    \item \textbf{Evitare conflitti}: prevenire modifiche non autorizzate o che possano entrare in conflitto con il sistema;
    \item \textbf{Mantenere la qualità}: garantire che le modifiche non compromettano l'integrità, la funzionalità o gli obiettivi generali del progetto.
\end{itemize}
\subsubsubsection{ITS\textsubscript{G}}
Per conseguire l'obiettivo di assicurare la tracciabilità delle modifiche è necessario creare degli audit trail dettagliati, ovvero dei registri che tracciano tutte le attività e le modifiche all'interno di un sistema. Per la creazione, la gestione ed il tracciamento di questi audit trail, PEBKAC utilizza l'Issue Tracking System Jira, sviluppato da Atlassian.
\subsubsubsubsection{Ticket}
Un ticket è una voce che rappresenta una singola attività, problema, richiesta o task all'interno di un progetto. \\
Esistono varie tipologie di ticket:
\begin{itemize}
    \item \textbf{Task}: questa tipologia di ticket rappresenta una comune attività che deve essere completata all'interno del progetto;
    \item \textbf{Sub-task}: questa tipologia di ticket rappresenta una parte di un ticket più grande (come un task) che viene suddivisa in azioni più piccole e gestibili;
    \item \textbf{Story}: questa tipologia di ticket rappresenta un requisito ed è generalmente scritto in un formato che descrive il risultato atteso dal punto di vista dell'utente;
    \item \textbf{Bug}: questa tipologia di ticket rappresenta un errore o difetto nel sistema che necessita di correzione;
\end{itemize}
Ogni ticket è dotato di campi per riportare i dettagli relativi all'attività, al problema, alla richiesta o alla task che rappresenta:
\begin{itemize}
    \item \textbf{Summary}: un riassunto breve in una sola riga del ticket;
    \item \textbf{Key}: un identificatore unico per ogni ticket, nella forma si SW-Key;
    \item \textbf{Epic}: epic a cui il ticket è associato;
    \item \textbf{Links}: un elenco di link a ticket correlati;
    \item \textbf{Assignee}: la persona o le persone a cui il ticket è attualmente assegnato;
    \item \textbf{Description}: una descrizione dettagliata del ticket;
    \item \textbf{Due}: la data entro cui questo ticket è programmato per essere completato;
    \item \textbf{Reporter}: la persona che ha inserito il ticket nel sistema;
    \item \textbf{Links}: un elenco di link alle commit e alle pull request effettuati nella repository di GitHub correlate al ticket;
    \item \textbf{Status}: la fase in cui si trova attualmente il ticket nel suo ciclo di vita, che può essere "To Do", "In Process", "Verify" ed infine "Approve \& Release";
    \item \textbf{Sprint}: sprint a cui il ticket è associato;
    \item \textbf{Fix Version}: la versione del progetto in cui il ticket è stato (o sarà) risolto;
    \item \textbf{Priority}: l'importanza del ticket rispetto ad altri ticket.
\end{itemize}
\subsubsubsubsection{Epic}
Un'epic è una raccolta di ticket che rappresenta uno degli obiettivi più ampi e significativi verso cui è diretto l'intero progetto. Si tratta di un concetto che aiuta a gestire e strutturare il lavoro più complesso, suddividendolo in parti più piccole e gestibili. Le epic sono utili per monitorare i progressi rispetto a funzionalità che richiedono tempo o che coinvolgono diverse aree del progetto. Le epic sono particolarmente utili nei processi Agile, poiché offrono una visione a lungo termine del progetto, anche mentre si adatta e si pianifica in modo incrementale, fornendo strumenti di tracciamento, come una scorebord che presenta le percentuali di ticket presenti in ogni stato, che monitorano lo stato generale di ogni epic, misurano i progressi e identificano eventuali ritardi. Inoltre, un'epic, oltre ai ticket che comprende, possiede tutti i campi precedentemente elecati per i ticket.
\subsubsubsubsection{Versioni}
Le versioni sono la modalità di organizzazione, pianificazione e monitoraggio del lavoro in base alle specifiche milestone di un progetto. Ogni versione ha a che fare con le funzionalità, rappresentate da epic e relativi ticket ad essa associati, da realizzare entro una scadenza. In pratica, permette di sapere chiaramente quali requisiti devono essere soddisfatti per la specifica milestone che rappresenta. Ciò rende semplice la tracciabilità dei progressi di ciascuna versione, oltre a ritardi o modifiche, usando una scoreboard che adotta la stessa logica di quella utilizzata dalle epic. In seguito al completamento di tutti i ticket associati ad una determina versione, questa può essere rilasciata. Inoltre, una versione, oltre ai ticket che comprende, possiede dei campi che specificano la data di inizio, la data di fine ed una breve descrizione.
\subsubsubsubsection{Backlog e Sprint}
In Jira sono integrati diversi strumenti per lo sviluppo secondo il metodo Agile, tra i quali è importante evidenziare Backlog e Sprint.\\
Il backlog contiene una lista di ticket da completare dal team, ed è ordinata in base alla priorità: i più importanti sono posti in cima, mentre i meno importanti sono disposti verso il fondo. La lista non è statica, ma è uno spazio dinamico all'interno del quale il team può aggiungere, eliminare, aggiornare e riorganizzare le priorità dei ticket al suo interno. Il backlog è inteso come punto di partenza per pianificare il lavoro: prima di ogni sprint, il team esamina il backlog per selezionare il lavoro da svolgere durante quello sprint.\\
Uno sprint è un periodo di tempo predefinito in cui vengono completati i ticket selezionati dal backlog prima del suo inizio. Ogni sprint è dotato di una data di inizio, una data di fine e di uno stato, che può essere "In Corso" o "Terminato".
\subsubsubsubsection{Timeline}
La timeline messa a diposizione in Jira è uno strumento realizzato tramite un diagramma di Gantt che aiuta a gestire le scadenze, le dipendenze e l'andamento del progetto, fornendo una panoramica d'insieme dello stato di avanzamento.\\
Essa mostra tutti i ticket associati ad una epic che sono stati inseriti al suo interno, evidenziandone le date di inizio e fine e le dipendenze con altri ticket. Ogni ticket viene visualizzato come un blocco che si estende lungo la timeline in base alla durata prevista.\\
Le dipendenze tra i vari ticket possono essere visualizzate tramite linee di collegamento, mostrando come il completamento di un'attività dipenda da un’altra. Inoltre, la timeline mostra chiamente anche lo stato delle attività, ovvero quali attività sono in corso, quali attività sono state completate e quali attività sono in ritardo.\\
Un'altra informazione mostrata nella timeline sono le versioni e, in particolare, quando sono state fissate le loro date di scadenza. Le versioni sono rappresentate graficamente come delle linee verticali posizionate proprio sulla data di scadenza corrispondente.\\
\'E inoltre possibile visualizzare gli sprint definiti all'interno di Jira nell'area superiore della timeline, permettendo così di determinare quali attività sono state svolte durante ciascuno sprint.\\
Infine, è utile notare che nella timeline e possibile effettuare delle operazioni di filtraggio dei ticket visualizzati, permettendo così di visualizzare anche l'organizzazione di specifici gruppi di attività.
\subsubsubsection{Pull Request}
Le pull request sono un meccanismo per la gestione controllata delle modifiche nei rilasci del prodotto. Quando un membro del team propone modifiche al repository, esse vengono raggruppate in una pull request, che consente ai verificatori di analizzarle prima di integrarle nel repository.\\
Il processo può essere riassunto nei seguenti passaggi:
\begin{itemize}
    \item \textbf{Creazione della pull request}: Non appena le modifiche vengono apportate in un branch dedicato, il contributore apre una pull request per proporre la loro integrazione nel branch develop o in un altro branch di destinazione;
    \item \textbf{Revisione}: I verificatori esaminano le modifiche proposte direttamente all'interno della pull request, visualizzando i dettagli di ciò che è stato modificato e avendo la possibilità di commentare su una singola riga o su più righe/sezioni. Se le modifiche richieste sono minime, il verificatore può correggerle direttamente eseguendo una commit sul branch in cui sono state suggerite le modifiche;
    \item \textbf{Feedback e suggerimenti}: Se le modifiche sono significative, i verificatori solitamente forniscono suggerimenti per miglioramenti, evidenziano possibili errori ed elencano tutte le modifiche necessarie per allineare il lavoro agli standard e alle linee guida del progetto;
    \item \textbf{Applicazione delle correzioni}: Il contributore che ha creato la pull request risponde al feedback ricevuto ed apporta le modifiche necessarie. Successivamente, la pull request viene aggiornata con le modifiche revisionate;
    \item \textbf{Decisione finale}: Quando i verificatori approvano la proposta, o quando tutte le questioni sollevate da questi ultimi sono state risolte, la pull request può essere considerata accettata e viene "unita" al branch di destinazione, garantendo che solo contributi di alta qualità facciano parte del progetto.
\end{itemize}
\subsubsection{Configuration status accounting}
\subsubsubsection{Scopo}
Il configuration status accounting è il processo di documentazione e monitoraggio di verifiche e cambiamenti alle caratteristiche del prodotto software. Grazie ad esso si ha una visione complessiva della sua evoluzione e della sua aderenza ai requisiti e agli standard.
\subsubsubsection{Version control}
Il sistema di version control adottato dal gruppo per tracciare l'evoluzione del software segue la seguqnte convenzione di numerazione delle versioni: [x].[y].[z]([build]). Si tratta di un sistema di versionamento a quattro componenti, ognuna delle quali ha un significato specifico:
\begin{itemize}
    \item \textbf{x}: Rappresenta la versione principale del software, e il suo valore viene incrementato per ogni fase significativa di revisione o avanzamento del progetto;
    \item \textbf{y}: Rappresenta cambiamenti significativi o aggiunte di nuove funzionalità, e il suo valore viene incrementato ogni volta che vengono apportati cambiamenti considerati rilevanti per il prodotto;
    \item \textbf{z}: Rappresenta piccole modifiche o correzioni di bug, e viene incrementato quando vengono svolte azioni come l'aggiornamento della documentazione o la correzione di errori minori;
    \item \textbf{build}: Indica il numero delle build eseguite per una determinata versione, e viene incrementato ogni volta che vengono apportate delle modifiche alla documentazione.
\end{itemize}
Il sistema prevede che la numerazione inizi dalla versione "0.0.1(0)". Ogni volta che il valore di x, y o z aumenta, tutti i valori alla sua destra vengono resettati a "0".
\subsubsection{Configuration evaluation}
\subsubsubsection{Scopo}
Il processo di configuration evaluation serve a garantire la correttezza, la coerenza e la conformità della configurazione del sistema, del software e dell'infrastruttura rispetto ai requisiti definiti.
\subsubsubsection{Tracciamento dei requisiti}
Il team PEBKAC ha deciso di tracciare i requisiti direttamente nel codice del prodotto software. Questo approccio offre un collegamento diretto e verificabile tra i requisiti di progettazione e il codice che soddisfa tali requisiti. Il tracciamento viene effettuato includendo un commento specifico prima di ogni blocco di codice che implementa un determinato requisito. Il commento contiene l'ID univoco del requisito, in modo da facilitare l'associazione tra i requisiti e le corrispondenti implementazioni nel codice.
\subsubsection{Release Management}
\subsubsubsection{Scopo}
Il processo di release management serve a pianificare, coordinare e gestire il rilascio, per far sì che qualsiasi nuova versione di un prodotto venga distribuita dal sistema in modo controllato.
\subsubsubsection{Automazione compilazione documenti}
Il gruppo si è dotato di una GitHub Action che provvede all'automazione della generazione e trasferimento di documenti LaTeX in PDF.\\
Essa si attiva quando viene unita una Pull Request nel branch develop e funziona nel seguente modo: 
\begin{enumerate}
    \item Prepara l'ambiente ed installa gli strumenti necessari;
    \item Esplora la directory contenente i documenti LaTeX;
    \item Converte eventuali immagini in PDF e compila i file LaTeX;
    \item Carica i PDF generati in una cartella organizzata;
    \item Trasferisce la cartella sul branch main;
    \item Sposta i PDF all'interno della cartella corretta nel \textit{main branch} mediante un commit automatico.
\end{enumerate}
Il componente del gruppo che si è occupato della redazione di un documento dovrà semplicemente occuparsi di aprire una Pull Request\textsubscript{G} verso il \textit{branch develop}. Essa verrà verificata dal membro del gruppo che ricopre attualmente il ruolo di verificatore e, in seguito, verrà approvata dal membro del gruppo che ricopre attualmente il ruolo di responsabile, che, infine, chiuderà la Pull Request, eseguendo l'automazione.\\
Questa automazione permette di eliminare il lavoro manuale ripetitivo e permette al team di concentrarsi sul contenuto invece che sulla gestione tecnica dei file.