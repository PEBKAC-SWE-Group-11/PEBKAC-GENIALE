\section{Metriche di qualità}
\subsection{Metriche di qualità del processo}
\subsubsection{Fornitura}
\begin{itemize}
    \item \textbf{CV (Cost Variance):} Misura la deviazione dei costi rispetto al budget, se il costo è negativo significa che si è sforato il limite del budget(SPI \( > \) 1: in anticipo rispetto ai tempi pianificati, SPI \( < \) 1: in ritardo rispetto ai tempi pianificati).
    \\CV = EV - AC.
    \item \textbf{PV (Planed Value):} Il valore pianificato, ovvero il costo stimato del lavoro previsto in un determinato momento del progetto.
    \item \textbf{EV (Earned Value):} Rappresenta il valore guadagnato, che rappresenta il costo stimato del lavoro effettivamente completato in quel momento.
    \item \textbf{AC (Actual Cost):} Rappresenta il costo effettivo, cioè quanto è stato realmente speso fino a quel punto.
    \item \textbf{CPI (Cost Performance Index):} Indica se il progetto sta spendendo meno o più del previsto (CPI \( > \) 1: sotto budget, CPI \( < \) 1: sopra budget). \\ La formula è \( \text{CPI} = \frac{\text{EV}}{\text{AC}} \).
    \item \textbf{SPI (Schedule Performance Index):} Rappresenta l'efficienza temporale con cui il lavoro pianificato è stato completato rispetto a quanto programmato.
    \\ La formula è \( \text{SPI} = \frac{\text{EV}}{\text{PV}} \).
    \item \textbf{BAC (Budget At Completion):} Rappresenta il budget totale pianificato per il completamento del progetto.
    \item \textbf{EAC (Estimated At Completion):} Rappresenta l'aggiornamento della stima del valore per la realizzazione del progetto, ovvero il BAC ricalcolato in base allo stato attuale del progetto.
    La formula è \( \text{EAC} = \frac{\text{BAC}}{\text{CPI}} \).
    \item \textbf{VAC (Variance At Completion):} Rappresenta la differenza tra il budget previsto e quello attuale alla fine del progetto.\\
    La formula è VAC = BAC - EAC.
    \item \textbf{ETC (Estimated To Completion):} Rappresenta la valutazione del costo supplementare richiesto per portare a termine il progetto.\\
    La formula è ETC = EAC - AC.
    \item \textbf{SV (Schedule Variance):} Indica se le attività pianificate del progetto sono in linea, anticipate o in ritardo rispetto alla programmazione.\\
    La formula è SV = EV - PV.
    \item \textbf{BV (Budget Variance):} Indica se, alla data attuale, le spese sostenute sono superiori o inferiori rispetto a quanto originariamente previsto nel budget.\\
    La formula è BV = PV - AC.
\end{itemize}
\newpage
\subsubsection{Sviluppo}
\begin{itemize}
    \item \textbf{SC (Statement Coverage):} Rappresenta la percentuale di istruzioni nel codice che vengono eseguite durante i test.
    La formula è \( \text{SC} = \frac{\text{N Statement eseguiti}}{\text{N Statement totali}}*100 \).
\end{itemize}
\subsubsection{Documentazione}
\begin{itemize}
    \item \textbf{IG (Indice Gulpease):} Rappresenta un indicatore per analizzare la facilità di lettura di un testo scritto in italiano. L’Indice Gulpease si basa su due variabili linguistiche principali: la lunghezza delle parole e quella delle frasi. \\
    La formula per determinarlo è: \( \text{IG} = 89+\frac{\text{300*NF-NL}}{\text{NP}} \), dove:
    \begin{itemize}
        \item \textbf{NF:} Indica il numero delle frasi.
        \item \textbf{NL:} Indica il numero delle lettere.
        \item \textbf{NP:} Indica il numero di parole.
    \end{itemize}
    Questo indice fornisce un punteggio che varia da 0 a 100. I possibili punteggi possono essere:
    \begin{itemize}
        \item \textbf{0-55:} Testo incomprensibile.
        \item \textbf{56-70:} Testo molto difficile.
        \item \textbf{71-80:} Testo difficile.
        \item \textbf{81-95:} Testo facile.
        \item \textbf{95-100:} Testo molto facile.
    \end{itemize}
    Per calcolarlo viene utilizzato un software online: \url{https://farfalla-project.org/readability_static/}
    \item \textbf{CO (Correttezza Ortografica):} Rappresenta il numero di errori grammaticali ed ortografici che presenta un documento.
\end{itemize}
\subsubsection{Gestione delle qualità}
\begin{itemize}
    \item \textbf{MNS (Metriche Non Soddisfatte):} Rappresenta le quantità di metriche che il progetto non riesce a soddisfare o mantenere.
\end{itemize}
\newpage
\subsection{Metriche per la qualità del prodotto}
\subsubsection{Funzionalità}
\begin{itemize}
    \item \textbf{ROS (Requisiti Obbligatori Soddisfatti):} Rappresenta la percentuale di requisiti obbligatori che sono stati soddisfatti durante la creazione del prodotto.\\
    La formula è \( \text{ROS} = \frac{\text{requisiti obbligatori soddisfatti}}{\text{requisiti obbligatori totali}}*100 \).
    \item \textbf{RDS (Requisiti Desiderabili Soddisfatti):} Rappresenta la percentuale di requisiti desiderabili che sono stati soddisfatti durante la creazione del prodotto.\\
    La formula è \( \text{RDS} = \frac{\text{requisiti desiderabili soddisfatti}}{\text{requisiti desiderabili totali}}*100 \).
    \item \textbf{RPS (Requisiti Opzionali Soddisfatti):} Rappresenta la percentuale di requisiti opzionali che sono stati soddisfatti durante la creazione del prodotto.\\
    La formula è \( \text{RPS} = \frac{\text{requisiti opzionali soddisfatti}}{\text{requisiti opzionali totali}}*100 \).
\end{itemize}

\subsubsection{Affidabilità}
\begin{itemize}
    \item \textbf{PTCP (Passed Test Cases Percentage):} Rappresenta la percentuale di casi di test completati con successo rispetto al numero totale di casi di test pianificati.\\
    La formula è \( \text{PTCP} = \frac{\text{test superati}}{\text{test totali}}*100 \).
    \item \textbf{CC (Code Coverage):} Rappresenta il numero di linee di codice Verificate con esito positivo all'interno di un processo di test.\\
    La formula è \( \text{CC} = \frac{\text{linee di codice scritte}}{\text{linee di codice totali}}*100 \).
\end{itemize}

\subsubsection{Manutenibilità}
\begin{itemize}
    \item \textbf{SFIN (Structure Fan IN):} Rappresenta la quantità di moduli o componenti che interagiscono direttamente o dipendono da un modulo o una funzione specifica. Un valore elevato suggerisce che molte parti del sistema fanno affidamento su quel particolare modulo.
    \item \textbf{SFOUT (Structure Fan Out):} Rappresenta la quantità di connessioni o relazioni che un componente o modulo ha con altri elementi del sistema. Questa misura riflette il numero di moduli che interagiscono o su cui si basa un determinato modulo. Un fan-out elevato può segnalare che un modulo è fortemente dipendente da molti altri.
\end{itemize}
\subsubsection{Efficienza}
\begin{itemize}
    \item \textbf{TDE (Tempo di Elaborazione):} Rappresenta il tempo di risposta dal momento in cui vengono inseriti dati all'interno del prodotto software al momento in cui vengono visualizzati dall'utente in questo caso l'installatore.
\end{itemize}