\section{Qualità di processo}
La qualità di processo si basa sull’idea che, per realizzare un prodotto conforme a specifici standard qualitativi, è essenziale monitorare e migliorare regolarmente i processi che lo generano. Questo principio si applica all’intera gamma di attività, pratiche e metodologie impiegate durante il ciclo di vita del software.
In altre parole, la qualità dei processi ha l'obiettivo di andare a conformare la qualità del prodotto in modo tale da garantire sempre che gli standard definidi nel documento \textit{Norme di Progetto} vengano rispettati ed eventualmente migliorati. Di seguito sono elencate le metriche che il team si impegna a rispettare per garantire l’eccellenza nei processi. La sigla \textit{MPC} sta ad indicare le metriche di processo.
\subsection{Processi primari}
\subsubsection{Fornitura}

\begin{table}[H]
    \centering
    \begin{tabularx}{\textwidth}{>{\hsize=0.7\hsize}X|X|>{\centering\arraybackslash}X|>{\hsize=0.8\hsize}>{\centering\arraybackslash}X}
        \textbf{Metrica} & \textbf{Descrizione} & \textbf{Valore Accettabile} & \textbf{Valore Ideale} \\ \hline
        
         MPC - CV& Cost variance& \(\ge -7.5\%\) & \(\ge 0\%\) \\ \hline
         MPC - PV& Planned Value& \(\ge 0\) & \(\le BAC\) \\ \hline
         MPC - EV& Earned Value& \(\ge 0\) & \(\le EAC\) \\ \hline
         MPC - AC& Actual Cost& \(\ge 0\)  & \(\le EAC\) \\ \hline
         MPC - CPI& Cost Performance Index & tra 0.95 e 1.05& 1 \\ \hline
         MPC - EAC& Estimated At Completion & deviazione del 5\% del BAC & BAC \\ \hline 
         MPC - VAC& Variance At Completion & deviazione del 10\% del BAC & 0\% \\ \hline
         MPC - ETC& Estimated To Completion & \(\ge 0\) & \(\le EAC\) \\ \hline
         MPC - SV& Schedule Variance & deviazione del 10\% del BAC & 0\% \\ \hline
         MPC - BV& Budget Variance & deviazione del 10\% del BAC & 0\% \\ 
         
    \end{tabularx}
    \caption{Documenti del ciclo di vita del prodotto SW}
\end{table}

\subsubsection{Sviluppo}
\subsubsubsection{Codifica}

\begin{table}[H]
    \centering
    \begin{tabularx}{\textwidth}{>{\hsize=0.7\hsize}X|X|>{\centering\arraybackslash}X|>{\hsize=0.8\hsize}>{\centering\arraybackslash}X}
   
        \textbf{Metrica} & \textbf{Descrizione} & \textbf{Valore accettabile} & \textbf{Valore ideale}  \\
        \hline
        MPC-SC & Statement Coverage &  0 & \(\ge 100\%\) \\
        
    \end{tabularx}
    \caption{Tabella riguardante le metriche per il processo di codifica}
\end{table}

\subsection{Processi di supporto}
\subsubsubsection{Documentazione}

\begin{table}[H]
    \centering
    \begin{tabularx}{\textwidth}{>{\hsize=0.7\hsize}X|X|>{\centering\arraybackslash}X|>{\hsize=0.8\hsize}>{\centering\arraybackslash}X}
   
        \textbf{Metrica} & \textbf{Descrizione} & \textbf{Valore accettabile} & \textbf{Valore ideale}  \\
        \hline
        MPC-IG & Indice Gulpease & \(\ge65\%\) & 100 \\
        \hline
        MPC-CO & Correttezza Ortografica & 0 & 0 \\

        
    \end{tabularx}
    \caption{Tabella riguardante le metriche per il processo di documentazione}
\end{table}

\subsubsubsection{Gestione della qualità}

\begin{table}[H]
    \centering
    \begin{tabularx}{\textwidth}{>{\hsize=0.7\hsize}X|X|>{\centering\arraybackslash}X|>{\hsize=0.8\hsize}>{\centering\arraybackslash}X}
   
        \textbf{Metrica} & \textbf{Descrizione} & \textbf{Valore accettabile} & \textbf{Valore ideale}  \\
        \hline
        MPC-MNS & Metriche Non Soddisfatte & \(\le3\) & 0\\
        
    \end{tabularx}
    \caption{Tabella riguardante le metriche per il processo di gestione delle qualità}
\end{table}

