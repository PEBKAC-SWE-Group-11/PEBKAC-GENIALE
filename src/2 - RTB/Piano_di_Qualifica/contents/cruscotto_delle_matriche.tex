\section{Cruscotto delle metriche}
\subsection{Qualità di processo - Fornitura}
\subsubsection{Earned Value, Actual Cost, Planned Value}
\begin{figure}[H]
    \centering
    \includegraphics[width=0.8\textwidth]{./images/EV-AC-PV.png}
    \caption{Earned Value, Actual Cost, Planned Value}
\end{figure}
\subsubsubsection*{Analisi}
Queste metriche indicano un buon andamento essenso sempre abbastanza sovrapposte su tutti gli sprint, anche se si nota in graduale distaccamento verso il quarto sprint a causa dell'aumento dei costi per il raggiungimento degli obiettivi.

\subsubsection{Estimated To Compliation, Estimated At Compliation}
\begin{figure}[H]
    \centering
    \includegraphics[width=0.8\textwidth]{./images/ETC-EAC.png}
    \caption{Estimated To Compliation, Estimated At Compliation}
\end{figure}
\subsubsubsection*{Analisi}
Dopo un inizio non ottimale è possibile notare un riallineamento. L'Estimation At Completion si è riallineato al Badget At Completion, ma è sempre rimasto superiore, anche se di poco, a causa del costo maggiore per il raggiungimento di tutti gli obiettivi degli sprint. L'Estimated To Commpletion è sempre stato gradualmente discendente.
\subsubsection{Budget Variance}
\begin{figure}[H]
    \centering
    \includegraphics[width=0.8\textwidth]{./images/BV.png}
    \caption{Estimated To Compliation, Estimated At Compliation}
\end{figure}
\subsubsubsection*{Analisi}
Anche se la Budget Variance è sempre rimasta nei limiti ammissibili, è evidente come nel quarto sprint, dove la variazione è notevole, le ore preventivate fossero insufficienti e questo deve spingere a una più corretta pianificazione basata sulle retrospettive precedenti.

\subsubsection{Cost Variance, Schedule Variance}
\begin{figure}[H]
    \centering
    \includegraphics[width=0.8\textwidth]{./images/CV-SV.png}
    \caption{Cost Variance, Schedule Variance}
\end{figure}
\subsubsubsection*{Analisi}
Per quanto i valori, soprattutto della Schedule Variance, siano praticamente sempre rimasti accettabili, non si notano dei netti miglioramenti, ma anzi un peggioramento, come in altre metriche, nel corso del quarto sprint. Questo deve spingere a una migliore gestione delle risorse per il raggiungimento degli obiettivi entro i costi preventivati.

\subsubsection{Cost Performace Index}
\begin{figure}[H]
    \centering
    \includegraphics[width=0.8\textwidth]{./images/CPI.png}
    \caption{Cost Performace Index}
\end{figure}
\subsubsubsection*{Analisi}
È evidente come il Cost Performace Index sia migliorato nel corso degli sprint, fino a rientrare nei limiti ammissibli, l'obiettivo deve essere quello di seguire il trend attuale per puntare al valore ideale.




\subsection{Qualità di processo - Gestione della qualità}
\subsubsection{Metriche non soddisfatte}
\begin{figure}[H]
    \centering
    \includegraphics[width=0.8\textwidth]{./images/MNS.png}
    \caption{Metriche non soddisfatte}
\end{figure}
\subsubsubsection*{Analisi}
La metrica che non risulta mai soddisfatta è EAC per la quale si rimanda all'analisi specifica. Altre metriche che risultano non sddisfatte sono ETC (primi due sprint), CPI (primo sprint) e CV (quarto sprint per il quale si rimanda all'analisi specifica)
.
\subsection{Qualità di prodotto - Funzionalità}
\subsubsection{Requisiti soddisfatti}
\begin{figure}[H]
    \centering
    \includegraphics[width=0.8\textwidth]{./images/requisitisoddisfatti.png}
    \caption{Requisiti soddisfatti}
\end{figure}
\subsubsubsection*{Analisi}
Il progetto fino ad ora si è concentrato sulla creazione di un \textit{Proof of Concept}\textsubscript{G}, e non su prodotto finale con il quale soddisfare i requisiti, per cui è normale che nel cruscotto non risulti soddisfatto alcun requisito.

\subsection{Qualità di prodotto - Documentazione}
\subsubsection{Indice di Gulpease}
\begin{figure}[H]
    \centering
    \includegraphics[width=0.8\textwidth]{./images/gulpease.png}
    \caption{Indice di Gulpease}
\end{figure}
\subsubsubsection*{Analisi}
Al termine del secondo sprint tutti i documenti erano in lavorazione ed entro i limiti minimi di leggibilità, si nota peraltro un miglioramento nella leggibilità di tutti i documenti ad esclusione del Piano di Progetto, sul quale il gruppo dovrà concentrarsi di più per garantire una scrittura comprensibile.

\subsection{Considerazioni finali in vista della revisione RTB}