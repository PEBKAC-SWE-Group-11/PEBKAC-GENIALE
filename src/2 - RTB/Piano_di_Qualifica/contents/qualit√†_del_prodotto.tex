\section{Qualità del prodotto}
La qualità del prodotto si concentra sulla valutazione del software sviluppato, ponendo l'accento su caratteristiche come facilità d'uso, funzionalità, affidabilità, capacità di manutenzione e, in senso più ampio, sulle prestazioni complessive del prodotto. L'obiettivo principale del team è anche quello non solo di soddisfare le attese del cliente fornendo un prodotto software che implementi le funzionalità volute, ma che lo faccia seguendo precisi standard di qualità. Vengono quindi fornite di seguito le metriche che il team si impegna a soddisfare nel contesto della qualità del prodotto. La sigla MPD sta per metriche di prodotto come definito nel documento \textit{Norme di progetto v 1.0.0}
\subsection{Funzionalità}

\begin{table}[H]
    \centering
    \begin{tabularx}{\textwidth}{>{\hsize=0.7\hsize}X|X|>{\centering\arraybackslash}X|>{\hsize=0.8\hsize}>{\centering\arraybackslash}X}
   
        \textbf{Metrica} & \textbf{Descrizione} & \textbf{Valore accettabile} & \textbf{Valore ideale}  \\
        \hline
        MPC-ROS & Requisiti Obbligatori Soddisfatti & \(100\%\) & 0\\
        \hline
        MPC-RDS & Requisiti Desiderabili Soddisfatti & \(\ge0\%\) & \(\ge75\%\)\\
        \hline
        MPC-RPS & Requisiti Opzionali Soddisfatti & \(\ge0\%\) & \(\ge50\%\)\\
        
    \end{tabularx}
    \caption{Tabella riguardante le metriche per la funzionalità del prodotto}
\end{table}

\subsection{Manutenibilità}

\begin{table}[H]
    \centering
    \begin{tabularx}{\textwidth}{>{\hsize=0.7\hsize}X|X|>{\centering\arraybackslash}X|>{\hsize=0.8\hsize}>{\centering\arraybackslash}X}
   
        \textbf{Metrica} & \textbf{Descrizione} & \textbf{Valore accettabile} & \textbf{Valore ideale}  \\
        \hline
        MPC-SFIN & Structure Fan In & da determinare & da determinare\\
        \hline
        MPC-SFOUT & Structure Fan Out & da determinare & da determinare\\
        
    \end{tabularx}
    \caption{Tabella riguardante le metriche per la manutenibilità del prodotto}
\end{table}

\subsection{Affidabilità}

\begin{table}[H]
    \centering
    \begin{tabularx}{\textwidth}{>{\hsize=0.7\hsize}X|X|>{\centering\arraybackslash}X|>{\hsize=0.8\hsize}>{\centering\arraybackslash}X}
   
        \textbf{Metrica} & \textbf{Descrizione} & \textbf{Valore accettabile} & \textbf{Valore ideale}  \\
        \hline
        MPC-PTCP & Passed Test Cases Percentage & \(\ge80\%\) & \(100\%\)\\
        \hline
        MPC-CC & Code Coverage & \(\ge80\%\) & \(100\%\)\\
        
    \end{tabularx}
    \caption{Tabella riguardante le metriche per la affidabilità del prodotto}
\end{table}

\subsection{Efficienza}

\begin{table}[H]
    \centering
    \begin{tabularx}{\textwidth}{>{\hsize=0.7\hsize}X|X|>{\centering\arraybackslash}X|>{\hsize=0.8\hsize}>{\centering\arraybackslash}X}
   
        \textbf{Metrica} & \textbf{Descrizione} & \textbf{Valore accettabile} & \textbf{Valore ideale}  \\
        \hline
        MPC-TDE & Tempo Di Elaborazione & da determinare & da determinare\\
        
    \end{tabularx}
    \caption{Tabella riguardante le metriche per l'efficienza del prodotto}
\end{table}