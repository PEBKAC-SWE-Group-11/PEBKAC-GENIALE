\section{Qualità del prodotto}
La qualità del prodotto si concentra sulla valutazione del \textit{software}\textsubscript{G} sviluppato, ponendo l'accento su caratteristiche come facilità d'uso, funzionalità\textsubscript{G}, affidabilità\textsubscript{G}, capacità di manutenzione\textsubscript{G} e, in senso più ampio, sulle prestazioni complessive del prodotto. L'obiettivo principale del \textit{team}\textsubscript{G} è anche quello non solo di soddisfare le attese del cliente fornendo un prodotto \textit{software}\textsubscript{G} che implementi le funzionalità\textsubscript{G} volute, ma che lo faccia seguendo precisi \textit{standard}\textsubscript{G} di qualità. Vengono quindi fornite di seguito le metriche\textsubscript{G} che il \textit{team}\textsubscript{G} si impegna a soddisfare nel contesto della qualità del prodotto. 

\subsection{Funzionalità}

\begin{table}[H]
    \centering
    \begin{tabularx}{\textwidth}{>{\hsize=0.5\hsize}X|X|>{\centering\arraybackslash}X|>{\hsize=0.8\hsize}>{\centering\arraybackslash}X}
   
        \textbf{Metrica} & \textbf{Descrizione} & \textbf{Valore accettabile} & \textbf{Valore ideale}  \\
        \hline
        ROS\textsubscript{G} & Requisiti Obbligatori Soddisfatti & \(100\%\) & 0\\
        \hline
        RDS\textsubscript{G} & Requisiti Desiderabili Soddisfatti & \(\ge0\%\) & \(\ge75\%\)\\
        \hline
        RPS\textsubscript{G} & Requisiti Opzionali Soddisfatti & \(\ge0\%\) & \(\ge50\%\)\\
        
    \end{tabularx}
    \caption{Metriche\textsubscript{G} per la funzionalità\textsubscript{G} del prodotto}
\end{table}

\subsection{Manutenibilità}

\begin{table}[H]
    \centering
    \begin{tabularx}{\textwidth}{>{\hsize=0.5\hsize}X|X|>{\centering\arraybackslash}X|>{\hsize=0.8\hsize}>{\centering\arraybackslash}X}
   
        \textbf{Metrica} & \textbf{Descrizione} & \textbf{Valore accettabile} & \textbf{Valore ideale}  \\
        \hline
        SFIN\textsubscript{G} & \textit{Structure Fan In}\textsubscript{G} & da determinare & da determinare\\
        \hline
        SFOUT\textsubscript{G} & \textit{Structure Fan Out}\textsubscript{G} & da determinare & da determinare\\
        
    \end{tabularx}
    \caption{Metriche\textsubscript{G} per la manutenibilità\textsubscript{G} del prodotto}
\end{table}

\subsection{Affidabilità}

\begin{table}[H]
    \centering
    \begin{tabularx}{\textwidth}{>{\hsize=0.5\hsize}X|X|>{\centering\arraybackslash}X|>{\hsize=0.8\hsize}>{\centering\arraybackslash}X}
   
        \textbf{Metrica} & \textbf{Descrizione} & \textbf{Valore accettabile} & \textbf{Valore ideale}  \\
        \hline
        PTCP\textsubscript{G} & \textit{Passed Test Cases Percentage}\textsubscript{G} & \(\ge80\%\) & \(100\%\)\\
        \hline
        CC\textsubscript{G} & \textit{Code Coverage}\textsubscript{G} & \(\ge80\%\) & \(100\%\)\\
        
    \end{tabularx}
    \caption{Metriche\textsubscript{G} per l'affidabilità\textsubscript{G} del prodotto}
\end{table}

\subsection{Efficienza}

\begin{table}[H]
    \centering
    \begin{tabularx}{\textwidth}{>{\hsize=0.5\hsize}X|X|>{\centering\arraybackslash}X|>{\hsize=0.8\hsize}>{\centering\arraybackslash}X}
   
        \textbf{Metrica} & \textbf{Descrizione} & \textbf{Valore accettabile} & \textbf{Valore ideale}  \\
        \hline
        TDE\textsubscript{G} & Tempo Di Elaborazione\textsubscript{G} & da determinare & da determinare\\
        
    \end{tabularx}
    \caption{Metriche\textsubscript{G} per l'efficienza del prodotto}
\end{table}
