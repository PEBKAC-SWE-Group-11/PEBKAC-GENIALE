\section{Riassunto della riunione}
Nella riunione svoltasi, il gruppo, insieme al proponente, si è concentrato sulle attività svolte nel periodo precedente e sulla pianificazione delle attività future. Per quanto riguarda il periodo trascorso, il gruppo:
\begin{itemize}
    \item Ha aggiornato i proponente degli attuali ruoli;
    \item Ha presentato un riassunto dell'incontro avvenuto con il Prof. Cardin in merito alla RTB;
\end{itemize}

Per quanto riguarda il futuro invece, il gruppo: 
\begin{itemize}
    \item Si prepara a redigere una presentazione per la RTB con il Prof. Vardanega;
    \item Si impegna a correggere gli errori nel documento di Analisi dei Requisiti riportati dal Prof. Cardin;
    \item Inizierà una fase di test di vari modelli di LLM in modo da fare una scelta ponderata su quale modello considerare definitivamente;
    \item Un'altra critica mossa dal Prof. Cardin è la scelta del framework Flask che, a sua detta, non sembra una scelta sufficientemente pesata. Nonostante la critica, il gruppo ha deciso comunque di mantenere il corrente framework per lo sviluppo del backend poiché fornisce delle API semplici ed efficienti che difficilmente si possono trovare in altri linguaggi;
    \item Si prepara ad aggiornare l'atteggiamento del modello LLM integrato nell'applicativo perché risponda come se fosse una persona esterna all'azienda. 
\end{itemize}

A livello organizzativo, invece, il gruppo si organizza in vista della consegna del PB (prevista per il 21 marzo). In merito, il gruppo, insieme al proponente, dopo quest'ultimo sprint di due settimane, si sottoporrà a un SAL ogni settimana ma di durata minore per garantire una qualità sufficiente dell'applicativo. Inoltre è richiesto di presentare una versione funzionante qualche tempo prima del PB così che alcuni membri dell'azienda possano testarlo.