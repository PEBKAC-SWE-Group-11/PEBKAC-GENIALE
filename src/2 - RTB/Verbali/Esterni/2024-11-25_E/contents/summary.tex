\section{Riassunto della riunione}
Dopo una breve premessa, si è proceduto con l'ordine del giorno previsto per le riunioni SAL:
\begin{enumerate}
    \item \textbf{Presentazione dei ruoli}: vengono elencati i ruoli assunti dai membri del gruppo in questo periodo:
    \begin{itemize}
        \item Alessandro Benin - Verificatore;
        \item Ion Bourosu - Analista;
        \item Matteo Gerardin - Analista;
        \item Derek Gusatto - Amministratore;
        \item Davide Martinelli - Analista;
        \item Matteo Piron - Analista;
        \item Tommaso Zocche - Responsabile.
    \end{itemize}
    \item \textbf{Panoramica ad ampio spettro}: sono state brevemente discusse le due revisioni necessarie per l'avanzamento e la conclusione del progetto, e sono state comunicate all'azienda le date in cui il gruppo ha stimato la candidatura per queste ultime:
    \begin{itemize}
        \item \textbf{RTB}: tra l'8 e il 24 gennaio 2025;
        \item \textbf{PB}: 14 marzo 2025.
    \end{itemize}
    \item \textbf{Attività completate ed in corso}: sono state esposte le attività che sono state svolte in questo periodo, suddividendole principalmente in stesura della documentazione e ricerca tecnologica.\\
    Le prime riguardano la redazione di:
    \begin{itemize}
        \item Piano di progetto;
        \item Norme di progetto;
        \item Glossario;
        \item Analisi dei requisiti.
    \end{itemize}
    Queste attività sono tuttora in corso, in quanto deve essere terminata la prima stesura di tutti i documenti sopraelencati. Si è trattata con attenzione l'analisi dei requisiti, ed in particolare i casi d'uso, per cui sono stati forniti dei chiarimenti.\\
    Le seconde invece riguardano lo studio individuale effettuato dai componenti del gruppo relativamente ai modelli LLM ed alle tecnologie di web scraping proposte dall'azienda. Nello specifico:
    \begin{itemize}
        \item Llama 3.1;
        \item Mistral;
        \item Bert;
        \item Phi;
        \item Scrapy;
        \item OCRmyPDF.
    \end{itemize}
    \item \textbf{Prossime attività da svolgere}: nel prossimo periodo si è deciso di:
    \begin{itemize}
        \item Proseguire con la redazione di piano di progetto, norme di progetto, glossario ed analisi dei requisiti;
        \item Effettuare un allineamento interno al gruppo relativamente alle tecnologie studiate individualmente;
        \item Assegnare le varie tecnologie necessarie al progetto ai vari membri e procedere con delle prove pratiche, volte a concretizzare ciò su cui ci si è informati.
    \end{itemize}
    \item \textbf{Discussione di dubbi e domande}: vengono riportate di seguito le domande che sono state poste durante la riunione con le relative risposte:
    \begin{enumerate}[label=\Alph*)]
        \item \textbf{Come attori principali sono stati individuati l'installatore, utente principale del sistema, e l'amministratore, che effettua un'attività di monitoraggio sul sistema. Devono essere riconosciuti come attori anche componenti come, ad esempio, il modello LLM e il database?}\\
        Nonostante sia possibile definire ed utilizzare degli attori che non rappresentino degli utenti ma dei servizi esterni al sistema, è sconsigliato l'utilizzo di questi ultimi, in quanto porterebbe ad un analisi dei casi d'uso troppo approfondita. Inoltre è stata già giustificata correttamente la scelta dei due attori principali menzionati.
        \item \textbf{Relativamente al confronto dei modelli LLM proposti dall'azienda, oltre alla precisione, esistono delle metriche specifiche per questo tipo di applicazione che possono essere utilizzate per valutarne l'output?}\\
        Le metriche che potranno essere utilizzate per valutare l'output fornito dal modello LLM comprendono:
        \begin{itemize}
            \item Verifica della corretta comprensione del contesto della domanda da parte del modello;
            \item Verifica della correttezza del informazioni recuperate dal modello;
            \item Verifica della comprensibilità dell'output.
        \end{itemize}
        \item \textbf{\'E consigliato maggiormente utilizzare un unico repository oppure crearne diversi, ad esempio, per documentazione e codice?}\\
        Nonostante l'utilizzo di repository multiple permetta la realizzazione di un numero arbitrario di queste ultime per gestire le diverse componenti del sistema, è sconsigliabile in quanto introduce una complessità considerevole nella fase di versionamento e allineamento.\\
        L'utilizzo di una singola repository, caratterizzato dalla separazione dei componenti in cartelle e dall'utilizzo di branch differenti per apportare modifiche a questi ultimi, è quindi fortemente consigliato, in quanto permette di avere una buona suddivisione a livello di file system ed a livello logico senza richiedere una gestione complessa.
    \end{enumerate}
    Sono stati inoltre discussi alcuni dubbi relativi all'implementazione di alcuni requisiti opzionali, che si è deciso di non considerare per il momento.
\end{enumerate}