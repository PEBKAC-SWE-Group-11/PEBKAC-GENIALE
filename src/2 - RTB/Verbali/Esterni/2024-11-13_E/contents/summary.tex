\section{Riassunto della riunione}
Dopo una prima fase di presentazioni ufficiali ed esposizione delle regole della riunione, sono state prese le seguenti decisioni:
\begin{enumerate}
    \item \textbf{Riunioni SAL}: le riunioni di stato avanazamento lavori avranno la durata di un'ora si terranno ogni 14 giorni, fino alla prima revisione con il completamento del PoC. I membri del team che dovranno di volta in volta confermare la presenza o segnalare l'eventuale assenza. Come momento più adatto per questi incontri è stato scelto il lunedì pomeriggio, ma ci potrebbero essere delle modifiche qualora fosse necessario. Le riunioni si terranno su Microsoft Teams o negli uffici di Vimar, facendone rischiesta;
    
    \item \textbf{Ordine del Giorno per i SAL}: viene concordato con l'azienda proponente il seguente ordine del giorno per le riunioni periodiche:
        \begin{enumerate}[label=\Roman* - ]
            \item Presentazione dei ruoli dei membri del gruppo in quello slot temporale
            \item Panoramica ad ampio spettro degli avanzamenti con i rispettivi tempi
            \item Attività completate o in corso 
            \item Prossime attività da svolgere
            \item Discussione di eventuali problemi riscontrati
        \end{enumerate}
    Se ci dovranno essere modifiche al precedente ordine del giorno andranno segnalate per tempo;

    \item \textbf{Condivisione documenti}: con lo scopo di evitare scambio di mail superflue i documenti che Vimar S.p.A. dovrà approvare verranno caricati su un cartella condivisa di Google Drive;

    \item \textbf{Condivisione codice sorgente}: per facilitare, quando necessario, lo scambio di opinioni sui codici sorgente verrà aggiunto Mariano Sciacco come utente in sola lettura nei repo contenenti sorgenti di progetto.

\end{enumerate}

Sono state poi presentate al gruppo alcune direttive da parte di Vimar S.p.A. per quanto riguarda le interazioni, nello specifico
\begin{enumerate}[label=\Alph*)]
    \item \textbf{Approfondimenti}: gli approfondimenti (possibili sia in presenza che da remoto) devono essere richiesti 2 settimane in anticipo e devono essere motivati. Inoltre il gruppo si impegna ad arrivare preparato agli approfondimenti per favorirne l'efficenza e l'efficacia;

    \item \textbf{Microsoft Teams}: deve essere usata la chat delle riunioni periodiche, solo per domande veloci o piccoli approfondimenti;

     \item \textbf{E-mail}: dovranno essere comunicate via mail le domande più articolate e le richieste di modifica degli appuntamenti. 
    
\end{enumerate}

Successivamente ci sono state delle richieste di chiarimenti sul Capitolato.