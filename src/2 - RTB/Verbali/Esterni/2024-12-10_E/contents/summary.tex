\section{Riassunto della riunione}
Come previsto, si è proceduto seguendo l'ordine del giorno previsto per le riunioni SAL:
\begin{enumerate}
    \item \textbf{Presentazione dei ruoli}: vengono elencati i ruoli assunti dai membri del gruppo in questo periodo:
    \begin{itemize}
        \item Alessandro Benin - Progettista;
        \item Ion Bourosu - Analista;
        \item Matteo Gerardin - Responsabile;
        \item Derek Gusatto - Analista;
        \item Davide Martinelli - Verificatore;
        \item Matteo Piron - Amministratore;
        \item Tommaso Zocche - Analista.
    \end{itemize}
    \item \textbf{Panoramica ad ampio spettro}: sono stati discussi gli avanzamenti effettuati sulla documentazione, prestando particolare attenzione ai casi d'uso, presenti nell'Analisi dei Requisiti, ed inoltre è stato effettuato un confronto con i rappresentanti dell'azienda proponente relativamente a quanto emerso dallo studio delle tecnologie.
    \item \textbf{Attività completate ed in corso}: sono state esposte le attività che sono state svolte in questo periodo, suddividendole principalmente in stesura della documentazione e ricerca tecnologica.\\
    Le prime riguardano i progressi effettuati con la stesura di:
    \begin{itemize}
        \item Piano di Progetto;
        \item Norme di Progetto;
        \item Glossario;
        \item Analisi dei Requisiti.
    \end{itemize}
    Inoltre, in questo periodo, è stata iniziata la redazione del Piano di Qualifica. Nonostante gli avanzamenti raggiunti, queste attività sono tuttora in corso, in quanto deve essere ancora terminata la prima stesura di tutti i documenti sopraelencati. Ci si è soffermati particolarmente sull'Analisi dei Requisiti, ed in particolare sui casi d'uso, che sono stati presi nuovamente in esame dopo che sono emerse ulteriori informazioni che li riguardano, e per cui si sono discusse ulteriori modifiche che sarà necessario applicare.\\
    Le seconde invece riguardano lo studio e la sperimentazione pratica delle tecnologie necessarie alla realizzazione del progetto. Nello specifico sono state oggetto di queste attività:
    \begin{itemize}
        \item Flask, Angular e Vue.js per la realizzazione del frontend;
        \item PostgreSQL con l'estensione pgvector (per la creazione degli indici vettoriali) per la realizzazione del database vettoriale;
        \item Scrapy e OCRmyPDF per il web scraping;
        \item Python per la realizzazione del backend;
        \item Llama 3.1, Mistral, Bert e Phi come modello LLM.
    \end{itemize}
    Oltre ad essere state analizzate e testate individualmente, si è anche provato ad integrare alcune di esse.
    \item \textbf{Prossime attività da svolgere}: nel prossimo periodo si è deciso di:
    \begin{itemize}
        \item Effettuare un'analisi e delle prove con un nuovo modello LLM proposto, ovvero OpenGPT-x;
        \item Effettuare delle prove di interrogazione dei modelli LLM utilizzando dei dataset presi da della documentazione di Vimar;
        \item Studiare ed effettuare delle prove con Docker per la progettazione e la realizzazione dell'architettura e realizzare un primo schema di quest'ultima;
        \item Effettuare ulteriori prove per le operazioni di scraping e di embedding per le immagini;
        \item Effettuare la redazione di un verbale interno in cui vengono elencate le scelte prese in merito alle tecnologie ed in cui vengono spiegate e motivate queste scelte;
        \item Proseguire con le prove di integrazione fra le varie tecnologie, in direzione della realizzazione del PoC;
        \item Proseguire con la redazione di Piano di Progetto, Norme di Progetto, Glossario, Analisi dei Requisiti e Piano di Qualifica.
    \end{itemize}
    \item \textbf{Discussione di dubbi e domande}: vengono riportate di seguito le domande che sono state poste durante la riunione con le relative risposte:
    \begin{enumerate}[label=\Alph*)]
        \item \textbf{Riguardo la scelta della tecnologia per il frontend, dopo un cofronto effettuato è risulta che Flask ed Angular svolgono due funzioni sostanzialmente diverse. \`E possibile utilizzare Flask insieme ad Angular? Si può scegliere solamente uno dei due scartando l'altro in tranquillità?}\\
        Mentre Angular è un linguaggio che viene utilizzato totalmente a livello frontend, Flask può essere utilizzato sia per realizzare delle API (quindi per costruire sostanzialmente un server di backend che fornisca delle risposte) sia per renderizzare una pagina web. Quindi è possibile utilizzare Angular che avrà poi bisogno di essere collegato tramite delle API, che possono essere realizzate anche con lo stesso Flask, oppure è possibile utilizzare direttamente Flask per la renderizzazione della pagina, senza doversi successivamente preoccupare di gestire delle API per realizzare questo tipo di implementazione.
        \item \textbf{Perché si dovrebbe preferire Angular a Vue.js?}\\
        Angular è un linguaggio che esiste da parecchi anni e che ha subito svariati aggiornamenti, ed è molto conosciuto dalla community, quindi in un futuro potrebbe essere utilizzato nel curriculum. Inoltre possiede una serie di pattern che sono in grado di semplificare il lavoro, al costo di saperli utilizzare. Infine è un linguaggio molto rigoroso. Vue.js, invece, è un linguaggio molto giovane che risulta essere particolarmente complesso per certi aspetti, ma è un'ottima alternativa se si vuole evitare la rigorosità di Angular.
        \item \textbf{Esiste un modo per verificare se le risposte fornite dall'LLM sono corrette e conformi alla documentazione presa dal sito di Vimar?}\\
        \`E possibile utilizzare due diverse strategie:
        \begin{itemize}
            \item \`E possibile caricare il pdf di un manuale su GPT e chiedere delle possibili domande con le relative risposte, andando poi a porre le stesse domande al nostro modello LLM e confrontando le sue risposte con quelle fornite da GPT;
            \item \`E possibile porre direttamente una domanda relativa alla documentazione su GPT, che dovrebbe essere in grado di rispondere dato che si tratta di documenti pubblici, e, successivamente, è possibile porre la stessa domanda al nostro modello LLM e confrontare le due risposte.
        \end{itemize}
        \item \textbf{Nella fase di web scraping è necessario estrerre tutte le informazioni disponibili nella documentazione? Ed è necessario farlo per tutte le lingue disponibili o solamente per una o due?}\\
        Alcune delle informazioni presenti nella documentazione sono ripetute in molteplici manuali, esattamente nello stesso modo, quindi sarà necessario ottimizzare il database per gestire queste ridondanze. Inoltre, relativamente alle versioni dei manuali in differenti lingue, è necessario estrarne solamente tre diverse, dato che le traduzioni sono letterali e non si rischia di avere perdita di informazione.
        \item \textbf{Come è possibile effettuare l'operazione di scraping delle immagini presenti nella documentazione? Un possibilità che abbiamo considerato è stata generare un'alternativa testuale tramite un modello LLM che esprima il contenuto informativo dell'immagine? \`E una soluzione valida?}\\
        Può essere una soluzione assolutamente valida, ma necessiterà di una pipeline automatizzata, in quanto effettuare questo procedimento manualmente per tutte le immagini richiederebbe un quantitativo enorme di tempo. Un'altra alternativa che potrebbe essere valutata è "fare una foto" all'intera pagina del manuale e trovare un modo per far sì che il modello LLM riconosca che in essa è presente un'immagine, permettendone così l'estrazione.
        \item \textbf{\`E necessario implementare l'embedding delle immagini da parte del modello LLM per renderle disponibili nel database e poi poterle utilizzare?}\\
        Per poter effettuare questa operazione è necessario che il modello sia multimodale, ovvero che sia in grado di comprendere sia testo che immagini, e potenzialmente anche audio. Sarebbe inoltre necessario capire come elaborare il tutto in maniera ottimale. Un'alternativa che può essere utilizzata per realizzare questo obiettivo è, una volta realizzata completamente la parte di embedding nel database, quando si passono gli indici che contengono le informazioni necessarie al modello per fornire la risposta, si può fare in modo che recuperi anche il link dell'immagine che "potrebbe tornare utile" all'utente che ha posto una determinata domanda, senza che il modello sappia effettivamente cosa rappresenta l'immagine in questione.
        \item \textbf{Se il modello LLM possiede la capacità di comprendere le immagini molto bene si può evitare di realizzare l'alternativa testuale che era stata discussa precedentemente?}\\
        In questo caso si potrebbe evitare di creare l'alternativa testuale, ma sarebbe più complesso gestire il reperimento delle immagini nel database. Una possibilità per la gestione di questa problematica potrebbe essere inserire in linguaggio naturale un contesto iniziale da passare al modello per istruirlo a svolgere questa operazione di ricerca dell'immagine oltre alle informazioni testuali. Viene comunque consigliato di analizzare per gradi il problema dell'embedding delle immagini e di trascurarlo per il momento in favore dell'embedding del testo, ritenuto più importante e semplice.
        \item \textbf{\`E giusto considerare come attori anche degli elementi interni al sistema?}\\
        Un attore può essere definito come chiunque o qualunque parte del sistema che abbia delle necessità che devono essere in qualche modo sopperite per raggiungere un obiettivo. In virtù di questa definizione, quindi, possono essere considerati attori anche componenti interni al sistema, come ad esempio il modello LLM, nonostante questa scelta debba essere chiaramente motivata.
        \item \textbf{\`E possibile utilizzare dei diagrammi di attività invece dei diagrammi di casi d'uso per rappresentare i casi d'uso che riguardano gli attori interni al sistema?}\\
        S\`i, è possibile in quanto, grazie alle regole a cui questo diagramma fa riferimento, è possibile capire chiaramente il flusso delle attività che si verificano e di quelle che non si verificano.
    \end{enumerate}
    \`E stato inoltre chiesto di spostare la data per la prossima riunione SAL dal 24/12/2024 al 23/12/2024 con orario da decidere da parte del gruppo, e che verrà comunicato ai referenti dell'azienda tramite email.
\end{enumerate}