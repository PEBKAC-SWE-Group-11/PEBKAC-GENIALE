
\documentclass[12pt, a4paper]{article}

\usepackage{graphicx}
\usepackage{xcolor}
\usepackage{float}
\usepackage{svg}
\usepackage[colorlinks=true, linkcolor=black, urlcolor=blue, citecolor=green]{hyperref}
\usepackage{enumitem}
\usepackage[italian]{babel}
\usepackage{lastpage}  % Pacchetto per ottenere il numero totale delle pagine
\usepackage{fancyhdr}  % Pacchetto per personalizzare l'intestazione e il piè di pagina
\usepackage[margin=1in]{geometry}
\usepackage{array}
\newcolumntype{C}[1]{>{\centering\arraybackslash}p{#1}}
\newcolumntype{L}[1]{>{\raggedright\arraybackslash}p{#1}}
\graphicspath{ {images/} {../shared/images/} }
\definecolor{unipd}{HTML}{B5121B}

\addto\captionsitalian{\renewcommand{\contentsname}{Indice}}


\pagestyle{fancy}% Imposta lo stile di pagina su "fancy"
\fancyhf{}% Cancella intestazioni e piè di pagina
\fancyfoot[C]{\thepage{} di \pageref{LastPage}} % Imposta il piè di pagina centrale come "numero pagina di totale pagine"
\renewcommand{\headrulewidth}{0pt} % Imposta la larghezza della linea di intestazione a 0 punti

\newcommand{\data}{16 dicembre 2024}
\newcommand{\titolo}{Verbale Interno}
\newcommand{\responsabile}{Alessandro Benin}
\newcommand{\verificatori}{Davide Martinelli\\&Tommaso Zocche}
\newcommand{\redattore}{Matteo Gerardin}
\newcommand{\uso}{Interno}
\newcommand{\destinatari }{
   %& Vimar S.p.A.  \\
    & Tullio Vardanega  \\
    & Riccardo Cardin  }
\newcommand{\abstractcontent}{Riunione di allineamento in risposta all'anticipazione della prossima riunione SAL: l’ordine del giorno ha riguardato la visone dei progressi nello sviluppo del Proof of Concept, la specifica dei dubbi emersi e la definizione dei prossimi passi da svolgere}

\begin{document}

\begin{minipage}[]{0.3\textwidth}
\includesvg[width=\linewidth]{pebkac.svg} 
\end{minipage}
\hspace{0.1\textwidth}
\begin{minipage}[]{0.6\textwidth}
  {\Large \textbf{PEBKAC}} \\
  Email: \href{mailto:pebkacswe@gmail.com}{pebkacswe@gmail.com} \\
  Gruppo: 11
\end{minipage}

\bigskip

\begin{minipage}[]{0.3\textwidth}
\includesvg[width=\linewidth]{logo_unipd.svg} 
\end{minipage}
\hspace{0.1\textwidth}
\begin{minipage}[]{0.6\textwidth}
  \textcolor{unipd}{
    \textbf{Università degli Studi di Padova} \\
    Corso di Laurea: Informatica \\
    Corso: Ingegneria del Software \\
    Anno Accademico: 2024/2025
  }
\end{minipage}


\bigskip
\bigskip
\bigskip
\begin{center}
  \Huge\textbf{Verbale Interno}

  \Large\textbf{\data}
\end{center}

\bigskip


\begin{center}
\textbf{Informazioni sul documento}: \\
\vspace{0.5cm}

\begin{tabular}{r|l}
    \textbf{Responsabile} & Tommaso Zocche \\ 
    \textbf{Verificatore} & Alessandro Benin \\ 
    \textbf{Redattore} & Tommaso Zocche \\ 
    \textbf{Uso} & Interno \\ 
    \textbf{Destinatari} & Tullio Vardanega \\ & Riccardo Cardin \\ 
\end{tabular}

\vfill

\textbf{Abstract}: \\
\vspace{0.5cm}
L'obiettivo dell'incontro è stato definire l'ordine di preferenza dei capitolati a seguito degli incontri avuti con le aziende interessate e iniziare a redigere il prospetto orario del gruppo.
\end{center}


\bigskip
\newpage

\section*{Registro delle modifiche}
\begin{table}[H]
    \begin{tabular}{|c|c|c|c|p{5cm}|}
        \hline
         \textbf{Versione} &  \textbf{Data} &  \textbf{Autore} &  \textbf{Ruolo} & \textbf{Descrizione} \\
          \hline
          &  &  & Responsabile & Approvazione e rilascio\\
          \hline
          0.1.0 & 17/11/2024 & Alessandro Benin & Verificatore  & Verificato \\
          \hline
          0.0.1 & 13/11/2024 & Derek Gusatto & Amministratore  & Stesura iniziale \\
          \hline
    \end{tabular}
\end{table}
\newpage
\tableofcontents
\newpage
% VERBALE
\section{Informazioni generali}
\begin{itemize}
  \item \textbf{Tipo riunione}: esterna
  \item \textbf{Luogo}: telematica, Teams
  \item \textbf{Data}: 25 novembre 2024
  \item \textbf{Ora inizio}: 16.00
  \item \textbf{Ora fine}: 17.00
  
  \item \textbf{Presenti}:
  \begin{itemize}
    \item Alessandro Benin
    \item Ion Bourosu
    \item Matteo Gerardin
    \item Derek Gusatto
    \item Davide Martinelli
    \item Matteo Piron
    \item Tommaso Zocche
    \item[$\star$] Mariano Sciacco (Vimar S.p.A.)
    \item[$\star$] Francesca Stival (Vimar S.p.A.)
  \end{itemize}

  \item \textbf{Assenti}:
 
\end{itemize}
\newpage
\section{Riassunto della riunione}
Nella presente riunione ci siamo allineati sugli ultimi task mancanti per la candidatura alla prima revisione RTB, che il gruppo si è impegnato a prensentare entro la fine della giornata. Risultano quindi mancanti:
\begin{itemize}
    \item La verifica del documento Norme di Progetto;
    \item L'aggiornamento del sito di presentazione ;
    \item La Lettera di Presentazione;
\end{itemize}

Viene quindi stilata dal gruppo la Lettera di Presentazione e il presente verbale.
\newpage
\section{Todo}
Durante la riunione sono emersi i seguenti task da svolgere.

\begin{center}
  \begin{tabular}{|p{5cm}|p{8cm}|}
    \hline
    \textbf{Assegnatario}       & \textbf{Task Todo} \\ \hline
     Derek Gusatto   &  Stesura Verbale Esterno 13/11/2024\\ \hline
     \textit{autoassegnazione}  & Condivisione del verbale e dei repo GitHub con Vimar S.p.A. \\ \hline
     \textit{autoassegnazione}  & Condivisione con Vimar S.p.A. dei nominativi e rispettive e-mail dei membri del gruppo \\ \hline
     Mariano Sciacco  & Invio invito riunione periodica per SAL \\ \hline
  \end{tabular}
\end{center}
\vspace{4cm}
\noindent Firma del referente Vimar S.p.A.: \underline{\hspace{5cm}}

%LISTA FIGURE
%\listoffigures 
%\newpage
%LISTA TABELLE
%\listoftables
%\newpage

\end{document}
