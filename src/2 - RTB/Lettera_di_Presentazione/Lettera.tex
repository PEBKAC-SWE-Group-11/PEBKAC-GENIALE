\documentclass[12pt, a4paper]{article}

\usepackage{graphicx}
\usepackage{array}
\usepackage{xcolor}
\usepackage{float}
\usepackage{svg}
\usepackage{eurosym}
\usepackage[colorlinks=true, linkcolor=black, urlcolor=blue, citecolor=green]{hyperref}

\graphicspath{ {images/} {../shared/images/} }
\definecolor{unipd}{HTML}{B5121B}
\newcommand{\data}{}


\begin{document}

\input{contents/header}
\begin{tabular}{@{}l l@{}}
  Egregio Prof. Tullio Vardanega, &  \\Egregio Prof. Riccardo Cardin,
\end{tabular}
\\

Con la presente lettera di presentazione il gruppo PEBKAC intende comunicare ufficialmente la decisione di presentarsi alla revisione di avanzamento RTB (\textit{Requirements and Technology Baseline}), relativa al progetto denominato:

{\huge\begin{center}\textbf{Vimar GENIALE}\end{center}}

da voi commissionato e proposto da 
{\begin{center}\textit{Vimar S.p.A.}\end{center}}

La documentazione richiesta per sostenere la revisione è disponibile al seguente link:

\begin{center}
\href{https://pebkac-swe-group-11.github.io}{https://pebkac-swe-group-11.github.io}
\end{center}

\newpage
Nello specifico si può trovare:
\begin{itemize}
    \item Documentazione interna
    \begin{itemize}
        \item Norme di Progetto V1.0.0
        \item Verbali interni            
    \end{itemize}
    \item Documentazione esterna
    \begin{itemize}
        \item Analisi dei Requisiti V1.0.0
        \item Piano di Progetto V1.0.0
        \item Piano di Qualifica V1.0.0
        \item Glossario V1.0.0
        \item Verbali esterni            
    \end{itemize}
    \item \textit{Proof of Concept}
\end{itemize}

Si prevede di effettuare la consegna finale entro e non oltre il 14/03/2024, confermando quanto affermato in precedenza al momento della candidatura. Viene confermato anche il budget totale necessario  a portare a termine il progetto di 12850,00\euro.


\bigskip
Di seguito si riportano i nominativi dei componenti del gruppo e i corrispondenti numeri di matricola:
\medskip
\begin{center}
\renewcommand{\arraystretch}{1.5} % Aumenta l'altezza delle righe
\begin{tabular}{| >{\centering\arraybackslash}m{0.5\textwidth} | >{\centering\arraybackslash}m{0.3\textwidth} |}
\hline
\textbf{Nominativo} & \textbf{Matricola} \\
\hline
Alessandro Benin & 2042356 \\
\hline
Ion Bourosu & 2010006 \\
\hline
Matteo Gerardin & 2075536 \\
\hline
Derek Gusatto & 2042330 \\
\hline
Davide Martinelli & 2077679 \\
\hline
Matteo Piron & 2076044 \\
\hline
Tommaso Zocche & 2075547 \\
\hline
\end{tabular}
\end{center}
\medskip
Per qualsiasi chiarificazione rimaniamo a Vostra completa disposizione.
\bigskip
\bigskip
\begin{flushright}
Cordiali Saluti, \\
il gruppo \textit{PEBKAC}
\end{flushright}


\end{document}
