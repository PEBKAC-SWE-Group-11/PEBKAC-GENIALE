\section{Introduzione}
\subsection{Scopo del Documento}
L'analisi dei Requisiti è un documento fondamentale per tutti i progetti di sviluppo software che vogliono essere alla stato dell'arte come definiti dalla materia di Ingegneria del Software.
Lo scopo del docuemento è quello di andare a descrivere e a definire le funzionalità che dovranno essere implementate nella creazione del prodotto software finale in modo tale da essere conformi alle richieste gestite dall'azienda proponente\textsubscript{G}.
L'analisi non deve fornire la soluzione al problema ma deve essere consapevole della sua fattibilità tecnologica(definendo un confine tra analisi del problema, il suo design e la soluzione). Lo scopo di questo documento può esssere riassunto in questi punti:
\begin{itemize}
    \item \textbf{Definire le esigenze dei proponenti:} \\
    Il documento di Analisi dei Requisiti nasce dalle esigenze del proponente, ovvero quanto richiesto in merito al prodotto software desiderato. Queste richieste vengono raccolte tramite il capitolato fornito dal proponente Vimar e a riunioni con il proponiente.
    \item \textbf{Tracciare i requisiti del sistema:} \\
    Una volta raccolte le esigenze del prodotto, il documento di analisi dei requisiti deve identificare tutti i casi d'uso necessari e i corrispettivi requisiti associati  ad ognuno di essi suddividendoli in funzionali,qualitativi e di vincolo.
    \item \textbf{Verificazione e validazione dei requisiti:}
    Il processo\textsubscript{G} di verifica dei requisiti\textsubscript{G} ha lo scopo di garantire che le attività vengano svolte secondo il way of working\textsubscript{G} del gruppo.\\
    La validazione dei requisiti invece consiste nella verifica che il prodotto software finale corrisponda alle attese fornite dall'analisi dei requisiti.
    \item \textbf{Fornire una base per la progettazione del sistema:}
    Un documento di Analisi dei Requisiti fornisce una solida base per la progettazione del sistema\textsubscript{G}, in quando definisce le funzionalità che il sistema deve offrire. I programmatori che poi dovranno andare a sviluppare il prodotto software utilizzeranno questo documento di analisi dei rischi per comprendere le esigenze del proponente e identificare le soluzioni più appropriate per soddisfare tali esigenze. 
    \item \textbf{Minimizzare i rischi e ottimizzare i costi:}
    Un documento di Analisi dei Requisiti completo e accurato può aiutare a ridurre i rischi del progetto e quantificare al meglio i suoi costi. Ciò è dovuto al fatto che il documento aiuta a garantire che i requisiti siano effettivamente corretti, evitando errori che possono ritardare lo sviluppo del sistema.
\end{itemize}
Arrivati ad una chiara visione sulle funzionalità del prodotto, dei suoi requisiti e degli attori del sistema software che si andrà a implementare, si andrà a dare una formale rappresentazione grafica utilizzando i diagrammi dei casi d'uso.
\subsection{Scopo del Prodotto} 
Il proponente del capitolato d'appalto C2 è l'azienda Vimar S.p.A. Lo scopo dell'azienda è quello di offrire prodotti di design e soluzioni tecnologicamente avanzate per gestire l'energia elettrica. L'azienda propone lo sviluppo di VIMAR GENIALE, , un applicativo che consiste in un interfaccia chatbot\textsubscript{G} dove gli installatori possono
interrogare per reperire informazioni testuali e grafiche sui prodotti Vimar presenti all’interno del
sito ufficiale.

\subsubsection{Funzionalità}
Questa interfaccia chat ha lo scopo di andare a reperire informazioni direttamente dai documenti forniti dall'azienda. L'applicativo deve essere: 
\begin{itemize}
    \item responsive e funzionare via a browser da smartphone, tablet e
    desktop.
    \item deve avere un sistema di conversazione libera attraverso cui l’installatore può
    fare domande e ricevere risposte in linguaggio naturale, similmente a una chat.
    \item deve avere un sistema di feedback attraverso cui un utente dopo ogni
    risposta può indicare se la risposta ottenuta è stata positiva o negativa.
    \item deve mostrare una sezione protetta da password che contenga un cruscotto
    informativo (dashboard) per amministratori.
    \item  deve prevedere un sistema di estrazione e raccolta delle informazioni dal sito
    web.
    \item deve prevedere un sistema di indicizzazione delle informazioni a partire dal
    database in cui sono stati salvati i dati precedentemente estratti dal sito web.
    \item deve prevedere un componente di interrogazione di un modello AI (LLM).
\end{itemize}

\subsection{Glossario}
Per evitare ambiguità relative al linguaggio utilizzato nei documenti, viene fornito il Glossario V1.0.0, nel quale si possono trovare tutte le definizioni di termini che hanno un significato specifico che vuole essere disambiguato. Tali termini sono marcati con una G a pedice.
\subsection{Riferimenti}
\subsubsection{Riferimenti  normativi}  
\begin{itemize}
    \item \textbf{Norme di Progetto}:\\
    \url{https://www.lipsum.com} 

    \item \textbf{Capitolato d'Appalto C2}: Vimar GENIALE (data di ultimo accesso: YYYY-MM-DD)\\
    \url{https://www.math.unipd.it/~tullio/IS-1/2024/Progetto/C2.pdf}
\end{itemize}
\subsubsection{Riferimenti informativi}
\begin{itemize}
    \item \textbf{T5 - Analisi dei Requisiti} (data di ultimo accesso: YYYY-MM-DD)\\
    \url{https://www.math.unipd.it/~tullio/IS-1/2024/Dispense/T05.pdf}
    
    \item \textbf{Glossario} \\
    \url{https://www.lipsum.com/}
\end{itemize}
