\section{Introduzione}
\subsection{Scopo del Documento}
Lo scopo del documento di analisi dei requisiti è quello di fornire una spiegazione di ciò che deve fare il prodotto per soddisfare i bisogni del Proponente. In particolare viene specificato cosa ci si aspetta faccia il prodotto e quali siano i suoi principali utilizzi. \\
In questo modo i requisiti offrono una rappresentazione corretta dei bisogni dell’utente descrivendo cosa deve succedere attraverso:
\begin{itemize}
    \item La descrizione di casi d’uso;
    \item La descrizione di User Stories;
\end{itemize}
Questi poi verranno raggruppati in scenari a loro volta ordinati in base alla loro affinità con le varie parti del sistema. In questo modo il documento chiarisce il passaggio dalla situazione da senza il prodotto (anche detta ”AS-IS”) a quella con il prodotto (“TO-BE”) ma soprattutto fa chiarezza sulla fattibilitá tecnologica di esso.

\subsection{Scenario di riferimento} 
Nel presente documento, il termine Proponente si riferisce all'azienda VIMAR, una delle principali realtà italiane nel settore elettrico. L’azienda offre prodotti per la distribuzione di elettricità nell’edilizia principalmente in ambienti casalinghi. Da qualche anno ormai tra i suoi prodotti ne sono comparsi alcuni che implementano diverse automazioni, pertanto, essendo cresciuta la complessità di questi, l’azienda ha deciso di rivolgere alcune risorse per la ricerca e lo sviluppo di soluzioni che potessero rendere piú accessibile la documentazione. La documentazione relativa al montaggio e la manutenzione dei dispositivi è disponibile sul sito internet dell’azienda, tuttavia, visto il grande numero di prodotti, il Proponente chiede di realizzare un sistema informativo in grado di reperire le istruzione più facilmente e velocemente. Viste le grandi potenzialità offerte dall’intelligenza artificiale, il Proponente vuole basare il funzionamento dell’app sull’interrogazione di uno di questi modelli.

\subsection{Glossario}
Per evitare ambiguità relative al linguaggio utilizzato nei documenti, viene fornito il Glossario V1.0.0, nel quale si possono trovare tutte le definizioni di termini che hanno un significato specifico che vuole essere disambiguato. Tali termini sono marcati con una G a pedice.
\subsection{Riferimenti}
\subsubsection{Riferimenti  normativi}  
\begin{itemize}
    \item \textbf{Norme di Progetto}:\\
    \url{https://www.lipsum.com} 

    \item \textbf{Capitolato d'Appalto C2}: Vimar GENIALE (data di ultimo accesso: YYYY-MM-DD)\\
    \url{https://www.math.unipd.it/~tullio/IS-1/2024/Progetto/C2.pdf}
\end{itemize}
\subsubsection{Riferimenti informativi}
\begin{itemize}
    \item \textbf{T5 - Analisi dei Requisiti} (data di ultimo accesso: YYYY-MM-DD)\\
    \url{https://www.math.unipd.it/~tullio/IS-1/2024/Dispense/T05.pdf}
    
    \item \textbf{Glossario} \\
    \url{https://www.lipsum.com/}
\end{itemize}
