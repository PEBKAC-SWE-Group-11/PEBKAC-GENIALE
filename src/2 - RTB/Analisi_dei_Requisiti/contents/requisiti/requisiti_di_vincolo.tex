\begin{table}[H]
\centering
    \begin{tabular}{|C{2.7cm}|L{7.2cm}|C{2.7cm}|C{2cm}|}
        \hline
    \textbf{ID requisito} & \textbf{Descrizione} & \textbf{Importanza} & \textbf{Fonti}  \\
    \hline
           RV.O.001 & Il sistema deve integrare un modello AI (LLM) open source. & Obbligatorio & Capitolato \\
          \hline 
          RV.O.002 & L’infrastruttura Cloud deve utilizzare Docker insieme a Docker Compose, al fine di rispettare il principio di Infrastructure as Code. & Obbligatorio & Capitolato \\
           \hline
          RV.D.003 & L'applicativo può essere ospitato su AWS. & Desiderabile & Capitolato \\
          \hline
          RV.O.004 & Il modello AI (LLM) deve essere Open Source.
         & Obbligatorio & Capitolato \\
        \hline
        RV.O.005 & Il componente di interrogazione deve essere in grado di interfacciarsi con il sistema di indicizzazione e con il modello AI (LLM).
         & Obbligatorio & Capitolato \\
        \hline
        RV.O.006 & Il componente di interrogazione deve poter essere contattato da un altro servizio sotto-forma di API autenticata (ad esempio tramite API-KEY)
         & Obbligatorio & Capitolato \\
         \hline
    \end{tabular}
    \caption{Requisiti di vincolo (1\textsuperscript{a}  parte)}
\end{table}
\begin{table}[H]
\centering
    \begin{tabular}{|C{2.7cm}|L{7.2cm}|C{2.7cm}|C{2cm}|}
        \hline
        RV.O.007 &  L’infrastruttura deve utilizzare la tecnologia dei container.
         & Obbligatorio & Capitolato \\
        \hline
         RV.O.008 & Il risultato atteso è che la parte applicativa possa essere costruita e replicata con un solo comando.
         & Obbligatorio & Capitolato \\
        \hline
        RV.O.009 & Il repository di lavoro deve essere versionato tramite Git e deve essere pubblicamente accessibile.
         & Obbligatorio & Capitolato \\
        \hline
        RV.O.010 & La licenza per i sorgenti dovrà essere open source.
         & Obbligatorio & Capitolato \\
        \hline
        RV.O.011 & Il modello AI (LLM) dovrà fare uso dell’approccio RAG.
         & Obbligatorio & Capitolato \\
        \hline
        RV.O.012 & L’applicazione deve essere compatibile con il browser Chrome dalla
        versione 108.
         & Obbligatorio & Interno \\
        \hline
        RV.O.013 & L’applicazione deve essere compatibile con il browser Edge dalla versione 94.0.992.31.
         & Obbligatorio & Interno \\
        \hline
        RV.O.014 & L’applicazione deve essere compatibile con il browser Opera dalla
        versione 95.
         & Obbligatorio & Interno \\
        \hline
        RV.O.015 & L’applicazione deve essere compatibile con il browser Firefox dalla
versione 109.
         & Obbligatorio & Interno \\
        \hline
        RV.O.016 & L’applicazione deve essere compatibile con il browser Safari dalla
versione 16.
         & Obbligatorio & Interno \\
        \hline
    \end{tabular}
    \caption{Requisiti di vincolo (2\textsuperscript{a}  parte)}
\end{table}