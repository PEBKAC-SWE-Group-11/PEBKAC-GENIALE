\section{Tracciamento Requisiti}
\begin{table}[H]
\centering
    \begin{tabular}{|C{3.0cm}|C{3.0cm}|}
        \hline
        \textbf{Fonte} & 
        \textbf{ID requisito} \\
        \hline
        \textit{Capitolato}\textsubscript{G} & 
        RF.D.006 \newline 
        RF.P.007 \newline 
        RF.D.011 \newline 
        RF.P.017 \newline 
        RF.O.019 \newline 
        RF.O.020 \newline 
        RF.O.024 \newline 
        RF.O.025 \newline 
        RF.D.030 \newline 
        RF.D.031 \newline 
        RF.D.045 \newline 
        RV.O.001 \newline 
        RV.O.002 \newline 
        RV.D.003 \newline 
        RV.O.004 \newline 
        RV.O.005 \newline 
        RV.O.006 \newline 
        RV.O.007 \newline 
        RV.O.008 \newline 
        RV.O.009 \newline 
        RV.O.010 \newline 
        RV.O.011 \\
        \hline
        Interno & 
        RQ.O.001 \newline
        RQ.O.002 \newline
        RQ.O.003 \newline
        RQ.O.004 \newline
        RV.O.012 \newline
        RV.O.013 \newline
        RV.O.014 \newline
        RV.O.015 \newline
        RV.O.016 \\
        \hline
        UC1 &
        RF.D.033 \newline
        RF.D.034 \\
        \hline
        UC2 &
        RF.O.014 \\
        \hline
        UC3 &
        RF.O.012 \\
        \hline
        UC4 &
        - \\
        \hline
        UC5 &
        RF.O.013 \newline
        RF.D.032 \\
        \hline
    \end{tabular}
    \caption{Suddivisione dei requisiti per fonte (1\textsuperscript{a} parte)}
\end{table}
\begin{table}[H]
\centering
    \begin{tabular}{|C{3.0cm}|C{3.0cm}|}
        \hline
        UC6 &
        RF.O.001 \newline
        RF.O.026 \newline
        RF.O.027 \newline
        RF.O.028 \newline
        RF.D.029 \\
        \hline
        UC7 &
        RF.O.009 \\
        \hline
        UC8 &
        RF.O.036 \\
        \hline
        UC9 &
        RF.O.010 \\
        \hline
        UC10 &
        - \\
        \hline
        UC11 &
        RF.O.004 \\
        \hline
        UC12 &
        RF.O.002 \\
        \hline
        UC13 &
        - \\
        \hline
        UC14 &
        RF.O.003 \\
        \hline
        UC14.1 &
        RF.P.015 \\
        \hline
        UC14.2 &
        RF.P.015 \\
        \hline
        UC14.3 &
        RF.P.016 \\
        \hline
        UC14.4 &
        RF.D.018 \\
        \hline
        UC14.5 &
        RF.D.018 \\
        \hline
        UC15 &
        RF.O.005 \newline
        RF.D.008 \newline
        RF.O.035 \\
        \hline
        UC16 &
        RF.O.001 \newline
        RF.O.023 \newline
        RF.O.026 \newline
        RF.O.027 \newline
        RF.O.028 \newline
        RF.D.029 \newline
        RF.O.038 \\
        \hline
        UC17 &
        RF.D.021 \newline
        RF.O.022 \\
        \hline
        UC18 &
        RF.O.037 \\
        \hline
        UC- &
        RF.P.039 \newline
        RF.P.040 \newline
        RF.P.041 \newline
        RF.P.042 \newline
        RF.P.043 \newline
        RF.P.044 \\
        \hline
    \end{tabular}
    \caption{Suddivisione dei requisiti per fonte (2\textsuperscript{a} parte)}
\end{table}

\subsection{Riepilogo}
\begin{table}[H]
\centering
    \begin{tabular}{|C{3.0cm}|C{2.5cm}|C{2.5cm}|C{2.5cm}|}
        \hline
         \textbf{Tipologia} &
         \textbf{Obbligatori} & 
         \textbf{Desiderabili} &
         \textbf{Opzionali} 
          \\
          \hline
          \textbf{Funzionali} & 25 & 09 & 11 \\
          \hline 
          \textbf{Qualità} & 05 & 01 & 00\\
          \hline
          \textbf{Vincolo} & 15 & 01 & 00\\
          \hline
          \textbf{Totale} & 45 & 11 & 11\\
          \hline
    \end{tabular}
    \caption{Suddivisione dei requisiti per tipologia}
\end{table}
