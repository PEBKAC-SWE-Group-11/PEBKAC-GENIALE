\section{Requisiti}
In questa sezione verranno elencati tutti i requisiti specificati nel capitolato, organizzandoli in categorie distinte per una maggiore chiarezza. Ogni requisito è identificato in modo univoco da un codice, strutturato secondo un formato predefinito, che ne facilita la consultazione e il riferimento all'interno della documentazione con il seguente formato: 
\begin{center}
\textbf{R[Tipo].[Importanza].[Codice]}
\end{center}
dove:
\begin{itemize}
    \item \textbf{Numero:} un numero progressivo a tre cifre che parte dal numero 001.
    \item \textbf{Tipo:} la tipologia di requisito che può essere tra le seguenti:
    \begin{itemize}[label=-]
        \item \textbf{F (funzionale):} Una funzione di sistema descrive il modo in cui un sistema utilizza determinati ingressi per generare specifiche uscite, seguendo una logica o una regola prestabilita che definisce il suo comportamento.
        \item \textbf{Q (qualità):} definisce le caratteristiche di qualità del prodotto software.
        \item \textbf{V (vincolo):} specifica i limiti e le restrizioni imposte dal capitolato, che il prodotto software deve rispettare.
\end{itemize}
    \item \textbf{Priorità:} la priorità viene assegnata al requisito con:
    \begin{itemize}
     \item \textbf{O}: requisito obbligatorio che deve essere soddisfatto per la realizzazione del prodotto software;
        \item \textbf{D}: requisito desiderabile il cui soddisfacimento è apprezzato dal committente;
        \item \textbf{P}: requisito facoltativo, la cui realizzazione è totalmente a discrezione del team in base all'andamento del progetto.
    \end{itemize}
\end{itemize}
In alcuni casi sarà specificato nella colonna delle fonti se il requisito è stato esplicitamente indicato nel Capitolato oppure se è stato dedotto implicitamente da altri requisiti obbligatori. In quest’ultimo caso, si farà riferimento a un requisito interno.
\subsection{Registro di Funzionalità}
\begin{table}[H]
    \begin{tabular}{|C{2.7cm}|L{7.2cm}|C{2.7cm}|C{2cm}|}
        \hline
        \textbf{ID requisito} & \textbf{Descrizione} & \textbf{Importanza} & \textbf{Fonti}  \\
        \hline
        RF.O.1 & Il sistema deve permettere all'installatore di effettuare ricerche testuali e ricevere informazioni dettagliate sui prodotti Vimar. & Obbligatorio & Capitolato \\
        \hline
        RF.O.2 & Il sistema deve prevedere un sistema di autenticazione tramite password per l'accesso alla dashboard per amministratori. & Obbligatorio & UC12 \\
        \hline
        RF.O.3 & Il cruscotto informativo deve includere una sezione per la visualizzazione di statistiche di utilizzo. & Obbligatorio & UC14 \\
        \hline
        RF.O.4 & Il sistema deve permettere agli utenti di fornire un feedback positivo o negativo dopo ogni risposta ricevuta. & Obbligatorio & UC14 \\
        \hline
        RF.O.5 & Il sistema deve essere in grado di identificare e bloccare le richieste che riguardano argomenti non pertinenti ai prodotti VIMAR. & Obbligatorio & UC15 \\
        \hline
        RF.D.6 & Il sistema potrebbe includere la possibilità di visualizzare link di riferimento alle fonti delle informazioni fornite. & Desiderabile & Capitolato \\
        \hline
        RF.P.7 & Il sistema deve fornire un'interfaccia con menu e sottomenù per costruire richieste specifiche in conversazioni guidate & Opzionale & Capitolato\\
        \hline
        RF.O.8 & Il sistema deve consentire solo conversazioni pertinendi ai prodotti Vimar e bloccare conversazioni su argomenti proibiti & Obbligatorio & Capitolato \\
        \hline
        \end{tabular}

\end{table}
\subsection{Registro di Qualità}
\begin{longtable}{|C{2.7cm}|L{7.2cm}|C{2.7cm}|C{2cm}|}
        \hline
        \textbf{ID requisito} & \textbf{Descrizione} & \textbf{Importanza} & \textbf{Fonti}  \\
        \hline
       
        \hline
        RQ.D.001 & L'interfaccia utente del sistema potrebbe essere responsive, adattandosi a diversi dispositivi. & Desiderabile & Capitolato \\
        \hline
        RQ.O.002 & Il sistema deve essere progettato per essere facilmente eseguibile su altri dispositivi utilizzando la tecnologia dei container. & Obbligatorio & Capitolato \\
        
        \hline
        RQ.O.003 & \'E necessario fornire un documento che descriva le attività di bug\textsubscript{G} reporting effettuate. & Obbligatorio & Interno \\
        \hline
        RQ.O.004 & Il progetto deve essere svolto seguendo le regole contenute nel documento Norme di Progetto. & Obbligatorio & Interno \\
        \hline
        RQ.O.005 & \'E necessario fornire al proponente il codice sorgente dell'applicativo in un
        repository\textsubscript{G} GitHub. & Obbligatorio & Interno \\
        \hline
        RQ.O.006 & \'E necessario fornire il Manuale Utente dell'applicativo. & Obbligatorio & Interno \\
        \hline
        
        


\end{longtable}
\subsection{Requisiti di Vincolo}

\begin{longtable}{|C{2.7cm}|L{7.2cm}|C{2.7cm}|C{2cm}|}
        \hline
    \textbf{ID requisito} & \textbf{Descrizione} & \textbf{Importanza} & \textbf{Fonti}  \\
    \hline
           RV.O.001 & Il sistema deve integrare un modello AI (LLM) open source. & Obbligatorio & Capitolato \\
          \hline 
          RV.O.002 & L’infrastruttura Cloud deve utilizzare Docker insieme a Docker Compose, al fine di rispettare il principio di Infrastructure as Code. & Obbligatorio & Capitolato \\
           \hline
          RV.D.003 & L'applicativo può essere ospitato su AWS. & Desiderabile & Capitolato \\
          \hline
          RV.O.004 & Il modello AI (LLM) deve essere Open Source.
         & Obbligatorio & Capitolato \\
        \hline
        RV.O.005 & Il componente di interrogazione deve essere in grado di interfacciarsi con il sistema di indicizzazione e con il modello AI (LLM).
         & Obbligatorio & Capitolato \\
        \hline
        RV.O.006 & Il componente di interrogazione deve poter essere contattato da un altro servizio sotto-forma di API autenticata (ad esempio tramite API-KEY)
         & Obbligatorio & Capitolato \\
        \hline
        RV.O.007 &  L’infrastruttura deve utilizzare la tecnologia dei container.
         & Obbligatorio & Capitolato \\
        \hline
         RV.O.008 & Il risultato atteso è che la parte applicativa possa essere costruita e replicata con un solo comando.
         & Obbligatorio & Capitolato \\
        \hline
        RV.O.009 & Il repository di lavoro deve essere versionato tramite Git e deve essere pubblicamente accessibile.
         & Obbligatorio & Capitolato \\
        \hline
        RV.O.010 & La licenza per i sorgenti dovrà essere open source.
         & Obbligatorio & Capitolato \\
        \hline
        RV.O.011 & Il modello AI (LLM) dovrà fare uso dell’approccio RAG.
         & Obbligatorio & Capitolato \\
        \hline
        RV.O.012 & L’applicazione deve essere compatibile con il browser Chrome dalla
        versione 108.
         & Obbligatorio & Interno \\
        \hline
        RV.O.013 & L’applicazione deve essere compatibile con il browser Edge dalla versione 94.0.992.31.
         & Obbligatorio & Interno \\
        \hline
        RV.O.014 & L’applicazione deve essere compatibile con il browser Opera dalla
        versione 95.
         & Obbligatorio & Interno \\
        \hline
        RV.O.015 & L’applicazione deve essere compatibile con il browser Firefox dalla
versione 109.
         & Obbligatorio & Interno \\
        \hline
        RV.O.016 & L’applicazione deve essere compatibile con il browser Safari dalla
versione 16.
         & Obbligatorio & Interno \\
        \hline


        
        

\end{longtable}