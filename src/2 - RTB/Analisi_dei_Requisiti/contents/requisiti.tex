\section{Requisiti}
In questa sezione verranno elencati tutti i requisiti specificati nel \textit{capitolato}\textsubscript{G}, organizzandoli in categorie distinte per una maggiore chiarezza. Ogni \textit{requisito}\textsubscript{G} è identificato in modo univoco da un codice, strutturato secondo un formato predefinito, che ne facilita la consultazione e il riferimento all'interno della \textit{documentazione}\textsubscript{G} con il seguente formato: 
\begin{center}
\textbf{R[Tipo].[Importanza].[Codice]}
\end{center}
dove:
\begin{itemize}
    \item \textbf{Codice:} un numero progressivo a tre cifre che parte dal numero 001.
    \item \textbf{Tipo:} la tipologia di \textit{requisito}\textsubscript{G} che può essere tra le seguenti:
    \begin{itemize}[label=-]
        \item \textbf{F (funzionale):} Una funzione di \textit{sistema}\textsubscript{G} descrive il modo in cui un \textit{sistema}\textsubscript{G} utilizza determinati ingressi per generare specifiche uscite, seguendo una logica o una regola prestabilita che definisce il suo comportamento.
        \item \textbf{Q (qualità):} definisce le caratteristiche di qualità del prodotto \textit{software}\textsubscript{G}.
        \item \textbf{V (vincolo):} specifica i limiti e le restrizioni imposte dal \textit{capitolato}\textsubscript{G}, che il prodotto \textit{software}\textsubscript{G} deve rispettare.
\end{itemize}
    \item \textbf{Importanza:} la priorità viene assegnata al \textit{requisito}\textsubscript{G} con:
    \begin{itemize}
     \item \textbf{O}: \textit{requisito}\textsubscript{G} obbligatorio che deve essere soddisfatto per la realizzazione del prodotto \textit{software}\textsubscript{G};
        \item \textbf{D}: \textit{requisito}\textsubscript{G} desiderabile il cui soddisfacimento è apprezzato dal \textit{committente}\textsubscript{G};
        \item \textbf{P}: \textit{requisito}\textsubscript{G} facoltativo, la cui realizzazione è totalmente a discrezione del team in base all'andamento del progetto.
    \end{itemize}
\end{itemize}
In alcuni casi sarà specificato nella colonna delle fonti se il \textit{requisito}\textsubscript{G} è stato esplicitamente indicato nel \textit{Capitolato}\textsubscript{G} oppure se è stato dedotto implicitamente da altri requisiti obbligatori. In quest’ultimo caso, si farà riferimento a un \textit{requisito}\textsubscript{G} interno.
\subsection{Registro di Funzionalità}


\begin{longtable}{|C{2.7cm}|L{7.2cm}|C{2.7cm}|C{2cm}|}
        \hline
        \textbf{ID requisito} & \textbf{Descrizione} & \textbf{Importanza} & \textbf{Fonti}  \\
        \hline
        RF.O.001 & Il sistema deve permettere all'installatore di effettuare ricerche testuali e ricevere informazioni dettagliate sui prodotti Vimar. & Obbligatorio & Capitolato \\
        \hline
        RF.O.002 & Il sistema deve prevedere un sistema di autenticazione tramite password per l'accesso alla dashboard per amministratori. & Obbligatorio & UC12 \\
        \hline
        RF.O.003 & Il cruscotto informativo deve includere una sezione per la visualizzazione di statistiche di utilizzo. & Obbligatorio & UC14 \\
        \hline
        RF.O.004 & Il sistema deve permettere agli utenti di fornire un feedback positivo o negativo dopo ogni risposta ricevuta. & Obbligatorio & UC14 \\
        \hline
        RF.O.005 & Il sistema deve essere in grado di identificare e bloccare le richieste che riguardano argomenti non pertinenti ai prodotti VIMAR. & Obbligatorio & UC15 \\
        \hline
        RF.D.006 & Il sistema potrebbe includere la possibilità di visualizzare link di riferimento alle fonti delle informazioni fornite. & Desiderabile & Capitolato \\
        \hline
        RF.P.007 & Il sistema deve fornire un'interfaccia con menu e sottomenù per costruire richieste specifiche in conversazioni guidate. & Opzionale & Capitolato\\
        \hline
        RF.D.008 & Il componente di interrogazione deve prevedere un controllo sull’output
        per verificare che il contenuto non vada in conflitto con argomenti proibiti. & Desiderabile & Capitolato \\
        \hline
        RF.O.009 & L’utente deve poter fare richieste testuali limitate a un certo numero di caratteri. & Obbligatorio & Capitolato \\
        \hline
        RF.O.010 & L’utente deve poter visualizzare uno storico dei messaggi nella stessa
        conversazione. & Obbligatorio & Capitolato \\
        \hline
        RF.D.011 & L’utente può fare richieste in più lingue.
         & Desiderabile & Capitolato \\
         \hline
         RF.O.012 & Le conversazioni avute possono essere salvate al termine della sessione. & Obbligatorio & Capitolato \\
        \hline
        RF.O.013 & Le conversazioni devono poter essere cancellate. & Obbligatorio & Capitolato \\
        \hline
        RF.O.014 & Ogni utente dovrà avere un limite massimo di conversazioni.
         & Obbligatorio & Capitolato \\
        \hline
        RF.P.015 & L'amministratore deve poter visualizzare nel cruscotto informativo il numero totale di richieste effettuate con conversazione libera o guidata.
         & Opzionale & Capitolato \\
        \hline
        RF.P.016 & L'amministratore deve poter visualizzare nel cruscotto informativo le statistiche sul numero di termini usati nelle richieste.
         & Opzionale & Capitolato \\
        \hline
        RF.P.017 & L'amministratore deve poter visualizzare nel cruscotto informativo le statistiche sulle parole più usate nelle richieste.
         & Opzionale & Capitolato \\
        \hline
        RF.P.018 & L'amministratore deve poter visualizzare nel cruscotto informativo il grafico con l'andamento giornaliero dell’utilizzo dell’applicativo.
         & Opzionale & Capitolato \\
        \hline
        RF.D.019 & L'amministratore deve poter visualizzare nel cruscotto informativo il numero di risposte positive o negative ricevute dal sistema di feedback.
         & Desiderabile & Capitolato \\
        \hline
        RF.O.020 &  L’applicativo deve prevedere un sistema di estrazione e raccolta delle informazioni dal sito web dell'azienda.
         & Obbligatorio & Capitolato \\
        \hline
         RF.O.021 & Il sistema deve essere in grado di navigare un elenco di prodotti, estrarre le informazioni utili e immagazzinare le informazioni correlandole in modo opportuno.
         & Obbligatorio & Capitolato \\
        \hline
         RF.D.022 & L'utente deve poter scaricare i file di istruzioni dei prodotti in formato PDF.
         & Desiderabile & Capitolato \\
        \hline
         RF.O.023 & Il sistema deve essere in grado di ricavare le informazioni utili dai PDF ed estrarre le immagini degli schemi elettrici.
         & Obbligatorio & Capitolato \\
        \hline
         RF.O.024 & Il sistema deve impiegare un database per collezionare le informazioni relative ai prodotti.
         & Obbligatorio & Capitolato \\
        \hline
         RF.O.025 & Il sistema di estrazione e raccolta deve essere realizzato sotto-forma di pipeline e automatizzato.
         & Obbligatorio & Capitolato \\
        \hline
         RF.O.026 & L’applicativo deve prevedere un sistema di indicizzazione delle informazioni a partire dal
        database in cui sono stati salvati i dati precedentemente estratti dal sito web.
         & Obbligatorio & Capitolato \\
        \hline
        RF.O.027 & Il componente di interrogazione deve fornire in output la risposta del modello AI (LLM).
         & Obbligatorio & Capitolato \\
        \hline
        RF.O.028 & Il modello AI (LLM) deve essere in grado di rispondere a domande sui prodotti appartenenti agli impianti Smart.
         & Obbligatorio & Capitolato \\
        \hline
        RF.O.029 & Il modello AI (LLM) deve essere in grado di rispondere a domande sui prodotti appartenenti agli impianti Domotici.
         & Obbligatorio & Capitolato \\
        \hline
        RF.D.030 & Il modello AI (LLM) deve essere in grado di rispondere a domande sui prodotti appartenenti agli impianti Tradizionali.
         & Desiderabile & Capitolato \\
        \hline
        RF.D.031 & L’utente deve poter visualizzare il numero della pagina del documento relativo alla risposta.
         & Desiderabile & Capitolato \\
        \hline
        RF.D.032 & L’utente deve poter visualizzare il nome del documento relativo alla
        risposta.
         & Desiderabile & Capitolato \\
        \hline
        RF.O.033 & L’utente deve poter confermare l'eliminazione di una conversazione con il chatbot.
         & Obbligatorio & Interno \\
        \hline
        RF.O.034 & L’utente deve poter visualizzare una lista delle
        sessioni di conversazione attive col chatbot.
         & Obbligatorio & Interno \\
        \hline
        RF.O.035 & L’utente deve poter creare una nuova sessione di conversazione col chatbot.
         & Obbligatorio & Interno \\
        \hline
        RF.O.036 & L’utente deve ricevere una risposta di cortesia quando pone una domanda relativa a dei contenuti proibiti.
         & Obbligatorio & Interno \\
        \hline
        RF.O.037 & L’utente deve ricevere una risposta di cortesia quando pone una domanda relativa a dei contenuti non presenti.
         & Obbligatorio & Interno \\
        \hline
        RF.O.038 & L’utente deve poter digitare la domanda da porgere al chatbot tramite tastiera.
         & Obbligatorio & Interno \\
        \hline
        RF.O.039 & L’utente deve poter inviare le domande da porgere al chatbot.
         & Obbligatorio & Interno \\
        \hline
        RF.P.040 & L’utente deve poter ricercare una conversazione salvata in precedenza.
         & Opzionale & Interno \\
        \hline
        RF.P.041 & L’utente può visualizzare la data di invio di un messaggio.
         & Opzionale & Interno \\
        \hline
        RF.P.042 & L’utente può visualizzare l'ora di invio di un messaggio.
         & Opzionale & Interno \\
        \hline
        RF.P.043 & L’utente deve poter visualizzare il contenuto del documento di interesse.
         & Opzionale & Interno \\
        \hline
        RF.P.044 & L'Admin può azzerare il conto dei feedback positivi ricevuti.
         & Opzionale & Interno \\
        \hline
        RF.P.045 & L'Admin può azzerare il conto dei feedback negativi ricevuti.
         & Opzionale & Interno \\
        \hline
        
        


\end{longtable}

\newpage
\subsection{Requisiti di Qualità}
\begin{table}[H]
\centering
    \begin{tabular}{|C{2.7cm}|L{7.2cm}|C{2.7cm}|C{2cm}|}
        \hline
        \textbf{ID requisito} & \textbf{Descrizione} & \textbf{Importanza} & \textbf{Fonti}  \\
        \hline
       
        \hline
        RQ.D.001 & L'interfaccia utente del sistema potrebbe essere responsive, adattandosi a diversi dispositivi. & Desiderabile & Capitolato \\
        \hline
        RQ.O.002 & Il sistema deve essere progettato per essere facilmente eseguibile su altri dispositivi utilizzando la tecnologia dei container. & Obbligatorio & Capitolato \\
        
        \hline
        RQ.O.003 & \'E necessario fornire un documento che descriva le attività di bug\textsubscript{G} reporting effettuate. & Obbligatorio & Interno \\
        \hline
        RQ.O.004 & Il progetto deve essere svolto seguendo le regole contenute nel documento Norme di Progetto. & Obbligatorio & Interno \\
        \hline
        RQ.O.005 & \'E necessario fornire al proponente il codice sorgente dell'applicativo in un
        repository\textsubscript{G} GitHub. & Obbligatorio & Interno \\
        \hline
        RQ.O.006 & \'E necessario fornire il Manuale Utente dell'applicativo. & Obbligatorio & Interno \\
        \hline
    \end{tabular}
    \caption{Requisiti di qualità}
\end{table}
\subsection{Requisiti di Vincolo}

\begin{longtable}{|C{2.7cm}|L{7.2cm}|C{2.7cm}|C{2cm}|}
        \hline
    \textbf{ID requisito} & \textbf{Descrizione} & \textbf{Importanza} & \textbf{Fonti}  \\
    \hline
           RV.O.001 & Il sistema deve integrare un modello AI (LLM) open source. & Obbligatorio & Capitolato \\
          \hline 
          RV.O.002 & L’infrastruttura Cloud deve utilizzare Docker insieme a Docker Compose, al fine di rispettare il principio di Infrastructure as Code. & Obbligatorio & Capitolato \\
           \hline
          RV.D.003 & L'applicativo può essere ospitato su AWS. & Desiderabile & Capitolato \\
          \hline
          RV.O.004 & Il modello AI (LLM) deve essere Open Source.
         & Obbligatorio & Capitolato \\
        \hline
        RV.O.005 & Il componente di interrogazione deve essere in grado di interfacciarsi con il sistema di indicizzazione e con il modello AI (LLM).
         & Obbligatorio & Capitolato \\
        \hline
        RV.O.006 & Il componente di interrogazione deve poter essere contattato da un altro servizio sotto-forma di API autenticata (ad esempio tramite API-KEY)
         & Obbligatorio & Capitolato \\
        \hline
        RV.O.007 &  L’infrastruttura deve utilizzare la tecnologia dei container.
         & Obbligatorio & Capitolato \\
        \hline
         RV.O.008 & Il risultato atteso è che la parte applicativa possa essere costruita e replicata con un solo comando.
         & Obbligatorio & Capitolato \\
        \hline
        RV.O.009 & Il repository di lavoro deve essere versionato tramite Git e deve essere pubblicamente accessibile.
         & Obbligatorio & Capitolato \\
        \hline
        RV.O.010 & La licenza per i sorgenti dovrà essere open source.
         & Obbligatorio & Capitolato \\
        \hline
        RV.O.011 & Il modello AI (LLM) dovrà fare uso dell’approccio RAG.
         & Obbligatorio & Capitolato \\
        \hline
        RV.O.012 & L’applicazione deve essere compatibile con il browser Chrome dalla
        versione 108.
         & Obbligatorio & Interno \\
        \hline
        RV.O.013 & L’applicazione deve essere compatibile con il browser Edge dalla versione 94.0.992.31.
         & Obbligatorio & Interno \\
        \hline
        RV.O.014 & L’applicazione deve essere compatibile con il browser Opera dalla
        versione 95.
         & Obbligatorio & Interno \\
        \hline
        RV.O.015 & L’applicazione deve essere compatibile con il browser Firefox dalla
versione 109.
         & Obbligatorio & Interno \\
        \hline
        RV.O.016 & L’applicazione deve essere compatibile con il browser Safari dalla
versione 16.
         & Obbligatorio & Interno \\
        \hline


        
        

\end{longtable}