\section{Descrizione del Prodotto}
\subsection{Obiettivo fissato}
L'obiettivo principale fissato dal capitolato proposto dall'azienda Vimar è un applicativo che gli installatori possono
interrogare per reperire informazioni testuali e grafiche sui prodotti Vimar presenti all’interno del
sito ufficiale.
\subsection{Caratteristiche utente}
\subsubsection{Utenti}
Gli utenti principali di questo prodotto software sono gli installatori che andranno a interrogare il chatbot per reperire informazioni testuali e grafiche sui prodotti Vimar presenti all'interno del sito ufficiale.
\subsection{Tecnologie  e analisi della struttura di progetto} 
\subsubsection{Vincoli tecnologici}
I vincoli tecnologici sono i seguenti:
\begin{itemize}
    \item L’infrastruttura Cloud deve usare Docker con docker-compose al fine di far valere il
    principio di infrastructure as code.
    \item Il repository di lavoro deve essere versionato tramite Git e deve essere pubblicamente
    accessibile e per i sorgenti la licenza dovrà essere open
    source
    \item La parte di intelligenza artificiale dovrà fare uso dell’approccio RAG e usare un modello LLM
\end{itemize}
Inoltre vengono suggerite le seguenti tecnologie da utilizzare:
\begin{itemize}
    \item A livello di applicativo web responsive, si consigliano framework come Flask (Python), Angular
    (TypeScript) o VueJS per lo sviluppo front-end.
    \item Per lo sviluppo (es. API, business-logic) è fortemente consigliato l’utilizzo di Python come
    linguaggio di programmazione, vista la semplicità di utilizzo e di apprendimento.
    \item Per il componente di estrazione e reperimento delle informazioni dal sito web, si consigliano
    librerie di Web Scraping e OCR, come ad esempio Scrapy e OCRmyPDF.
    \item A livello di database, si consiglia l’utilizzo di database relazionali come PostgreSQL per
    immagazzinare i dati, unito all’uso dell’estensione pgvector per realizzare indici vettoriali (cfr.
    embeddings \#1, embeddings \#2) da sfruttare col componente di interrogazione. In alternativa si
    possono utilizzare database NoSQL come TimescaleDB o InfluxDB.
    \item Per il componente di interrogazione si consiglia di lavorare con modelli open source di LLM come
    ad esempio Llama 3.1, Mistral, Bert o Phi. Il sito di HuggingFace riporta anche una lista di LLM
    (sotto al tab Retrieval) che usino il RAG.
    \item In ottica di soluzione a container con sviluppo su AWS, si consiglia di utilizzare il servizio AWS
    LightSail (ref. guida a LightSail Containers) oppure AWS EC2 (Elastic Compute Cloud).
    \item Nell’ambito del software development lifecycle, si raccomanda l’utilizzo di una CI (es. GitHub
    Runners) per automatizzare l’esecuzione di test e l’analisi statica del codice.
    \item Nell’ambito dello sviluppo software, si consiglia l’utilizzo di GitHub Copilot (vedi GitHub Student
    Pack) o Amazon Q (versione gratuita).
\end{itemize}
\subsubsection{Test}
Il prodotto software inoltre dovrà essere sottoposto ad alcuni test, in particolare:
\begin{itemize}
    \item Test di unità, per la verifica delle componenti back-end e front-end;
    \item Test di integrazione, con le componenti infrastrutturali e applicative;
    \item Test End-to-End, per validare i requisiti iniziali e provare la soluzione nella sua interezza.
\end{itemize}
