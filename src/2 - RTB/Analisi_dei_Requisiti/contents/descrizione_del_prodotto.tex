\section{Descrizione del Prodotto}
\subsection{Obiettivo fissato}
L’obiettivo del progetto é quello di fornire al Proponente un’app in grado di fornire supporto in tempo reale agli installatori di dispositivi offerti dall’azienda allo scopo di agevolare e velocizzare la raccolta di informazioni richieste. Per rendere efficiente e veloce la raccolta di informazioni, l’azienda sceglie di basare la ricerca principalmente sull’interrogazione di LLM. La soluzione potrà essere realizzata tramite un sistema di chat dove l’utente, tramite linguaggio naturale, fa delle richieste ed ottiene delle risposte da un’intelligenza artificiale.

\subsection{Funzioni del prodotto}
Il software deve essere in grado di recuperare informazioni dal sito web di Vimar e indicizzarle in un database vettoriale\textsubscript{G} dal quale un LLM possa ricavare le informazioni per rispondere alle domande poste dagli installatori.
Secondariamente il software deve permettere la gestione di queste chat.

\subsection{Tecnologie  e analisi della struttura di progetto} 
\subsubsection{Vincoli tecnologici}
I vincoli tecnologici sono i seguenti:
\begin{itemize}
    \item L’infrastruttura Cloud deve usare Docker con docker-compose al fine di far valere il
    principio di infrastructure as code.
    \item Il repository di lavoro deve essere versionato tramite Git e deve essere pubblicamente
    accessibile e per i sorgenti la licenza dovrà essere open
    source
    \item La parte di intelligenza artificiale dovrà fare uso dell’approccio RAG e usare un modello LLM
\end{itemize}
Inoltre vengono suggerite le seguenti tecnologie da utilizzare:
\begin{itemize}
    \item A livello di applicativo web responsive, si consigliano framework come Flask (Python), Angular
    (TypeScript) o VueJS per lo sviluppo front-end.
    \item Per lo sviluppo (es. API, business-logic) è fortemente consigliato l’utilizzo di Python come
    linguaggio di programmazione, vista la semplicità di utilizzo e di apprendimento.
    \item Per il componente di estrazione e reperimento delle informazioni dal sito web, si consigliano
    librerie di Web Scraping e OCR, come ad esempio Scrapy e OCRmyPDF.
    \item A livello di database, si consiglia l’utilizzo di database relazionali come PostgreSQL per
    immagazzinare i dati, unito all’uso dell’estensione pgvector per realizzare indici vettoriali (cfr.
    embeddings \#1, embeddings \#2) da sfruttare col componente di interrogazione. In alternativa si
    possono utilizzare database NoSQL come TimescaleDB o InfluxDB.
    \item Per il componente di interrogazione si consiglia di lavorare con modelli open source di LLM come
    ad esempio Llama 3.1, Mistral, Bert o Phi. Il sito di HuggingFace riporta anche una lista di LLM
    (sotto al tab Retrieval) che usino il RAG.
    \item In ottica di soluzione a container con sviluppo su AWS, si consiglia di utilizzare il servizio AWS
    LightSail (ref. guida a LightSail Containers) oppure AWS EC2 (Elastic Compute Cloud).
    \item Nell’ambito del software development lifecycle, si raccomanda l’utilizzo di una CI (es. GitHub
    Runners) per automatizzare l’esecuzione di test e l’analisi statica del codice.
    \item Nell’ambito dello sviluppo software, si consiglia l’utilizzo di GitHub Copilot (vedi GitHub Student
    Pack) o Amazon Q (versione gratuita).
\end{itemize}

\subsection{Caratteristiche utente}
Gli utenti principali di questo prodotto software sono gli installatori: professionisti che si occupano della progettazione, messa in funzione e manutenzione di un impianto elettrico e domotico. Questi utenti andranno a interrogare il chatbot per reperire informazioni testuali e grafiche sui prodotti Vimar presenti all'interno del sito ufficiale. 