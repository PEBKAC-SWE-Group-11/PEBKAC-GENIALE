\section{Descrizione del Prodotto}
\subsection{Obiettivo fissato}
L’obiettivo del progetto è sviluppare un’app in grado di fornire supporto in tempo reale agli installatori di dispositivi offerti dall’azienda allo scopo di agevolare e velocizzare la raccolta di informazioni richieste. Per rendere efficiente e veloce la raccolta di informazioni, l’azienda sceglie di basare la ricerca principalmente sull’interrogazione di \textit{LLM}\textsubscript{G}. La soluzione potrà essere realizzata tramite un \textit{sistema}\textsubscript{G} di chat dove l’utente, tramite linguaggio naturale, fa delle richieste ed ottiene delle risposte da un’intelligenza artificiale.

\subsection{Funzioni del prodotto}
Il \textit{software}\textsubscript{G} deve essere in grado di recuperare informazioni dal sito web di \textit{Vimar}\textsubscript{G} e indicizzarle in un \textit{database}\textsubscript{G} vettoriale\textsubscript{G} dal quale un \textit{LLM}\textsubscript{G} possa ricavare le informazioni per rispondere alle domande poste dagli installatori.
Secondariamente il \textit{software}\textsubscript{G} deve permettere la gestione di queste chat.

\subsection{Tecnologie  e analisi della struttura di progetto} 
\subsubsection{Vincoli tecnologici}
I vincoli tecnologici sono i seguenti:
\begin{itemize}
    \item L’infrastruttura \textit{Cloud}\textsubscript{G} deve usare \textit{Docker}\textsubscript{G} con \textit{docker-compose}\textsubscript{G} al fine di far valere il
    principio di infrastructure as code.
    \item Il \textit{repository}\textsubscript{G} di lavoro deve essere versionato tramite \textit{Git}\textsubscript{G} e deve essere pubblicamente
    accessibile e per i sorgenti la licenza dovrà essere open
    source
    \item La parte di intelligenza artificiale dovrà fare uso dell’approccio \textit{RAG}\textsubscript{G} e usare un modello \textit{LLM}\textsubscript{G}
\end{itemize}
Inoltre vengono suggerite le seguenti tecnologie da utilizzare:
\begin{itemize}
    \item A livello di applicativo \textit{Web Responsive}\textsubscript{G} , si consigliano \textit{framework}\textsubscript{G} come Flask (\textit{Python}\textsubscript{G}), \textit{Angular}\textsubscript{G}
    (\textit{Typescript}\textsubscript{G}) o \textit{VueJS}\textsubscript{G} per lo sviluppo front-end.
    \item Per lo sviluppo (es. \textit{API}\textsubscript{G}, business-logic) è fortemente consigliato l’utilizzo di \textit{Python}\textsubscript{G} come
    linguaggio di programmazione, vista la semplicità di utilizzo e di apprendimento.
    \item Per il componente di estrazione e reperimento delle informazioni dal sito web, si consigliano
    librerie di \textit{Web Scraping}\textsubscript{G}  e OCR, come ad esempio \textit{Scrapy}\textsubscript{G} e \textit{OCRmyPDF}\textsubscript{G}.
    \item A livello di \textit{database}\textsubscript{G}, si consiglia l’utilizzo di \textit{database}\textsubscript{G} relazionali come \textit{PostgreSQL}\textsubscript{G} per
    immagazzinare i dati, unito all’uso dell’estensione pgvector per realizzare indici vettoriali (cfr.
    \textit{embeddings}\textsubscript{G} \#1, \textit{embeddings}\textsubscript{G} \#2) da sfruttare col componente di interrogazione. In alternativa si
    possono utilizzare \textit{database}\textsubscript{G} \textit{NoSQL}\textsubscript{G} come \textit{TimescaleDB}\textsubscript{G} o \textit{InfluxDB}\textsubscript{G}.
    \item Per il componente di interrogazione si consiglia di lavorare con modelli \textit{Open Source}\textsubscript{G} di \textit{LLM}\textsubscript{G} come
    ad esempio \textit{Llama 3.1}\textsubscript{G}, \textit{Mistral}\textsubscript{G}, \textit{Bert}\textsubscript{G} o \textit{Phi}\textsubscript{G}. Il sito di HuggingFace riporta anche una lista di \textit{LLM}\textsubscript{G}
    (sotto al tab \textit{Retrieval}\textsubscript{G}) che usino il \textit{RAG}\textsubscript{G}.
    \item In ottica di soluzione a \textit{container}\textsubscript{G} con sviluppo su \textit{AWS}\textsubscript{G}, si consiglia di utilizzare il servizio \textit{AWS}\textsubscript{G}
    LightSail (ref. guida a \textit{LightSail Containers}\textsubscript{G}) oppure \textit{AWS EC2}\textsubscript{G} (Elastic Compute \textit{Cloud}\textsubscript{G}).
    \item Nell’ambito del \textit{Software Development Lifecycle}\textsubscript{G}, si raccomanda l’utilizzo di una CI (es. \textit{GitHub}\textsubscript{G}
    Runners) per automatizzare l’esecuzione di test e l’analisi statica del codice.
    \item Nell’ambito dello sviluppo \textit{software}\textsubscript{G}, si consiglia l’utilizzo di \textit{GitHub}\textsubscript{G} Copilot (vedi \textit{GitHub}\textsubscript{G} Student
    Pack) o \textit{Amazon Q}\textsubscript{G} (versione gratuita).
\end{itemize}

\subsection{Caratteristiche utente}
Gli utenti principali di questo prodotto \textit{software}\textsubscript{G} sono gli installatori, professionisti che si occupano della progettazione, messa in funzione e manutenzione di un impianto elettrico e domotico, che andranno a interrogare il \textit{chatbot}\textsubscript{G} per reperire informazioni testuali e grafiche sui prodotti \textit{Vimar}\textsubscript{G} presenti all'interno del sito ufficiale. 