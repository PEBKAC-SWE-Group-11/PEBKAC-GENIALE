\section{Introduzione}
\subsection{Scopo del Documento}
Questo documento si propone di descrivere la pianificazione e il coordinamento delle attività indispensabili per l'esecuzione del progetto. Vengono trattati in dettaglio elementi fondamentali come l'analisi dei rischi, il modello di sviluppo scelto, la programmazione delle attività, l'assegnazione dei ruoli, oltre a una stima dei costi e delle risorse richieste.
\subsection{Scopo del Prodotto} 
L'obiettivo del prodotto è consentire agli installatori che utilizzano le soluzioni dell'azienda \textit{proponente$_G$}, Vimar S.p.A., di accedere rapidamente a informazioni testuali e grafiche relative ai prodotti presenti sul sito ufficiale.
\subsection{Glossario}
Per garantire chiarezza e prevenire fraintendimenti legati alla terminologia utilizzata nel documento, si è scelto di includere un \texttt{Glossario} che raccolga le definizioni dei termini.\\ I termini presenti nel glossario saranno scritti in \textit{corsivo} e marcati con una $_G$ a pedice.
\subsection{Riferimenti}
\subsubsection{Riferimenti  normativi}  
\begin{itemize}
    \item \textbf{Norme di Progetto v1.0.0}
    \item \textbf{PD1 - Regolamento del progetto didattico} \\
    \url{https://www.math.unipd.it/~tullio/IS-1/2024/Dispense/PD1.pdf} 
    \item \textbf{Capitolato d'Appalto C2}: Vimar GENIALE \\
    \url{https://www.math.unipd.it/~tullio/IS-1/2024/Progetto/C2.pdf}
\end{itemize}
\subsubsection{Riferimenti informativi}
\begin{itemize}
    \item \textbf{T2-I Processi di Ciclo di Vita del Software} \\
    \url{https://www.math.unipd.it/~tullio/IS-1/2024/Dispense/T02.pdf}
    \item \textbf{T4-Gestione di Progetto} \\
    \url{https://www.math.unipd.it/~tullio/IS-1/2024/Dispense/T04.pdf}
    \item \textbf{Glossario v1.0.0} 
\end{itemize}


\subsection{Preventivo Iniziale}
Il preventivo iniziale, presentato durante la fase di candidatura, è disponibile al seguente link: \href{https://pebkac-swe-group-11.github.io/assets/pdf/Preventivo_Costi_Assunzione_Impegni_V2.0.0.pdf}{Preventivo Iniziale}.\\Nel documento si specifica che il costo stimato per il progetto ammonta a \textbf{12850€} e che il gruppo prevede di completare il prodotto entro la data \textbf{14 marzo 2025}.