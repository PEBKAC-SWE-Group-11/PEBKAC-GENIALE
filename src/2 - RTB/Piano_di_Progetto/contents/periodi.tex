\section{Periodi}

%%%%%%%%%%%%%%%%%%%%%%%%%%%%%%%%%%%%%%%%%%%%%%%%%%%%%%%%%%%%%%
% Semplicemente un'idea di come potrebbe essere, da valutare %
%%%%%%%%%%%%%%%%%%%%%%%%%%%%%%%%%%%%%%%%%%%%%%%%%%%%%%%%%%%%%%
%    \subsection{RTB}
%        \subsubsection{I Periodo (dal 2024-10-XY al 2024-10-XY)}
%        Punti salienti/Obiettivi: \\
%        Rischi incontrati e intervento:
%        \subsubsection{II Periodo (dal 2024-10-XY al 2024-11-XY)}
%        Punti salienti/Obiettivi: \\
%        Rischi incontrati e intervento:
%        \subsubsection{III Periodo (dal 2024-11-XY al 2024-11-XY)}
%        Punti salienti/Obiettivi: \\
%        Rischi incontrati e intervento:
%    \subsection{PB}
%        \subsubsection{IV Periodo (dal 2024-12-XY al 2025-01-XY)}
%       Punti salienti/Obiettivi: \\
%        Rischi incontrati e intervento:
%        \subsubsection{V Periodo (dal 2025-01-XY al 2025-01-XY)}
%        Punti salienti/Obiettivi: \\
%        Rischi incontrati e intervento:
%        \subsubsection{VI Periodo (dal 2025-01-XY al 2025-02-XY)}
%        Punti salienti/Obiettivi: \\
%        Rischi incontrati e intervento:
%\\[20pts]

%%%%%%%%%%%%%%%%%%%%%%%%%%%%%%%%%%%%%%%%%%%%%%%%%%%%%%%%%
% Qui sotto invece e' riportato quello fatto dai Mattei %
%%%%%%%%%%%%%%%%%%%%%%%%%%%%%%%%%%%%%%%%%%%%%%%%%%%%%%%%%
Per ogni periodo vengono riportati di solito le seguenti informazioni:
\begin{itemize}
    \item Data inizio, data fine prevista, data di fine attuale ed eventuale giorni di ritardo;
    \item Pianificazione delle attività da svolgere al suo interno con eventuali rischi;
    \item Tempo stimato per poter completare le attività previste (preventivo);
    \item Confronto tra il lavoro svolto e quello preventivato, con annessa analisi dei costi;
    \item Rischi riscontrati durante ogni periodo;
    \item Retrospettiva del periodo per capire cosa mantenere, cosa cambiare e cosa migliorare in futuro
\end{itemize}

I periodi vengono a loro volta divisi in 3 macroperiodi che corrispondono alle revisioni di avanzamento del progetto:
\begin{itemize}
    \item RTB (Requirement and Technology Baseline)
    \item PB (Product Baseline)
    \item CA (Customer Acceptance)
\end{itemize}
Alla fine di ogni collettivo si trovano una revisione del calendario rimanente e un aggiornamento della stima dei costi finali, oltre ad un riepilogo del lavoro svolto.
\subsection{RTB}
\subsubsection{Periodo I}
\begin{itemize}
    \item Inizio: 2024-11-14
    \item Fine prevista: 2024-11-25
    \item Fine attuale: 2024-11-26
    \item Giorni di ritardo: (+1)
\end{itemize}
\paragraph{Pianificazione} \hspace{1cm}
\\ \hspace{1cm} \\
L'inizio di questo periodo è sancito dalla prima riunione in presenza con i rappresentati dell'azienda proponente Vimar S.p.A. a seguito dell'aggiudicazione dell'appalto da parte del gruppo. Come prima attività è stato pianificato di sopperire alle mancanze riportate in fase di candidatura. Successivamente, le azioni programmate per questo periodo riguardano principalmente la definizione di un template di base per i documenti richiesti per la prima revisione (RTB) e l'inizio della loro stesura. In aggiunta, si prevede di cominciare a studiare ed analizzare i vari modelli LLM proposti nel capitolato per poterli confrontare e stabilire quale sia il più valido per l'ambito applicativo in esame.
I compiti che sono previsti per questo periodo possono quindi essere riassunti come segue:
\begin{itemize}
    \item Riorganizzare il repository in una maniera tale da non permettere l'ingresso di elementi non verificati in esso;
    \item Creare il template LaTeX necessario ad uniformare lo stile di tutti i documenti che verranno redatti in futuro;
    \item Iniziare la stesura del Piano di Progetto, definendone l'organizzazione e la strutturazione;
    \item Iniziare la stesura delle Norme di Progetto, permettendo così ai membri del gruppo di essere dotati di un riferimento scritto per il Way of Working adottato fino a questo punto e che verrà successivamente migliorato ed integrato;
    \item Iniziare la stesura dell'Analisi dei Requisiti, prestando particolare attenzione all'analisi e alla redazione dei casi d'uso;
    \item Iniziare la stesura dell'Glossario ed inserire le prime voci;
    \item Ricercare delle informazioni riguardo i modelli LLM proposti nel capitolato e riguardo a delle metriche per confrontarli, in modo da stabilire quale verrà utilizzato per il progetto.
\end{itemize}
\paragraph{Rischi attesi} \hspace{1cm}
\\ \hspace{1cm} \\
In questo periodo si prevede che verranno corsi diversi rischi, in particolare:
\begin{itemize}
    \item \textbf{\hyperlink{RT1}{RT1 - Inesperienza}}
    \item \textbf{\hyperlink{RO1}{RO1 - Imprecisione nella pianificazione}}
    \item \textbf{\hyperlink{RO2}{RO2 - Impegni personali (e accademici)}}
    \item \textbf{\hyperlink{RO4}{RO4 - Mal interpretazione dei requisiti}}
    \item \textbf{\hyperlink{RG1}{RG1 - Problemi di comunicazione interna}}
    \item \textbf{\hyperlink{RG2}{RG2 - Rischio di conflitti interni}}
\end{itemize}
Questi rischi possono verificarsi anche a causa del fatto che questo sarà il primo periodo, e quindi l'inizio del progetto. \'E dunque necessario che i membri acquisiscano dimestichezza con gli strumenti utilizzati, acquisiscano le capacità di pianificazione richieste, imparino a conciliare i propri impegni personali con il tempo necessario allo svolgimento delle attività di progetto, aprendano le modalità di interpretazione dei requisiti ed imparino a collaborare in maniera efficace adottando un buona strategia di suddivisione del lavoro.
\paragraph{Preventivo} \hspace{1cm} 
\\ \hspace{1cm} \\
Ruoli coinvolti: Responsabile, Amministratore, Verificatore, Analista.
\begin{table}[H]
\centering
\resizebox{1.0\textwidth}{!}{%
    \begin{tabular}{|c|c|c|c|c|c|c|c|}
        \hline
         \textbf{} & \textbf{Re.} &  \textbf{Am.} &  \textbf{Ve.} &  \textbf{An.} & \textbf{Pg.} &  \textbf{Pr.} & \textbf{Totale per persona}\\
         \hline \colorrow
           \cellcolor{lightgray}{\textbf{Benin}} & 0 & 0 & 5 & 0 & 0 & 0 & 5\\
        \hline
           \textbf{Bourosu} & 0 & 0 & 0 & 5 & 0 & 0 & 5\\
        \hline \colorrow
           \cellcolor{lightgray}{\textbf{Gerardin}} & 0 & 0 & 0 & 5 & 0 & 0 & 5\\
        \hline 
           \textbf{Gusatto} & 0 & 5 & 0 & 0 & 0 & 0 & 5\\
        \hline \colorrow
           \cellcolor{lightgray}{\textbf{Martinelli}} & 0 & 0 & 0 & 5 & 0 & 0 & 5\\
        \hline 
           \textbf{Piron} & 0 & 0 & 0 & 5 & 0 & 0 & 5\\
        \hline \colorrow
           \cellcolor{lightgray}{\textbf{Zocche}} & 5 & 0 & 0 & 0 & 0 & 0 & 5\\
        \hline 
           \textbf{Totale per ruolo} & 5 & 5 & 5 & 20 & 0 & 0 & 35\\
        \hline
    \end{tabular}
}
\caption{Preventivo dell'impegno orario di ciascun membro durante il periodo I}
\end{table}

\paragraph{Consuntivo} \hspace{1cm} 
\\ \hspace{1cm} \\
I compiti previsti sono stati tutti completati.
Confrontando i prospetti orari di preventivo e consuntivo è possibile osservare che:
\begin{itemize}
    \item Responsabili e Amministratori hanno avuto bisogno di un numero maggiore di ore rispetto a quanto preventivato;
    \item Verificatori e Analisti hanno avuto bisogno di un numero minore di ore rispetto a quanto preventivato;
\end{itemize}

\paragraph{Prospetto orario} \hspace{1cm}
\begin{table}[H]
\centering
\resizebox{1.0\textwidth}{!}{%
    \begin{tabular}{|c|c|c|c|c|c|c|c|}
        \hline
         \textbf{} & \textbf{Re.} &  \textbf{Am.} &  \textbf{Ve.} &  \textbf{An.} & \textbf{Pg.} &  \textbf{Pr.} & \textbf{Totale per persona}\\
         \hline \colorrow
           \cellcolor{lightgray}{\textbf{Benin}} & 0 & 0 & 3 (-2) & 0 & 0 & 0 & 3\\
        \hline
           \textbf{Bourosu} & 0 & 0 & 0 & 6 (+1) & 0 & 0 & 6\\
        \hline \colorrow
           \cellcolor{lightgray}{\textbf{Gerardin}} & 1 (+1) & 0 & 0 & 4 (-1) & 0 & 0 & 5\\
        \hline 
           \textbf{Gusatto} & 0 & 7 (+2) & 0 & 0 & 0 & 0 & 7\\
        \hline \colorrow
           \cellcolor{lightgray}{\textbf{Martinelli}} & 0 & 0 & 0 & 5 & 0 & 0 & 5\\
        \hline 
           \textbf{Piron} & 2 (+2) & 0 & 0 & 3 (-2) & 0 & 0 & 5\\
        \hline \colorrow
           \cellcolor{lightgray}{\textbf{Zocche}} & 5 & 0 & 0 & 0 & 0 & 0 & 5\\
        \hline 
           \textbf{Totale per ruolo} & 8 & 7 & 3 & 18 & 0 & 0 & 36\\
        \hline
    \end{tabular}
}
\caption{Consuntivo dell'impegno orario di ciascun membro durante il periodo I}
\end{table}

\paragraph{Prospetto economico} \hspace{1cm}

\begin{table}[H]
    \centering
    \begin{tabular}{|c|c|c|c|}
            \hline
             \textbf{Ruolo} &  \textbf{Ore} &  \textbf{Costo} &  \textbf{Differenza}  \\
             \hline \colorrow
               \cellcolor{lightgray}{\textbf{Responsabile}} & 8 (+3) & 240€ & (+90€) \\
            \hline
               \textbf{Amministratore} & 7 (+2) & 140€ & (+40€) \\
            \hline \colorrow
               \cellcolor{lightgray}{\textbf{Verificatore}} & 3 (-2) & 45€ & (-30€) \\
            \hline 
               \textbf{Analista} & 18 (-2) & 450€ & (-50€) \\
            \hline \colorrow
               \cellcolor{lightgray}{\textbf{Progettista}} & - & - & - \\
            \hline 
               \textbf{Programmatore} & - & - & - \\
            \hline \colorrow
               \cellcolor{lightgray}{\textbf{Totale preventivo}} & 35 & 825€ & - \\
            \hline 
               \textbf{Totale consuntivo} & 36 (+1) & 875€ & (+50€) \\
            \hline
        \end{tabular}
    \caption{Aggiornamenti economici del progetto al termine del periodo I, riflettendo le variazioni tra preventivo ed ore effettive di lavoro}
\end{table} 

\paragraph{Rischi occorsi, impatto e la loro mitigazione} \hspace{1cm} 
\\ \hspace{1cm} \\
Nella durata di questo primo periodo, il gruppo si è trovato ad affrontare delle situazioni che hanno rallentato leggermente l'avanzamento del progetto.\\
In primo luogo, a causa della redazione in contemporanea da parte di diverse persone, ci siamo ritrovati ad essere di possesso di documenti dotati di convenzioni stilistiche differenti; questa prima problematica può essere attribuita ad inesperienza e problemi di comunicazione interna, compresi nei rischi attesi dal gruppo, ed ha cusato dei rallentamenti, ma è stata risolta prontamente tramite la definizione di un template generico utilizzato per la stesura di tutti i documenti. \'E stato necessario utilizzare una maggiore quantità di tempo per effettuare il riallineamento dello stile della documentazione, ma una volta effettuata questa operazione, l'avanzamento del progetto è ripreso come previsto.\\
In secondo luogo, il gruppo si è trovato in difficoltà riguardo a come organizzare il repository per far sì che permetta l'ingresso solamente a documenti che sono stati precedentemente sottoposti a operazione di verifica, garantendo così la correttezza del versionamento; questa seconda problematica può essere attribuita ad inesperienza ed imprecisione nella pianificazione, compresi nei rischi attesi dal gruppo, ed ha causato dei rallentamenti, ma è stata risolta prontamente tramite la definizione di una nuova modalità di ingresso dei documenti nel repository e la modifica della sezione corrispondente nelle Norme di Progetto. \'E stato necessario utilizzare una maggiore quantità di tempo per stabilire la nuova modalità da utilizzare, ma una volta decisa, l'avanzamento del progetto è ripreso come previsto.\\
In definitiva le difficoltà incontrate non hanno posto grandi freni all'avanzamento del progetto, e sono state prontamente risolte tramite il confronto tra i membri del gruppo e la tempestività con cui le soluzioni sono state realizzate ed applicate.

\paragraph{Retrospettiva} \hspace{1cm} 
\\ \hspace{1cm} \\
Analizzando lo svolgimento del periodo appena terminato è possibile delineare chiaramente degli elementi che devono essere modificati al fine di migliorare lo svolgimento del prossimo periodo:
\begin{itemize}
    \item Si è notato che definire delle linee guida per lo svolgimento dei processi produttivi è essenziale, dato che, altrimenti, si rischia che, come in questo caso nella stesura della documentazione, ogni membro utilizzi uno stile diverso. Il problema specifico riscontrato in questo periodo è stato risolto tramite la definizione di un template per tutta la documentazione, ma si è definito di procedere maggiormente con la stesura delle Norme di Progetto per eliminare che questo rischio si ripresenti;
    \item Durante la redazione di un unico documento da parte di più persone, come in questo caso l'Analisi dei Requisiti, si sono verificate incomprensioni e conflitti tra i membri adibiti a questo compito. \'E quindi necessario migliorare la comunicazione tra i membri del gruppo a cui sono assegnate delle attività strettamente collegate tra loro;
    \item Durante la pianificazione di questo periodo sono stati effettuati alcuni errori, che hanno portato a richiedere un numero diverso di ore rispetto a quelle stimate inizialmente. Si procederà dunque ad uno studio più approfondito e ad una conseguente pianificazione migliore per i periodi successivi.
\end{itemize}
Dall'analisi, oltre ad elementi che necessitano di essere rivisti e modificati, sono emerse anche delle situazioni che hanno portato a buoni risultati e che ci si assicurerà di mantenere anche nello svolgimento dei periodi successivi:
\begin{itemize}
    \item La suddivisione ed assegnazione del lavoro che è stato svolto in questo primo periodo è stata ritenuta dai membri del gruppo particolarmente buona e soddisfacente. Per questo si pianifica di seguire lo stesso ragionamento per il lavoro pianificato per i prossimi periodi;
    \item La metodologia utilizzata dai membri del gruppo per lo studio dei modelli LLM proposti nel capitolato d'appalto si è rivelata molto efficace ed efficiente, permettendo di ricercare una quantità di informazioni soddisfacente con tempistiche brevi. \'E quindi intenzione dei membri del gruppo utilizzare questa stessa metodologia anche per lo studio delle altre tecnologie necessarie per la realzzazione del progetto.
\end{itemize}
In conclusione il gruppo, nel corso dei prossimi periodi, continuerà a seguire i comportamenti "virtuosi" e cercherà di apportare delle modifiche per migliorare quelli che invece non sono stati ritenuti all'altezza.

%\paragraph{Punto di avanzamento raggiunto e prospettive di completamento}
%\hspace{1cm} \\ \\ Barra di avanzamento \\ \\
%\begin{tikzpicture}
%    \draw[gray, fill=gray!20] (0,0) rectangle (14,1);
%    \draw[black, fill=black!70] (0,0) rectangle (10.5,1); % 75% completato
%    \node at (7,-0.5) {75\%};
%\end{tikzpicture}

\subsubsection{Periodo II}
\begin{itemize}
    \item Inizio: 2024-11-27
    \item Fine prevista: 2024-12-10
    \item Fine attuale: 2024-12-12
    \item Giorni di ritardo: (+2)
\end{itemize}
\paragraph{Pianificazione} \hspace{1cm}
\\ \hspace{1cm} \\
Durante lo svolgimento del secondo periodo si prevede di continuare con la redazione della documentazione e di effettuare la revisione e la modifica di alcune parti di essa, dedicando una quantità di tempo maggiore all'Analisi dei Requisiti, ed in particolare all'analisi dei casi d'uso, che è stata sottoposta ad una prima stesura nel periodo precedente e per cui è stata evidenziata la necessità di correzioni al seguito delle attività denominate "Diario di Bordo" ed in seguito a dei chiarimenti ricevuti durante una delle riunioni di "Stato Avanzamento Lavori" con l'azienda proponente.
Inoltre si procederà anche con la stesura delle Norme di Progetto, ed in particolare delle metriche di qualità, del Glossario e del Piano di Progetto, in cui si andranno ad inserire il consuntivo e la retrospettiva del primo periodo oltre alla pianificazione del periodo corrente.
Come ultimo punto per quanto riguarda la documentazione, in questo periodo si inizieranno la strutturazione e la scrittura del Piano di Qualifica.
Per quanto riguarda lo studio delle tecnologie, nel periodo corrente, si prevede di iniziare uno studio teorico anche delle tecnologie proposte per la realizzazione di frontend, backend, web scraping e database vettoriale, affiancado sia lo studio dei modelli LLM sia quello delle tecnologie riportate sopra a delle prove pratiche individuali e di integrazione.
I compiti che sono previsti per questo periodo posso quindi essere riassunti come segue:
\begin{itemize}
    \item Revisione, correzione ed integrazione dell'Analisi dei Requisiti, ed in particolare dell'analisi dei casi d'uso;
    \item Intregrazione delle Norme di Progetto, in particolare aggiungendo le metriche di qualità;
    \item Arricchimento del Glossario con i termini emersi durante la redazione della documentazione durante il periodo precedente;
    \item Stesura del consuntivo di ore e costi e della retrospettiva del periodo precedente nel Piano di Progetto, oltre alla scrittura della pianificazione del periodo corrente;
    \item Inizio della stesura del Piano di Qualifica, definendone l'organizzazione e la strutturazione;
    \item Studio, effettuazione di prove pratiche e scelta della tecnologia che sarà utilizzata per la realizzazione del frontend, selezionata tra quelle proposte nel capitolato d'appalto, ovvero Flask, Angular e Vue.js;
    \item Studio, effettuazione di prove pratiche e scelta della tecnologia che sarà utilizzata per la realizzazione del database vettoriale, selezionata dal capitolato d'appalto, ovvero PostgreSQL con l'estensione pgvector per la realizzazione degli indici vettoriali;
    \item Studio, effettuazione di prove pratiche e scelta della tecnologia che sarà utilizzata per il web scraping, selezionata tra quelle proposte nel capitolato d'appalto, ovvero Scrapy e OCRmyPDF;
    \item Studio ed effettuazione di prove pratiche della tecnologia che sarà utilizzata per la realizzazione del backend, selezionata dal capitolato d'appalto, ovvero Python;
    \item Effettuazione di prove pratiche e scelta del modello LLM, selezionato tra quelli proposti nel capitolato d'appalto, ovvero Llama 3.1, Mistral, Bert e Phi;
    \item Esecuzione di prove per l'integrazione tra le diverse tecnologie studiate.
\end{itemize}
\paragraph{Rischi attesi} \hspace{1cm}
\\ \hspace{1cm} \\
In questo periodo si prevede che verranno corsi diversi rischi, in particolare:
\begin{itemize}
    \item \textbf{\hyperlink{RT1}{RT1 - Inesperienza}}
    \item \textbf{\hyperlink{RT2}{RT2 - Problemi con software di terze parti}}
    \item \textbf{\hyperlink{RO1}{RO1 - Imprecisione nella pianificazione}}
    \item \textbf{\hyperlink{RO2}{RO2 - Impegni personali (e accademici)}}
    \item \textbf{\hyperlink{RO4}{RO4 - Mal interpretazione dei requisiti}}
    \item \textbf{\hyperlink{RG1}{RG1 - Problemi di comunicazione interna}}
    \item \textbf{\hyperlink{RG2}{RG2 - Rischio di conflitti interni}}
\end{itemize}
I rischi di questo periodo non divergono eccessivamente da quelli elencati per il periodo precedente in quanto, trattandosi del secondo periodo, è ancora chiaramente palpalbile l'inesperienza dei membri del gruppo e sono ancora presenti le stesse problematiche identificate in passato, benché in forma più lieve, segnalando quindi un miglioramento graduale della situazione. 
La novità di questo periodo, per quanto riguarda i rischi, sono invece i problemi con software di terza parti, che emergo a causa del fatto che cominceranno le prove pratiche delle tecnologie studiate nel periodo precedente e in quello corrente.
\paragraph{Preventivo} \hspace{1cm} 
\\ \hspace{1cm} \\
Ruoli coinvolti: Responsabile, Amministratore, Verificatore, Analista, Progettista.
\begin{table}[H]
\centering
\resizebox{1.0\textwidth}{!}{%
    \begin{tabular}{|c|c|c|c|c|c|c|c|}
        \hline
         \textbf{} & \textbf{Re.} &  \textbf{Am.} &  \textbf{Ve.} &  \textbf{An.} & \textbf{Pg.} &  \textbf{Pr.} & \textbf{Totale per persona}\\
         \hline \colorrow
           \cellcolor{lightgray}{\textbf{Benin}} & 0 & 0 & 0 & 0 & 0 & 0 & 0\\
        \hline
           \textbf{Bourosu} & 0 & 0 & 0 & 4 & 0 & 0 & 4\\
        \hline \colorrow
           \cellcolor{lightgray}{\textbf{Gerardin}} & 5 & 0 & 0 & 0 & 0 & 0 & 5\\
        \hline 
           \textbf{Gusatto} & 0 & 0 & 0 & 4 & 0 & 0 & 4\\
        \hline \colorrow
           \cellcolor{lightgray}{\textbf{Martinelli}} & 0 & 0 & 4 & 0 & 0 & 0 & 4\\
        \hline 
           \textbf{Piron} & 0 & 5 & 0 & 0 & 0 & 0 & 5\\
        \hline \colorrow
           \cellcolor{lightgray}{\textbf{Zocche}} & 0 & 0 & 0 & 4 & 0 & 0 & 4\\
        \hline 
           \textbf{Totale per ruolo} & 5 & 5 & 4 & 12 & 0 & 0 & 26\\
        \hline
    \end{tabular}
}
\caption{Preventivo dell'impegno orario di ciascun membro durante il periodo II}
\end{table}

\paragraph{Consuntivo} \hspace{1cm} 
\\ \hspace{1cm} \\


\paragraph{Prospetto orario} \hspace{1cm}
\begin{table}[H]
\centering
\resizebox{1.0\textwidth}{!}{%
    \begin{tabular}{|c|c|c|c|c|c|c|c|}
        \hline
         \textbf{} & \textbf{Re.} &  \textbf{Am.} &  \textbf{Ve.} &  \textbf{An.} & \textbf{Pg.} &  \textbf{Pr.} & \textbf{Totale per persona}\\
         \hline \colorrow
           \cellcolor{lightgray}{\textbf{Benin}} & 0 & 0 & 0 & 0 & 0 & 0 & 0\\
        \hline
           \textbf{Bourosu} & 0 & 0 & 0 & 0 & 0 & 0 & ?\\
        \hline \colorrow
           \cellcolor{lightgray}{\textbf{Gerardin}} & 7 & 0 & 0 & 0 & 0 & 0 & 7\\
        \hline 
           \textbf{Gusatto} & 0 & 0 & 0 & 2.5 & 0 & 0 & 2.5\\
        \hline \colorrow
           \cellcolor{lightgray}{\textbf{Martinelli}} & 0 & 0 & 0 & 0 & 0 & 0 & ?\\
        \hline 
           \textbf{Piron} & 0 & 4 & 0 & 0 & 0 & 0 & 4\\
        \hline \colorrow
           \cellcolor{lightgray}{\textbf{Zocche}} & 0 & 0 & 0 & 5 & 0 & 0 & 5\\
        \hline 
           \textbf{Totale per ruolo} & 7 & 4 & ? & 7.5+? & 0 & 0 & 18.5+?\\
        \hline
    \end{tabular}
}
\caption{Consuntivo dell'impegno orario di ciascun membro durante il periodo II}
\end{table}

\paragraph{Prospetto economico} \hspace{1cm}

\begin{table}[H]
    \centering
    \begin{tabular}{|c|c|c|c|}
            \hline
             \textbf{Ruolo} &  \textbf{Ore} &  \textbf{Costo} &  \textbf{Differenza}  \\
             \hline \colorrow
               \cellcolor{lightgray}{\textbf{Responsabile}} & 0 & 0€ & (+0€) \\
            \hline
               \textbf{Amministratore} & 0 & 0€ & (+0€) \\
            \hline \colorrow
               \cellcolor{lightgray}{\textbf{Verificatore}} & 0 & 0€ & (+0€) \\
            \hline 
               \textbf{Analista} & 0 & 0€ & (+0€) \\
            \hline \colorrow
               \cellcolor{lightgray}{\textbf{Progettista}} & 0 & 0€ & (+0€) \\
            \hline 
               \textbf{Programmatore} & 0 & 0€ & (+0€) \\
            \hline \colorrow
               \cellcolor{lightgray}{\textbf{Totale preventivo}} & 0 & 0€ & - \\
            \hline 
               \textbf{Totale consuntivo} & 0 & 0€ & (+0€) \\
            \hline
        \end{tabular}
    \caption{Aggiornamenti economici del progetto al termine del periodo II, riflettendo le variazioni tra preventivo ed ore effettive di lavoro}
\end{table} 

\paragraph{Rischi occorsi, impatto e la loro mitigazione} \hspace{1cm} 
\\ \hspace{1cm} \\


\paragraph{Retrospettiva} \hspace{1cm} 
\\ \hspace{1cm} \\


\subsubsection{Periodo III}
