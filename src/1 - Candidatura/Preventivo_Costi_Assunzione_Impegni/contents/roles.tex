\section{Analisi dei ruoli}
In questa sezione si analizzano le motivazioni della suddivisione delle ore assegnate a ciascun ruolo (proposta in Preventivo costi). 


\subsection{Responsabile}
Il responsabile di un progetto è colui che ne garantisce il completamento in modo efficiente e in linea con gli obiettivi concordati con il committente. Altro compito del responsabile è garantire che la rotazione dei ruoli avvenga in modo corretto. Questo ruolo richiede competenze nella gestione delle risorse, nel problem-solving, nella comunicazione e nella pianificazione. 
\\
Essendo il ruolo con il costo orario più alto il numero di ore nel ruolo di responsabile è piuttosto contenuto, anche grazie alla promesssa del gruppo di svolgere questo ruolo nel modo più efficiente possibile, per avere la massima resa per un costo medio.

\subsection{Amministratore}

L'amministratore garantisce che l'ambiente e l'infrastruttura necessari per lo sviluppo del progetto siano affidabili e sicuri, contribuendo alla stabilità del progetto.
\\
Il numero di ore destinate al ruolo di amministratore è il più basso perché si ritiene che non richieda elevato dispendio in termini di ore produttive.

\subsection{Analista}
Questo ruolo è fondamentale nelle prime fasi del progetto, in particolare nell'Analisi dei Requisiti, ed è meno presente, se non addirittura assente, nelle seguenti fasi del progetto. Per questo il numero di ore stimate per questo ruolo è il più basso.

\subsection{Progettista}
Il progettista ha il compito di creare una struttura coerente per il progetto e di identificare le risorse necessarie. Si tratta di una figura indispensabile per garantire un progretto strutturato, affidabile ed efficiente, che soddisfi i requisiti.
\\
Al ruolo di progettista è quindi dedicato un numero di ore medio-alto.

\subsection{Programmatore}
Quello del programmatore è un ruolo chiave nella fase di sviluppo, perché questa figura ha il compito di tradurre le specifiche e i requisiti del progetto in codice sorgente e di ottimizzare il codice per migliorare le prestazioni del progetto. Il programmatore è inoltre responsabile delle funzionalità più specifiche all'interno del software e deve quindi avere approfondita conoscenza e comprensione delle tecnologie utilizzate e dei linguaggi di programmazione.
\\
Alla figura del programmatore quindi vengono assegnate più ore per ruolo in assoluto. 

\subsection{Verificatore}
Il verificatore ha un ruolo primario all'interno dello sviluppo del progetto, poiché deve garantire affidabilità e robustezza nei test ed occuparsi del controllo di qualità del software e della documentazione, in base agli standard del committente e del gruppo stesso e ciò richiede approfondita conoscenza del Way of Working. 
\\
Al ruolo di verificatore è quindi assegnato un numero alto di ore.