\subsection{Capitolato C6 - Sistema di Gestione di un Magazzino
Distribuito}
    \subsubsection{Informazioni generali}
        \begin{itemize}
            \item \textbf{Titolo}: Sistema di Gestione di un Magazzino
Distribuito
            \item \textbf{Proponente}: M31 S.r.l.
            \item \textbf{Committente}: Prof. Vardanega T., Prof. Cardin R.
        \end{itemize}
    \subsubsection{Obiettivo}
    Sviluppare un sistema di gestione dell'inventario per una rete di magazzini distribuiti.
La gestione dell’inventario in una rete logistica con magazzini distribuiti è essenziale per garantire un flusso costante di risorse lungo la catena operativa. In particolare, con magazzini geograficamente distanti, la disponibilità dei prodotti deve essere monitorata e bilanciata per evitare interruzioni di servizio. Un sistema centralizzato di gestione delle scorte è quindi necessario per minimizzare i tempi di risposta e ottimizzare la distribuzione delle risorse.

     \subsubsection{Dominio Applicativo}
    Il progetto consiste nella creazione di un sistema di gestione dell'inventario per magazzini distribuiti, con l’obiettivo di ottimizzare la disponibilità di prodotti e la gestione delle scorte su una rete di magazzini geograficamente separati.
Le funzioni che dovrà possedere questo sistema sono:

\begin{itemize}
    \item Ottimizzare i livelli di scorte, tramite il monitoraggio costante dell'inventario e il suggerimento o l'automatizzazione di riassortimenti o trasferimenti tra magazzini;
    \item Gestire la condivisione dei dati in tempo reale, attraverso una sincronizzazione continua che permetterà di avere una visione chiara e centralizzata delle scorte di ogni magazzino;
    \item Implementare il riassortimento predittivo, basato sull’utilizzo di algoritmi di machine learning che permetteranno di prevedere la domanda futura sulla base di dati passati;
    \item Risolvere i conflitti di aggiornamento simultaneo dell’inventario, provenienti da magazzini differenti.
\end{itemize}
    \subsubsection{Tecnologie}
    Il progetto utilizza:

\begin{itemize}
    \item Node.js e Nest.js (TypeScript) per lo sviluppo dei microservizi.
    \item Go per componenti ad alte prestazioni, come servizi di sincronizzazione.
    \item NATS o Apache Kafka per la comunicazione asincrona tra microservizi.
    \item Google Cloud Platform e Kubernetes per l’orchestrazione e gestione centralizzata del sistema.
    \item MongoDB per la memorizzazione di dati non strutturati.
    \item PostgreSQL per la memorizzazione di dati strutturati.
    \item Redis, tramite il caching, per migliorare prestazioni e ridurre latenza.
    \item Angular e Single Page Applications (SPAs) per un’interfaccia utente simile ad un’app desktop.
\end{itemize}
    \subsubsection{Punti di forza}
   \begin{itemize}
    \item Il riassortimento predittivo permette di prevedere la domanda aiuta a ottimizzare i livelli di inventario e ridurre i costi di stoccaggio;
    \item L’architettura a microservizi favorisce scalabilità e indipendenza operativa dei magazzini, riducendo il rischio di guasti a cascata.
\end{itemize}
    \subsubsection{Punti deboli}
    \begin{itemize}
    \item L’efficacia del riassortimento predittivo dipende dalla qualità e quantità di dati storici disponibili;
    \item L’adozione di tecnologie specifiche può richiedere competenze avanzate e può limitare la flessibilità in fase di sviluppo;
    \item La gestione di aggiornamenti simultanei e la risoluzione dei conflitti possono essere tecnicamente complesse.
\end{itemize}
    \subsubsection{Conclusioni}
    Questo capitolato non è stato preso in considerazione durante la fase di selezione dei capitolati dato che l’ambito a cui appartiene non ha interessato particolarmente i membri del gruppo, facendo sì che l’attenzione ricadesse su altri capitolati.
