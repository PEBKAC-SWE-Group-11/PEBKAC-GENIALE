\subsection{Capitolato C7 - LLM: ASSISTENTE VIRTUALE}
     \subsubsection{Informazioni generali}
        \begin{itemize}
            \item \textbf{Titolo}: LLM: ASSISTENTE VIRTUALE
            \item \textbf{Proponente}: Ergon Informatica S.r.l.
            \item \textbf{Committente}: Prof. Vardanega T., Prof. Cardin R.
        \end{itemize}
    \subsubsection{Obiettivo}
    L’obiettivo del capitolato è quello di sviluppare un Assistente Virtuale che supporti i clienti nella ricerca di informazioni su prodotti gestiti da aziende esterne, tra le quali quelle riguardanti la vendita di beni alimentari, prodotti per la pulizia o giocattoli. Questo aiuta a semplificare l’accesso alle informazioni sui prodotti, aiutando il cliente e riducendo l’interazioni con gli specialisti. L’assistente virtuale sarà in grado di rispondere alle domande frequenti  e fornire dettagli specifici sui prodotti.
    \subsubsection{Dominio Applicativo}
    Il progetto mira alla creazione di un Assistente Virtuale in aiuto ad aziende esterne specializzate nel settore della vendita al dettaglio e gestione dei prodotti. Gli aspetti principali includono:
    \begin{itemize}
        \item Machine Learning e LLM (Large Language Models): L'assistente utilizzerà modelli avanzati di apprendimento automatico per aiutare il cliente con risposte e informazioni dettagliate sui vari prodotti.
        \item Analisi Predittiva: Il sistema ha lo scopo di applicare tecniche di analisi predittiva per cercare di anticipare future domande proposte dal cliente con l’intento di migliorare l’efficenza e l'esperienza dell’utente durante l’utilizzo del prodotto. Inoltre questa analisi può essere utile anche ai fornitori per gestioni di scorte o approvvigionamenti.
        \item User Experience: L'interfaccia utente mobile sarà progettata per essere user-friendly, facilitando l’esperienza dell’utente durante l’uso dell’app.
        \item Sicurezza e Affidabilità: L'assistente dovrà includere meccanismi di sicurezza per garantire che le informazioni fornite siano accurate e pertinenti.
    \end{itemize}
    \subsubsection{Tecnologie}
        \begin{itemize}
            \item Sql Server Express, MySqL o MariaDB per il database relazionale.
            \item BLOOM, FALCON IA,PYTHIA,ITALIA by iGENIUS, Minerva come LLM.
            \item API REST per la comunicazione tra il modello LLM e l’applicazione di interazione con l’utente.
            \item JSON per la comunicazione da/per il database.
            \item MAUI per lo sviluppo dell’interfaccia  utente.
        \end{itemize}
    \subsubsection{Aspetti positivi}
        \begin{itemize}
            \item L’utilizzo di modelli LLM semplifica l’accesso alle informazioni per il cliente, riducendo la necessità di interazioni umane che può essere un vantaggio sia per eventuali fornitori che per il cliente.
            \item La progettazione del sistema consente di integrare ulteriori modelli di LLM, mantenendo l’estensibilità dell’applicazione.
            \item L’utilizzo di dati pre-esistenti riduce i costi e i tempi di implementazione.
            \item Il meccanismo di feedback con gli utenti permette un continuo miglioramento dell’applicazione oltre a un dialogo con l’utente a scopo migliorativo.
        \end{itemize}
    \subsubsection{Punti deboli}
    \begin{itemize}
            \item L'efficacia del sistema dipende fortemente dalla qualità e completezza dei dati forniti nel database. Dati insufficienti o inaccurati possono portare a risposte errate o fuorvianti.
            \item Sebbene il sistema possa gestire domande comuni, potrebbe avere difficoltà con domande più complesse o situazioni non previste, richiedendo comunque l'intervento umano.
        \end{itemize}
    \subsubsection{Conclusioni}
    Il capitolato ha subito riscosso grande successo all’interno del gruppo, riscontrando grande interesse nella quasi totalità dei membri.  Anche questo capitolato è stato inserito ai vertici della classifica stilata dal gruppo, proprio perchè la potenzialità di aiutare le aziende nel rivoluzionare la comunicazione con il cliente hariscontrato entusiasmo nel gruppo.
