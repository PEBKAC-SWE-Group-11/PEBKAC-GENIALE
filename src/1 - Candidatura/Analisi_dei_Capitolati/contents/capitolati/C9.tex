\subsection{Capitolato C9 - BuddyBot}
     \subsubsection{Informazioni generali}
        \begin{itemize}
            \item \textbf{Titolo}: BuddyBot
            \item \textbf{Proponente}: Azzurrodigitale S.p.A.
            \item \textbf{Committente}: Prof. Vardanega T., Prof. Cardin R.
        \end{itemize}
    \subsubsection{Obiettivo}
    Il progetto ha come obiettivo la creazione di un assistente virtuale basato su un LLM che miri ad ottimizzare il knowledge management aziendale. L’assistente è pensato per semplificare e centralizzare l’accesso alle informazioni distribuite su diverse  piattaforme.\\
    L’obiettivo è migliorare la produttività riducendo il tempo necessario per il recupero delle informazioni e facilitando l’onboarding dei nuovi membri.
     \subsubsection{Dominio Applicativo}
    \begin{itemize}
        \item Integrazione delle piattaforme:
        \begin{itemize}
            \item GitHub per la gestione del codice sorgente.
            \item Confluence per documentazione e specifiche di progetto.
            \item Jira come piattaforma di ticketing per la gestione dei task.
            \item Slack e Telegram (opzionali) per la comunicazione aziendale.
        \end{itemize}
        \item Funzionalità principali:
        \begin{itemize}
            \item Risposte in linguaggio naturale a domande su codice, task e specifiche di progetto.
            \item Centralizzazione delle informazioni aziendali su un’unica piattaforma accessibile tramite chat.
            \item Accesso rapido alle informazioni critiche per migliorare la produttività e la collaborazione.
            \item Assistenza ai nuovi membri del team con domande frequenti e risposte automatiche.
        \end{itemize}
    \end{itemize}
    \subsubsection{Tecnologie (consigliate)}
    \begin{itemize}
        \item Tecnologie IA:
        \begin{itemize}
            \item OpenAI per l'elaborazione del linguaggio naturale.
            \item Langchain per l’integrazione di modelli di IA.
        \end{itemize}
        \item Front-end:
        \begin{itemize}
            \item Angular per costruire un'interfaccia utente modulare e dinamica.
        \end{itemize}
        \item Back-end:
        \begin{itemize}
            \item Node/NestJS o Spring Boot per lo sviluppo server-side, con supporto per API RESTful.
        \end{itemize}
        \item Persistenza dei dati:
        \begin{itemize}
            \item Database per la conservazione di storico chat, domande e risposte.
        \end{itemize}
    \end{itemize}
    \subsubsection{Punti di forza}
    \begin{itemize}
        \item BuddyBot rende disponibile in un’unica piattaforma dati provenienti da fonti diverse, riducendo il tempo necessario per reperire informazioni.
        \item Il bot fornisce assistenza ai nuovi membri del team, migliorando la qualità e la velocità del loro inserimento.
        \item Con l’architettura modulare e l’uso di API, il bot può essere facilmente ampliato per supportare altre piattaforme aziendali.
    \end{itemize}
    \subsubsection{Punti deboli}
    \begin{itemize}
        \item Usare modelli di linguaggio closed source come quelli di OpenAI può portare a costi elevati, problemi di privacy e in generale dipendenza dai fornitori del servizio.
        \item La compatibilità con tante piattaforme richiede manutenzione costante.
    \end{itemize}
    \subsubsection{Conclusioni}
    Il progetto non ha suscitato particolare interesse nel gruppo, essendo simile ad altri capitolati ritenuti più stimolanti.


% à, è, ì, ò, ù,
% À, È, Ì, Ò, Ù

% á, é, í, ó, ú, ý
% Á, É, Í, Ó, Ú, Ý