\subsection{Capitolato C3 - Automatizzare le routine digitali tramite l’intelligenza generativa}
    \ \subsubsection{Informazioni generali}
        \begin{itemize}
            \item \textbf{Titolo}:Automatizzare le routine digitali tramite l’intelligenza generativa
            \item \textbf{Proponente}: Var Group S.p.A.
            \item \textbf{Committente}: Prof. Vardanega T., Prof. Cardin R.
        \end{itemize}
     \subsubsection{Obiettivo}
   Automatizzare le routine digitali tramite l’intelligenza generativa. Le aziende sono sempre alla ricerca di modi più efficienti per svolgere operazioni quotidiane, spesso ripetitive e dispendiose in termini di tempo. Con l’introduzione dell’intelligenza generativa in cloud diventa possibile l’automazione delle routine digitali, che permette la riduzione dell’intervento umano e l’aumento della produttività.

     \subsubsection{Dominio Applicativo}
    Il progetto prevede la creazione di un sistema di automazione che integra l’intelligenza artificiale generativa in cloud, per semplificare varie attività digitali quotidiane, grazie alla creazione di un’interfaccia utente intuitiva e alla possibilità di impostare workflow personalizzati.
Gli obiettivi di questo sistema sono:

\begin{itemize}
    \item Creare un’infrastruttura cloud che sfrutti i sistemi Generative AI di AWS.
    \item Creare un client per Windows/Mac per disegnare e personalizzare i flussi di automazione tramite drag-and-drop e linguaggio naturale.
    \item Sviluppare le logiche di integrazione tra ambiente locale e servizi cloud.
    \item Mappare i limiti e le criticità di queste soluzioni.
\end{itemize}
    \subsubsection{Tecnologie}
    Il progetto utilizza:

\begin{itemize}
    \item MongoDB per la memorizzazione di dati non strutturati e per l’archiviazione locale.
    \item Python o C\# per lo sviluppo del frontend per Windows.
    \item React per lo sviluppo dell’interfaccia utente per Windows.
    \item Swift per lo sviluppo del frontend per Mac.
    \item SwiftUI per lo sviluppo dell’interfaccia utente per Mac.
    \item Node.js, Python e TypeScript per lo sviluppo di API cloud.
    \item AWS come infrastruttura cloud per i propri sistemi di Generative AI.
\end{itemize}
    \subsubsection{Punti di forza}
    \begin{itemize}
    \item Gli utenti possono creare flussi di lavoro personalizzati tramite un’interfaccia drag-and-drop e linguaggio naturale, rendendo il sistema flessibile e facile da usare.
    \item Il sistema può automatizzare attività ripetitive, consentendo alle aziende di migliorare la gestione del tempo e la produttività.
    \item L’applicativo verrà realizzato in ottica modulare per permettere estensioni delle funzioni della piattaforma in futuro.
\end{itemize}
    \subsubsection{Punti deboli}
    \begin{itemize}
    \item Il progetto necessità dell’utilizzo di AWS, limitando così la portabilità in altre infrastrutture cloud.
    \item Il progetto richiede di sviluppare client sia per Windows che per Mac utilizzando linguaggi differenti, comportando un aumento del tempo di sviluppo, dato che ogni versione deve essere sviluppata, testata e aggiornata separatamente.
    \item Richiedendo competenze specifiche in un grande numero di linguaggi, è necessario un tempo più lungo di formazione per gli sviluppatori.
    \item La gestione delle funzionalità e dell’esperienza utente tra le due versioni del client può risultare complessa, aumentando il rischio di avere delle incoerenze e dei disallineamenti.
    \item Uno degli obiettivi del progetto è mappare le criticità e i limiti di queste soluzioni, ma il capitolato si limita a menzionare la cosa, senza dare ulteriori informazioni.
\end{itemize}
    \subsection{Conclusioni}
   Questo capitolato è risultato inizialmente interessante per alcuni membri del gruppo, ma è stato successivamente scartato perché ritenuto particolarmente complesso e perché richiede la conoscenza di molti linguaggi differenti, che avrebbero richiesto parecchio tempo di training per la maggior parte del gruppo.
