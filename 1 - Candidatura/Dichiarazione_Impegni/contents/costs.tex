\section{Preventivo costi}

Sulla base della precedente analisi è emersa la seguente distribuzione orari con il corrispondente prospetto economico:

\begin{table}[H]
    \centering
    \renewcommand{\arraystretch}{1.5}
    \arrayrulecolor{black} % Colore del bordo della tabella
    % Definiamo la colonna grigio chiaro per la prima e l'ultima colonna
    \begin{tabular}{|>{\bfseries}c|c|c|c|c|c|c|>{\bfseries}c|}
        \hline
        % Prima riga - Grigio molto scuro, testo bianco in grassetto
        \rowcolor{gray!70} 
        \color{white}\textbf{Ruolo} & \color{white}\textbf{Costo orario (\euro/h)} & \color{white}\textbf{Ore (h)} & \color{white}\textbf{Costo (\euro)} \\
        \hline
        % Righe alternate con bianco e grigio chiaro
        \color{black}\textbf{Responsabile} & 30 & 68 & 2040  \\ 
        \hline
        \rowcolor{gray!10} % Grigio molto chiaro per riga alternata
        \color{black}\textbf{Amministratore} & 20 & 53 & 1060 \\ 
        \hline
        \color{black}\textbf{Analista} & 25 & 57 & 1425  \\ 
        \hline
        \rowcolor{gray!10} % Grigio molto chiaro per riga alternata
        \color{black}\textbf{Progettista} & 25 & 106 & 2650  \\ 
        \hline
         \color{black}\textbf{Programmatore} & 15 & 180 & 2700 \\ 
        \hline
        \rowcolor{gray!10} % Grigio molto chiaro per riga alternata
        \color{black}\textbf{Verificatore} & 15 & 180 & 2700 \\ 
        \hline

        % Ultima riga - Grigio molto scuro, testo bianco in grassetto
        \rowcolor{gray!70} 
        \color{white}\textbf{Totale} & \color{white} & \color{white}644 & \color{white}12575  \\ 
        \hline
    \end{tabular}
    \caption{Tabella con il prospetto economico}
\end{table}