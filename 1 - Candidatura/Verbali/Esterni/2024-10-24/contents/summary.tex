\section{Riassunto della riunione}

    La riunione aveva l'obiettivo di chiarire alcuni punti del capitolato C7 - LLM: ASSISTENTE VIRTUALE di Ergon Informatica. Le domaande e relative risposte sono state:
\begin{itemize}
    \item \textbf{L'applicazione deve essere disponibile in italiano o anche in altre lingue?} L'applicazione è richiesta come requisito minimo in italiano, senza eslcudere la possibilità di implemetare altre lingue
    \item \textbf{Dal capitolato si evince che l'applicazione comunicherà con un Web Server, questo sembrerebbe escludere la possibilità di utilizzo dell'applicazione offline.} Non è richiesto che l'applicazione funzioni offline, la comunicazione con un Web Server è necessaria.
     \item \textbf{Per l'applicazione è consigliato l'utilizzo di .NET MAUI, come mai questa scelta? L'azienda ha già lavorato con MAUI?} L'azienda lavora molto con il framework .NET e ha un team specilizzato in .NET MAUI, divenuto per l'azienda la tecnologia principale per lo sviluppo di applicazioni multi-piattaforma dopo la fine del supporto di Xamarin. Comunque l'utilizzo di MAUI ed il tipo di applicazione da creare è trattabile. 
            \item \textbf{Quindi potrebbe essere valutabile anche un Web Application, magari sviluppata con Blazor per rimanere all'interno del framework .NET?} Si, anche se è una soluzione che non era stata presa in considerazione se ne può discutere.

\end{itemize}


