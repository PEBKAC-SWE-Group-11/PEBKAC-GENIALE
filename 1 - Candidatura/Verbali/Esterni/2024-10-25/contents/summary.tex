\section{Riassunto della riunione}
L'obiettivo dell'incontro  è stato quello di chiarire e/o approfondire alcuni temi emersi da una prima analisi del capitolato C5 - 3Dataviz di Sanmarco Informatica.
Vengono riportate di seguito le domande principali insieme ad un riassunto delle loro risposte:
\begin{itemize}
    \item \textbf{Oltre agli istogrammi 3D è richiesta la visualizzazione, ad esempio, di grafici a dispersione 3D e/o grafici di funzione a 2 variabili?}\\
    No. Il progetto è orientato ai soli istogrammi, infatti alcune funzionalità richieste trovano applicazione tangibile solo nel ``campo" degli istogrammi.
    \item \textbf{È richiesto che l'interfaccia Web sia accessibile alle persone con disabilità visiva?}\\
    No, in questo progetto la visualizzazione del grafico è un elemento essenziale. Uno screen reader potrebbe essere in grado di leggere i dati inseriti in tabella ma all'aumentare di questi diventerebbe comunque inutilizzabile.\\
    Ad oggi esistono sistemi basati sull'IA che aiutano le persone con problemi di vista ad interpretare i dati di una pagina, ma questo da solo potrebbe essere un progetto a parte.
    \item \textbf{La visualizzazione deve essere responsive per dispositivi diversi?}\\
    No, per la natura stessa del progetto si presuppone l'utilizzo di un computer.\\
    Similmente a prima, su uno smartphone potrebbe essere fattibile visualizzare la tabella dei dati ma sarebbe complicato usare le funzioni tridimensionali.
    \item \textbf{Se si sceglie di utilizzare le API come fonte dei dati, è importante facilitare l'integrazione dell'alternativa, un database, e viceversa?}\\
    Sì, questo fa anche parte di ciò che studierete nel corso di Ingegneria del Software. È anche importante che il progetto sia facilmente adattabile ad API e Database diversi.
    \item \textbf{Come dovranno essere gestiti gli errori nei dati inseriti?}\\
    Potete vedere l'interfaccia come ``dummy" e assumere che la gestione degli errori sia fatta tutta dal lato backend. In pratica potete assumere che le API e i Database vi condividano solo dati corretti.
\end{itemize}


% à, è, ì, ò, ù,
% À, È, Ì, Ò, Ù

% á, é, í, ó, ú, ý
% Á, É, Í, Ó, Ú, Ý