\section{Riassunto della riunione}
\begin{itemize}
    \item Scelta del nome ``\textit{PEBKAC - Problem Exists Between Keyboard And Chair}", commissionamento del logo e creazione della casella di posta elettronica \href{mailto:pebkacswe@gmail.com}{pebkacswe@gmail.com}
  \item Discusso  l'utilizzo degli strumenti per la gestione del progetto
  \begin{itemize}
    \item \textbf{Slack} come principale piattaforma di comunicazione e interscambio tra i membri del gruppo;
    \item \textbf{Trello}, integrato in Slack, per l'assegnazione dei task,  per quanto riguarda i documenti precedenti all'assegnazione di un capitolato, in una board divisa nella colonne \textit{To do, In Progress, Verify, Approve \& Release}. Ogni task in \textit{Todo} o \textit{Verify} può essere presa in carico da qualsiasi membro del gruppo, in linea con stato della task, ruolo e/o altri accordi preesistenti. Ogni nuovo task non può essere eliminata ma solo avanzare fino alla chiusura;
    \item \textbf{Google Calendar}, integrato con Slack, per l'organizzazione degli impegni;
    \item \textbf{Latex}, per la redazione dei documenti di progetto;
  \item \textbf{GitHub}, per la condivisione e il versionamento della/e repository.
  \end{itemize}
    \item Discusso i capitolati preferiti dal gruppo, emergono, nell'ordine:
    \begin{enumerate}
        \item C2 - Vimar GENIALE (Vimar)
        \item C5 - 3Dataviz (Sanmarco Informatica)
        \item C7 - LLM: Assistente Virtuale (ERGON)
    \end{enumerate}
  \item Assegnati dei ruoli ai componenti, individuando un responsabile, un amministratore e due verificatori. Per la prima fase di lavoro, prima dell'assegnazione del capitolato, si sceglie di dare la possibilità a tutti i membri di analizzare i capitolati.
  \item Fissato un prossimo incontro certo con tutti i membri del gruppo per il 27/10/2024, in previsione del primo diario di bordo.
  

\end{itemize}